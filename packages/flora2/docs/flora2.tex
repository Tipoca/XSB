\documentclass[11pt]{article}

\usepackage{epsf,epsfig,subfigure,latexsym,makeidx,latexsym,xspace,amssymb,alltt}
\usepackage{hyperref}

\pagestyle{headings}
\bibliographystyle{plain}

%%\setlength{\topmargin}{-0.25in}
\setlength{\headheight}{10pt}
\setlength{\headsep}{30pt}
\setlength{\oddsidemargin}{0.0in}
\setlength{\evensidemargin}{0.0in}
\setlength{\textheight}{8.5in}
\setlength{\textwidth}{6.5in}
\setlength{\footskip}{50pt}


\setlength{\parskip}{2mm}               % space between paragraphs

\def\cut{\mbox{\tt '!'/0}}

\newtheorem{example}{Example}[section]

\newenvironment{Prog}{\begin{tt}\begin{tabular}[c]{l}}{\end{tabular}\end{tt}}

\newcommand{\comment}[1]{}

\newcommand{\demo}[1]{\hspace*{1.5cm}{\tt #1}}
\newcommand{\desc}[1]{\item[{\tt #1}]\hspace*{1mm}\newline}
\newcommand{\desce}[1]{\item[{\tt #1}]}
\newcommand{\ourrepeatitem}[1]{\item[{\mbox{\tt #1}}]\ \\ \vspace*{-.35in}}
\newcommand{\ouritem}[1]{\item[{\mbox{\tt #1}}]\ \\}
\newcommand{\ournewitem}[2]{\item[{\mbox{\tt #1}}]\hspace*{\fill}{\mbox{\tt #2}}\ \\}

\newcommand{\stuff}[1]{
        \begin{minipage}{4in}
        {\tt \samepage
        \begin{tabbing}
        \hspace{8mm} \= \hspace{6mm} \= \hspace{10mm} \= \hspace{55mm} \= \kill
        #1 \hfill
        \end{tabbing}
        }
        \end{minipage}
}

\newcommand{\longline}{\noindent\rule{\textwidth}{.01in}}


\newenvironment{qrules}{\begin{quote}\tt\begin{tabular}[t]{l}}%
{\end{tabular}\end{quote}}


\newcommand{\obj}{\textit{obj}\xspace}
\newcommand{\db}[1]{\ensuremath{\mathcal{#1}}}

\newcommand{\xany}{\textsf{any}}

\newcommand{\xplus}{\ensuremath{^+}}
\newcommand{\xstar}{\ensuremath{^*}}
\newcommand{\xinv}{\ensuremath{^{-1}}}
\newcommand{\xopt}{\ensuremath{^{?}}}

\newcommand{\xto}[1]{\ensuremath{^{#1}}}
\newcommand{\xcond}[1]{\ensuremath{\textsf{if}(#1)}}
\newcommand{\xif}[1]{\ensuremath{\textsf{if}(#1)}}
\newcommand{\xmu}[1]{\ensuremath{\tcmu(#1)}}
\newcommand{\xmuif}[2]{\ensuremath{\tcmu(#1,#2)}}


\newcommand{\xconc}{\ensuremath{{\cdot}}}
\newcommand{\xor}{\ensuremath{|}}

\newcommand{\nnot}{\mbox{$\neg$}}                           % negation
\newcommand{\query}{\mbox{$\, ?\! - \, $}}                  % query
\newcommand{\impl}                                          % implication
  {\mbox{\Large $\; {\bf \leftarrow} \;$}}  
\newcommand{\isa}{\,{\bf{:}}\,}
\newcommand{\subcl}{\,{\bf{::}}\,}
\newcommand{\eq}{\ensuremath{\doteq}}                           % equation

% f-logic arrows

\newcommand{\fd}{\ensuremath{{\rightarrow}}}                   % scalar
\newcommand{\bfd}{\ensuremath{{\bullet\!\!\!\fd}}}            % " + inheritable
\newcommand{\mvd}{\ensuremath{{\rightarrow\!\!\!\!\rightarrow}}}  % multivalued
\newcommand{\bmvd}{\ensuremath{{\bullet\!\!\!\mvd}}}              % " + inheritable
\newcommand{\Fd}{\ensuremath{{\Rightarrow}}}                      % scalar signature
\newcommand{\Mvd}{\ensuremath{{\Rightarrow\!\!\!\!\Rightarrow}}}  % multiv signature



% curved f-logic arrows

\newcommand{\anyd}{\ensuremath{\leadsto}}                       % noninheritable
\newcommand{\bleadsto}{\ensuremath{\bullet\!\!\!\leadsto}}     % inheritable
\newcommand{\banyd}{\bleadsto}                              % "
\newcommand{\Leadsto}{\ensuremath{\approx}\!\!{>}}            % signature
\newcommand{\Anyd}{\Leadsto}                                % "

\newcommand{\FdConstr}{\ensuremath{\stackrel{constr}{\Fd}}}
\newcommand{\MvdConstr}{\ensuremath{\stackrel{constr}{\Mvd}}}

\newlength{\flogicindent}


\newlength{\flength}
\newlength{\counterlength}


\newcommand{\la}{\ensuremath{\,\leftarrow\,}}

\newcommand{\anon}{\_}

\newcommand{\note}[1]{\textit{[[#1]]}}
\newcommand{\nterm}[1]{\ensuremath{\langle}\textit{#1}\ensuremath{\rangle}}



\newcommand{\bs}{\ensuremath{\backslash}}
\newcommand{\FLIP}{{\mbox{\sc Flip}}\xspace}
\newcommand{\FLORA}{{\mbox{${\cal F}${\small\it LORA}\rm\emph{-2}}}\xspace}
\newcommand{\FLORAone}{{\mbox{${\cal F}${\sc lora}}}\xspace}
\newcommand{\FLORID}{{\mbox{\sc Florid}}\xspace}
\newcommand{\fl}{\mbox{F-logic}\xspace}
\newcommand{\NAF}{{$\tt\backslash +$}\xspace}


\newcommand{\consts}{\ensuremath{\mathcal{C}}}
\newcommand{\funcs}{\ensuremath{\mathcal{F}}}
\newcommand{\preds}{\ensuremath{\mathcal{P}}}
\newcommand{\vars}{\ensuremath{\mathcal{V}}}

\newcommand{\HU}{\ensuremath{U}}
\newcommand{\HB}{\ensuremath{\mathcal{HB}}}
\newcommand{\ext}{\ensuremath{^{\star}}}

\newcommand{\bksl}{\symbol{92}}
\newcommand{\dq}{\symbol{34}}


\title{\FLORA User's Manual}

\author{
  {Guizhen Yang
  \hspace{3cm}
  Michael Kifer}
  \\\\
  Department of Computer Science\\
  University at Stony Brook\\
  Stony Brook, NY 11794-4400
  }
  
\makeindex
\begin{document}

\maketitle
\thispagestyle{empty}

\newpage
\pagenumbering{roman}
\setcounter{page}{1}

\tableofcontents

\newpage

\pagenumbering{arabic}
\setcounter{page}{1}


\section{Introduction}

\FLORA is a sophisticated compiler and application development platform
that translates a unified language of \fl \cite{KLW95}, HiLog
\cite{hilog-jlp}, and Transaction Logic \cite{trans-tcs94} into XSB. It
takes a program written in the \fl language with HiLog and Transaction
Logic extensions (which must be in a file with extension {\tt .flr}, {\it
  e.g.}, {\tt file.flr}) and outputs a regular XSB program (with extension
{\tt .P}).  This program is then passed to XSB for compilation (which
produces {\tt file.O}) and execution.

\index{FLIP}
\index{FLORID}
%%
\FLORA was implemented by Guizhen Yang, but its origins trace back to the
\FLIP compiler developed by Bertram Lud\"aescher.  The programming language
supported by \FLORA is a dialect of \fl with numerous extensions.  Some
extensions are borrowed from \FLORID, a C++-based \fl system developed at
Freiburg University.\footnote{
  %%
  See {\tt http://www.informatik.uni-freiburg.de/$\sim$dbis/florid/} for more
  details.
  %%
  }
%%
In particular, \FLORA fully supports the versatile syntax of \FLORID path
expressions. Other important extensions are motivated by the need to
support a flexible module system (and enable modular software development
in \fl) and in order to support HiLog \cite{hilog-jlp} and Transaction
Logic \cite{trans-dbpl93,trans-iclp93,trans-tcs94}, both of which are
smoothly integrated with \fl. Extensions aside, the syntax of \FLORA also
differs in some important ways from \FLORID, from the original version of
\fl, as described in \cite{KLW95}, and from an earlier implementation
of \FLORAone. These syntactic changes were needed in order to bring the
syntax of \FLORA closer to that of Prolog and make it possible to include
typical Prolog programs into \FLORA programs without choking the compiler.
Other syntactic deviations from the original F-logic syntax are a direct
consequence of the added support for HiLog, which obviates the need for the
``@'' sign in method invocations (this sign is now used to denote calls to
other modules).

\FLORA is part of the official distribution of XSB beginning with version
2.4. It is organized as an XSB package and lives in the directory
%%
\begin{quote}
 \verb|<xsb-installation-directory>/packages/flora2/|  
\end{quote}
%%
\FLORA is fully integrated into the XSB system, including its module
system. In particular, \FLORA modules can invoke predicates defined in
other XSB modules, and regular XSB modules can query the objects defined in
\FLORA modules. At present, XSB is the only platform where \FLORA can run,
because it heavily relies on tabling and the well-founded semantics for
negation, both of which are available only in XSB.

Due to certain problems with XSB, \FLORA runs best when XSB is configured
with SLG-WAM and local scheduling. This problem will be fixed in a future
release of XSB. Under the default XSB configuration, certain \fl programs might
run 30 times slower than under the suggested configuration. Some \FLORA
programs might also give errors when run under the default configuration.
To configure XSB for optimal \FLORA performance, build XSB as follows:
%%
{\tt
\begin{quote}
 configure --with-local-scheduling --enable-slg-wam --config-tag=\\
 makexsb
\end{quote}
}
%%
In the current release of \FLORA, it does not come pre-built, so you must
build it after installing and compiling XSB:
%%
\begin{verbatim}
   cd <xsb-installation-directory>/packages/flora2/
   make clean
   make
\end{verbatim}
%%

The fastest and easiest way to get a feel of the system
is to start \FLORA shell and begin to enter queries interactively.  To
this end, you must first invoke XSB and then load the {\tt flora2}
package:
%%
\begin{quote}
  \tt
foo>~~xsb  \\
\tt
... XSB loading messages omitted ...\\
\tt
| ?- [flora2].\\
\tt
[flora2 loaded]\\
\tt
| ?-
\end{quote}
%%
At this point, it is possible to use a limited number of \FLORA
commands, but to run queries you must enter the \FLORA command loop:
%%
\begin{quote}
  \tt
| ?- flora\_shell.  \\
 \tt
... FLORA messages omitted ... \\
 \tt
flora2 ?-
\end{quote}
%%

At this point, \FLORA takes over and \fl syntax becomes the
norm. To get back to the XSB command loop, type {\tt Control-D} or 
%%
\begin{quote}
  \tt
| ?- end.  
\end{quote}
%%

If you are using \FLORA shell frequently, it pays to define an alias, say,
%%
\begin{quote}
 {\tt
   alias flora2='xsb -e "[flora2], flora\_shell."'
   }
\end{quote}
%%
\FLORA can then be invoked directly from the shell prompt by typing
\begin{quote}
  \tt
foo>~~flora2
\end{quote}
%%
It is even possible to tell it to execute commands on start-up.
For instance, 
%%
\begin{quote}
 \tt
 foo>~~flora2 -e "flHelp."
\end{quote}
%%
will cause the system to execute the help command right after the start.
Then the usual \FLORA shell prompt is displayed.

\noindent
\FLORA comes with a number of demo programs that live in
%%
\begin{quote}
 \verb|<xsb-installation-directory>/packages/flora2/demos/|  
\end{quote}
%%
The demos can be run by issuing the command
``\verb|flDemo(demo-filename).|''
at the \FLORA prompt, {\it e.g.},
%%
\begin{quote}
 \verb|flora2 ?- flDemo(flogic_basics).|
\end{quote}
%%
There is no need to change to the demo directory, as {\tt flDemo} knows
where to find the demos.


\section{\FLORA Shell Commands} \label{sec-shell-commands}

The most common shell command you might need to execute is loading and
compiling a program:
%%
\begin{quote}
  flora2 ?-  {\tt [programfile].}
\end{quote}
%%
or 
%%
\begin{quote}
  flora2 ?- {\tt flLoad programfile.}
\end{quote}
%%
Here {\tt program-file} can contain a \FLORA program or an XSB program. If
{\tt program-file.flr} exists, it is assumed to be a \FLORA program. The
system will compile the program, if necessary, and then load it. The
compilation process is two-stage: first, the program is compiled into a
Prolog program (one or more files with extensions {\tt .P} and {\tt .fdb})
and then into an executable byte-code, which has the extension {\tt .O}.

If there is no {\tt program-file.flr} file, the file is assumed to contain
an XSB program and the system will look for the file named {\tt
  program-file.P}. This file then is compiled into {\tt program-file.O} and
loaded.

By default, all \FLORA programs are loaded into the module {\tt main}, but
you can also load them into other modules using the following syntax:
%%
{\tt
\begin{quote}
 flora2 ?-  [file>>modulename].
\end{quote}
}
%%
Understanding \FLORA modules is very important in order to be able to take
full advantage of the system; we will discuss the module system in \FLORA
in Section~\ref{sec:flora-modules}.  Once the program is loaded, you can
pose queries and invoke methods for the objects defined in the program.

There is an important special of the {\tt flLoad} and {\tt [...]} command
when the file name is {\tt user}. In that case, instead of looking for the
program file {\tt user.flr}, \FLORA starts reading user input. At this
point, the user can start typing in program clauses, which the system saves
in a temporary file. When the user is done and types {\tt Control-D} (end
of file), the file is compiled and loaded. It is also possible to load such
a program into a designated module, rather than the default one,
using one of the following commands:
%%
{\tt
\begin{quote}
  flora2 ?- [file>>module].\\
  flora2 ?- flLoad file>>module.
\end{quote}
}
%%

\index{don't care variable}
\index{anonymous variable}
\index{variable!don't care}
\index{variable!anonymous}
%%
When the user types in a query to the shell, the query is evaluated and the
results are returned. A result is a tuple of values for each variable
mentioned in the query, except for the \emph{anonymous variables}
represented as ``{\tt \_}'' and named {\rm don't care variables}, which are
preceded with the underscore, {\it e.g.}, {\tt \_abc}.

By default, \FLORA prints out all answers. If only one is desired, type in
the following command: {\tt flOne}. You can revert back to the all-answers
mode by typing {\tt flAll}.

\FLORA shell includes many more commands beyond those mentioned above.
These commands are listed below. However, at this point the purpose of some
of these commands might seem a bit cryptic, so it is a good idea to come
back here after you become more familiar with the various concepts
underlying the system.

In the following command list, the suffixes {\tt .flr} {\tt .P}, {\tt .O}
are optional. If the file suffix is specified explicitly, the system uses
the file with the given name without any modification. The {\tt .flr}
suffix denotes a \FLORA program, the {\tt .P} suffix indicates that it is an
XSB program, and {\tt .O} means that it is a bytecode file, which can be
executed by XSB.  If no suffix is given, the system assumes it is dealing
with a \FLORA program and adds the suffix {\tt .flr}. If the file with such a
name does not exist, it assumes that the file contains an XSB program and
tries the suffix {\tt .P}. Otherwise, it tries {\tt .O} in the hope that an
executable XSB bytecode exists. If none of these tries are successful, an
error is reported.
%
\begin{itemize}
\item {\tt flHelp}:
    Show the help info.
\item {\tt flCompile FILE}:
    Compile FILE.flr for the default module `main'.
\item {\tt flCompile FILE>>Module}:
    Compile FILE.flr for the module `{\tt Module}'.
\item {\tt flLoad FILE>>Module}:
    Load FILE.flr into the module `{\tt Module}'. If you specify 'FILE.P'
    or 'FILE.O' then will load these files.
\item {\tt flLoad FILE}:
    Load FILE.flr into the default module `main'. If you specify 'FILE.P'
    or 'FILE.O' then will load these files.
\item {\tt [FILE.\{P$|$O$|$flr\} $>>$ Module,...]}:
    Load the files in the specified list into the module `{\tt Module}'.
\item {\tt flDemo(FILE)}:
    Consult a demo from \FLORA demos directory.
\item {\tt equality \{none$|$basic$|$flogic\}}:
    Set the level of support for the equality predicate {\tt :=:} in the
    shell module `main'.
    {\tt none}  means that {\tt :=:} is treated as a regular
    predicate; {\tt basic} means that only standard first-order equality is
    supported ({\it i.e.}, the usual congruence rules); {\tt flogic} means
    that F-logic style equality is supported ({\it i.e.}, congruence plus
    the axiom for scalar methods).
\item {\tt abolish\_all\_tables}:
    Flush all tabled data. This is sometimes needed when XSB's tabling 
    gets in the way. We describe tabling (as it pertains to \FLORA) in
    Section~\ref{sec-tabling-flora}.
\item {\tt firstorder Functor/Arity}:
    Define Functor/Arity as non-HiLog in shell mode.
\item {\tt arguments Functor(\{oid$|$form\}, ...)}:
    Define the predicate meta signature in shell mode.
\item {\tt op(Precedence,Associativity,Operator)}:
    Define an operator in shell mode.
\item {\tt flReset(\{firstorder|arguments|op\})}:
    Clear all dynamic {\tt firstorder/arguments/op}  definitions in the
    \FLORA shell.
\item {\tt flAll}:
    Show all solutions (default).
\item {\tt flOne}:
    Show solutions one by one.
\item {\tt flMaxerr(all$|$N)}:
    Set/show the maximum number of errors \FLORA reports.
\item {\tt flTrace/flNoTrace}:
    Turn on/off \FLORA trace.
\item {\tt flChatter/flNoChatter}:
    Turn on/off the display of the number of solutons at the end of query
    evaluation.
\item {\tt flEnd}:
    Say Ciao to \FLORA, stay in XSB.
\item {\tt flHalt}:
    Quit both \FLORA and XSB.
\end{itemize}
%

All commands with a FILE argument passed to them use the XSB
{\tt library\_directory} predicate to search for the file, except that the
command {\tt flDemo(FILE)} first looks for {\tt FILE} in the \FLORA demo
directory. The search path typically includes the standard system's
directories used by XSB followed by the current directory. 

All XSB commands can be executed from \FLORA shell, if the corresponding
XSB library has already been loaded.

After a syntax, parsing, or compilation error, \FLORA shell will
discard tokens read from the current input stream until the end of file or a
rule delimiter (``.'') is encountered. If \FLORA shell seems to be hanging
after the message
\begin{quote}
\begin{verbatim}
++FLORA Warning: discarding tokens (rule delimeter `.' or EOF expected)
\end{verbatim}
\end{quote}
%%
hit the {\tt Enter} key once, type ``.'', and then {\tt Enter} again.  This
should reset the current input buffer and you should see the \FLORA command
prompt:
\begin{quote}
\begin{verbatim}
flora2 ?-
\end{verbatim}
\end{quote}

 
\section{\fl and \FLORA by Example}


In the future, this section will contain a number of small
introductory examples illustrating the use of \fl and \FLORA. Meanwhile, the
reader is referred to the excellent tutorial written by the members of the
\FLORID project.\footnote{
  %%
  See {\tt http://www.informatik.uni-freiburg.de/$\sim$dbis/florid/} for more
  details.
  %%
  }
%%
Since \FLORA and \FLORID share much of the same syntax, most examples in that
tutorial can be made into valid \FLORA programs by changing the separator
``;'' used in F-molecules into ``,'' and by eliminating the ``@''
sign in method invocations.



\section{Basic \FLORA Syntax}

In this section we describe the basic syntactic structures used to build
\FLORA programs. Subsequent sections describe the various advanced features
that are needed to build practical applications.


\subsection{\fl Vocabulary}\label{sec-basic-flogic}


\begin{itemize}
\item \emph{Symbols}: The \fl alphabet of \emph{object constructors}
  \index{object constructor}
  consists of the sets \funcs (function symbols), \preds (predicate symbols
  including $=$), and \vars (variables).  Variables begin with a
  capitalized letter or an underscore, followed by zero or more letters
  and/or digits and/or underscores (e.g., $\tt X, Name, \_, \_v\_5$).
  All other symbols, including the constants (which are 0-ary object
  constructors), are symbols that start with a lowercase letter (e.g., {\tt a,
  john}). Constants can also be any string of symbols enclosed in single
  quotes (e.g., {\tt 'AB@*c'}). 
  In addition to the usual first-order connectives and symbols, there is a
  number of special symbols:
  ], [, \}, \{, ``,'', ``;'', \#, \_\#, \fd, \mvd, \Fd,
  \Mvd, \isa, \subcl. Later we will explain other symbols introduced by
  the inheritance mechanism.
  
\item \emph{Variables}: The variable ``\_'' is called \emph{anonymous}
  variable. It is used whenever a \emph{unique} new variable is needed.
  In particular, two different occurrences of ``\_'' in the same clause are
  treated as different variables. The variables that start with an
  underscore, e.g., {\tt \_foo}, are \emph{not} anonymous and two
  different occurrences of such a variable in the same clause refer to the
  same variable. Nevertheless, such variables have special status as far as
  error checking is concerned. The practice of logic programming shows that
  a singleton occurrence of a variable in a clause is often a mistake
  due to misspelling. Therefore, \FLORA issues a warning when
  it finds that some variable is mentioned only once in a clause. If such
  an occurrence is truly intended, it must be replaced by an anonymous
  variable or a variable that begins with the underscore to avoid the
  warning message from \FLORA.

  %%
  \index{Id-term}
  \index{oid}
  \index{object identifier}
\item \emph{Id-Terms/Oids}:
    First-order terms over \funcs\ and \vars\ are called \emph{Id-terms},
    and are used to name objects, methods, and classes.  Ground Id-terms
    (i.e., terms with no variables) correspond to \emph{logical
      object identifiers} (\emph{oid}s), also called object \emph{names}.
    Numbers (including integers and floats) can also be used as Id-terms,
    but such use might be confusing and is not recommended.
  \index{atomic formula!in \fl}
\item \emph{Atomic formulas}: Let $\tt O,M,R_{i},X_{i},C,D,T$ be Id-terms.  In
  addition to the usual first-order atomic formulas, like
  $p(X_1,\dots,X_n)$, there are the following basic types of formulas:
  \medskip

  \begin{enumerate}
    \item \label{eq-scalar-atom} $\tt O[M\fd V]$
    \item $\tt O[M\mvd \{V_1,\dots,V_n\}]$
    \item $\tt C[M\Fd T]$
    \item $\tt C[M\Mvd \{T_1,\dots,T_m\}]$
  \end{enumerate}
  
  \index{data atom}
  \index{atom!data}
  \index{method}
  \index{method!single-valued}
  \index{method!scalar}
  \index{method!multi-valued}
  %
  In all of the above cases, {\tt O}, {\tt C}, {\tt M}, ${\tt V_i}$, and
  ${\tt T_i}$ are HiLog terms, {\it i.e.}, expressions of the form, $\tt a$,
  $\tt f(X)$, $\tt X(s,Y)$, $\tt X(f,Y)(X,g(k))$, etc., where $\tt X$
  and $\tt Y$ are variables and lowercase letters $\tt f$, $\tt s$, etc., are
  constants.
  
  Expressions (1) and (2) above are \emph{data atoms}, which specify that a
  \emph{method expression} $\tt M$ applied to an object $\tt O$ yields the
  result object $\tt V$ in case (1), or a set of objects, $\tt V_1$, ...,
  $\tt V_n$, in case (2). Thus, in (1), $\tt M$ is said to be a
  \emph{single-valued} (or \emph{scalar}) method expression, i.e., there is
  at most one $\tt V$ such that $\tt O[M\fd V]$ holds.  In contrast, in (2),
  $\tt M$ is \emph{multi-valued}, so the result contains several objects,
  which \emph{includes} $\tt V_1$, $\tt V_2$, ..., $\tt V_n$. Note that we
  emphasized ``includes'' rather than ``equals'', because other facts and
  rules in the program can specify additional objects that must be
  considered part of the method result.
  
  When $n=1$ in set-valued data atoms, the curly braces can be omitted. For
  instance, $\tt O[M\mvd V_1]$. In fact, the single expression (2) is
  equivalent to a the following set of expressions, where the result set is
  split into singletons:
  %%
  \begin{quote}
  $\tt O[M\mvd V_1]$    \\
  $\tt O[M\mvd V_2]$    \\
  $~~~\dots$\\
  $\tt O[M\mvd V_n]$
  \end{quote}
  %%
  
  When $\tt M$ is a constant, {\it e.g.}, {\tt abc}, then we say that it is
  an \emph{attribute}; for example, {\tt john[name\fd 'John']}. When $\tt
  M$ has the form {\tt f(X,Y,Z)} then we refer to it as a method, {\tt f},
  with arguments {\tt X}, {\tt Y}, and {\tt Z}; for example, {\tt
  john[salary(1998)\fd 50000]}.  However, as we saw
  earlier, method expressions can be much more general than these two
  possibilities, as they can be arbitrary HiLog terms.


  \medskip

  \index{atom!signature}
  \index{signature!in \fl}
  %%
  Expressions (3) and (4) above denote \emph{signature atoms}. They specify
  that the method expression, $\tt M$, when applied to objects that belong
  to class $\tt C$, must yield objects that belong to class $\tt T$.  In (3),
  $M$ is declared as single-valued, while in (4) it is set-valued. In a
  set-valued signature, the intention is that the method expression $\tt M$
  must return objects that belong \emph{simultaneously} to the classes
  $\tt T_1$, ..., $\tt T_m$. As with data atoms, a single set-valued
  signature expression of the form (4) is equivalent to the set of
  signature expressions
  %%
  \begin{quote}
      $\tt O[M\Mvd T_1]$    \\
      $\tt O[M\Mvd T_2]$    \\
      $~~~\dots$\\
      $\tt O[M\Mvd T_m]$
  \end{quote}
  %%
  and the curly braces \{ and \} can be omitted when only one class appears
  on the right of $\Mvd$.

  Note that it is allowed for the same method to have both a single-valued
  signature and a set-valued one. The single-valued signature controls the
  data atoms that use $\fd$, and the set-valued signatures control data
  atoms that use $\mvd$.
  
  \medskip

  \index{atom!isa}
  %%
  Objects are grouped into classes using \emph{ISA-atoms}:
  \medskip

  \begin{enumerate}
  \item[5.] $\tt O\isa C$
  \item[6.] \label{eq-subclass} $\tt C\subcl D$
  \end{enumerate}

  \index{class}
  \index{subclass}
  \index{class!subclass}
  \index{class!instance}
  %%
  The expression (5) states that $\tt O$ is an \emph{instance} of class $\tt C$,
  while (6) states that $\tt C$ is a \emph{subclass} of $D$.
\item
  \index{F-molecule}
  \emph{F-molecules} provide a convenient way to shortcut specifications
  related to the same object. For instance, the conjunction of the atoms
  {\tt john{\isa}person}, {\tt john[age{\fd}31]}, {\tt
  john[children\mvd\{bob,mary\}]}, and {\tt john[children\mvd john]}
  is equivalent to the following single F-molecule:
  %%
  \begin{quote}
    {\tt john{\isa}person[age{\fd}31, children\mvd\{bob,mary,john\}]} 
  \end{quote}
  %%
  Note the use of the ``,'' that separates the expression for the {\tt age}
  attribute from the expression for the {\tt children} attribute. This is a
  departure from the original \fl syntax in \cite{KLW95}, which uses ``;'' 
  to separate such expressions.
  
\item \emph{Rules} are, as usual, the constructs of the form $head :-
  body$, where $head$ is an F-molecule and \emph{body} is a conjunction of
  F-molecules or negated F-molecules. (Negation is specified using {\NAF}
    or {\tt tnot} --- the difference will be explained later.)
  Each rule must be terminated with a ``.''.
  
  Conjunction is specified as in Prolog, using the ``,'' symbol. Like in
  Prolog, \FLORA also allows disjunction in the rule body, which is denoted
  using ``;''. As usual in logic languages, a single rule of the form
  %%
  \begin{equation}\label{eq-disjunction}
  {\tt
    {\it head}~:-~john[age{\fd}31],~(john[children\mvd\{bob,mary\}]~;~
    john[children\mvd john]).
    }
  \end{equation}
  %%
  is equivalent to the following pair of rules:
  %%
  \begin{quote}
  {\tt
    {\it head}~:-~john[age{\fd}31],~john[children\mvd\{bob,mary\}].
    }
  \\
  {\tt
    {\it head}~:-~john[age{\fd}31],~john[children\mvd john].
    }
  \end{quote}
  %%
  Disjunction is also allowed inside F-molecules. For instance, the rule
  (\ref{eq-disjunction}) can be equivalently rewritten as:
  %%
  \begin{quote}
 {\tt
   {\it head}~:-~john[age{\fd}31,~(children\mvd\{bob,mary\}~;~children\mvd john)].
   }
  \end{quote}
  %%
  Note that conjunction ``,'' binds stronger than disjunction ``;'', so the
  parentheses in the above example are essential.
  
\item \emph{Programs and queries}: A \emph{program} is a set of rules. A
  \emph{query} is a rule without the head. In \FLORA, such headless rules
  use {\tt ?-} instead of {\tt :-}, {\it e.g.}, 
  %%
  \begin{quote}
    {\tt 
    ?-~john[age->X].    
    }
  \end{quote}
  %%
  The symbol {\tt :-} in headless \FLORA expressions is used for various
  directives, which are plenty and will be introduced in due course.
\end{itemize}



\begin{example}
  {\bf (Publications Database)} \rm Figure~\ref{fig-flogic-model} depicts
  a fragment of a \FLORA program that represents a database of scientific
  publications.
\end{example}


\begin{figure}[htb]
\begin{tabular}{c}
  \begin{tabular}{l}
    {\bf Schema:}\\
    conf\_p\subcl paper. \\
    journal\_p\subcl paper.\\
    paper[authors\Mvd  person, title\Fd string].\\
    journal\_p[in\_vol\Fd volume]. \\
    conf\_p[at\_conf\Fd conf\_proc].\\
    journal\_vol[of \Fd journal, volume\Fd integer, 
               number\Fd integer, year\Fd integer].\\  
    journal[name\Fd string, publisher\Fd string,
            editors\Mvd person]. \\
    conf\_proc[of\_conf\Fd conf\_series, year\Fd integer,
               editors\Mvd person]. \\
    conf\_series[name\Fd string]. \\
    publisher[name\Fd string].\\
    person[name\Fd string, affil(integer)\Fd institution]. \\
    institution[name\Fd string, address\Fd string].\smallskip\\

    {\bf Objects:}\\
    $o_{j1}$\isa journal\_p[%
      title\fd 'Records, Relations, Sets, Entities, and Things',
      authors\mvd$\{o_{mes}\}$, in\_vol\fd $o_{i11}$]. \\
    $o_{di}$\isa conf\_p[
      title\fd 'DIAM II and Levels of Abstraction',
      authors\mvd$\{o_{mes},o_{eba}\}$, at\_conf\fd $o_{v76}$]. \\
    $o_{i11}$\isa journal\_vol[of\fd $o_{is}$, number\fd 1, volume\fd 1, year\fd1975]. \\
    $o_{is}$\isa journal[name\fd'Information Systems', editors\mvd $\{o_{mj}\}$]. \\
    $o_{v76}$\isa conf\_proc[of\fd vldb, year\fd 1976, editors\mvd $\{o_{pcl},o_{ejn}\}$].\\
    $o_{vldb}$\isa conf\_series[name\fd'Very Large Databases']. \\
    $o_{mes}$\isa person[name\fd'Michael E. Senko']. \\
    $o_{mj}$\isa person[name\fd'Matthias Jarke', affil(1976)\fd $o_{rwt}$]. \\
    $o_{rwt}$\isa institution[name\fd'RWTH\_Aachen'].
\end{tabular}
\end{tabular}
\caption{A Publications Object Base and its Schema in \FLORA}
  \label{fig-flogic-model}
\end{figure}



\subsection{Symbols, Strings, and Comments}


\index{symbol}
%%
\paragraph{Symbols.}
\FLORA symbols (that are used for the names of constants, predicates, and
object constructors) begin with a lowercase letter followed by zero or more
letters ($\tt A \ldots Z, a \ldots z$), digits ($\tt 0 \ldots 9$), or underscores
(\_), e.g., \texttt{student}, \texttt{apple\_pie}. Symbols can also be
\emph{any} sequence of characters enclosed in a pair of single quotes,
e.g., \texttt{'JOHN SMITH'},\texttt{'default.flr'}.  Internally, \FLORA
symbols are represented as \emph{XSB symbols},\footnote{
  %%
  Symbols are called ``atoms'' in XSB, which contravenes the use of this
  term for atomic formulas in classical logic and \fl.
  We avoid the use of the term ``atom'' in reference to symbols.
  %%
  }
%%
which are used there as names of predicates and function symbols.

\begin{table}[htb]
\center
\texttt{ \small
\begin{tabular}{|c|r@{\hspace{1.5cm}}|@{\hspace{5mm}}l@{\hspace{5mm}}|}
\hline
Escaped String &
  \multicolumn{1}{c|@{\hspace{5mm}}}{ASCII (decimal)} &
  \multicolumn{1}{c|}{Symbol} \\ \hline
{\bksl}{\bksl} &  92 & {\bksl} \\ \hline
{\bksl}n &  10 &		 NewLine \\ \hline
{\bksl}N &  10 &		 NewLine \\ \hline
{\bksl}t &   9 &		 Tab \\ \hline
{\bksl}T &   9 &		 Tab \\ \hline
{\bksl}r &  13 &		 Return \\ \hline
{\bksl}R &  13 &		 Return \\ \hline
{\bksl}v &  11 &		 Vertical Tab \\ \hline
{\bksl}V &  11 &		 Vertical Tab \\ \hline
{\bksl}b &   8 &		 Backspace \\ \hline
{\bksl}B &   8 &		 Backspace \\ \hline
{\bksl}f &  12 &		 Form Feed \\ \hline
{\bksl}F &  12 &		 Form Feed \\ \hline
{\bksl}e &  27 &		 Escape \\ \hline
{\bksl}E &  27 &		 Escape \\ \hline
{\bksl}d & 127 &		 Delete \\ \hline
{\bksl}D & 127 &		 Delete \\ \hline
{\bksl}s &  32 &		 Whitespace \\ \hline
{\bksl}S &  32 &		 Whitespace \\
\hline
\end{tabular}
}
\caption{Escaped Character Strings and Their Corresponding Symbols}
\label{tab:tab-esc-str}
\end{table}

\index{escaped character}
\FLORA also recognizes escaped characters inside single quotes
(\texttt{'}).  An escaped character normally begins with a backslash
(\texttt{\bksl}).  Table~\ref{tab:tab-esc-str} lists the special escaped
character strings and their corresponding special symbols. An escaped
character may also be any ASCII character. Such a character is preceded
with a backslash together with a lowercase \texttt{x} (or an uppercase
\texttt{X}) followed by one or two hexadecimal symbols representing its
ASCII value. For example, \texttt{{\bksl}xd} is the ASCII character
Carriage Return, whereas \texttt{{\bksl}x3A} represents the semicolon. In
other cases, a backslash is recognized as itself.

One exception is that inside a quoted symbol, a single quote character is
escaped by another single quote, e.g., \texttt{'isn''t'}.

\paragraph{Strings (character lists).}

\index{string}
\index{character list}
%
Like XSB strings, \FLORA strings are enclosed in a pair of double quotes
(\texttt{\dq}).  These strings are represented internally as lists of
ASCII characters. For instance, \mbox{\texttt{[102,111,111]}} is the same
as \texttt{{\dq}foo{\dq}}.

Escape characters are recognized inside \FLORA strings similarly to
\FLORA symbols.  However, inside a string, a single quote character does
not need to be escaped. A double quote character, however, needs to be
escaped by another double quote, e.g.,
\texttt{{\dq}{\dq}{\dq}foo{\dq}{\dq}{\dq}}.

\paragraph{Numbers.}

\index{number}
%%
\index{integer}
%%
\index{floating number}
%%
Normal \FLORA integers are decimals represented by a sequence of digits,
e.g., \texttt{892, 12}.  \FLORA also recognizes integers in other bases (2 through
36). The base is specified by a decimal integer followed by a single quote
(\texttt{'}). The digit string immediately follows the single quote. The
letters $\tt A \ldots Z$ or $\tt a \ldots z$ are used to represent digits greater
than 9.  Table~\ref{tab:tab-int-rep} lists a few example integers.
%%
\begin{table}[htb]
\center
\texttt{ \small
\begin{tabular}{|r@{'}l|r@{\hspace{1.5cm}}|c|}
\hline
  \multicolumn{2}{|c|}{Integer} &
  \multicolumn{1}{c|}{Base (decimal)} &
  \multicolumn{1}{c|}{Value (decimal)} \\ \hline
\multicolumn{1}{|r}{} & \multicolumn{1}{@{}l|}{1023} &  10 & 1023 \\ \hline
2 & 1111111111 & 2 & 1023 \\ \hline
8 & 1777 & 8 & 1023 \\ \hline
16 & 3FF &  16 & 1023 \\ \hline
32 & vv & 32 & 1023 \\
\hline
\end{tabular}
}
\caption{Representation of Integers}
\label{tab:tab-int-rep}
\end{table}

Underscore (\texttt{\_}) can be put inside any sequence of digits as
delimiters. It is used to partition some long numbers. For instance,
$\texttt{2'11\_1111\_1111}$ is the same as $\texttt{2'1111111111}$.
However, ``\texttt{\_}'' cannot be the first symbol of an integer, since
variables can start with an underscore. For example, $1\_2\_3$ represents
the number $123$ whereas $\_12\_3$ represents a variable named $\_12\_3$.

Floating numbers normally look like {\tt 24.38}. The decimal point
must be preceded by an integral part, even if it is 0, e.g., {\tt 0.3}
must be entered as {\tt 0.3}, but not as {\tt .3}. Each floating
number may also have an optional exponent. It begins with a lowercase
{\tt e} or an uppercase {\tt E} followed by an optional minus sign
({\tt -}) or plus sign ({\tt +}) and an integer. This exponent is
recognized as in base 10. For example,
\mbox{\tt 2.43E2 is 243} whereas
\mbox{\tt 2.43e-2 is 0.0243}.

\paragraph{Comments.}

\index{comment}
%
\FLORA supports three kinds of comments: (1) all characters following
{\tt \%} until the end of the line; (2) all characters following
{\tt //} until the end of the line; (3) all characters inside a pair of
{\tt /*} and {\tt */}. Note that only (3) can span multiple lines.

Comments are recognized like whitespaces by the compiler.  Therefore,
tokens can also be delimited by comments.


\subsection{Operators}


\index{operators}
%%
As in Prolog, \FLORA allows the user to define operators, to liven up the
otherwise boring syntax.  There are three kinds of operators: infix,
prefix, and postfix. An infix operator appears between its two arguments,
while a prefix operator before its single argument and a postfix operator
after its single argument. For instance, if {\tt foo} is defined as an
infix operator, then {\tt X foo a} will be parsed as {\tt foo(X,a)} and if
{\tt bar} is a postfix operator then {\tt X bar} is parsed as {\tt bar(X)}. 

\index{operators!precedence level}
\index{operators!type}
%
Each operator has a \emph{precedence level}, which is a positive integer.
Each operator also has a \emph{type}. The possible types for infix operatos
are: {\tt xfx}, {\tt xfy}, {\tt yfx}; the possible types for prefix
operators are: {\tt fx}, {\tt fy}; and the possible types for postfix
operators are: {\tt xf}, {\tt yf}. In each of these type expressions, {\tt
  f} stands for the operator, and {\tt x} and {\tt y} stand for the
arguments.  The symbol {\tt x} in a type expression means that the
precedence level of the corresponding argument should be \emph{smaller}
than that of the operator, while {\tt y} means that the precedence level of
the corresponding argument should be \emph{less or equal} than that of the
operator.

The precedence level and the type together determine the way the operators
are parsed. The general rule is that precedence of a constant or a functor
symbol that has not been defined as an operator is zero. Precedence of a
Prolog term is the same as the precedence of its main functor. 
An expression that contains several operators is parsed in such a way that
the operator with the highest precedence level becomes the main functor of
the parsed term, the operator with the next-highest precedence
level becomes the main functor of one of the arguments, and so on.
If an expression cannot be parsed according to this rule, a parse error is
reported.

It is not our goal to cover the use of operators in any detail, since this
information can be found in any book on Prolog. Here we just give an
example that illustrates the main points.  For example, in \FLORA, {\tt -}
has precedence level {\tt 800} and type {\tt yfx}, {\tt *} has precedence
level {\tt 700} and type {\tt yfx}, {\tt ->} has precedence level 1100 and
type {\tt xfx}.  Therefore, {\tt 8-2-3*4} is the same as {\tt
  -(-(8,2),*(3,4))} in prefix notation, and {\tt a -> b -> c} will generate
a parsing error.


\index{compiler directive!{\tt op}}
%
Any symbol can be defined as an operator. The general syntax is
\begin{qrules}
{\tt :- op(\emph{Precedence},\emph{Type},\emph{Name}).}
\end{qrules}
%%
For instance, 
%%
\begin{quote}
 {\tt
   :- op(800, xfx, foo)
   }
\end{quote}
%%
As a notational convenience, the argument {\tt Name} can also be a list of
operator names of the same type and precedence level, for instance,
\begin{qrules}
{\tt :- op(800,yfx,[+,-]).}
\end{qrules}
%%
It is possible to have more than one operator with the same name provided
they have different use ({\it e.g.}, one infix and the other postfix).
However, the \FLORA built-in operators are not allowed to be redefined.
In particular, any symbol that is part of \fl syntax, such as ``,', ``.'',
``[``, ``:'', etc., as well as any name that begins with {\tt flora} or
{\tt fl} followed by a capital letter should be considered as reserved for
internal use.

Although this simple rule is sufficient, in most cases, to keep you out of
trouble, you should be aware of the fact that symbols such as ``{\tt ,}'',
``{\tt ;}'', ``{\tt +}'', ``{\tt .}'', ``{\tt ->}'', ``{\tt ::}'', and many
other parts of \FLORA syntax are operators. Therefore, there is a chance
that precedence levels chosen for the user-defined operators conflict with
those of \FLORA and, as a result, your program might not parse. If in
doubt, check the declarations in the file {\tt flroperator.P} in the \FLORA
source code.


\subsection{Logical Expressions}


\index{logical expressions}
%
In a \FLORA program, any combination of conjunction, disjunction, and
negation of literals can appear wherever a logical formula is allowed,
e.g., in a rule body.

Conjunction is represented through the infix operator ``{\tt ,}'' and
disjunction is made using the infix operator ``{\tt ;}''.  Negation is made
through the prefix operators ``\NAF'' and ``{\tt
  tnot}''.\footnote{
  %%
  In brief, ``{\tt $\backslash$+}'' represents negation as
  failure and can be applied only to non-tabled Prolog, \FLORA, or HiLog
  predicates. ``{\tt tnot}'', on the other hand, is negation that
  implements the well-founded semantics.  Refer to
  Section~\ref{sec:negation} for more information on the difference between
  negation operators. 
  %%
  }
%%
When parentheses are omitted, conjunction binds stronger than disjunction
and the negation operators bind their arguments stronger than the other
logical operators.  For example, in \FLORA the following expression:
\verb|a, b; c, not d|, is equivalent to the the logical formula: $\tt (a
\wedge b) \vee (c \wedge (\neg d))$.

\index{molecule!logic expressions}
%
Logical formulas can also appear inside the specification of an object. For
instance, the following F-molecule:
\begin{qrules}
o[tnot att1{\fd}val1, att2{\mvd}val2; meth{\fd}res]
\end{qrules}
is equivalent to the following formula:
\begin{qrules}
(tnot o[att1{\fd}val1], o[att2{\mvd}val2]) ; o[meth{\fd}res]
\end{qrules}


\subsection{Arithmetic Expressions}


\index{arithmetic expression}
%%
In \FLORA arithmetic expressions are \emph{not} always evaluated. As in
XSB, the arithmetic operators such as {\tt +}, {\tt -}, {\tt /}, and {\tt
  *}, are defined as normal binary functors. However, to evaluate
arithmetic expressions, \FLORA provides two operators: {\tt is} and {\tt
  :=}, which synonymous.  For example, {\tt X := 3+4} will bind {\tt X} to
the value {\tt 7}.

When dealing with arithmetic expressions, the order of literals is
important.  In particular, all variables appearing in an arithmetic
expression must be instantiated at the time of evaluation. Otherwise, a
runtime error will occur. For instance, 
%%
\begin{qrules}
  \tt
?- X > 1, X := 1+1.
\end{qrules}
%%
will produce an error, while
%%
\begin{qrules}
  \tt
?- X := 1+1, X > 1.
\end{qrules}
%%
will evaluate to true.

As in Prolog, the operands of an arithmetic expression can be any variable
or a constant. However, in \FLORA, an operand can also be a \emph{path
  expression}. For the purpose of this discussion, a path expression of the
form $\tt p.q$ should be understood as a shortcut for {\tt p[q$\fd$X]}, where
$\tt X$ is a new variable, and $\tt p.q.r$ is a shortcut for {\tt p[q$\fd$X],
  X[r$\fd$Y]}. For set-valued formulas, the notation ``..'' is used. For
instance, {\tt p..q} stands for {\tt p[q$\mvd$X]}. More detailed discussion
of path expressions appears in Section~\ref{sec-pathexpr}.

Both single-valued and multi-valued path expressions are allowed in
arithmetic expressions, and all variables are considered to be
existentially quantified. For example, the following query
%%
\begin{qrules}
?- john..bonus $+$ mary..bonus $>$ 1000.
\end{qrules}
%%
should be understood as
%%
\begin{qrules}
?- john[bonus{\mvd}{\tt \_V1}], mary[bonus{\mvd}{\tt \_V2}], ${\tt \_V1}+{\tt \_V2} > 1000$.
\end{qrules}
%%
Note that in first query does not have any variables, so after the
evaluation the system would print either yes or no. To achieve the same
behavior, we use \emph{semi-anonymous variables}, {\tt \_V1} and {\tt
  \_V2}. If we used {\tt V1} and {\tt V2} instead, the values of these
variables would have been printed out.

\FLORA recognizes numbers as oids and, thus, it is perfectly normal to have
allows arithmetic expressions inside path expressions such as this:
{\tt 1.2.(3+4*2).7}. When parentheses are omitted, this might lead to
ambiguity.
For instance, is the meaning of
%%
\begin{qrules}
1.m+2.n.k
\end{qrules}
%%
represented by
the arithmetic expression {\tt (1.m)+(2.n.k)}, or by
the path expressions {\tt (1.m+2.n).k}, by {\tt (1.m + 2).n.k}, or by {\tt
  1.(m+2).n.k}? To disambiguate such expressions, we must remember that the
operators ``.'' and ``..'' used in path expressions bind stronger than the
arithmetic operators $+$, $-$, etc.

Even more interesting is the following example: {\tt 2.3.4}. Does it
represent the path expression {\tt (2).(3).(4)}, or {\tt (2.3).4}, or {\tt
  2.(3.4)} (where in the latter two cases 2.3 and 3.4 are interpreted as
decimal numbers)? The answer to this puzzle is {\tt (2.3).4}: when
tokenizing, \FLORA first tries to classify tokens into meaningful
categories. Thus, when 2.3 is first found, it is identified as a
decimal. Thus, the parser receives the expression (2.3).4, which it
identifies as a path expression that consists of two components, the oids
2.3 and 4.

Another ambiguous situation arises when the symbols {\tt -} and {\tt +} are
used as minus and plus
signs, respectively. \FLORA follows the common arithmetic interpretation of
such expressions, where the {\tt +/-} signs bind stronger than the infix
operators and thus
{\tt 4--7} and {\tt 4-+7} are interpreted as {\tt 4-(-7)} and {\tt 4-(+7)},
respectively.

%%
\begin{table}[tb]
\center
\texttt{ \small
\begin{tabular}{|c|c|c|c|c|}
\hline
%%
Precedence  & Operator & Use & Associativity & Arity \\ \hline
not applicable & () & parentheses & not applicable & not applicable\\ \hline
not applicable & . & decimal point & not applicable & not applicable \\ \hline
            & .   & single-valued object reference & left & binary \\ \cline{2-5}
300         & ..   & multi-valued object reference & left & binary \\ \cline{2-5}
            & :    & ISA specification & left & binary \\ \cline{2-5}
            & ::   & subclass specification & left & binary \\ \hline
600         & - & minus sign & right & unary \\ \cline{2-5}
            & + & plus sign & right & unary \\ \hline
700         & * & multiplication & left & binary \\ \cline{2-5}
            & / & division & left & binary \\ \hline
800         & - & subtraction & left & binary \\ \cline{2-5}
            & + & addition & left & binary \\ \hline
            & =< & less than or equals to & not applicable & binary \\ \cline{2-5}
            & >= & greater than or equals to & not applicable & binary \\ \cline{2-5}
1000        & =:= & equals to & not applicable & binary \\ \cline{2-5}
            & ={\bksl}= & unequal to & not applicable & binary \\ \cline{2-5}
            & := & assignment & not applicable & binary \\ \cline{2-5}
            &is & \multicolumn{3}{c|}{same as :=} \\
\hline
\end{tabular}
}
\caption{Operators in Non-Increasing Precedence Order and Their Associativity and Arity}
\label{tab:tab-op-pre}
\end{table}

Table~\ref{tab:tab-op-pre} lists various operators in decreasing precedence
order, their associativity, and arity.  When in doubt, use parentheses.
Here are some more examples of valid arithmetic expressions:
{\tt
\begin{quote}
o1.m1+o2.m2.m3~~~~~~~~~{\rm same as} (o1.m1)+(o2.m2)\\
2.(3.4)~~~~~~~~~~~~~~~~{\rm the value of the attribute} 3.4 {\rm on object} 2\\
3 + - - 2~~~~~~~~~~~~~~{\rm same as} 3+(-(-2))\\
5 * - 6~~~~~~~~~~~~~~~~{\rm same as} 5*(-6)\\
5.(-6)~~~~~~~~~~~~~~~~~{\rm the value of the attribute} -6 {\rm on object} 5
\end{quote}
}
%%
Note that the parentheses in {\tt 5.(-6)} are needed,
because otherwise ``{\tt .-}'' would be recognized as a single token.
Similarly, the whitespace around ``{\tt +}'', ``{\tt -}'', and ``{\tt *}''
are also needed in these examples to avoid {\tt *-} and {\tt +--} being
interpreted as distinct token.


\section{Multifile Programs}

\FLORA supports many ways in which a program can be modularized.  First, an
\fl program can be split into many files with separate namespaces. Each
such file can be considered an independent library, and the different
libraries can call each other. In particular, the same method name (or a
predicate) can be used in different files and the definitions will not
clash.  Second, a program file can be split of several files, and these
files can be included by the preprocessor prior to the compilation. In this
case, all files share the same namespace in the sense that the different
rules that define the same method name (or a predicate) in different files
are assumed to be part of one definition. Third, \FLORA programs can call
XSB modules and vice versa. In this way, a large system can be built partly
in Prolog and partly in \FLORA.

We discuss each of the aforesaid modularization methods in turn.


\subsection{\FLORA Modules} \label{sec:flora-modules}

\index{module}
%%
A \emph{\FLORA module} is a programming abstraction that allows a large
program to be split into separate libraries that can be reused in multiple
ways in the same program. Formally, a module is a pair that consists of a
\emph{name} and a \emph{contents}. The name is just an alphanumeric symbol,
and the contents consists of the program code that is typically loaded from
some file (but it can also be constructed dynamically by inserting
facts\footnote{
  %%
  Dynamic insertion of rules will be implemented in the future.
  %%
  }
%%
into another module).

The basic idea behind \FLORA modularization is that reusable code libraries
are to be placed in separate files.  To use a library, it must be
\emph{loaded into a module}. Other parts of the program can then invoke
this library's methods by providing the name of the module (and the
method/predicate names, of course).  There is no need to export anything
from a library --- any public method or predicate can be called by other
parts of the program.\footnote{
  %%
  At present, all methods are public, but encapsulation will be implemented
  in the future.
  %%
  }
%%
In this way, the library loaded into a module becomes that module's content.

Note that there is no a priori association between files and modules.  Any
file can be loaded into any module and one program file can even be loaded
into two different modules at the same time. The same module can be reused
during the same program run by loading another file into that module. In
this case, the old contents is erased and the module gets new contents from
the second file.

In \FLORA, modules are completely decoupled from file
names. A \FLORA program knows only the module names it needs to call, but
not the file names. Specific files can be loaded into modules by another,
unrelated bootstrapping program. Moreover, a program can be written in such
a way that it calls a method of some module without knowing that module's
name. The name of the module can be passed as a parameter or in some other
way and the concrete binding of the method to the module will be done at
runtime.

This dynamic nature of \FLORA modules stands in sharp contrast to the module
system of XSB, which is static and associates modules with files at compile
time. Moreover, to call a predicate from another module, that predicate
must be imported explicitly and referred to by the same name.


\subsection{Calling Methods and Predicates Defined in Other Modules}


\index{module!reference}
%
If \emph{literal} is an F-molecule or a predicate defined in another
module, it can be called using the following syntax:
%%
\begin{quote}
\emph{literal} @ \emph{module} 
\end{quote}
%%
The name of the module can be any alphanumeric symbol.\footnote{
  %%
  In fact, any symbol is allowed. However, it cannot contain the quote
  symbol, ``{\tt '}''.
  %%
  }
%%
For instance, \verb|foo(a) @ foomod| tests whether {\tt foo(a)} is true in
the module named {\tt foomod}, and {\tt mary[children\mvd X]@genealogy}
queries the information on {\tt mary}'s children available in the module
{\tt genealogy}. More interestingly, the module specifier can be a variable
that gets bound to a module name at run time. For instance, 
%%
\begin{quote}
 {\tt
   ..., Agent=zagat, ..., newyork[dinner(italian) \mvd X]@Agent.
   }
\end{quote}
%%
A call to a literal with an unbound module specification or one that is not
bound to a symbol will result in a runtime error.

\index{module!rules}
%%
When calling the literals defined in the same module, the {\tt @{\it
    module}} notation is not needed, of course. (In fact, since a program
does not know where it will be loaded, using the @-notation to call a
literal in the same module is hard. However, it is possible with the help
of the special predicate {\tt flThisModule/1}, described later, and is left
as an exercise.)

The following rules apply when calling a literal defined in another module:
%%
\begin{enumerate}
\item Literal reference cannot appear in a rule head or be specified as
  a fact. For example, the following program will generate
  a parsing error
  %%
  \begin{quote}
    \verb|john[father->smith] @ foomod.| \\
    \verb|foo(X) @ foomod :- goo(X).|
  \end{quote}
  %%
  because defining a literal that belongs to another module does not make
  sense.
  
\item Module specification is distributive over logical connectives,
  including the conjunction operator, ``\verb|,|'', the disjunction,
  ``\verb|;|'', and the negation operators, ``\NAF'' and
  ``\verb|tnot|''. For example, the formula below:
  %%
  \begin{quote}
    \verb|(john[father->smith], tnot smith[spouse->mary]) @ foomod|
  \end{quote}
  is equivalent to the following formula:
  \begin{quote}
    \verb|john[father->smith] @ foomod, tnot (smith[spouse->mary] @ foomod)|
  \end{quote}

\item Module specifications can be nested. The one closest to a literal
  takes effect. For example,
  \begin{quote}
    \verb|(foo(a), goo(b) @ goomod, hoo(c)) @ foomod|
  \end{quote}
  is equivalent to
  \begin{quote}
    \verb|foo(X) @ foomod, goo(b) @ goomod, hoo(c) @ foomod|
  \end{quote}
  
\item The module specification propogates to any F-molecule appearing
  in the argument of a predicate for which the module is
  specified. For example,
  \begin{quote}
    \verb|foo(a.b[c->d]) @ foomod|
  \end{quote}
  %%
  is equivalent to
  %%
  \begin{quote}
    \verb|a[b->X] @ foomod, X[c->d] @ foomod, foo(X) @ foomod|
  \end{quote}
  
\item Module specifications do not affect function terms that are not
  predicates or method names, unless such a specification is explicitly
  attached to such a term. For instance, in
  %%
  \begin{quote}
    \verb|?- foo(goo(a)) @ foomod.|
  \end{quote}
  %%
  {\tt goo/1} refers to the same functor both in module {\tt foomod} and in
  the calling module. However, if the module is attched explicitly, as in
  \begin{quote}
    \verb|?- foo(goo(a) @ goomod) @ foomod.|
  \end{quote}
  %%
  then {\tt foo/1} is assumed to be a meta-predicate that receives the
  predicate {\tt goo/1}  as a parameter.
\end{enumerate}


\subsection{Finding the Current Module Name}

\index{module!{\tt flThisModule}}
%
Since a \FLORA program can be loaded into any module, the program does not
have a priori knowledge of the module it will be executing in. However, the
program can determine its module at runtime using the special predicate
{\tt flThisModule/1}.  This predicate accepts a variable argument and binds
it to the name of the \FLORA module into which the program was loaded.




\subsection{Loading Files into Modules}\label{sec-loading-mods}

%

\index{module!compilation}
%%
\FLORA provides commands for compiling and loading program files into
specified modules.
The
command 
%%
\begin{quote}
  {\tt ?- flCompile({\it file}>>{\it module}).}
\end{quote}
%%
generates the byte code for the program to be loaded into the module named
{\tt module}. The name of the byte code for the program in \emph{file}.flr,
which can later be loaded into the specified module. In practice this means
that the compiler generates files named \emph{file\_module.P} and
\emph{file\_module.O} with symbols appropriately renamed to avoid clashes.

If no module is specified, the command
%%
\begin{quote}
 \tt
 ?- flCompile({\it file}).
\end{quote}
%%
compiles {\it file}.flr for the default module {\tt main}.


\index{loading files}
%
Like XSB, \FLORA also allows the user to compile and load program files at
the same time: If the file was not compiled before (or if the program file
is newer), the program is compiled before being loaded.
For instance, the following command:
\begin{quote}
\verb|[myprog1, myprog2]|
\end{quote}
will loads both {\tt myprog1} and {\tt myprog2} into the default module
{\tt main}. If either {\tt myprog1.flr} or {\tt myprog2.flr} is stale, it
is compiled first.

Both XSB files and \FLORA files can appear in the list. Note
that the names of XSB source files have {\tt .P} as extension,
while XSB byte code files have {\tt .O}, and \FLORA source files
have {\tt .flr}. If a file name is not specified with an extension,
\FLORA will search the follwing files in order until one is found
or none: \FLORA source files, XSB source files, and XSB byte code
files.

An optional module name can be used to load the program into a specified
module:
\begin{quote}
  \tt
[..., {\it program\/} >> {\it module}, ...]
\end{quote}
%%
For instance, the following command:
%%
\begin{quote}
\verb|[myprog >> foomod]|
\end{quote}
%%
loads the \FLORA program {\tt myprog.flr} into the module named {\tt
  foomod}, compiling it if necessary.

Mixed module specifications are also possible. For instance, 
\begin{quote}
\verb|['myprog.flr', mydb >> foomod]|
\end{quote}
loads {\tt myprog.flr} into module {\tt main} and
{\tt mydb.flr} into module {\tt foomod}.

Note that the {\tt [...]} command can also load and compile XSB programs.
The overall algorithm is as follows. If the file suffix is specified
explicitly, the corresponding file is assumed to be a \FLORA file, an XSB
file, or a byte code depending on the suffix: {\tt .flr}, {\tt .P}, or {\tt
  .O}. If the suffix is not given explicitly, the compiler first checks if
{\it file}.{\tt flr} exists. If so, the file assumed to be a \FLORA program
and is compiled as such. If {\it file}.{\tt flr} is not found, but {\it
  file}.{\tt P} or {\it file}.{\tt O} is, the file is passed to XSB for
compilation.


\subsection{Calling XSB from \FLORA}

XSB predicates can be called from \FLORA through the \FLORA module system
\FLORA models XSB programs as collections of static modules, {\it i.e.},
from \FLORA's point of view, XSB modules are always available
and do not need to be loaded. The syntax to call XSB predicates is
%%
\index{module!{\tt @prolog(module)}}
%%
\begin{quote}
  \tt
  ?- {\it predicate}@prolog({\it module})  
\end{quote}
%%
For instance, since the predicate {\tt member/2} is defined in the XSB
module {\tt basics}, we can call it as follows:
%%
\begin{quote}
 \tt
 ?- member(abc,[cde,abc,pqr])@prolog(basics).
\end{quote}
%%
To use this mechanism, you must know which module the particular predicate
is defined in. Some predicates are defined by programs that do not belong
to any module. When such an XSB program is loaded, the corresponding
predicates become available in the special module called {\tt usermod} and
\FLORA can call such predicates as follows:
%%
\begin{quote}
 \tt
 ?- foo(X)@prolog(usermod).
\end{quote}
%%
Note that variables are not allowed in the module specifications of XSB
predicates, {\it i.e.},
%%
\begin{quote}
 \tt
 ?- M=usermod, foo(X)@prolog(M).
\end{quote}
%%
will cause a compilation error.

Some XSB predicates are considered ``well-known'' and, even though they are
defined in various obscure modules, the user can just use those predicates
without remembering the corresponding XSB module names. These predicates
(that are listed in the XSb manual) can be called from \FLORA with
particular ease:
%%
\begin{quote}
 \tt
 ?- writeln('Hello')@prolog()
\end{quote}
%%
{\it i.e.}, we can simply omit the XSB module name (but parentheses must be
preserved). 




\subsection{Calling \FLORA from XSB}

Since XSB does not understand object-oriented syntax, it can call only use
predicates defined in \FLORA programs. These predicates can be HiLog or
first-order predicates (See Section~\ref{sec:hilog}). To call predicates
defined in \FLORA programs, they must be imported by the XSB program.
To do this, the XSB program must execute the query
%%
\begin{quote}
 ?- bootstrap\_flora.  
\end{quote}
%%
somewhere near the top of the file and then include the appropriate
{\tt flImport} statements described below:
%%
\begin{quote}
  \tt
   ?- flImport \{hilog|firstorder\} {\it flora-predicate/arity} as {\it
     xsb-name}(\_,\_,...,\_)\\
   \hspace*{5cm}from {\it filename} >> {\it flora-module-name}
   \\
   ?- flImport \{hilog|firstorder\} {\it flora-predicate/arity} as {\it 
     xsb-name}(\_,\_,...,\_)\\
   \hspace*{5cm}from {\it flora-module-name}
\end{quote}
%%
The first syntax is used to both import the predicate and also load the
program file defining it into a given module. The second syntax is used
when the flora program is already loaded and we only need to import the
corresponding predicate.

In both cases, the option {\tt hilog} is used to import a HiLog predicate
and {\tt firstorder} is used to import first-order predicates. The imported
predicate must be given a name by which the imported predicate will be
known in XSB.  (This name can be the same as the name used in \FLORA.)  It
is important, however, that the XSB name be specified as shown, {\it i.e.},
as a predicate skeleton with the same number of arguments as in the
first-order predicate. For instance, {\tt foo(\_,\_,\_)} will do, but {\tt
  foo/3} will not.

Once the predicate is imported, it can be used under its XSB name as a
regular predicate.

XSB programs can also load and compile \FLORA programs using the following
queries: 
%%
\begin{quote}
 \tt
 :- import flLoad/1, flCompile/1 from flora2.\\
 ?- flLoad  {\it flora-file} >> {\it flora-module}.\\
 ?- flLoad  {\it flora-file}\\
 ?- flCompile {\it flora-file} >> {\it flora-module}.\\
 ?- flCompile  {\it flora-file}
\end{quote}
%%
The first query loads the file {\it flora-file\/} into the given module and
compiles it, if necessary. The second query loads the program into the
default module {\tt main}. The last two queries compile the file for
loading into the module {\it flora-module} and {\tt main}, respectively,
but do not load it.

Finally, an XSB program can check if a certain \FLORA module has been
loaded using the following call:
%%
\begin{quote}
 \tt
 :- import flModule/1 from flora2.\\
 ?- flModule({\it flora-module-name}).
\end{quote}
%%


\subsection{Including Files into \FLORA Programs}

The last and the simplest way to construct multi-file \FLORA 
programs is by using the {\tt \#include} preprocessing directive.
For instance if file {\tt foo.flr} contains the following instructions:
%%
\begin{quote}
  \#include file1 \\
  \#include file2\\
  \#include file3
\end{quote}
%%
the effect is the same as if the above three files were concatenated
together and stored in {\tt foo.flr}.



\section{Path Expressions}\label{sec-pathexpr}

\subsection{Path Expressions in the Rule Body}


\index{path expression}
\index{path expression!in rule body}
%%
In addition to the basic \fl syntax, the \FLORA  system also supports
\emph{path expressions} to simplify object navigation along
single-valued and multi-valued method applications, and to avoid
explicit join conditions \cite{frohn-lausen-uphoff-VLDB-94}.  The
basic idea is to allow the following \emph{path expressions} wherever
Id-terms are allowed:
%%

  \medskip

\begin{enumerate} 
\item[7.]\label{eq-path-fun} ~~ {\tt O.M}
\item[8.]\label{eq-path-set} ~~ {\tt O..M} 
\end{enumerate} \medskip

\noindent
The path expression in (7) is \emph{single-valued}; it refers to the unique
object $\tt R_0$ for which $\tt O[M\fd R_0]$ holds; (8) is a
\emph{multi-valued} path expression; it refers to each $R_i$ for which $\tt
O[M\mvd\{R_i\}]$ holds.  The symbols $\tt O$ and $\tt M$ stand for an
Id-term or path a expression.  Moreover, $\tt M$ can be a method that takes
arguments, in which case $\tt O.M(P_1,\dots,P_k)$ and $\tt
O..M(P_1,\dots,P_k)$ are a valid path expressions.
  
In order to disambiguate the syntax and to specify the desired order of
method applications, parentheses can be used. By default, path expressions
associate to the left, so $\tt a.b.c$ is equivalent to $\tt (a.b).c$, which
specifies the object $\tt o$ such that $\tt a[b\fd x] \land x[c\fd o]$
holds (note that $\tt x=a.b$). In contrast, $\tt a.(b.c)$ is the object
$\tt o1$ such that $b[c\fd x1] \land a[x1\fd o1]$ holds (note that in this
case, $\tt x1=b.c$). In general, $o$ and $o1$ can be different objects.
Note also that in $\tt (a.b).c$, $\tt b$ is a method name, whereas in $\tt
a.(b.c)$ it is used as an object name and {\tt b.c} as a method.  Observe
that function symbols can also be applied to path expressions, since path
expressions, like Id-terms, represent objects. Thus, $\tt f(a.b)$
is a valid expression.

As path expressions and F-molecules can be arbitrarily nested, this leads
to a concise and flexible specification language for object properties, as
illustrated in the following example.

\begin{example}[Path Expressions]\label{Ex:PathExpr}
  \rm Consider again the schema given in Figure~\ref{fig-flogic-model}.  If
  $n$ is the name of a person, the following path expression is a query
  that returns all editors of conferences in which $n$ had a paper:
  %%
  \begin{qrules}
    ?- P\isa conf\_p[authors\mvd\{\anon [name\fd $n$]\}].at\_conf..editors
  \end{qrules}
  %%
  Likewise, the answer to the query
  %%
  \begin{qrules}
    ?- P\isa conf\_p[authors\mvd\{\anon [name\fd
    $n$]\}].at\_conf[editors\mvd\{E\}].
  \end{qrules}
  %%
  is the set of all pairs (\textsf{P},\textsf{E}) such that \textsf{P} is
  (the logical oid of) a paper written by $n$, and \textsf{E} is the
  corresponding proceedings editor.  If we also want to see the
  affiliations of the above editors, we only need to modify our query
  slightly:
  %%
  \begin{qrules}
    ?- P\isa conf\_p[authors\mvd\{\anon [name\fd
    $n$]\}].at\_conf[year\fd Y]..editors[affil(Y)\fd A].
  \end{qrules}
\end{example}
%%
Thus, \FLORA path expressions support navigation 
along the method application dimension using the operators
``.''  and
``..''. In addition, intermediate objects through which such navigation
takes place can be selected by specifying the properties of such objects
inside square brackets.\footnote{
  %%
  A similar feature is used in other languages, e.g., XSQL \cite{xsql-92}.
  %%
  }
%%

\index{method!self}
To access intermediate objects that arise implicitly in the middle
of a path expression, one can define the method \textsf{self} as
%%
\begin{quote}
  {\tt X[self{\fd}X].} 
\end{quote}
%%
and then simply write $\dots${\tt [self{\fd}O]}$\dots$ anywhere in a
complex path expression. This would bind the Id of the current object to
the variable {\tt O}.

\begin{example}[Path Expressions with \textsf{self}]\label{ex-path-self}
  \rm{
    %%
    To illustrate convenience afforded by the use of the {\tt self}
    attribute in path expressions, consider the second query in
    Example~\ref{Ex:PathExpr}. If, in addition, we want to obtain the names
    of the conferences where the respective papers were published, that
    query can be reformulated as follows:
    %%
    }
  \begin{qrules}
    X[self\fd X].\\
    ?- P\isa conf\_p[authors\mvd\anon [name\fd
    $n$]].at\_conf[self\fd C, year\fd Y]..editors[affil(Y)\fd A]. 
  \end{qrules}
\end{example}


\subsection{Path Expressions in the Rule Head}\label{sec-pathexp-head}


Only single-valued path expressions are allowed in a rule head. Set-valued
path expressions are not allowed because the semantics is not always clear
in such cases.

\index{path expression!in rule head}
The following is an example of a path expression in rule head. For any
person X, this rule defines the grandchildren for X's mother.
\begin{qrules}
X.mother[grandson{\mvd}Y] :- X{\isa}person[son{\mvd}Y].
\end{qrules}
%%
Here X's mother is treated as an unknown object. Should this object become
known in the future, complications may arise. For instance, suppose that
later on the following facts were added or derived:
%%
\begin{qrules}
  john[mother{\fd}mary]. \\
  john[son{\mvd}david].
\end{qrules}

Because John's mother is now identified as {\tt mary}, it follows that
${\tt mary}$ and ${\tt john.mother}$ represent the same things, since the
attribute {\tt mother} is scalar. To deal with single-valued path
expressions in rule heads, \FLORA \emph{skolemizes} ${\tt john.mother}$ and
adds the requisite equalities to record that the Skolem term that
corresponds to {\tt john.mother} equals {\tt mary}.  All this is done by
the \FLORA compiler transparently to the user: if a path expression in rule
head is detected, \FLORA replaces this expression with a Skolem function.
The problem here is that maintenance of such equalities is costly
(sometimes causing a slowdown by the factor of 2--10 times). Since the user
often knows that equations of the above kind will never be derived, \FLORA
leaves it to the user to tell the compiler whether equality maintenance is
needed.  Equality maintenance is discussed in Section~\ref{sec-eqmaintain}.


\section{Truth Values, Object Values, and Predicate Meta Signatures}
\label{sec-references}


Id-terms, \fl atoms, and path expressions can all be used as objects. This
is obvious for Id-terms and the object interpretation of path expressions
of the form (7) and (8) on page~\pageref{eq-path-fun} was discussed
earlier. The \fl atoms (1) through (6) on pages~\pageref{eq-scalar-atom}
through~\pageref{eq-subclass} are typically viewed as formulas and, thus,
they are assumed to have a truth value only.  However, there also is a
natural way to give them object interpretation.  For example, {\tt
  o{\isa}c[m{\fd}r]} has object value {\tt o} and some truth value.
However, unlike the object value, the truth value depends on the database
(on whether {\tt o} belongs to class {\tt c} in the database and whether
the value of the attribute {\tt m} is, indeed, {\tt r}.

Although previously we discussed only the object interpretation for path
expressions, it is easy to see that they have truth values as well, because
a path expression corresponds to a conjunction of F-logic atoms.
Consequently, all F-molecules of the form (1) through (8) have dual
reading: As logical formulas (\emph{the deductive perspective}), and as
expressions that represent one or more objects (\emph{the object-oriented
  perspective}).  Given an intended model, \db I, of an \fl program an
expression has:
%%
\begin{itemize}
  \index{molecule!object value}
\item An \emph{object value}, which yields the Id(s) of the object(s)
  that are reachable in \db I by the corresponding expression, and 
  \index{molecule!truth value}
\item A \emph{truth value}, like any other literal or molecule of the
  language. 
\end{itemize}
%%
An important property that relates the above interpretations is: a
molecule, $r$, evaluates to \emph{false} if \db I has no object
corresponding to $r$.


Consider the following path expression and an equivalent, decomposed
expression:

\begin{equation}\label{eq-decomp}
\tt
a..b[c\mvd\{d.e\}] \quad\ \Leftrightarrow \quad\  a[b\mvd X_{ab}]
\land d[e\fd X_{de}] \land X_{ab}[c\mvd X_{de}]. 
\end{equation}

\noindent
Such decomposition is used to determine the truth value of arbitrarily complex
path expressions in the \emph{body} of a rule.  Let $\tt \obj(path)$ denote
the Ids of all objects represented by the path expression. Then, for
(\ref{eq-decomp}) above, we have:

\begin{displaymath} \tt
\obj(a..b) = \{x_{ab} \mid \db I \models a[b\mvd x_{ab} ]\}
\qquad\textrm{ and }\qquad \obj(d.e) = \{x_{de} \mid \db I \models d[e\fd 
x_{de}]\} 
\end{displaymath}
%
where $\db I \models \varphi$ means that $\varphi$ holds in \db I.  Observe
two formulas can be equivalent, but their object values might be different.
For instance, $\tt d[e\fd f]$ is equivalent to {\tt d.e} as a formula.
However, $\tt \obj(d.e)$ is $f$, while $\tt \obj(d[e\fd f])$ is {\tt d}.

In general, for an \fl\ database \db I, the object values of ground path
expressions are given by the following mapping, \obj, from ground molecules
to sets of ground oids ($t$, $o$, $c$, $d$, $m$ can be oids or path
expressions):
%
\begin{displaymath}
  \begin{array}{cll@{\hspace{4em}}c}
    \obj(t) & := & \{t \mid  \db I\models t[] \}, 
     \textrm{ for a ground Id-term $t$}  \\   
                                %
    \obj(o[\dots]) & := & \{o1 \mid o1\in\obj(o), \db I \models o1[\dots]
    \} \\  
                                %
    \obj(o\isa c) & := & \{o1 \mid o1\in\obj(o), \db I \models o1\isa c\}
     \\ 
                                %
    \obj(c\subcl d) & := & \{c1 \mid c1\in\obj(c), \db I \models c1\subcl
    d\} \\ 
                                %
    \obj(o.m) & :=  & \{r1 \mid r1\in\obj(r), \db I \models o[m\fd
    r]\} \\ 
                                %
    \obj(o..m) & := &  \{r1 \mid  r1\in\obj(r), \db I \models
    o[m{\mvd}\{r\}] \}
  \end{array}
\end{displaymath}

Observe that if $\tt t[]$ does not occur in \db{I}, then $\obj(t)$ is
$\emptyset$.  Conversely, a ground molecule $r$ is called \emph{active} if
$\obj(r)$ is not empty. A molecule, $r$, can be 
single-valued or multi-valued:
%%
\begin{itemize}
\item $r$ is called \emph{multi-valued} if
 \begin{itemize}
  \item it has the form $o..m$, or 
  \item it has one of the forms $\underline{o}[\dots]$,
    $\underline{o}\isa c$, $\underline{c}\subcl d$, or
    $\underline{o}.\underline{m}$, and any of the underlined
    subexpressions is multi-valued;
 \end{itemize}
\item in all other cases, $r$ is \emph{single-valued}.
\end{itemize}

\paragraph{Dual representation and meta-predicates.}
Since path expressions can appear wherever Id-terms are allowed, the
question arises whether a path expression is intended to indicate a truth
value or an object value. For instance, we may want to call the XSB
aggregate predicate {\tt findall/3} to retrieve the oids of the managers
who like to play tennis:
%%
\begin{qrules}
findall(P,P{\isa}manager[hobbies{\mvd}tennis],L)
\end{qrules}
%%
If all arguments are treated as objects, then this expression would mean
%%
\begin{quote}
 \tt
 P{\isa}manager[hobbies{\mvd}tennis], findall(P,P,L)
\end{quote}
%%
and a meaningless result --- the list of all instances of {\tt P} that
evaluate to true --- will be returned. Since, due to the first F-molecule,
these instances of {\tt P} must be the oids of the tennis-playing managers,
such as {\tt john}, the list {\tt L} would contain those oids that happen
to be true as 0-ary facts. (For example, {\tt john} would end up on such a
list if the fact {\tt john} is true as a 0-ary predicate).

The upshot of this problem is that the interpretation of \fl expressions as
objects is not always suitable. In our example, we need to indicate to the
compiler that the middle argument of {\tt findall/3} ought to be translated
into XSB as follows:
%%
\begin{quote}
 \tt
 findall(P,F(P),L)
\end{quote}
%%
where {\tt F(P)} is an formula into which {\tt
  P{\isa}manager[hobbies{\mvd}tennis]} is translated by the \FLORA
compiler.\footnote{
  %%
  Something like {\tt isa(P,manager),mvd(P,hobbies,tennis)}.
  %%
  }
%%
Under this translation, findall/3 will find all instances of {\tt
  P} that make {\tt F(P)} true and return a list of such instances in the
list {\tt L}. This feat is accomplished using the {\tt arguments} compiler
instruction.

\index{predicate meta signature}
\index{meta signature of a predicate}
\index{compiler directive!{\tt arguments}}
%
By default, when a truth-valued expression, $E$, such as an \fl molecule or
a path expression, appears as an argument to a predicate or function
symbol, {\it p}, the \FLORA compiler will pass the object value of the
expression to {\it p}. In order to tell the compiler that the entire
formula obtained by translating $E$ into Prolog is to be passed to $p$, we
need to define the \emph{meta signature} of $p$ using the compiler
directive {\tt arguments}.  For instance, the following directive will make
{\tt findall/3} work as intended:
%%
\begin{qrules}
:- arguments findall(oid,bform,oid).
\end{qrules}
%%
This signature declaration tells \FLORA that the first and the third
arguments to {\tt findall/3} should be translated as oids, while the second
argument of {\tt findall/3} should be treated as a formula of the kind that
might occur in the body of a rule.



\section{HiLog and Related Issues} \label{sec:hilog}


\index{HiLog}
%
HiLog \cite{hilog-jlp} is the default syntax that \FLORA uses to represent
functor terms (including object Ids) and predicates.  In HiLog, complex
terms can appear wherever a function symbol is allowed. For example, {\tt
  group(X)(Y,Z)} is a HiLog term where the functor is no longer a symbol
but rather a complex term {\tt group(X)}. Variables in HiLog can range over
terms, predicate and function symbols, and even over atomic formulas. For
instance,
%%
\begin{quote}
 ?- p(X), X(p), X.  
\end{quote}
%%
is a perfectly legal query. If {\tt p(a(b))}, {\tt a(b)(p)},
and {\tt a(b)} are all true in the database, then $X=a$ is one of the
answers to the query.

\index{HiLog!translation}
%
Although HiLog has a higher order syntax, its semantics is first order
\cite{hilog-jlp}. Any HiLog term can be consistently translated into a
Prolog term. For instance, {\tt group(X)(Y,Z)} can be represented by the
Prolog term {\tt apply(apply(group,X),Y,Z)}. The translation scheme is
pretty straightforward and described in \cite{hilog-jlp}.

In \FLORA any Id-term, including function symbols and predicate symbols,
are considered to be HiLog terms and therefore are subject to translation.
That is, even a normal Prolog term will by default be represented using the
HiLog translaiton, e.g., {\tt foo(a)} will be represented as {\tt
  apply(foo,a)}. This guarantees that HiLog unification will work correctly
at runtime. For instance, {\tt foo(a)} will unify with {\tt F(a)} and bind
the variable {\tt F} to {\tt foo}.


\index{HiLog!unification}
%
HiLog has one interesting peculiarity: If {\tt foo}  is a symbol,
then {\tt foo} and {\tt foo()} are not the same, and both are different
from {\tt foo()()}. In terms of the translation, this means that {\tt foo}
is different from {\tt apply(foo)} is different from {\tt apply(apply(foo))}.

This, it would seem that the following queries are different:
%%
{\tt
\begin{quote}
   ?- p.\\
   ?- p().
\end{quote}
}
%%
\noindent
However, considering that calling 0-ary predicates is quite common in logic
programming, we decided to make {\tt p} and {\tt p()} the same {\em when
  they occur as predicates in the rule head or body}. Thus, in the following
program,
%%
{\tt
\begin{quote}
 p. \\
 q().\\
 ?- p().\\
 ?- q.\\
 ?- r = r().
\end{quote}
}
%%
\noindent
the first two queries will succeed, but the last one will fail.
This grace does not extend to {\tt p()()}. In the following program,
%%
{\tt
\begin{quote}
 p.\\
 q().\\
 ?- p()().\\
 ?- q()().\\
 ?- p = p()().\\
 ?- q() = q()().
\end{quote}
}
%%
\noindent
all queries fail.


\subsection{Meta-programming}


\index{meta-programming}
%
\fl together with HiLog is powerful stuff. In particular, it lends itself
naturally to meta-programming. For instance, it is easy to examine the
methods and types defined for the various classes.  Here are some simple
examples:
%%
\begin{quote}
\begin{verbatim}
// all unary scalar methods defined for John
?- john[M(_) -> _].

// all unary scalar methods that apply to John,
// for which a signature was declared
?- john[M(_) => _].

// all method signatures that apply to John,
// which are either declared explicitly or inherited
?- john[M => _].

// all method invocations defined for John
?- john[M -> _].
\end{verbatim}
\end{quote}
%%

However, a number of meta-programming primitives are still needed
since they cannot be directly expressed in \fl. Many such features are
provided by the underlying XSB system and \FLORA simply takes advantage of
them: 
%%
\begin{quote}
\begin{verbatim}
?- functor(X,f,3).
X = f(_h455,_h456,_h457)
Yes.

?- compound(f(X)).
X = _h472
Yes.

?- X =.. [f,a,b].
X = f(a,b)
Yes.
\end{verbatim}
\end{quote}
%%
Note that these primitives are used for Prolog terms only and
are described in the XSB manual. Meta-programming support for HiLog terms
and F-molecules will be provided through future enhancement.


\subsection{Passing Arguments between \FLORA and XSB}\label{sec-passing-args}

The native HiLog support in \FLORA causes some tension when crossing the
border from one system to another. The reason is that \FLORA terms and XSB
terms have different internal representation. Even though XSB supports
HiLog (according to the manual, anyway), this support is incomplete and is
not integrated well into the system --- most notably into its module
system. As a result, XSB does not recognize terms passed to it from \FLORA
as HiLog terms and, thus, many useful primitives will not work correctly.
(Try {\tt ?- writeln(foo(abc))@prolog()} and see what happens.)

\index{flP2H/2}
\index{Prolog to HiLog conversion}
\index{HiLog to Prolog conversion}
%%
To cope with the problem, \FLORA provides a primitive, {\tt
  flP2H(Plg,Hlg)}, which does the translation. If the first argument,
{\tt Plg}, is bound, the primitive binds the second argument to the
Hilog representation of the term. If {\tt Plg} is already bound to a
Hilog term, then {\tt Hlg} is bound to the same term without conversion.
Similarly, if {\tt Hlg} is bound to a HiLog term, then {\tt Plg} gets
bound to the Prolog representation of that term. If {\tt Hlg} is bound to
a non-HiLog term, then {\tt Plg} gets bound to the same term without
conversion. In all these cases, the call to flP2H/2 succeeds. If {\tt both}
arguments are bound, then the call succeeds if and only if
%%
\begin{itemize}
\item {\tt Plg} is a Prolog term and {\tt Hlg} is its HiLog
  representation.
\item Both {\tt Plg} and {\tt Hlg} are identical HiLog terms.
\end{itemize}
%%
Note that if both {\tt Plg} and {\tt Hlg} are bound to the same
\emph{Prolog term} then the predicate \emph{fails}. Thus, if you type the
following queries into the \FLORA shell, they both succeed:
%%
{\tt
\begin{quote}
  ?- flP2H(X,f(a)), flP2H(X,f(a)).  \\
  ?- flP2H(f(a),f(a)).
\end{quote}
}
%%


Not all arguments passed back and forth to XSB need conversion. For
instance, {\tt sort/2}, {\tt ground/1}, {\tt compound/1}, and many others
do not need conversion because they work the same for Prolog and HiLog
representations. On the other hand, most I/O predicates require conversion.
In a future release, \FLORA will provide libraries of useful predicates and
methods that do appropriate conversion without the user having to do
this explicitly.


\subsection{First-Order Predicates and HiLog Predicates}
\label{sec:flora-modules-predicates}

\index{predicate!first-order}
\index{predicate!HiLog}
%%
\FLORA is an object-relational language in the sense that it supports both
the object-oriented syntax of \fl and the predicate-based ({\it i.e.},
relational) syntax of Prolog. Incorporation of predicates is conceptually
simple, but is somewhat burdened by pragmatic considerations, namely,
integration with HiLog. As a result, \FLORA has two types of predicates,
{\em first-order predicates\/} and \emph{HiLog predicates}. 

Recall that the main reason for using HiLog as the native syntax and
semantics in \FLORA is meta-programming. However, this is also one of the
reasons for adding the first-order predicates. To explain, consider the
following query:
%%
\begin{quote}
 \tt
 ?- X(a,b).
\end{quote}
%%
which returns all names of binary predicates that contain
the tuple {\tt a,b}. However, it is sometimes useful to exclude certain
predicates from the domain of the meta-variables, such as {\tt X} above.
For instance, if {\tt write/2}  is an I/O predicates, should the above
query return the binding {\tt X=write}? If so, if {\tt a} happens to be the
name of a file and {\tt b} a string, should this string be written out to
that file as a side effect of the evaluation of this query?

To avoid these semantic problems, \FLORA supports the {\tt firstorder}
declaration, which defines certain symbols as first-order predicates. For
instance,
%%
\index{compiler directive!{\tt firstorder}}
%%
\begin{quote}
 \tt
 :- firstorder foo/1, moo/3.
\end{quote}
%%
Predicates defined in this way are outside of the range of HiLog variables.
If you want all predicates to be shielded from meta programming, you can use
the following compiler directive:
%%
\index{compiler directive!{\tt firstorderall}}
%%
\begin{quote}
  \tt
  :- firstorderall.
\end{quote}

First-order predicates are module-aware. This means that the predicates defined
in different modules are treated as unrelated entities.

There is one other reason why we might sometimes choose first-order
predicates over HiLog predicates --- predicate tabling. This issue
is discussed in Section~\ref{sec-tabling-flora}.



\subsection{Abolishing First-Order Predicates}


\index{module!{\tt expunge}}
%
To expunge a predicate of the form {\tt functor/arity} from the module that
a program is loaded into, the following compiler directive can be used:
\begin{quote}
\verb!:- expunge functor/arity.!
\end{quote}

\noindent
{\tt expunge} can also be called like a normal predicate,
either from the shell or from within a rule body, to expunge
predicates in one specific module. The syntax is shown below:
%%
\begin{quote}
\verb!:- expunge functor/arity, ..., functor/arity in modulename.!
\end{quote}
where {\tt in} is a special keyword, and {\tt module} stands for the
name of the module from which to expunge the list of predicates.

Note that when {\tt expunge functor/arity} is used as a normal predicate
(not as a compiler directive), it will expunge the predicate for the
default module.




\section{Equality Maintenance}\label{sec-eqmaintain}


Unlike regular Prolog, \FLORA terms can become {\tt equal}, which is a side
effect of the \fl semantics for scalar methods.  For instance, consider the
following \fl facts:
%%
\begin{quote}
mary[spouse{\fd}john]. \\
mary[spouse{\fd}joe]. \\
john[son{\mvd}frank].
\end{quote}
%%
Since {\tt spouse} is a scalar attribute, it can have at most one value for
any given object. This implies that the oids {\tt john} and {\tt joe} must
refer to the same object. Therefore, whatever is true about {\tt john}
should be also true about {\tt joe}, and vice versa.  Thus we should be
able to derive that {\tt joe[son{\mvd}frank]}.

What is illustrated above is just a very simple scenario of equality
maintenance. Another scenario arises due to the path expressions in the
rule heads (Section~\ref{sec-pathexp-head}).  Furthermore, users can define
equality \emph{explicitly} in the source program using the predicate {\tt
  :=:}, e.g.,
%%
\begin{quote}
\verb|john:=:batman.|
\end{quote}

Once oids are made equal, this fact may need to be propogate to all \fl
structures, including the subclass hierarchy, the ISA hierarchy, etc.,
depending on the application domain. Although equality is a powerful
feature, its maintenance can slow the program down quite significantly.
In order to be able to eat the cake and have it at the same time, \FLORA
allows the user to control how equality is handles.
by providing the following three compiler directives:
%%
\index{equality}
\index{compiler directive!equality}
%%
{\tt
\begin{quote}
:- equality none.  (default)\\
:- equality basic.\\
:- equality flogic.
\end{quote}
}

\noindent
These directives affect the particular modules in which they are included.
Thus, equality can be treated differently in different modules. This allows
to compartmentalize the problem and, if used judiciously, can lead to
significant gain in performance.

To first directive, \mbox{\tt equality none}, does not maintain equality
and {\tt :=:} is considered just like any other user-defined predicate
without any special semantics.  The directive \mbox{\tt equality basic}
guarantees that {\tt :=:} obeys the usual congruence rules for equality,
i.e., transivity, reflexity, and symmetry, and substitution, but not the
special \fl equality rule for scalar methods. Finally, \mbox{\tt equality
  flogic} means that the full equality theory for \fl is in force.  In
particular, programs with path expressions in the rule heads might require
this level of equality to work correctly (if there is a possibility that
the Skolem terms introduced to represent the path expressions might be
equal to some real objects).

Note that equality can also be maintained when working with the \FLORA
shell. Furthermore, \FLORA allows one module to set the level of equality
maintenance in another module:
%%
\begin{quote}
\verb!equality {none|basic|flogic} in modulename!
\end{quote}
%%
\index{dynamic module}
%%
This might be useful for \emph{dynamic} modules, {\it i.e.}, modules that
are not associated with any files and whose content is generated completely
dynamically. (See Section~\ref{sec-updates}.)

One final advice regarding equality. In many cases, programmers tend to use
equality as an aliasing technique for long messages, numbers, etc. In this
case, we recommend to use the preprocessor commands, which achieve the same
result without loss of performance. For instance,

%%
\begin{verbatim}
  #define YAHOO  'http://yahoo.com'  

  ?- YAHOO[fetch \fd X].
\end{verbatim}
%%
\noindent
Assuming that {\tt fetch} is a method that applies to strings that
represent WWW sites and that fetches the corresponding Web pages, the above
program will fetch the page at the Yahoo site, because \FLORA compiler will
replace YAHOO with the corresponding string that represents a URL.


\section{Inheritance}\label{sec-inheritance}


\index{inheritance!structural}
\index{inheritance!behavioral}
%%
\fl supports two types of inheritance: \emph{structural} and
\emph{behavioral}.  Structural inheritance applies to signatures only. For
instance, if {\tt student::person} and a program defines the signature
{\tt person[name{\Fd}string]} then the query {\tt ?- student[name{\Fd}X]}
succeeds with {\tt X=string}.

\index{inheritance!non-monotonic}
Behavioral inheritance is much more complicated. The problem  is that it is
\emph{non-monotonic}. That is, the addition of new facts might obviate previously
established inferences.

\index{attribute!inheritable}
\index{attribute!non-inheritable}
\fl (and \FLORA) distinguishes between attributes and methods that can
inherit values from superclasses and those that do not. The syntax that we
have seen so far applies to \emph{non-inheritable} attributes only.
\emph{Inheritable attributes} are declared using the {\tt *=>} and {\tt *=>>}
style arrows and defined using the {\tt *->} and {\tt *->>} style arrows. For
instance, the following is a \FLORA program for the classical {\tt Royal
Elephant} example:
%%
\begin{quote}
\begin{verbatim}
elephant[color*=>color].
royal_elephant::elephant.
clyde:elephant.
elephant[color*->gray].
\end{verbatim}
\end{quote}
%%
The question is what is the color of {\tt clyde}?
{\tt clyde}'s color has not been defined in the above program. However, since
{\tt clyde} is an elephant and the default color for elephants is gray,
{\tt clyde}
must be gray. Thus, we can derive:
%%
\begin{quote}
\begin{verbatim}
clyde[color->gray].  
\end{verbatim}
\end{quote}
%%
Observe that when inheritable methods are inherited from a class by its
members, the attribute becomes non-inheritable. On the other hand, when
such a method is inherited by a subclass from its superclass, then the
method is still inheritable, so it can be further inherited by the members
of that subclass or by its subclasses. For instance, if we have
%%
\begin{quote}
\begin{verbatim}
circus_elephant::elephant.
\end{verbatim}
\end{quote}
%%
then we can derive 
%%
\begin{quote}
\begin{verbatim}
circus_elephant[color*->gray].  
\end{verbatim}
\end{quote}
%%

Non-monotonicity of behavioral inheritance becomes apparent when certain new
information gets added to the knowledge base. For instance, suppose we
learn that
%%
\begin{quote}
\begin{verbatim}
royal_elephant[color*->white].  
\end{verbatim}
\end{quote}
%%
Although we have previously established that {\tt clyde} is gray, this
new information renders our earlier conclusion invalid. Indeed, Since
{\tt clyde} is a royal elephant, he must be white, while being an
elephant he must be gray.  The conventional wisdom in object-oriented
languages, however, is that inheritance from more specific classes
must take precedence. Thus, we must withdraw our earlier conclusion
that {\tt clyde} is gray and infer that he is white:
%%
\begin{quote}
\begin{verbatim}
clyde[color->white].    
\end{verbatim}
\end{quote}
%%

Behavioral inheritance in \fl is discussed at length in \cite{KLW95}.
The above problem of non-monotonicity is just a tip of the iceberg. Much
more difficult problems arise when inheritance interacts with the regular
deduction. To illustrate, consider the following program:
%%
\begin{quote}
\begin{verbatim}
b[m*->>c].
a:b.
a[m->>d] :- a[m->>c].
\end{verbatim}
\end{quote}
%%
In the beginning, it seems that \verb|a[m->>c]| should be derived by
inheritance, and so we can derive \verb|a[m->>d]|. Now, however, we can
reason in two different ways:
%%
\begin{enumerate}
\item \verb|a[m->>c]| was derived based on the belief that attribute {\tt
    m} is not defined for the object {\tt a}. However, once inherited,
  necessarily we must have \verb|a[m->>{c,d}]|. So, the value of
  attribute {\tt m} is not really the one produced by inheritance. In other
  words, inheritance of \verb|a[m->>c]| negates the very premise on which
  the original inheritance was based, so we must undo the operation and the
  ensuing rule application.
\item We did derive \verb|a[m->>d]| as a result of inheritance, but
  that's OK --- we should not really be looking back and undo previously
  made inheritance inferences. Thus, the result must be \verb|a[m->>{c,d}]|.
\end{enumerate}
%%
A similar situation (with similarly conflicting conclusions) arises when
the class hierarchy is not static. For instance, 
%%
\begin{verbatim}
 d[m*->e]
 d::b.
 b[m*->c].
 a:b.
 a:d :- a[m->c].  
\end{verbatim}
%%
If we inherit {\tt a[m \fd c]}  from {\tt b} (which seems to be OK in the
beginning, because nothing overrides this inheritance), then we derive {\tt
  a:d}, {\it i.e.}, we get the following: {\tt a:d::b}. This means that
\emph{now} {\tt d} seems to be negating the reason why {\tt a[m \fd c]} was
inherited in the first place. Again, we can either undo the inheritance or
adopt the principle that inheritance is never undone.

A semantics that favors the second interpretation was proposed in
\cite{KLW95}. This approach is based on a fixpoint computation of
non-monotonic behavioral inheritance.  However, this semantics is very hard
to implement efficiently, especially using a top-down deductive engine
provided by XSB. It is also unsatisfactory in many respects because it is
not based on any model-theory. \FLORA uses a different, more cautious
semantics for inheritance, which favors the first interpretation above.
The basic idea can be summarized using the following rules, which define how
class instances inherit from the classes they belong to:
%%
\begin{quote}
\begin{verbatim}
// inheritance rules for scalar attributes
:- table defined/2, overwritten/3, conflict/3.

Obj[A->V] <- Obj:Class, Class[A*->V], tnot defined(Obj,A),
             tnot overwritten(Obj,Class,A), tnot conflict(Obj,Class,A).

defined(Obj,A) <- Obj[A->V].

overwritten(Obj,Class,A) <- Obj:Class1, Class1::Class,
                            Class1[A*->W], Class1 \= Class.

conflict(Obj,Class,A) <- Obj:Super, Super[A*->V],
                         tnot Super::Class, tnot Class::Super.

// inheritance rules for multi-valued attributes
:- table definedSet/2, overwrittenSet/3, conflictSet/3.

Obj[A->>V] <- Obj:Class, Class[A*->>V], tnot definedSet(Obj,A),
              tnot overwrittenSet(Obj,Class,A),
              tnot conflictSet(Obj,Class,A).

definedSet(Obj,A) <- Obj[A->>V].

overwrittenSet(Obj,Class,A) <- Obj:Class1, Class1::Class,
                               Class1[A*->>W], Class1 \= Class

conflictSet(Obj,Class,A) <- Obj:Super, Super[A*->V],
                            tnot Super::Class, tnot Class::Super.
\end{verbatim}
\end{quote}
%%
\index{well-founded semantics}
%%
Negation here is implemented using the {\em well-founded semantics}
for negation \cite{gelder-alternating-89,gelder-ross-schlipf-91} (as
indicated by the {\tt tnot} operator).  Similar rules are needed to
describe how classes inherit from superclasses.

Under this semantics, {\tt clyde} will still be white, but in the other two
examples {\tt a[m->c]} is \emph{not} inherited. Details of this semantics
and its model theory will be presented in a future paper.

In the examples that we have seen so far, path expressions use only
non-inheritable attributes. Clearly, there is no reason to disallow
inheritable attributes in such expressions. To distinguish inheritable
attributes from non-inheritable ones, \FLORA uses the symbols
{\tt !} and {\tt !!} in its path expressions. For instance,
%%
\begin{quote}
\begin{verbatim}
clyde!color           means: some X, such that clyde[color*->X]}.
obj!!attr             means: some Y, such that obj[attr*->>Y].
\end{verbatim}
\end{quote}
%%



\section{Aggregates}


The syntax for aggregates is similar to that in \FLORID. An
aggregate has the following form:
%%
\begin{qrules}
agg\{X[Gs] $|$ {\it body}\}
\end{qrules}
%%
\index{aggregation!aggregate operator}
\index{aggregation!grouping}
%
where {\tt agg} represents the aggregate operator, {\tt X} is called the
aggregation variable, {\tt Gs} is a list of comma-separated grouping
variables, and {\it body} is a logical formula that specifies the
query conditions. The grouping variables, {\tt Gs}, are optional. {\it body}
can be any combinaiton of conjunction, disjunction, and negation of literals.

All the variables appearing in {\it body} but not in {\tt X} or {\tt Gs} are
considered to be existentially quantified. Furthermore, the syntax of an
aggregate must satisfy the following conditions:
%%
\begin{enumerate}
\item All names of variables in both {\tt X} and {\tt Gs} must
appear in {\it body};
\item {\tt Gs} should not contain {\tt X}.
\end{enumerate}

Aggregates are evaluated as follows: First, the query
condition specified in {\it body} is evaluated to obtain all the bindings
for the template of the form {\tt \texttt{<}X, Gs\texttt{>}}. Then, these
tuples are grouped according to each distinct binding for
{\tt \texttt{<}Gs\texttt{>}}. Finally, for each group, the aggregate operator
is applied to the list of bindings for the aggregate variable $\tt X$.


\index{aggregates!min}
\index{aggregates!max}
\index{aggregates!count}
\index{aggregates!sum}
\index{aggregates!avg}
\index{aggregates!collectset}
\index{aggregates!collectbag}
%
The following aggregate operators are supported in \FLORA: {\tt min}, {\tt max},
{\tt count}, {\tt sum}, {\tt avg}, {\tt collectset} and {\tt collectbag}.

The operators {\tt min} and {\tt max} can apply to any list of
terms. The order among terms is defined by the XSB operator {\tt @=<}.  In
contrast, the operators {\tt sum} and {\tt avg} can take numbers only. If
the aggregate variable is instantiated to something other than a
number, {\tt sum} and {\tt avg} will discard it and generate a runtime
warning message.

For each group, the operator {\tt collectbag} collects all the bindings of
the aggregation variable into a list. The operator {\tt collectset} works
similarly to {\tt collectbag}, except that all the duplicates are removed
from the result list.

In general, aggregates can appear wherever a number or a list is
allowed. Therefore, aggregates can be nested. The following examples
illustrate the use of aggregates (some borrowed from the \FLORID manual):
%%
\begin{quote}
\begin{verbatim}
?- Z = min{S|john[salary(Year)->S]}.
?- Z = count{Year|john.salary(Year) < max{S|john[salary(Y)->S], Y < Year}}.
?- avg{S[Who]|Who:employee[salary(Year)->S]} > 20000. 
\end{verbatim}
\end{quote}
%%
If an aggregate contains grouping variables that are \emph{not} bound
by a preceding subgoal, then this aggregate would backtrack over such
grouping variables (In other words, grouping variables are considered to be
existentially quantified). For instance, in the last query above, the
aggregate will backtrack over the variable {\tt Who}. Thus, if
{\tt john}'s and {\tt mary}'s average salary is greater than {\tt 20000},
this query will backtrack and return both {\tt john} and {\tt mary}.

The following query returns, for each employee, a list of years when this
employee had salary less than 60. This illustrates the use of the {\tt
  collectset} aggregate.
%%
\begin{quote}
\begin{verbatim}
?- Z = collectset{Year[Who]|Who[salary(Year)->X], X < 60}.
Z = [1990,1991]
Who = mary

Z = [1990,1991,1997]
Who = john
\end{verbatim}
\end{quote}
%%

\subsection {Aggregation and Set-Valued Methods}

\index{aggregates!multi-valued methods}
\index{multi-valued methods!aggregation}
\index{->->}
\index{*->->}
%%
Aggregation is often used in conjunction with set-valued methods, and
\FLORA provides several shortcuts to facilitate this use.
In particular, the operators
{\tt ->->} and {\tt *->->}, for non-inheritable and
inheritable multivalued methods, collects all the values of the given
method for a given object in a set. The semantics of these operators is
as follows:

\begin{quote}
\begin{verbatim}
O[M->->L] :- L=collectset{V|O[M->>V]}

O[M*->->L] :- L=collectset{V|O[M*->>V]}
\end{verbatim}
\end{quote}

\noindent
Note that in {\tt O[M->->L]}  and {\tt O[M*->->L]}  {\tt L} is a list of oids.

Having special meaning for {\tt ->->} and {\tt *->->} means that these
constructs \emph{cannot} appear in the head of a rule.
One other caveat: recursion through aggregation is not supported and can
produce incorrect results.



\index{\tt +>>}
\index{\tt *+>>}
  %%
Sets collected in the above manner often  need to be compared to other
sets. For this, \FLORA provides another pair of primitives: {\tt +>>} and
{\tt *+>>} for non-inheritable and inheritable methods, respectively.
The atom of the form {\tt o[m+>>s]} is true if the set of all values of the
non-inheritable attribute {\tt m} for object {\tt o} \emph{contains} every
element in the list {\tt s}. 

For instance, the following query tests whether all Mary's children are
also John's children:
%%
\begin{quote}
\begin{verbatim}
?- mary[children->->L], john[children+>>L].
\end{verbatim}
\end{quote}

As with {\tt ->->} and {\tt *->->},
the use of {\tt +>>} and {\tt *+>>} is limited to rule bodies.



\section{Boolean Methods}


\index{method!boolean}
%
As a syntactic sugar, \FLORA introduces boolean methods, which can be
considered as scalar methods that return some fixed value, e.g.,
{\tt void}. For example, the following facts:
\begin{quote}
\verb|john[is_tall -> void].| \\
\verb|john[loves(tennis) -> void].|
\end{quote}
can be simplified as boolean methods as follows:
\begin{quote}
\verb|john[is_tall].| \\
\verb|john[loves(tennis)].|
\end{quote}

Conceptually, boolean methods are statements about objects whose truth
value is the only concern. Boolean methods do not return any value (not
even the value {\tt void}). Therefore, boolean methods {\tt cannot} appear
in path expressions. For instance, \mbox{\tt john.is\_vegetarian} is
illegal.

Like scalar methods, boolean methods can also be inheritable. To make a
boolean method inheritable, the ``\verb|*|'' sign is added before
the method:
%%
\begin{quote}
\begin{verbatim}
buddhist[*is_vegetarian].
john:buddhist.
\end{verbatim}
\end{quote}
%%
The above says that all Buddhists are vegetarian and John (the object with
oid {\tt john}) is a Buddhist. Since \verb|is_vegetarian| is inheritable,
it follows that John is also a vegetarian, i.e.,
\verb|john[is_vegetarian]|.


\section{Anonymous Oid}


\index{anonymous oid}
%
For applications where oids are not important, \FLORA provides the
compiler directive \verb|_#| to automatically generate a new
oid. \verb|_#| can be used wherever an Id-term is allowed. Like the
anonymous variable \verb|_|, each occurrence of \verb|_#| represents
an anonymous oid. The difference is that such an oid is not only
unique in each rule, but in the source program as well.

Of course, uniqueness is achieved through the use of special weird naming
schema for such oids. However, as long as the user does not use a similar
naming convention (who, on earth, would give names that begin with lots of
{\tt ``\_\$''}s?), uniqueness is guaranteed.


For example, in the following program:
%%
\begin{quote}
\begin{verbatim}
_#[ssn->123, father->_#[name->john, spouse->_#[name->mary]]].
foo[_#(X)->Y] :- foo(X,Y).
\end{verbatim}
\end{quote}
%%
the compiler will generate unique oids for each occurrence of {\tt \_\#}.
Note that, in the second clause, only one oid is generated and it serves as
a method name.



\section{\FLORA and Tabling}\label{sec-tabling-flora}


\subsection{Tabling in a Nutshell}


\index{tabling}
%
Tabling is a technique that enhances top-down query evaluation with a
mechanism that remembers the calls made previously in the process.  This
technique is known to be essentially equivalent to the Magic Sets method
for bottom-up evaluation. However, tabling combined with top-down
evaluation has the advantage of being able to utilize highly optimized
compilation techniques developed for Prolog. The result is a very efficient
deductive engine.

XSB lets the user specify which predicates must be tabled.  The \FLORA
compiler automatically tables the predicates used to represent F-molecules.
However, the user is responsible for telling the system which other
predicates --- first-order and HiLog --- must be tabled.  \FLORA
understands one tabling directive for first-order predicates and another
for HiLog.  (Section~\ref{sec-comp-directives} lists all the compiler
directives.)

\index{table}
\index{compiler directive!{\tt table}}
%
To table a first-order predicate, the following can be used:
\begin{quote}
\begin{verbatim}
:- firstorder tc/2.
:- table tc/2.

tc(X,Y) :- edge(X,Y).
tc(X,Y) :- edge(X,Y), tc(Y,Z).
\end{verbatim}
\end{quote}
%%
which renders {\tt tc/2}  as a tabled first-order predicate.
(In the future, the tabling directive will imply {\tt firstorder}.)

\index{hilogtable}
\index{compiler directive!{\tt hilogtable}}
%%
To table HiLog predicates, \FLORA requires you to table \emph{all}
predicates of a given arity: For instance,
%%
\begin{quote}
 \tt
 :- hilogtable 3.
\end{quote}
%%
tables all HiLog predicates of arity 3.

It is important to keep in mind that XSB does not do reordering of \fl
molecules and predicates during joins. Instead, all joins are performed
left-to-right.  Thus, program clauses must be written in such a way as to
ensure that smaller predicates and classes appear early on in the join.
Also, even though XSB tables the results obtained from previous queries,
the current tabling engine has several limitations. In particular, when a
new query comes in, XSB tries to determine if this query is ``similar'' to
one that already has been answered (or is in the process of being
evaluated).  Unfortunately, the default notion of similarity used by XSB is
fairly weak, and many unnecessary recomputations might result. Recently, a
new technique, called \emph{subsumptive tabling}, has been implemented in
XSB. It is known that subsumptive tabling can speed up certain queries by
an order of magnitude. A future version of \FLORA might take advantage of
this technique.


\subsection{Procedural Methods}\label{sec-proc-methods}

\index{abolish\_all\_tables}
%%
When XSB (and \FLORA) evaluate a program, all tabled predicates are
partially materialized and all the computed tuples are stored in XSB
tables. Thus, if you change the set of facts, the existing tables must be
discarded in order to allow XSB to recompute the results. This is
accomplished by issuing the predicate {\tt abolish\_all\_tables/0}
described in the XSB manual. We discuss updates in
Section~\ref{sec-updates}.

\index{method!procedural}
%
Because tabling is not integrated with the update mechanism in XSB, it can
have undesirable effect on predicates with non-logical ``side effects''
(e.g., writing or reading a file) and predicates that change the state of
the database.  If a tabled predicate has a side effect, the first time the
predicate is called the side effect is performed, but the second time the
call simply returns with success or failure (depending on the outcome of
the first call), because XSB will simply look it up in a table.  Thus, if
the predicate is intended to perform the side effect each time it is
called, it will not operate correctly.

Object-oriented programms often rely on methods that produce side effects
or make updates.  In \FLORA we call such methods \emph{procedural}.
Because by default \FLORA tables everything that looks like an F-molecule,
these procedural methods are potentially subject to the aforesaid problem.

To sidestep this program, \FLORA introduces a new syntax to identify
procedural methods --- by allowing the ``\verb|#|'' sign in front of a
procedural method. For instance, the following rule defines an output
method that, for every object, writes out its oid:
%%
\begin{quote}
\verb|O[#output] :- write(O)@prolog().|
\end{quote}
%%
Like boolean methods, procedural methods can take arguments, but do not
return any values.  The only difference is that procedural methods are
\emph{not} tabled, while boolean methods are.


\subsection{Cuts}


\index{cuts in \FLORA}
\index{cutting across tables}
%%
No discussion of a logic programming language is complete without a few
words about the infamous Prolog cut (!). Although Prolog cut has been
(mostly rightfully) excommunicated as far as Database Query Languages are
concerned, it is sometimes indispensable when doing ``real work'', like
pretty-printing \FLORA programs or implementing a pattern matching
algorithm.  To facilitate this kind of tasks, \FLORA lets the programmer
use cuts.  However, the current implementation of XSB has a limitation that
Prolog cuts cannot ``cut across tabled predicates.''  If you get an error
message telling something about cutting across the tables --- you know that
you have cut too much!

The basic rule that can keep you out of trouble is: do not put a cut in the
body of a rule \emph{after} any F-molecule or tabled first-order or HiLog
predicate. However, it is OK to to put a cut before any F-molecule. It is
even OK to have a cut in the body of a rule that \emph{defines} an
F-molecule (again, provided that the body has no F-molecule to the left of
that cut). If you need to use cuts, plan on using procedural methods or
non-tabled predicates.

In a future release, XSB will implement a different tabling schema. While
cutting across tables will still be prohibited, it will provide an
alternative mechanism that achieves many of the goals a cut is used to
achieve.


\section{Updates}\label{sec-updates}


\index{update}
\index{predicate!derived part of}
\index{predicate!base part of}
\index{object!base part of}
\index{object!derived part of}
\index{derived part of predicate}
\index{base part of predicate}
%
\FLORA provides primitives to update the runtime database. Unlike Prolog,
\FLORA does not require the user to define a predicate as dynamic in order
to update it. Instead, every predicate (HiLog or first-order) and object
has a \emph{base part} and a \emph{derived part}.  Updates directly update
only the base parts and only indirectly the derived parts.

\FLORA updates can be \emph{non-backtrackable,} as in Prolog,
or backtrackable, as in Transaction Logic \cite{trans-tcs94}.
We first describe non-backtrackable updates.


\subsection{Non-backtrackable Updates} \label{sec:non-backtrackable-updates}

\index{non-backtrackable update}
\index{update!non-backtrackable}
%
The effects of non-backtrackable updates persist even if a subsequent failure
causes the system to backtrack.

\FLORA supports the following non-backtrackable update primitives:
{\tt insert}, {\tt insertall}, {\tt delete}, {\tt deleteall},
{\tt erase}, {\tt eraseall}. These primitives use special syntax (the curly
braces) and are \emph{not} predicates. Thus, it is allowed to have a
user-defined predicate such as {\tt insert}.

\index{non-backtrackable update!insert}
\index{non-backtrackable update!insertall}
%
\paragraph{Insertion.} The syntax of an insertion is as follows (note the
\{,\}s!):
{\tt
\begin{quote}
\emph{insop}\{\emph{literals} [| \emph{query}]\}
\end{quote}
}
where {\it insop} stands for either {\tt insert} or {\tt insertall}.
The
{\it literals} part represents a comma separated list of literals,
which can include
predicates and \mbox{F-molecules}. The optional part, {\tt |}{\it query},
is an additional condition that must be satisfied in order for
\emph{literals} to be deleted.
The semantics is that \emph{query} is posed first and, if it is
satisfied, \emph{literals} is inserted (note that the query may affect the
variable binding and thus the particular instance of \emph{literals} that
will be inserted). For instance, in
\begin{quote}
\begin{verbatim}
flora2 ?- insert{p(a),mary[spouse->smith,children->>frank]}
flora2 ?- insertall{P[spouse->S] | S[spouse->P]}
\end{verbatim}
\end{quote}
%%
the first statement inserts a particular molecule. In the second case, the
query {\tt S[spouse->P]} is posed and one answer (a binding for {\tt P} and
{\tt S}) is obtained. If there is no such binding, nothing is inserted and
the statement fails. Otherwise, the instance of {\tt P[spouse->S]} is
inserted for that binding and the statement succeeds.

The difference between {\tt insert} and {\tt insertall} is that {\tt
  insert} inserts only one instance of \emph{literals} that satisfies the
formula, while {\tt insertall} inserts \emph{all} instances of the literals
that satisfy the formula. In other words, \emph{query} is posed first and
\emph{all} answers are obtained. Each answer is a tuple of bindings for
some (or all) of the variables that occur in \emph{literals}.  To
illustrate the difference between {\tt insert} and {\tt insertall},
consider the following statements:
%%
{\tt
\begin{quote}
  flora2  ?- p(X,Y), insert\{q(X,Y,Z)|r(Y,Z)\}.\\
  flora2  ?- p(X,Y), insertall\{q(X,Y,Z)|r(Y,Z)\}.
\end{quote}
}
%%
In the first case, if {\tt p(x,y)} and {\tt r(y,z)} are true, then the fact
{\tt q(x,y,z)} is inserted. In the second case, if {\tt p(x,y)} is true,
then the update means the following:
%%
\begin{quote}
  For each z such that r(y,z) holds, insert(x,y,z).
\end{quote}
%%
\index{bulk insert}
\index{insert!bulk}
%%
The primitive {\tt insertall} is also known as a bulk-insert operator.

Note that literals appearing inside the update predicates are treated as
facts and should follow the syntactic rules for facts and literals in the
rule head. In particular, multi-valued path expressions are not allowed.
However, \FLORA permits databases updates whose target is a module other
than the current one.  Thus, flora module specifications \emph{are}
allowed. As before, module specification is distributing through
parentheses, so
%%
\begin{quote}
\begin{verbatim}
flora2 ?- insert{(mary[children->>frank], john[father->smith]) @ foomod}
\end{verbatim}
\end{quote}
is equivalent to the following one:
\begin{quote}
\verb|flora2 ?- insert{mary[children->>frank] @ foomod, john[father->smith] @ foomod}|
\end{quote}

However, @prolog(...) is not allowed: If you want to update a dynamic XSB
predicate, use {\tt assert/retract}:
%%
\begin{quote}
 {\tt flora2 ?- ?- assert(foo(a,b,c))@prolog().  }
\end{quote}
%%
Well, not quite. Recall Section~\ref{sec-passing-args} on the issues
concerning the difference between Prolog representation of terms and the
one used in XSB. The problem is that {\tt f(a,b,c)} is a HiLog term that
XSB does not understand and will not associate it with the predicate {\tt
  foo/3}  that it might have. To do it right, use explicit conversion:
%%
\begin{quote}
 {\tt ?- flP2H(PrologRepr,foo(a,b,c)) assert(PrologRepr)@prolog().  }
\end{quote}
%%
This will insert {\tt foo(a,b,c)} into the catch-all module of XSB called
{\tt usermod}. If {\tt foo/3} is defined in another XSB module and is
imported by {\tt usermod}, then the above statement will not do anything
useful --- the XSB module system is too involved and requires more brain cells
that we can afford to spend explaining it.

We should also mention one important difference between insertion of facts
in \FLORA and Prolog. Prolog treats facts as a \emph{bag}, so duplicates
are allowed. In contrast, \FLORA treats the database as a \emph{set} of
facts with no duplicates. Thus, insertion of a fact that is already in the
database has no effect.


\index{non-backtrackable update!delete}
\index{non-backtrackable update!deleteall}
\index{non-backtrackable update!erase}
\index{non-backtrackable update!eraseall}
%
\paragraph{Deletion.} The syntax of a deletion primitive is as follows:
\begin{quote}
{\tt \emph{delop}\{\emph{literals} [| \emph{query}]\}}
\end{quote}
%%
where {\it delop} can be {\tt delete}, {\tt deleteall}, {\tt erase}, and
{\tt eraseall}. The {\it literals} part is a comma separated list of
F-molecules and predicates. The optional part, {\tt |}\emph{query},
represents an additional constraint or a restricted quantifier, similarly
to the one used in the insertion primitive.

For instance, the following predicate:
\begin{quote}
\begin{verbatim}
flora2 ?- deleteall{john[Year(Semester)->>Course] | Year < 2000}
\end{verbatim}
\end{quote}
will delete John's course selection history before the year 2000.

Note that the semantics of a {\tt delete}\{\emph{literal}{\tt |}{\it query}\}
statement is that first the query \emph{literal} $\wedge$ \emph{query} should
be asked. If it succeeds, then deletion is performed. For instance, if the
database is
%%
\begin{quote}
\begin{verbatim}
p(a). p(b). q(a). q(c).
\end{verbatim}
\end{quote}
then the statement below:
\begin{quote}
\begin{verbatim}
flora2 ?- delete{p(X)|q(X)}
\end{verbatim}
\end{quote}
will succeed with the variable {\tt X} bound to {\tt a} and {\tt p(a)} will
be deleted.  However, if the database contains only the facts {\tt p(b)}
and {\tt q(c)}, then the above predicate will fail and the database will
stay unchanged.

\index{bulk delete}
\index{delete!bulk}
%%
\FLORA provides four deletion primitives: {\tt delete}, {\tt deleteall},
{\tt erase}, and {\tt eraseall}. The primitive {\tt delete} removes at most
one fact from the database. The primitives {\tt deleteall} and {\tt
  eraseall} are {\tt bulk delete} operations; {\tt erase} is kind of a
hybrid: it starts slowly, by deleting one fact, but may go on a joy ride
and end up deleting much of your data. These primitives are described
below.
%%
\begin{enumerate}
\item If there are serveral bindings or matches for the literals to be
  deleted, then {\tt delete} will choose only one of them
  nondeterministically, and delete it.  For instance, suppose the database
  contains the following facts:
  \begin{quote}
    \verb|p(a). p(b). q(a). q(b).|
  \end{quote}
  then
  \begin{quote}
    \verb|flora2 ?- delete{p(X),q(X)}|
  \end{quote}
  will succeed with {\tt X} bound to either {\tt a} or {\tt b},
  depending on the ordering of facts in the database at runtime.
      
\item In contrast to the plain {\tt delete} primitive, {\tt deleteall} will
  try to delete all bindings or matches.  Namely, for each binding of
  variables produced by \emph{query} it deletes the corresponding instance
  of \emph{literal}. If \emph{query} $\wedge$ \emph{literal} is false, the
  {\tt deleteall} primitive fails.  To illustrate, consider the following:
  %%
  \begin{quote}
    {\tt flora2 ?- p(X,Y), deleteall\{q(X,Y,Z)|r(Y,Z)\}.}
  \end{quote}
  %%
  and suppose {\tt p(x,y)} is true. Then the above statement will, for
  each {\tt z} such that {\tt r(y,z)} is true,
  delete {\tt q(x,y,z)}.
  
  For another example, suppose the database contains the following facts:
  %%
  \begin{quote}
    \verb|p(a). q(b). q(c).|
  \end{quote}
  %%
  and the query is {\tt ?- deleteall\{p(a),q(X)\}.} The effect will be 
  the deletion of {\tt p(a)} and of all the facts in {\tt q}. (If you
  wanted to delete just one fact in {\tt q}, {\tt delete} should
  have been used.
  
  \smallskip
  
  Unlike the {\tt delete} predicate, {\tt deleteall} \emph{always}
  succeeds. However, when it succeeds, {\tt deleteall} will leave
  all variables unbound.
  
\item {\tt erase} works like {\tt delete}, but with an object-oriented
  twist: For each \fl fact, $f$, that it deletes, {\tt erase} will
  traverse the object tree by following $f$'s methods and delete all
  objects reachable in this way. It is a power-tool that can cause
  maiming and injury. Safety glasses and protective gear are
  recommended.
  
  Note that only the base part of the objects can be erased. If the
  object has a part that is derived from the facts that still exist, this
  part will not be erased.
  
\item {\tt eraseall} is the take-no-prisoners version of {\tt erase}.
  Just like {\tt deleteall}, it first computes \emph{query} and for each
  binding of variables it deletes the corresponding instance of
  \emph{literal}. For each deleted object, it then finds all objects it
  references through its methods and deletes those. This continues
  recursively until nothing reachable is left.  This primitive always
  succeeds.
\end{enumerate}


\subsection{Backtrackable Updates}


\index{backtrackable update}
\index{update!backtrackable}
%
The effects of backtrackable updates are undone upon backtracking, i.e., if
some post-condition fails and the system backtracks, a previously inserted
item will be removed from the database, and a previously deleted item will be
put back.

\index{backtrackable update!btinsert}
\index{backtrackable update!btinsertall}
\index{backtrackable update!btdelete}
\index{backtrackable update!btdeleteall}
\index{backtrackable update!bterase}
\index{backtrackable update!bteraseall}
%
The syntax of backtrackable update primitives is similar to that of
non-backtrackable ones and the names are similar, too.  The syntax for
backtrackable insertion is:
%%
\begin{quote}
\emph{btinsop}\{\emph{literals} [$\mid$ \emph{formula}]\}
\end{quote}
%%
while the syntax of a backtrackable deletion is:
%%
\begin{quote}
\emph{btdelop}\{\emph{literals} [$\mid$ \emph{query}]\}
\end{quote}
%%
where {\tt btinsop} stands for either {\tt btinsert} or {\tt btinsertall},
and {\tt btdelop} stands for either of the following four deletion
operations: {\tt btdelete}, {\tt btdeleteall}, {\tt bterase}, and {\tt
  bteraseall}. The meanings of {\it literals} and {\it query} is
the same as in Section~\ref{sec:non-backtrackable-updates}.

{\tt btinsert}, {\tt btinsertall}, {\tt btdelete}, {\tt btdeleteall}, {\tt
  bterase}, and {\tt bteraseall} work similarly to {\tt insert}, {\tt
  delete}, {\tt deleteall}, {\tt erase}, and {\tt eraseall}, respectively,
except that the new operations are backtrackable.  Refer to
Section~\ref{sec:non-backtrackable-updates} for details of these
operations.

To illustrate the difference between backtrackable and
non-backtrackable updates, consider the following execution trace
immediately after the \FLORA system starts:
\begin{quote}
\begin{verbatim}
flora2 ?- insert{p(a)}, fail.

No.

flora2 ?- p(a).

Yes.

flora2 ?- btinsert{q(a)}, fail.

No.

flora2 ?- q(a).

No.
\end{verbatim}
\end{quote}
In the above example, when the first \verb|fail| executes, the system
backtracks to {\tt insert\{p(a)\}} and does nothing. Thus the insertion
of {\tt p(a)} persists and the following query \verb|p(a)| returns
with {\tt Yes}. However, when the second \verb|fail| executes, the
system backtracks to {\tt btinsert\{q(a)\}} and removes {\tt q(a)} that
was previously inserted into the database. Thus the next query
\verb|q(a)| returns with {\tt No}.

Keep in mind that some things that Prolog programmers routinely do with
{\tt assert} and {\tt retract} goes against the grain of the use of
backtrackable updates. In particular, {\tt fail}-loops are not going to
work (will leave the databases unchanged) for obvious reasons. In a future
release, we will provide {\tt while}-loops that will work correctly with
such updates.

There are certain less obvious pitfalls in using backtrackable predicates
in \FLORA. For instance, the query
%%
\begin{quote}
 {\tt flora2 ?- p(X), btinsert\{q(X)\}}  
\end{quote}
%%
will succeed (if {\tt p(X)} is true for some {\tt X}), but will leave the
database unchanged! The reason has to do with the fact that, by default,
\FLORA returns all answers to a query.  In the current release, this
involves a {\tt fail}-loop and, obviously, hurts backtrackable updates
badly. In a future release, these updates will be integrated with the query
answering machinery much better. For now, we recommend using the {\tt
  flOne} primitive, which turns the all-answer mode off:
%%
{\tt
\begin{quote}
  flora2 ?- flOne.\\
  flora2 ?- p(X), btinsert\{q(X)\}.
\end{quote}
}
%%
\noindent
(Note: {\tt flOne,p(X),btinsert\{q(X)\}} will not work: {\tt flOne} must be
issued in a separate query.)


\subsection{Updates and Tabling}


We have earlier remarked in Section~\ref{sec-proc-methods} that tabling and
database updates do not mix well.  The problem is that the results from
previous queries are stored in XSB tables, and database updates do not
modify XSB tables.  Thus, a user might get the following counterintuitive
result:
%%
\begin{quote}
\begin{verbatim}
flora2 ?- insert{o[m->v]}.

Yes.

flora2 ?- o[m->v].

Yes.

flora2 ?- delete{o[m->v]}, o[m->v].

Yes.
\end{verbatim}
\end{quote}
The last positive answer is a consequence of the fact that XSB tables remember
that the query {\tt o[m->v]} is true. So, when the same query is asked
after {\tt delete}, a ``stale'' answer is returned from the tables. Similarly,
tabling might interact poorly with {\tt insert}:
\begin{quote}
\begin{verbatim}
flora2 ?- o[m->v].

No.

flora2 ?- insert{o[m->v]}, o[m->v].

No.
\end{verbatim}
\end{quote}
The reason for this result is, again, that XSB tables remember that {\tt
  o[m->v]} is false. However, this result becomes stale after the
  insertion.

\index{abolish\_all\_tables}
%%
In a future release, \FLORA will provide a workaround for these
problems (and it is even possible that a future release of XSB will
start doing the right thing in these situations). For now, the only
remedy is to use a call to {\tt abolish\_all\_tables}, which will
clear all tables.  However, at present, the only safe way to do this
is by executing {\tt abolish\_all\_tables} as a \emph{separate} query.

One other temporal solution could be to design your program in such a way
that tabled predicates and F-molecules will not depend on the facts that
are dynamically inserted or deleted.

Another, related, issue is that a tabled predicate might have an update in
its body. In this case, the update will be executed the first time the
predicate is evaluated. Subsequent calls will return the predicate truth
value from the tables, without invoking the predicate definition.
Such a dependency is most likely a mistake, which should be caught by the
compiler. This enhancement will be supported in the future.


\section{Negation} \label{sec:negation}


\FLORA supports two kinds of negation: negation based on SLD resolution,
and negation based on well-founded semantics, which relies on the
underlying XSB tabling system.

For negation based on SLD resolution, it is specified using the operators
\NAF and {\tt not}.

For negation based on well-founded semantics, it is specified using the
operator {\tt tnot}. Usually, {\tt tnot} is applied to predicates that
are tabled. Note that except procedural methods, all \fl methods are
tabled, thus {\tt tnot} can be applied to them.

All tabled predicates should use {\tt tnot} for negation. The
semantics is not clear when negation based on SLD resolution is
applied to a tabled predicate.

In \FLORA, {\tt tnot} can also be specified for a non-tabled
predicate, or a logic formula. The effect is such that the system will
first generate a new tabled predicate to hold the results from the
non-tabled predicate or the logic formula, then apply {\tt tnot} to
this new predicate.

In fact, {\tt tnot} in \FLORA is implemented using the XSB operator
\verb|sk_not|. Refer to the XSB manual for more information on
negation, well-founded semantics, and the XSB operators {\tt tnot} and
\verb|sk_not|.



\section{Control Constructs}

For user convenience, \FLORA provides a number of control structures
borrowed from other languages. One often used such construct is
{\tt if-then-else}. For instance, 
%%
\begin{quote}
 \tt
 ?- if (foo(a),foo2(b)) then (abc(X),cde(Y)) else (qpr(X),rts(Y)).
\end{quote}
%%
Here the system first evaluates {\tt foo(a),foo2(b)} and, if true,
evaluates {\tt abc(X),cde(Y)}. Otherwise, it evaluates {\tt qpr(X),rts(Y)}.
The entire clause succeeds or fails depending on whether the {\tt
  then}-part or the {\tt else}-part succeeds or fails (whichever applies).
The {\tt if-then} version (without the {\tt else}-part) is also provided.
In this case, when the {\tt if}-condition succeeds, the {\tt then}-part is
evaluated. The entire clause is then succeeds or fails depending on the
{\tt then}-part. When the {\tt if}-condition is false, the entire clause fails.

Note that {\tt if}, {\tt then}, and {\tt else} are implemented as
operators, and they bind stronger than the conjunction ``,'', the
disjunction ``;'', etc. This is why the parentheses are needed in the above
example.

In the current version, the \FLORA parser does not yet handle all the
syntax errors associated with the use of {\tt if-then-else} and nested
clauses are not always parsed the way one would expect (say, as in C).
Therefore, explicit parentheses are recommended around the entire clause. 



\section{Type Checking}


\index{type checking}
%%
Although \FLORA allows specification of object types through signatures,
type correctness is not being checked automatically. So, what are the
signatures good for then? One answer is that future versions of \FLORA
might support some forms of type checking. However, because \fl can
naturally support powerful meta-programming, even the current level
of support for signatures is useful. For instance, users can write
simple queries to check the types of methods that might look suspicious.
Here is one way to construct such a type-checking query:
%%
\begin{verbatim}
scalar_type_incorrect(O,M,R) :- O[X->R], O:C, C[X=>D], tnot R:D.
?- scalar_type_incorrect(obj,meth,Result).
\end{verbatim}
%%
Here, we define what it means to violate type checking using the usual
\fl semantics. The corresponding predicate can then be queried. A
``no'' answer means that the corresponding attribute \emph{does not}
violate the typing rules.

In this way, one can easily consruct special purpose type checkers.  This
feature is particularly important when dealing with \emph{semistructured}
data. (Semistructured data has object-like structure but normally does not
need to conform to any type; or if it does, the type would normally cover
only certain portions of the object structure.)


\section{Summary of \FLORA Compiler Directives} \label{sec-comp-directives}

\index{compiler directive}
%
\index{compiler directive!{\tt index}}
\index{compiler directive!{\tt firstorder}}
\index{compiler directive!{\tt firstorderall}}
\index{compiler directive!{\tt expunge}}
\index{compiler directive!{\tt equality}}
\index{compiler directive!{\tt arguments}}
\index{compiler directive!{\tt op}}
%
Like XSB compiler, \FLORA compiler can take compiler directives. All
such directives must begin with {\tt :-} (while all queries must begin with
{\tt ?-}). The following is a list of all the compiler directives supported
by \FLORA:
\begin{itemize}
\item {\tt expunge} {\em functor/arity}, ..., {\em functor/arity}
  \\
  Abolishes the listed first-order predicates in the current program's module,
  {\it i.e.}, deletes the contents as well as the definition of the
  corresponding predicate. By program's module we mean the module into
  which the program that contains this particular {\tt expunge} directive
  is loaded.
\item {\tt expunge} {\em functor/arity}, ..., {\em functor/arity} {\tt
    in} \emph{module}\\
  Same as above, except the predicates are expunged in the specified
    module rather than in the current program's module.
  \item {\tt equality none|basic|flogic}
    \\
    Sets the equality maintenance level in the current program's module.
    With {\tt none}, equality is not maintained, and the symbol {\tt :=:}
    works like an ordinary predicate.  With {\tt basic}, the predicate {\tt
      :=:} is treated as the equality, but only the usual congruence axioms
    for equality are enforced. With {\tt flogic}, the usual congruence
    axioms are maintained \emph{plus} the additional equality axiom for
    single-valued methods in \fl.
\item {\tt equality none|basic|flogic} {\tt in}  \emph{module}
  \\
  Same as above, except that equality maintenance is set for the specified
  module.
\item {\tt firstorder} {\em functor/arity}, ..., {\em functor/arity}
  \\
  Declare the listed predicates as first-order predicates (as opposed to HiLog
  predicates).
\item {\tt firstorderall}
  Makes all predicates into first-order predicates. (The default for predicates
  is HiLog.)
\item {\tt arguments} \emph{functor}(\emph{type}, ..., \emph{type}), where
  \emph{type} is either {\tt oid} {\tt bform}
  \\
  Specifies the meta signature to the given predicate, which tells how the
  arguments of the predicate are to be compiled. {\tt oid} means that the
  oid of the corresponding argument is computed and passed to the predicate
  as an argument. {\tt bform} means the argument is compiled as a body
  subgoal and passed to the predicate as such. This type of compilation is
  useful in meta-predicates, such as {\tt findall/3}, which need to
  consider the truth value of some of their arguments rather their oid.
\item {\tt import {\it functor/arity}, ..., {\it functor/arity} from
    {\it module}}
  \\
  Imports the given predicate from the specified XSB module.
\item {\tt index {\it functor/arity-argNumber}}
  \\
  Specifies that the given predicate must be indexed on the specified
  argument.
\item {\tt op({\it precedence},{\it type},{\it operator})}
  \\
  Defines \emph{operator} as a \FLORA operator with the given precedence
  and type. The \emph{type} is the same as in Prolog operators, {\it i.e.},
  {\tt fx}, {\tt xf}, {\tt xfy}, etc.
\item {\tt op({\it precedence},{\it type},[{\it operator}, ..., {\it operator}])}
  \\
  Same as above, except that this directive defines a list of operators
  with the same precedence and type.
\item {\tt table {\it functor/arity}, ..., {\it functor/arity}}
\end{itemize}


\section{\FLORA Libraries}

\FLORA provides a number of useful libraries that other programs can
use. We describe the functionality of these libraries below.

\paragraph{Pretty printing.}


\section{\FLORA Debugger}
\index{debugging}


*********** Must be fixed when the debugger is adapted **********


\FLORA debugger is essentially a presentation layer on top of the XSB
debugger, so familiarity with the latter is highly recommended (XSB Manual,
Part I). Here we sketch only a few basics.

The debugger has two facilities: tracing and spying. Tracing allows the
user to watch the program being executed step by step, and spying allows
one to tell \FLORA that it must pose when execution reaches certain 
predicates or object methods. The user can trace the execution from then
on. At present, only the tracing facility has been implemented.

\index{tracing}
To start tracing, you must issue the command {\tt flora\_trace} at the
\FLORA prompt. It is also possible to put the subgoal {\tt flora\_trace} in
the middle of the program. In tat case, tracing will start after this
subgoal gets executed. This is useful when you know where exactly you want
to start tracing the program. To stop tracing, type {\tt flora\_notrace}.

During tracing, the user is normally prompted at the four ports of subgoal
execution: {\tt Call} (when a subgoal is first called), {\tt Exit} (when
the call exits), {\tt Redo} (when the subgoal is tried with a different
binding on backtracking), and {\tt Fail} (when a subgoal fails).
At each of the prompts, the user can issue a number of commands. The most
common ones are listed below. See the XSB manual for more.
%%
\begin{itemize}
  \item {\tt carriage return (creep)}:  to go to the next step  
  \item {\tt s (skip)}: execute this subgoal non-interactively; prompt
    again when the call exits (or fails)
  \item {\tt S (verbose skip)}: like {\tt s}, but also show the trace
    generated by this execution
  \item {\tt l (leap)}: stop tracing and execute the remainder of the
    program
\end{itemize}
%%
The behavior of the debugger is controled by the predicate {\tt
  debug\_ctl}. For instance, executing {\tt debug\_ctl(profile, on)} at the
\FLORA prompt tells XSB to measure the CPU time it takes to execute each
call. This is useful for tuning your program for performance. Other useful
controls are: {\tt debug\_ctl(prompt, off)}, which causes the trace to be
generated without user intervention; and {\tt debug\_ctl(redirect,
  foobar)}, which redirects debugger output to the file named {\tt foobar}.
The latter feature is usually useful only in conjunction with the
  aforesaid prompt-off mode. See the XSB manual for additional information
  on debugger control.


\section{Emacs Support}

Editing and debugging \FLORA programs can be greatly simplified with the
help of \emph{flora-mode}, a special Emacs editing mode designed
specifically for \FLORA programs. Flora-mode provides support for syntactic
highlighting, automatic indentation, and the ability to run \FLORA programs
right out of the Emacs buffer.


\subsection{Instalation}


To install \emph{flora-mode}, you must perform the following steps. Put the
file
%%
\begin{quote}
  {\tt XSB/packages/flora2/emacs/flora.el} 
\end{quote}
%%
found in your XSB distribution on the load path of Emacs or XEmacs
(whichever you are using). The best way to work with Emacs is to make a
separate directory for Emacs libraries (if you do not have one), and put
{\tt flora.el} there. Such a directory can be added to emacs search path by
putting the following command in the file \verb|~/.emacs| (or
\verb|~/.xemacs|, if you are running one of the newer versions of XEmacs):
%%
\begin{quote}
  \tt
   (setq load-path (cons "your-directory" load-path)) 
\end{quote}
%%
It is also a good idea to compile emacs libraries. To compile flora.el,
use this:
%%
\begin{quote}
  \tt
   emacs -batch -f batch-byte-compile flora.el 
\end{quote}
%%
This will produce the file {\tt flora.elc} --- a compiled byte code.
If you are using XEmacs, use {\tt xemacs} instead of {\tt emacs} above ---
the two emacsen use incompatible byte code, and you cannot use {\tt
  flora.elc} compiled under one system for editing files under another.

Finally, you must tell X/Emacs how to recognize \FLORA program files, so
Emacs will be able to invoke the Flora major mode automatically when you
are editing such files:
%%
\begin{verbatim}
(setq auto-mode-alist (cons '("\\.flr$" . flora-mode) auto-mode-alist))
(autoload 'flora-mode "flora" "Major mode for editing Flora programs." t)
\end{verbatim}
%%$

To enable syntactic highlighting of Emacs buffers (not just for \FLORA
programs), you can do the following:
%%
\begin{itemize}
  \item  {\sf In Emacs:} select {\tt Help.Options.Global Font Lock} on
    the menubar.  To enable highlingting permanently, put 
    %%
    \begin{quote}
      \tt
      (global-font-lock-mode t)
    \end{quote}
    %%
    in \verb|~/.emacs|.
  \item {\sf In XEmacs:} select {\tt Options.Syntax
        Highlighting.Automatic} in the menubar. To enable this permanently, put
      %%
      \begin{quote}
        \tt
        (add-hook 'find-file-hooks 'turn-on-font-lock)
      \end{quote}
      %%
      in \verb|~/.emacs| or \verb|~/.xemacs| (whichever is used by your
      XEmacs).
\end{itemize}
%%


\subsection{Functionality}


\paragraph{Menubar menu.}
Once \FLORA editing mode is installed, it provides a number of functions.
First, whenever you edit a \FLORA program, you will see the ``Flora'' menu
in the menubar. This menu provides commands for controlling the Flora
process (i.e., XSB with the \FLORA shell). You can start and stop
this process, type queries to it, and you can tell it to consult regions of
the buffer you are editing, the entire buffer, or some other file.

Because Emacs provides automatic file completion and allows you to edit
what you typed, performing these functions right out of the buffer takes
much less effort than typing the corresponding commands to the \FLORA
shell.

\paragraph{Keyboard functions.}
In addition to the menu, \emph{flora-mode} lets you execute most of the
menu commands using the keyboard. Once you get the hang of it, keyboard
commands are much faster to invoke:
%%
\begin{verbatim}
Load file:                 Ctl-c Ctl-f
Load file dynamically:     Ctl-u Ctl-c Ctl-f
Load buffer:               Ctl-c Ctl-b
Load buffer dynamically:   Ctl-u Ctl-c Ctl-b
Load region:               Ctl-c Ctl-r
Load region dynamically:   Ctl-u Ctl-c Ctl-r
\end{verbatim}
%%
When you invoke any of the above commands, a \FLORA process is started,
unless it is already running. However, if you want to invoke this process
explicitly, type
%%
\begin{verbatim}
ESC x run-flora  
\end{verbatim}
%%
You can control the \FLORA process using the following commands:
%%
\begin{verbatim}
Interrupt Flora Process:      Ctl-c Ctl-c
Quit Flora Process:           Ctl-c Ctl-d
Restart Flora Process:        Ctl-c Ctl-s
\end{verbatim}
%%
Interrupting \FLORA is equivalent to typing {\tt Ctl-c} at the \FLORA
prompt. Quitting the process stops XSB, and restarting the process shuts
down the old XSB process and starts a new one with \FLORA shell running.

\paragraph{Indentation.}
Flora editing mode understands some aspects of the \FLORA syntax, which
enables it to provide correct indentation of program lines (in many cases).
In the future, flora-mode will know more about the syntax, which will let
it provide even better support for indentation.

The most common use of \FLORA indentation facility is by typing the {\tt
  TAB}-key. If \emph{flora-mode} manages to understand where the cursor is,
it will indent the line accordingly. Another way is to put the following in
your emacs startup file (\verb|~/.emacs| or \verb|~/.xemacs|):
%%
\begin{verbatim}
    (setq flora-electric t)  
\end{verbatim}
%%
In this case, whenever you type the return key, the next line will be
indented automatically.



\appendix

\section{Inside \FLORA}


\subsection{The File Structure}

\FLORA consists of the following modules:
\begin{itemize}
\item \texttt{flrshell.P}: The top level module that provides the \FLORA shell
  commands for compiling and consulting \FLORA programs, for setting the
  output mode, and -- last but not the least -- for directly issuing
  queries against the loaded database/program (see
  Section~\ref{sec-shell-commands} for a full description of shell
  commands).
\item \texttt{flrlexer.P}: The \FLORA lexer.
\item \texttt{flrcomposer.P}: The \FLORA composer that parses tokens 
  according to the operator grammar (converts operator syntax into Prolog
  terms) and also does other magic.
\item \texttt{flrparser.P}: The \FLORA parser.
\item \texttt{flrcompiler.P}: The \FLORA compiler that translates \fl to XSB.
\item \texttt{flrcoder.P}: The \FLORA coder that generates XSB code.
\item \texttt{flrutils.P}: Miscellaneous utility predicates.
\end{itemize}
%%
Additional system libraries are located in the {\tt syslib/} subdirectory.
The sbudirectory {\tt trailer/} contains trailers that implement the
various axioms implied by \FLORA semantics.  There is also a number of
files in the {\tt closure/} subdirectory that serve as headers and trailers
that are automatically included into the {\tt .P} files by the \FLORA
compiler (explained later).

There are several subdirectories that hold the various files that contain
definitions included at compile time. These will be described in a
technical document.


\subsection{How \FLORA Works}


As an \fl-to-XSB compiler, \FLORA first parses its source file,
compiles it into XSB syntax and then outputs XSB code. For instance the command
\begin{quote}
\verb|flora ?- flLoad(myprog).|
\end{quote}
would compile the \FLORA program ``{\tt myprog.flr}'' and generate the
following files: ``{\tt myprog.flh}'', ``{\tt myprog.P}'',
``{\tt myprog\_main.P}'', and ``{\tt myprog.fdb}'' (if ``{\tt myprog.flr}''
contains \fl facts).  By default, {\tt flLoad(myprog)} loads the
program into the default module named ``main''. ``{\tt myprog.flh}''
contains this module name definition.  If ``{\tt myprog.flr}''
contains \fl facts, all these facts will be compiled separately into
the XSB file ``{\tt myprog.fdb}'' that is to be dynamically loaded at
runtime. When the compilation process continues, the file ``{\tt
myprog\_main.P}'' will be generated and passed to the XSB compiler,
yielding byte code ``{\tt myprog\_main.O}'', which is then loaded and
executed.  If ``{\tt myprog.flr}'' contains queries, they are
immediately executed by XSB (provided there are no errors).

In the module system of \FLORA, a program can be loaded into any
arbitrarily named module. Even the same program can loaded into
different modules at the same time. For each module, a different byte
code will be generated (module name is appended to the source file
name to generate the name of the file containing the byte code). This
is achieved by encoding the name of the module into all predicates
(not function symbols) appearing in the program. Take a look at
``{\tt myprog\_main.P}'' to see what has become of your \FLORA program!

The main purpose of the \FLORA shell, however, is to allow the evaluation
of ad-hoc \fl queries. For example, after having consulted the
the \texttt{'default.flr'} file from the demo directory by launching
the command \texttt{flora2~?-~flDemo(default).}, you may ask
\begin{verbatim}
    flora2 ?-  X..kids[                 % Whose kids
                 self -> K,             % ... (list them by name)
                 hobbies ->>            % ... have hobbies
                   H:dangerous_hobby    % ... that are dangerous?
    ]. 
\end{verbatim}
\FLORA will parse, flatten, and evaluate this query in the same way as
those queries in a source program.


\paragraph{\FLORA compilation.}
The basic idea behind the implementation of \fl by translating it into
predicate calculus is described in \cite{KLW95}. It consists of two parts:
translation of F-molecules into various kinds of Prolog predicates, and
addition of appropriate ``closure rules'' that implement the
object-oriented semantics of the logic.

Consider, for instance, the following complex F-molecule, representing
facts about the object \texttt{mary} (the syntax of \fl is given in
Section \ref{sec-basic-flogic}):

\begin{quote}
{\small\begin{verbatim}
mary:employee[age->29, kids->>{tim,leo}, salary(1998)->a_lot].
\end{verbatim}}
\end{quote}

As described in \cite{KLW95}, any complex F-molecule can be
decomposed into a conjunction of simpler \fl atomic formulas. These
latter atoms can be directly represented using Prolog syntax.  For
different kinds of \fl atoms we use different Prolog predicates. For
instance, the result of translating the above F-molecule might be:

\begin{quote}
{\small
\begin{verbatim}
isa(mary,employee).           % mary:employee.
fd(mary,age,29).              % mary[age->29].
mvd(mary,kids,tim).           % mary[kids->>{tim}].
mvd(mary,kids,leo).           % mary[kids->>{leo}].
fd(mary,salary(1998),a_lot).  % mary[salary(1998)->a_lot].
\end{verbatim}
  }
\end{quote}
%%

\index{closure axioms}
The closure axioms are intended to provide the following semantic features:
%%
\begin{itemize}
\item Compute transitive closure of ``\subcl'' (the subclass relationship).  A
  runtime check warns about cycles in the subclass hierarchy.
\item Compute closure of ``\isa'' with respect to ``\subcl'', i.e., if $X\isa C,
  C\subcl D$ then $X\isa D$.
\item Perform monotonic and non-monotonic inheritance.
\item Make sure that scalar methods are, indeed, scalar.
\end{itemize}
%%
Templates for these axioms reside in the subdirectory \texttt{closure/}.

The above is a much simplified picture of the inner-workings of \FLORA. The
actual translation into Prolog and the form of the closure rules is very
complex.  Some of this complexity exists to ensure good performance.  Other
complications come from the need to provide a module system of \FLORA and
integrate it into the XSB system.  The module system serves two purposes.
First, it promotes modular design for \FLORA programs, making it possible
to split the code into separate files and import objects defined in other
modules. Second, it allows \FLORA programs to communicate with XSB by using
the predicates defined in XSB programs and letting XSB programs use \FLORA
objects.  Some of these implementation issues are described in
\cite{guiz-flora-00}.



\bibliography{../../../docs/userman/manual}

\printindex

\end{document}
