
\chapter{Undocumented Features}

\begin{itemize}

\item Higher-order relations

\item DL Rules

\item coversAttr/3

\mycomment{
\ourpreditem{coversAttr/4}, {\tt
coversAttr(SCid,Rid,TCid)} defines the inheritable relation predicate,
from an object to all elements in a set.  {\tt SCid} is the source
class or object ID, {\tt Rid} is the relation ID, and {\tt TCid} is
the target class or object ID.  For these inheritable relations, the
taxonomy is traversed and variables are bound only to values that
carry the most information.

It is useful to contrast \pred{coversAttr/3} to \pred{hasAttr/3}.  In
the case where Source and Target are class ids, the semantics of
{\tt hasAttr(Source,Relation,Target)} is

\begin{center}
forall Elt [isa(Elt,Source) implies  exists Targ [rel(X,Relation,Targ)
and elt(Targ,Target)]] 
\end{center}

Note that from this axiom and the inheritance axioms in CDF, hasAttr/3
inheraits "upward" in its third argument.  That is if
{\tt isa(Target,LargerTarget)} holds, and
{\tt hasAttr(Source,Relation,Target)}, then
{\tt hasAttr(Source,Relation,LargerTarget)} also holds.  However,
{\tt hasAttr(Source,Relation,SmallerTarget)} does not necessarily hold
if SmallerTarget is a proper subset of, or object in, LargerTarget.

This is not always the desired semantics for all applications.  For
instance, if a shipper ships to anywhere in New York State, then he
ships to Long Island, but not necessarily to anywhere in the North
Eastern US.  If we want to make Long Island places a subclass of New
York State places which are a subclass of North Eastern US places,
then we need a different kind of inheritance.  This is provided for by
{\tt coversAttr(Source,Relation,Target)} defined as 

\begin{center}
forall Elt [isa(Elt,Source) implies forall Targ [elt(Targ,Target)
                                              implies rel(Elt,Relation,Targ) ]]
\end{center}

It is easy to see that {\tt coversAttr/3} has a form of 3rd argument
inheritance that can properly model the above transportation example.
In addition, it can also be seen that first and second argument
inheritance in {\tt coversAttr/3} works in the same manner as
{\tt hasAttr}.

An index I is used only if all arguments that I uses consist of ground
atomic identifiers.
}

\end{itemize}