\section{Product Classes}

The above predicates allow the definition of various named binary
relations between classes.  However, binary definitions can sometimes
be inconvenient to use.  For instance, in the part equivalency
matching example, (Example~\ref{ex:strel}), it may be desirable to
make explicit the weight of the match as an indication of the strength
of the match.  The weight could be made explicit by a series of
definitions
%-------------------------------------------
{\small 
{\tt 
\begin{tabbing}
foo\=foo\=foo\=foo\=foo\=foo\=foooo\=foooooooooooooooo\=\kill
\> allAttr(\cid{dlaPart},\rid{suturesRusMatch\_low},\cid{suturesRusPart}) \\
\> : \\
\> allAttr(\cid{dlaPart},\rid{suturesRusMatch\_high},\cid{suturesRusPart})
\end{tabbing}
} } 
%----------------------------------
\noindent
indicting that a given part has a match of weight {\em low} through
{\em high}.  However, for a scale with a large number of values,
defining matches in this way is time-consuming and prone to errors.
To address this, we first define a new class {\tt \cid{matchScale}}
containing as subclasses the various match levels.  We then combine
{\tt \cid{matchScale}} with the class {\tt \cid{suturesRusPart}} in a
product with a {\em product identifier}, as in the following fact
%-------------------------------------------
{\small 
{\tt 
\begin{tabbing}
foo\=foo\=foo\=foo\=foo\=foo\=foooo\=foooooooooooooooo\=\kill
 
allAttr(\cid{dlaPart},\rid{suturesRusMatch}, \\
\> \> \> \> \cid{partMach(\cid{suturesRusPart},\cid{matchScale})})
\end{tabbing}  } }
\noindent 
which indicates that a {\tt \cid{dlaPart}} can have a {\tt
\cid{suturesRusMatch}} to an object in the {\tt partMatch/2}  class,
which has both a {\tt \cid{suturesRusPart}} component and a {\tt
\cid{matchScale}} component.

%-------------------------------------------------------------------

We capture the intuition behind product classes through the following
axiom schemas.  The first indicates that product identifiers 
are constructed from {\em constituent identifiers} of the same sort.

\begin{axiom}[Downward Closure] \label{ax:downcl}
\index{axioms!downward closure}
\ \\
For each product identifier $\cid{f(x_1,\ldots,x_n)}$,
$\oid{f((x_1,\ldots,x_n)}$, $\rid{f(x_1,\ldots,x_n)}$, and
$c\rid{f((x_1,\ldots,x_n)}$ the following axiom is added,
\begin{tabbing}
foo\=foo\=foo\=foo\=foo\=foo\=foooo\=foooooooooooooooo\=\kill
\> $isClass(\cid{f(x_1,\ldots,x_n)}) \Rightarrow 
	isClass(x_1) \wedge \ldots \wedge isClass(x_n) $\\
\> $isObj(\oid{f(x_1,\ldots,x_n)}) \Rightarrow 
	isObj(x_1) \wedge \ldots \wedge isObj(x_n) $\\
\> $isRel(\rid{f(x_1,\ldots,x_n)}) \Rightarrow 
	isRel(x_1) \wedge \ldots \wedge isRel(x_n) $\\
\> $isCrel(\crid{f(x_1,\ldots,x_n)}) \Rightarrow 
	isCrel(x_1) \wedge \ldots \wedge isCrel(x_n) $
\end{tabbing}
\end{axiom}

With this axiom, along with the use of $IdType/1$ in the various
instance axioms, we will sometimes refer to a CDF identifier $I$ as a
class identifier if its outer functor is {\tt cid/1}, an object
identifier if its outer functor is {\tt oid/1} etc.  The next axiom
associates product classes with the objects they contain.

\begin{axiom}[Implicit Subclassing] \label{ax:implsc}
\index{axioms!implicit subclassing} 

\begin{enumerate}
\item For each product identifier $cid(f(x_1,\ldots,x_n))$ or 
$oid(f(x_1,...,x_n))$, the following axioms are added for $x_i, 1 \leq i
\leq n$:
\[
(\forall E) [(elt(E,y_i) \Ra elt(E,x_i)) \Ra \\
	(\forall E') [elt(E',f(x_1,...x_n))[x_i/y_i] \Rightarrow elt(E',f(x_1,...x_n))]]
\]

\item For each product identifier $rid(f(x_1,\ldots,x_n))$ the
following axioms are added for $x_i, 1 \leq i 
\leq n$:
\begin{tabbing}
foo\=foo\=foo\=foo\=foo\=foo\=foooo\=foooooooooooooooo\=\kill
\> $(\forall E_1,E_2) [(rel(E_1,y_i,E_2) \Ra rel(E_1,x_i,E_2)) \Ra$ \\
\> \> $(\forall E'_1,E'_2) [rel(E'_1,f(x_1,...x_n),E'_2)[x_i/y_i] \Ra rel(E'_1,f(x_1,...x_n),E'_2)]]$
\end{tabbing}
\end{enumerate}
\end{axiom}

%-------------------------------------------
\mycomment{
\begin{axiom}[Implicit Subclassing] \label{ax:implsc}
\begin{enumerate}
\item For each product identifier $\oid{f(x_1,\ldots,x_n)}$ the
following axiom is added:  
\[(\forall O) [elt(O,\cid{f(x_1,\ldots,x_n)})
	\Ra (O = \oid{f(y_1,\ldots,y_n)} \wedge 
  		  elt(y_1,x_1) \wedge \ldots \wedge elt(y_n,x_n))] \]

\item For each product identifier $\cid{f(x_1,\ldots,x_n)}$ and for
each atomic identifier $\cid{c}$ the following axiom is
added:
\[ (\forall C) 
   ([elt(\oid{f(x_1,\ldots,x_n)},C)] \Ra (C = \cid{f(y_1,\ldots,y_n)}
   \vee C = \cid{c})) \]
\end{enumerate}
\end{axiom}
}
%-------------------------------------------

\begin{example} \rm
Axiom~\ref{ax:downcl} simply states that identifier types cannot be
mixed within a product identifier.  For instance, if {\tt
\oid{matchLevelN}} is an object in the {\tt \cid{matchScale}},
then the identifier 

{\tt \cid{partMatch(\cid{suturesRusMatch},\oid{matchLevelN})}}

\noindent
is improperly formed.  On the other hand, if {\tt \oid{sutureU245H}} is in
the class {\tt \cid{suturesRusPart}}, then the identifier {\tt
\oid{partMatch(\oid{sutureU245H},\oid{matchLevelN})}} is a product
identifier that is also an object identifier.

Axiom~\ref{ax:implsc} also means that the inheritance relation of a
product class is partly determined by the inheritance relation of its
constituent elements.
\end{example}

\mycomment{
Core Axiom~\ref{ax:implsc} can be used either to set up equality
constraints on a model of a CDF instance, or it can be used to
determine which CDF instances have free models (as explained in the
next section).  The following example illustrates the first use.

\begin{example} \rm \label{ex:equality}
Suppose we have the CDF instance
{\small 
{\tt 
\begin{tabbing}
foo\=foo\=foo\=foo\=foo\=foo\=foooo\=foooooooooooooooo\=\kill
\> isa(\cid{a},\cid{f(a)}.
\end{tabbing} } } 
\noindent
By Core Axiom~\ref{ax:nonnull} (Non-empty Classes), {\tt \cid{a}} has
at least one element, which can be called {\tt \oid{a1}}.  By the
instance axiom for the above fact, {\em elt(\oid{a1},\cid{f(a)}}.  By
Core Axiom~\ref{ax:implsc}(1), it must be the case that 
%
\[ \oid{a1} = \oid{f(Y)} \wedge elt(Y,\cid{a}) \]
%
If we take $Y = \oid{a1}$, then this means that 
%
$ \oid{a1} = \oid{f(\oid{a1})} $
% 
so that the above fragment has a model $\cM$ with a single individual
(in the universe of $\cM$) to which all object identifiers are mapped,
and that has a $elt/2$ relation to each individual to which a class
identifier is mapped (among other models).
\end{example}
}

\section{Models of Type-0 Instances} \label{sec:type0models}

Type-0 instances have been designed so that checking their consistency
--- whether they have a model --- is easily done.  Models of Type-0
instances are discussed in \cite{SwiW03b}, here we review some of the
main points without proofs.  We begin by defining a model of a Type-0
instance.

\begin{definition} \label{def:ont-th}
Let $\cO$ be a CDF instance.  We define as $\cL_{\cO}$, the ontology
language whose functions and constant symbols are restricted to the
identifiers in $\cO$.  $TH(\cO)$ is the ontology theory over
$\cL_{\cO}$ whose non-logical axioms consist of the core axioms and
the instance axiom for each fact in $\cO$.
% while $\cO^{\cI}$ is the theory over $\cL_{\cO}$ containing only the
% instance axioms.  
We say that an ontology structure $\cM$ is a model of $\cO$ if it is a
model of $TH(\cO)$.

\index{well-sorted}
If $\cO$ is Type-0 it is {\em well-sorted} if each instance axiom
cojoined with Core Axioms~\ref{ax:distinct} (Distinct Identifiers) and
Axiom Schema~\ref{ax:downcl} (Downward Closure) is consistent.
\end{definition}

Intuitively, a CDF fact is well-sorted if its arguments have class,
relation, class relation identifiers, etc. in the right place.  There
is no room for contradiction in a well-sorted Type-0 instance $\cO$
unless it contains {\tt minAttr/4} and {\tt maxAttr/4} facts.  If it
does, inconsistency can arise if {\tt minAttr(C$_1$,R,C$_2$,M)}$^{\cI}$
holds, {\tt maxAttr(C$_1$,R,C$_2$,N)}$^{\cI}$ holds, and $M > N$.
Thus, consistency of Type-0 instances is easy to check, and can in
fact be done in polynomial time.

However, CDF is a logic programming system, so it is important to know
whether an instance $\cO$ over a language $\cL_{\cO}$ has a {\em
Herbrand} model \cite{Lloy84} --- essentially one in which the
universe $\cU$ of the model are the terms of $\cL_{\cO}$, and where
the mapping of terms from $\cL_{\cO}$ to $\cU$ is the identity
mapping.  Herbrand models are the basis of Prolog semantics, so that
if CDF instances have Herbrand models, they can be easily implemented
in Prolog without resorting to a constraint system or to
meta-interpretation.  Care has been taken to ensure that any Type-0
CDF instance does in fact have a Herbrand model, so that the Type-0
interface in \refchap{sec:impl} can be supported~\footnote{This is one
of the reasons why object identifiers in CDF denote singleton classes.
If two classes $c_1$ and $c_2$ are ``equal'', this simply means that a
model must satisfy the sentence $(\forall X) elt(X,c_1)
\leftrightarrow elt(X,c_2)$ which has a Herbrand model.}.
