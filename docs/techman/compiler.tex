\section{Compiler}
%=================

The compiler of XSB is written in (XSB)-Prolog. It transforms
a source program module into an object code (byte code) module.
We begin by presenting the various phases of the compilation
process, the major modules of the compiler and their functionality
(in approximate order of use) and we finally describe the various
data structures that the modules of the compiler operate on.

\subsection{The Compilation Process and its Phases}
%==================================================
The compiler accomplishes the compilation process roughly in the
following three major phases:
\begin{enumerate}
\item	Module (global) level compilation phase.
\item	Predicate level compilation phase.
\item	Clause level compilation phase.
\end{enumerate}

Each of these phases is further divided in steps (subphases) 
which we describe in detail below.

\subsubsection{Module level compilation phase}
%---------------------------------------------
The module level compilation phase is the front-end of the XSB
compiler, where various global analyses and compilation techniques
take place.  All steps of this phase operate on the module as a whole
(encoded in the {\it internal source form} that is introduced later).
The following list describes possible steps of this phase.  We note
that some of the described steps are currently not implemented but their
position in the following list reflects various dependencies between them.
\begin{description}
\item[parsing]
	Reads in the source program files (in {\it external source form})
	and stores the information in the {\it internal source form}.
	A symbol table is included in the internal source form.
\item[use inference]
	Performs necessary inference such as detecting undefined predicates,
	implicit exported/global/local symbols, symbols that are never
	used, etc.  This step should return the list of exported
	predicates (symbols) of the module, so that they are used in
	the following steps.
\item[mode inference]
	Performs mode inference, annotating predicates with mode declarations
	(pragmatic information).  This step is currently not implemented,
	but should be of considerable use in (some of) the following steps.
\item[determinacy analysis]
	Performs determinacy analysis.  Currently not implemented.
\item[call specialisation]
	Specialises predicates according to (partially instantiated)
	call patterns that appear in the module.  One instance of this
	specialisation is for HiLog predicates.
\item[table declaration]
	Provided that the {\tt table\_all} option of the compiler is
	on, this step automatically chooses a subset of predicates that
	should be tabled so that all possible cycles that cannot be proven
	terminating contain at least one tabled predicate.  This step
	currently employs a very simple termination checker to infer
	termination proofs for immediately recursive predicates for which
	mode information is known.
	A related step with the declaration of tabled predicates is a step
	that checks for semantic violations in the use of tabling
	(i.e. uses of cuts or nots accross tabled predicates).
\end{description}
By the end of the module level compilation phase, the symbol table
should be complete and it could be dumped into the byte code and the
optional assembly file.  All symbols introduced by the following
phases should be {\em internal\/} to the module under compilation.

\subsubsection{Predicate level compilation phase}
%------------------------------------------------
The various steps in this phase share the property that they operate
on individual predicates, rather than on the whole module.  Some of
the operations performed in this phase, however, may produce a
predicate block (set of related predicates) out of a single predicate. 
\begin{description}
\item[cut transformation]
	Mimics a source-level transformation that cut-transforms predicates
	that contain cuts, nots and conditional statements.
\item[transformational indexing]
	Performs transformational indexing (unification factoring).
\item[index]
	Generates indexing instructions in {\it internal byte code
	form}, the Prolog representation of byte code (XWAM instructions).
\item[clause compilation]
	Performs the clause level compilation phase, returning a list of
	XWAM instructions in internal byte code form.
\item[peephole optimisation]
	Performs peephole optimisation on the internal byte code form.
\item[assemble]
	If the compiler option {\tt dumpasm} is on, outputs the byte code
	to the assembly file, in the {\it assembly form}.
\item[object code generation]
	Converts the internal byte code form into {\it byte code form} which 
	is written into the object file.
\end{description}

\subsubsection{Clause level compilation phase}
%---------------------------------------------
This phase is essentially the back-end of the XSB compiler and its
purpose is to generate (optimal) XWAM code for each clause.  The
substeps of this phase are (very roughly) the following:
\begin{description}
\item[clause flattening]
	Flattens each clause so that no nested structures occur in it.
\item[clause processing]
	Transforms each clause into an {\it intermediate form}, where
	pragmatic information for its variables is collected.
\item[XWAM code generation]
	Translates the intermediate form into the {\it internal byte 
	code form}.
\end{description}

As we have seen, a Prolog program can be in following forms:
\begin{itemize}
\item	external source form, (usually in {\tt *.P} and {\tt *.H} files);
\item	internal source form;
\item	intermediate form;
\item	internal byte code form;
\item	assembly form, in {\tt *.A} files;
\item	byte code, in {\tt *.O} files.
\end{itemize}
Finally, there is yet another form:
\begin{itemize}
\item	memory form, the memory image after loading.
\end{itemize}


\subsection{Compiler modules and their functionality}
%====================================================
\begin{description}
\item[compile.P] The top-level driver of the compiler.
\item[parse.P]	Uses Prolog's builtin predicate {\tt read/2} to parse
		the source module and performs initial symbol table registry.
		Uses {\bf singleton.P} to check for singleton variables in
		clauses, and the methods of {\bf symtab.P} (the symbol table
		manipulation module).
\item[useinfer.P]
		Performs inference of exported predicates/structures/constants
		marking them as exported/local/global.  It also prints out
		warnings for unused/undefined predicates.
\item[spec.P]	Specialises predicates according to partially instantiated
		call patterns that appear in the module.  One instance of this
		specialisation is for calls to HiLog predicates.  Calls to
		these predicates are usually partially instantiated.
\item[tabdef.P] Builds the call graph for the module and automatically chooses
		a subset of predicates that will be tabled.
\item[build\_graph.P]
		Checks for semantic violations in the use of cuts,
		if-then-elses, or nots in combination with tabled predicates.
\item[preprocess.P]
		Preprocesses predicate definitions that contain cuts, nots,
		or conditional statements.  It essentially performs
		a source to source cut transformation.  It uses the
		following module:
	\begin{description}
	\item[pre\_cond.P] Handles conditional statements (with inline tests)
		that need no choice points.  Currently, it is limited to
		branches of a disjunction with mutually exclusive inline
		tests, e.g. {\tt '>='/2} and {\tt '<'/2} with the same
		arguments.
	\end{description}
\item[tp\_index.P] Indexes a predicate, generating {\tt try/retry/trust}
		and {\tt switchon*} instructions for the predicate's clauses.
\item[tprog.P]	Translates clauses into XWAM code.  It is the ``heart''
		of the clause-level compilation phase. Calls in approximately
		the following order:
	\begin{description}
	\item[flatten.P] Flattens out structures in clauses along the lines of
		{\tt p(f(g(D))) => p(f(Y)),Y = g(D)}.  The code in that module 
		is much more general since it can flatten to any level by
		simply redefining the predicate {\tt level\_allowed/1}.
	\item[inprog.P] Transforms a clause in internal source form into the
		intermediate form, collecting pragmatic information
		about the clause.
	\item[varproc.P] Processes the variables of the clause collecting
		pragmatic information about their occurrences.
	\item[tp\_geninline.P] Generates code for inline predicates.
	\item[tp\_cond.P] Generates special code for conditionals with
		inline test parts.  Uses {\bf tp\_comp.P} and {\bf
		tp\_eval.P}.
	\item[tp\_goal.P] Generates {\tt get*/put*/bld*/uni*} XWAM instructions
 		for each goal (including the head).  Uses {\bf tp\_var.P}.
	\item[tp\_eval.P] Generates code for arithmetic expressions.
	\end{description}
\item[peephole.P]
		Performs peephole optimisation for one predicate at a time.
\item[asm.P, asm\_pass2.P]
		Generates XWAM byte code from internal byte code
		representation.  Use information in the instruction table 
		found in module {\bf asm\_inst.P}.
\item[asm\_opt.P] Further optimises the byte code for a predicate block
		by changing some {\tt call} and {\tt excute} XWAM instructions
		to {\tt calld} and {\tt executed} instructions respectively.
\end{description}

\subsection{Internal Source Form}
%================================

The internal source form is represented by the following Prolog data structure:

\demo{    Module = module(SymTab, DcList, ClList, QrList) }

{\sf SymTab} is the symbol table. Because of the complexity of its
structure, and the fact that the symbol entry is also used in
several other program forms, its form will be introduced in the
next subsection.

{\sf DcList} is currently not used, but it might be used for storing
directives found in the module, so that they can be easily dumped out.

{\sf ClList} represents the normal clauses and is a list of:

\demo{   pred(Symbol, [Clause1, Clause2, \ldots, ClauseN], Pragma)        }

which represents all the clauses of the predicate {\tt Symbol}
(a symbol table entry, see next subsection).

Each {\sf Clause} has the form:

\demo{	clause(Arguments, Body, Pragma)				}

where {\sf Arguments} is a list of {\tt Term}s (see below) and {\sf Body}
is a {\tt Goal} which can be one of the following:

\begin{description}
\desc{and(Goal1, Goal2)}
	Represents a conjunction of two goals {\tt Goal1} and {\tt Goal2}.
\desc{or(Goal1, Goal2)}
	Represents a disjunction of two goals {\tt Goal1} and {\tt Goal2}.
\desc{if(Goal1, Goal2)}
	Represents the conditional execution of {\tt Goal2} provided
	that {\tt Goal1} succeeds.
\desc{not(Goal)}
	Represents the {\em compiled} negation of a {\tt Goal}.
 	For maximal developer's confusion, compiled negations are calls
	to the standard predicates \not\ and {\tt fail\_if/1}, but not
	to the {\tt not/1} standard predicate.
\desc{goal(Symbol, Arguments)}
	{\tt Symbol} is a symbol table entry representing the predicate
	symbol of the atomic goal, and {\tt Arguments} is a list of
	{\tt Term}s.
\desc{inlinegoal(Name, Arity, Arguments)}
	Represents an atomic goal whose predicate, {\tt Name/Arity},
	is a predicate that is expanded inline when compiled;
	{\tt Arguments} is a list of {\tt Term}s, representing the
	arguments of the goal.  The inline predicates of XSB are given
	in table~\ref{inlinepredicatetable}.
\end{description}

\begin{table}[htbp]\centering{\tt
\begin{tabular}{lllll}
\verb|'='/2|	&\verb|'<'/2|	&\verb|'=<'/2|	&\verb|'>='/2| &\verb|'>'/2| \\
\verb|'=:='/2|	&\verb|'=\='/2|	&is/2		&\verb|'@<'/2| &\verb|'@=<'/2|\\
\verb|'@>'/2|	&\verb|'@>='/2|	&\verb|'=='/2|	&\verb|'\=='/2|&fail/0 \\
true/0		&var/1		&nonvar/1	&halt/0        &'!'/0 \\
'\_\$cutto'/1	&'\_\$savecp'/1	&'\_\$builtin'/1
\end{tabular}}
\caption{The Inline Predicates of XSB}\label{inlinepredicatetable}
\end{table}

A {\tt Term} can be one of the following:

\begin{description}
\desc{varocc(Vid)}
	A variable; {\tt Vid} is an integer or a constant that identifies
	the variable.  Different variables of the same clause have distinct
	identifiers.  The XSB parser assigns integers as variable identifiers,
	while modules like {\tt spec} and {\tt preprocess} assign atoms of
	the forms {\tt '\_v\#'} as variable identifiers.
\desc{constant(Symbol)}
	A constant symbol represented by the symbol table entry {\tt Symbol}.
\desc{integer(Int)}
	An integer having value {\tt Int}.
\desc{real(Real)}
	A floating number having value {\tt Real}.
\desc{structure(Symbol, Arguments)}
	A structure whose primary functor is {\tt Symbol} and whose
	arguments are elements in the list {\tt Arguments}.
\end{description}

{\sf QrList} represents queries (commands) of the module and has the
same form as {\sf ClList} except that it contains only one predicate
definition, and the predicate has the system supplied name \verb+'_$main'/0+.


\subsection{Symbol Table Format}
%===============================

The symbol table is implemented as an {\em abstract data type} by the
module {\tt symtab}.

Abstractly, the symbol table is a table of symbols. A symbol
is an abstract object that {\em can be accessed only by the methods
(interface routines)} introduced later.  It is considered a bad
programming habit to access the symbol table by any other means. 
The structure of the table and symbols are hidden from the users
of the module.  To put more emphasis on our argument, we avoid
presenting the internal structure of the symbol table and we only
do so for the symbol table entries.

\subsubsection{Internal structure of the entries}
%------------------------------------------------

Each symbol table entry is internally represented as:

\demo{		sym(Name, Arity, Properties, Index)		}

where {\sf Properties} has the form:

\demo{     prop(Category, Scope, Defined, Used, EP, PragmaList)	}

{\sf Category} is either undetermined (free variable) or one of the
category properties.  {\sf Scope} is one of the scope properties.
{\sf Defined} is either {\tt defined} or {\tt undef}. 
Similarly {\sf Used} is either {\tt used} or {\tt unused}.
{\sf EP} is {\tt ep(Offset)}, and finally {\sf PragmaList} is a
list of other pragma properties.  All of these are described below.

\subsubsection{Symbol properties}
%--------------------------------
As mentioned, each symbol has a {\sf Name} attribute and an {\sf Arity}
attribute that can be accessed through methods.  A symbol is
also associated with a unique ordinal number called {\sf Index} that
represents the location of the symbol in the output assembly and
object files.  In addition, a symbol can have one or more of the
following optional attributes called {\sf Properties}:

\begin{enumerate}
\item	Category (or type) of the symbol (a symbol can have at most one
	property in this group). This attribute tells the type of usage
	of the symbols, namely, as a predicate symbol, a module name, etc.
	\begin{description}
	\item[{\tt pred}]	The symbol is a predicate.
	\item[{\tt module}]	The symbol is a module name.
	\item[{\tt dynamic}]	The symbol is a dynamic predicate, currently
				not used (though it should be).
	\end{description}
	Such information will be passed into byte code files and then to the
	emulator when the files are loaded. See Table \ref{t:symcat} on page
	\pageref{t:symcat} for the corresponding categories of symbols
	in the emulator.

\item	Scope.  A symbol can have at most one property in this group,
	and this is ensured by the {\tt sym\_propin/2} method described below.
	\begin{description}
	\item[{\tt ex}]		The symbol is exported by this module.
	\item[{\tt im(Module)}]	The symbol is imported from the module
				{\tt Module}.
	\item[{\tt global}]	A global symbol that can be accessed
				(is visible) by any module.
	\item[{\tt local}]	The symbol is local to the module.
	\item[{\tt internal}]	The symbol is generated internally
				during compilation.
	\end{description}
\item	Defined checking.  The following properties are only used during usage 
	checking, and have meaning only for predicate symbols.
	\begin{description}
	\item[{\tt defined}]	The symbol is a predicate that is defined in
				this module.
	\item[{\tt undef}]	A predicate symbol that is not defined in
				this module, or imported from another module.
	\end{description}
\item	Usage checking.  The following properties are only used during usage 
	checking, but are applicable to all symbols.
	\begin{description}
	\item[{\tt used}]	A symbol that is used by some predicate
				definitions in this module.
	\item[{\tt unused}]	A symbol that is not {\tt used}.
	\end{description}
\item	Entry point of predicate (used by the assembler):
	\begin{description}
	\item[{\tt ep(Location)}]
		{\tt Location} is an offset of the starting point of the
		code to the beginning of the entire text segment.
	\end{description}
\item	Pragmatic information
	\begin{description}
	\item[{\tt index(Position)}]
		Indicates the fact that the predicate will be indexed on
		its {\tt Position}-th argument.  The default position is 1.
	\item[{\tt tabled(Tabind,Chs,Rhs,Arity)}]
		Used for tabled predicates.  {\tt Tabind} is the table
		identification number, {\tt Chs} and {\tt Rhs} are the call
		and return hash sizes for the table, while {\tt Arity} is
		the arity of the predicate symbol.
	\item[{\tt mode(p(+,.,?,..,-))}]
		Indicates mode usage of the predicate {\tt p/N}.  A predicate
		can have several different instances of this information.
	\end{description}
\end{enumerate}


\subsubsection{Methods for accessing the Symbol Table}
%-----------------------------------------------------
The following are all the methods implemented:

\begin{description}
\desc{sym\_insert(+Name, +Arity, +PropList, $\pm$SymTab, -Sym)}
	Inserts a symbol with name {\tt Name} and arity {\tt Arity} into 
	the symbol table {\tt SymTab}, and returns the symbol entry (the
	abstract object) in {\tt Sym}.  {\tt PropList} is a list of
	properties that are associated with the symbol.  If the symbol
	already exists in the symbol table, only the properties are
	added, and the symbol entry is returned.
\desc{sym\_count(+SymTab, -Number\_of\_Entries)}
	Returns the number of symbol entries in the symbol table.
\desc{sym\_scan(-Sym, +SymTab, -SymTabRest)}
	Gets an (arbitrary) symbol entry {\tt Sym} from the symbol table 
	{\tt SymTab}, and returns the rest of the symbol table in 
	{\tt SymTabRest}.  This procedure can be used to obtain
	all the symbols through backtracking, though we recommend the
	use of the more efficient {\tt sym\_gen/2} method, below.
\desc{sym\_gen(-Sym, +SymTab)}
	Gets an (arbitrary) symbol entry {\tt Sym} from the symbol table 
	{\tt SymTab}.  Best used to generate all symbols through
	backtracking.
\desc{sym\_empty(+SymTab)}
	Succeeds iff the symbol table is empty.
\desc{sym\_propin(+PropList, +Sym)}
	Inserts a list of additional properties into the symbol table
	entry, while checking for errors (inconsistencies).
	{\tt PropList} can either be a list or a single property.
\desc{sym\_prop(+Prop, +Sym)}
	Succeeds when the symbol table entry {\tt Sym} possesses the
	given property.
\desc{extract\_symlist(+Prop, -SymList, +SymTab)}
	{\tt Prop} is a single or a list of properties.
	The predicate returns a list of symbols ({\tt SymList}) that
	have the property {\tt Prop} (or have all the properties listed
	in {\tt Prop}).
\desc{sym\_name(+Sym, -Name, -Arity)}
	Returns the name and the arity of the symbol.
\desc{sym\_offset(+Sym, -Index)}
	Returns the index number of the symbol.
\desc{sym\_category(+Sym, -Category)}
	Returns the category of the symbol.
\desc{sym\_env(+Sym, -Scope)}
	Returns the scope of the symbol.
\desc{sym\_type(+Sym, -Tabled)}
	Returns the tabled properties of the symbol, or alternatively
	succeeds iff the symbol is a tabled predicate symbol.
\desc{sort\_by\_index(+SymTab, -Sorted\_SymList)}
	Sorts the symbols in the symbol table {\tt SymTab} based on their
	index field, returning them in the list {\tt Sorted\_SymList}.
\desc{sym\_complete($\pm$SymTab)}
	Completes the symbol table.
\end{description}



\subsection{Intermediate Program Form}
%=====================================

The intermediate program form is generated mainly by the {\tt
preprocess} module that performs cut and not transformations, and by
the predicate {\tt inprog/2} (in module {\tt inprog}).  The intermediate
program form is only used in the last two phases of the compiler.
After the gradual evolution of the XSB compiler, the intermediate program
form is not much different from the internal source form, except
pragma information being added to variable occurrences.  Also, {\tt if}
and {\tt not} constructs are changed by the cut-transformation.

The form of predicates is the same as the one of the internal source
form, while each {\tt Clause} has the form

\demo{	clause(Arguments, Body, CPrag)				}

where {\sf Arguments} is a list of {\tt Term}s (see below), {\sf CPrag}
is a clause-pragma structure (whose form will given later), and
{\sf Body} is a {\tt Goal} which can be one of the following:

\begin{description}
\desc{nil}
	An empty body.  Used when the clause is a fact that needs no
	flattening.
\desc{and(Goal1, Goal2)}
	Represents a conjunction of two goals {\tt Goal1} and {\tt Goal2}.
\desc{or(Goal1, Goal2)}
	Represents a disjunction of two goals {\tt Goal1} and {\tt Goal2}.
\desc{if\_then\_else(TestGoal, TrueGoal, FalseGoal)}
	Represents an if-then-else construct (maybe constructed by analysis),
	in which {\tt TestGoal} is a goal that contains only inline tests,
	while {\tt TrueGoal} and {\tt FalseGoal} are {\tt Goal}s.
\desc{goal(Symbol, Arguments)}
	{\tt Symbol} is a symbol table entry representing the predicate
	symbol of the atomic goal, and {\tt Arguments} is a list of
	{\tt Term}s.
\desc{inlinegoal(Name, Arity, Arguments)}
	Represents an atomic goal whose predicate is a predicate
	{\tt Name/Arity} that is expanded inline when compiled;
	{\tt Arguments} is a list of {\tt Term}s, representing the
	arguments of the goal.
\end{description}

A {\tt Term} can be one of the following:

\begin{description}
\desc{varocc(Vid, VPrag)}
	A variable; {\tt Vid} is an integer or a constant that identifies
	the variable. Different variables of the same clause have distinct ids.
	{\tt VPrag} is a variable-pragma structure whose form is given below.
\desc{constant(Symbol)}
	A constant symbol represented by the symbol table entry {\tt Symbol}.
\desc{integer(Int)}
	An integer having value {\tt Int}.
\desc{real(Real)}
	A floating number having value {\tt Real}.
\desc{structure(Symbol, Arguments)}
	A structure whose primary functor is {\tt Symbol} and whose
	arguments are elements in the list {\tt Arguments}.
\end{description}

The variable-pragma structure is now implemented by an abstract datatype.
The attributes and their accessing methods are given in Table~\ref{t:vprag}.

\begin{table}\centering
\begin{tabular}{|l|l|l|}
\hline
Attribute	& Value			& Access method			\\
\hline \hline
{\sf type}	& {\tt p} - permanent	& vprag\_type(VPrag, Type)	\\
		& {\tt t} - temporary	&				\\
		& {\tt vh} - void head	&				\\
{\sf location}	& an integer		& vprag\_loc(VPrag, Loc)	\\
{\sf use}	& see below		& vprag\_use(VPrag, Use)	\\
{\sf nouse}	& see below		& vprag\_nouse(VPrag, NoUse)	\\
\hline
{\sf context}	& {\tt h} - head	& vprag\_context(VPrag, Context)\\
		& {\tt i} - inline goal &				\\
		& {\tt b} - other goal	&				\\
{\sf level}	& {\tt t} - top	level	& vprag\_level(VPrag, Level)	\\
		& {\tt s} - second level&				\\
{\sf argument
     position}	& an integer		& vprag\_argno(VPrag, Argno)	\\
{\sf occurrence}& {\tt f} - first	& vprag\_occ(VPrag, Occ)	\\
		& {\tt l} - last	& 				\\
		& {\tt v} - void	& 				\\
		& {\tt s} - other	& 				\\
{\sf in last
     chunk}	& {\tt 1} - yes		& vprag\_lastchunk(VPrag, Yes)	\\
		& {\tt 0} - no		&				\\
\hline
{\sf type1}	& see below		& vprag\_type1(VPrag, Type)     \\
\hline
\end{tabular}
\caption{Variable Pragma Access Methods}
\label{t:vprag}
\end{table}

The attributes {\sf type, location, use} and {\sf nouse} are shared
by all occurrences of the same variable, while the other attributes
depend on the specific occurence.

For temporary variables, the {\sf location} attribute records the
register number, while for permanent variables it is the offset in
the activation record (the first variable has offset 1).

For temporary variables, the {\sf use} attribute stores a list of
integers that represent the register numbers the variable is preferred
to be allocated, and the {\sf nouse} attribute stores a list of 
registers that the variable should not be allocated.

For permanent variables, occurrences in the last chunk may be
{\it unsafe} (first occurrence of a variable that does not appear in
the head), or may need to be {\it dereferenced} (all subsequent
occurrences).  This information is stored in the {\sf use} attribute
as {\tt u} and {\tt d} respectively, or {\tt p} when neither case
applies (variables that appear in the head).  The attribute
{\sf type1} is a derived attribute that is the same as the attribute
{\sf type} except when the variable is permanent and occurs in the
last chunk, in which case {\sf type1} is the same as the {\sf use}
attribute.  {\sf nouse} attributes are not used for permanent
variables.

The internal structure of variable-pragmas has the form:

\demo{	vrec(Context,Level,Argno,Occ,v(Type,Loc,Use,NoUse),LastChunk)	}

where the {\sf v(\ldots)} part is shared by all occurrences of
the variable.

The clause-pragma structure takes the following form

\demo{		crec(ARSize, Label)		}

where {\sf ARSize} is the size of the activation record, and {\sf Label}
is the label of the clause.

%and {\sf VarList} is a list of {\sf variable(Vid, VCPrag)}, one
%for each variable (not each occurrence).  {\sf VCPrag} is the shared
%portion of the variable-pragmas of all occurrences of the variable.


\subsection{Compiler Options}
%============================

Besides the compiler options introduced in the {\it XSB Programmer's Manual},
there are some other options that exist mainly for debugging and profiling
purposes.  Here is a list of compiler options.

\begin{itemize}
\item   Inference Options:
	\begin{description}
        \item[nocompile] Do not compile, but do inferencing only.
	\item[spec\_off] Do not perform specialisation of partially
			 instantiated calls.
        \item[modeinfer] Performs mode inferencing (currently not implemented).
	\end{description}
\item   Debugging Options:
	\begin{description}
        \item[verbo]	Prints out intermediate information during compiling.
        \item[dump\_asm] Generates a {\tt *.A} file.  Known to {\em not\/}
			work when predicates are cut-transformed.
	\item[dump\_spec] Generates a {\tt *.spec} file, containing the result
			of specialising partially instantiated calls to
			predicates defined in the module.  The result is in
			Prolog source code form.
	\item[dump\_ti] Generates a {\tt *.ti} file, containing the result
			of applying transformational indexing (unification
			factoring) to some subset of the predicates defined
			in the module.  The result is in Prolog source code
			form.
	\end{description}
\item   Pragma Options:
	\begin{description}
        \item[sysmod]	Mainly used for the compilation of system modules.
			Standard predicates are not assumed to be automatically
			imported, and the compiler does not generate warnings
			for the redefinition of standard predicates.
			Also, all primitive predicates are compiled inline
			(calling {\tt '\_\$builtin'/1}).  Because of that,
			there are some modules (like {\tt syslib/standard.P},
			or {\tt lib/setof.P}) that {\em must\/} be compiled
			with this option.
        \item[index\_off] The compilation phase will not generate indices.
	\end{description}
\end{itemize}

