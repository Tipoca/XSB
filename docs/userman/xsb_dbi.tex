
\chapter{The New XSB-Database Interface} \label{db_interface}
%====================================================

\begin{center}
{\Large {\bf By Saikat Mukherjee, Michael Kifer and Hui Wan }}
\end{center}

\section{Introduction}
%=====================

The XSB-DB interface is a package that allows XSB users to access
databases through various drivers. Using this interface, information
in different DBMSs can be accessed by SQL queries. The interface
defines Prolog predicates which makes it easy to connect to databases,
query them, and disconnect from the databases. Central to the concept
of a connection to a database is the notion of a \emph{handle}.  A
connection handle describes a particular connection to a database.
Similar to a connection handle is the notion of a query handle which
describes a particular query statement. As a consequence of the
handles, it is possible to open multiple database connections (to the
same or different databases) and keep alive multiple queries (again
from the same or different connections). The interface also supports
dynamic loading of drivers. As a result, it is possible to query
databases using different drivers concurrently~\footnote{In
  \version{}, this package has not been ported to the multi-threaded
  engine.}.

Currently, this package provides drivers for ODBC, a native MySQL driver,
and a driver for the embedded MySQL server.

\section{Configuring the Interface}

Generally, each driver has to be configured separately, but if the database
packages such as ODBC, MySql, etc., are installed in standard places then
the XSB configuration mechanism will do the job automatically.

Under Windows, first make sure that XSB is configured and built correctly
for Windows, and that it runs. As part of that building process, the
command
%%
\begin{verbatim}
    makexsb_wind  
\end{verbatim}
%%
must have been executed
in the directory {\tt XSB$\backslash$build}. It will normally configure the ODBC
driver without problems. For the MySQL driver one has to edit the file
%%
\begin{verbatim}
    packages\dbdrivers\mysql\cc\NMakefile.mak  
\end{verbatim}
%%
to indicate where MySQL is installed. To build the embedded MySQL driver
under Windows, the file
%%
\begin{verbatim}
    packages\dbdrivers\mysqlenbedded\cc\NMakefile.mak  
\end{verbatim}
%%
might need to be edited. Then you should either rebuild XSB using the
{\tt makexsb\_wind} command or by running
%%
\begin{verbatim}
    nmake /f NMakefile.mak
\end{verbatim}
%% 
in the appropriate directories (\verb|dbdrivers\mysql\cc| or
\verb|dbdrivers\mysqlenbedded\cc|). Note that you need a C++ compiler and
{\tt nmake} installed on your system for this to work.\footnote{
  {\tt http://www.microsoft.com/express/vc/}\\
  {\tt
    http://download.microsoft.com/download/vc15/Patch/1.52/W95/EN-US/Nmake15.exe} 
  }

Under Unix, the {\tt configure} script will build the drivers automatically
if the {\tt --with-dbdrivers} option is specified.  If, however, ODBC and
MySQL are not installed in their standard places, you will have to provide
the following parameters to the configure script:
%%
\begin{itemize}
\item  {\tt --with-odbc-libdir=LibDIR} -- {\tt LibDIR} is the directory
  where the library libodbc.so lives on your system. 
\item  {\tt --with-odbc-incdir=IncludeDIR} -- {\tt IncludeDIR} is the
  directory where the ODBC header files, such as {\tt sql.h} live.
\item {\tt --with-mysql-libdir=MySQLlibdir} -- {\tt MySQLlibdir} is the
  directory where MySQL's shared libraries live on your system.
\item {\tt --with-mysql-incdir=MySQLincludeDir} -- {\tt MySQLincludeDir} is
  the directory where MySQL's header files live.  
\end{itemize}
%%
If you are also using the embedded MySQL server and want to take advantage
of the corresponding XSB driver, you need to provide the following directories
to tell XSB where the copy of MySQL that supports the embedded server is installed.
This has to be done \emph{only} if that copy is not in a standard place,
like {\tt /usr/lib/mysql}.  

\begin{itemize}%% 
\item {\tt --with-mysqlembedded-libdir=MySQLlibdir} -- {\tt MySQLlibdir} is the
  directory where MySQL's shared libraries live on your system.
  This copy of MySQL must be configured with support for the embedded
  server.
\item {\tt --with-mysqlembedded-incdir=MySQLincludeDir} -- {\tt MySQLincludeDir} is
  the directory where MySQL's header files live.  
\end{itemize}
%% 

Under Cygwin, the ODBC libraries come with the distribution; they are
located in the directory {\tt /cygdrive/c/cygwin/lib/w32api/} and are called
{\tt odbc32.a} and {\tt odbccp32.a}. (Check if your installation is complete
and has these libraries!)
Otherwise, the configuration of the interface under Cygwin is
same as in unix (you do not need to provide any ODBC-specific parameters to
the configure script under Cygwin).


If at the time of configuring XSB some database packages ({\it e.g.},
MySQL) are not installed on your system, you can install them later and
configure the XSB interface to them then. For instance, to configure the
ODBC interface separately, you can type
%%
\begin{verbatim}
    cd packages/dbdrivers/odbc
    configure
\end{verbatim}
%%
Again, if ODBC is installed in a non-standard location, you might need to
supply the options {\tt --with-odbc-libdir} and {\tt --with-odbc-incdir} to
the configure script. Under Cygwin ODBC is always installed in a
standard place, and {\tt configure} needs no additional parameters.

Under Windows, separate configuration of the XSB-DB interfaces is also
possible, but you need Visual Studio installed. For instance, to configure
the MySQL interface, type
%%
\begin{verbatim}
    cd packages\dbdrivers\mysql\cc  
    nmake /f NMakefile.mak
\end{verbatim}
%%
As before, you might need to edit the NMakefile.mak script to tell the
compiler where the required MySQL's libraries are. You also need the file
{\tt packages$\backslash$dbdrivers$\backslash$mysql$\backslash$mysql\_init.P} with the following content:
%%
\begin{verbatim}
:- export mysql_info/2.
mysql_info(support, 'yes').
mysql_info(libdir,  '').
mysql_info(ccflags, '').
mysql_info(ldflags, '').
\end{verbatim}
%%

Similarly, to configure the ODBC interface, do
%%
\begin{verbatim}
    cd packages\dbdrivers\odbc\cc  
    nmake /f NMakefile.mak
\end{verbatim}
%%
You will also need to create the file {\tt packages$\backslash$dbdrivers$\backslash$odbc$\backslash$odbc\_init.P}
with the following contents:
%%
\begin{verbatim}
:- export odbc_info/2.
odbc_info(support, 'yes').
odbc_info(libdir,  '').
odbc_info(ccflags, '').
odbc_info(ldflags, '').
\end{verbatim}
%%


\section{Using the Interface}
%========================================

We use the {\tt student} database as our example to illustrate 
the usage of the XSB-DB interface in this manual. The schema of the
student database contains three columns viz. the student name, 
the student id, and the name of the advisor of the student.

The XSB-DB package has to be first loaded before using any of the
predicates. This is done by the call:

\begin{verbatim}
| ?- [dbdrivers].
\end{verbatim}

Next, the driver to be used for connecting to the database has to 
be loaded. Currently, the interface has support for a native MySQL driver
(using the MySQL C API), and an ODBC driver. For example, to load 
the ODBC driver call:

\begin{verbatim}
| ?- load_driver(odbc).
\end{verbatim}

Similarly, to load the mysql driver call:

\begin{verbatim}
| ?- load_driver(mysql).
\end{verbatim}
%%
or
%%
\begin{verbatim}
| ?- load_driver(mysqlembedded).
\end{verbatim}

\subsection{Connecting to and Disconnecting from Databases}
%=================================================

There are two predicates for connecting to databases, {\tt db\_connect/5}
and {\tt db\_connect/6}.
The {\tt db\_connect/5} predicate is for ODBC connections, while {\tt
  db\_connect/6} is 
for other (non-ODBC) database drivers.

\begin{verbatim}
| ?- db_connect(+Handle, +Driver, +DSN, +User, +Password).
| ?- db_connect(+Handle, +Driver, +Server, +Database, +User, +Password).
\end{verbatim}

The {\tt db\_connect/5} predicate
assumes that an entry for a data source name (DSN) exists in the {\tt
  odbc.ini} file. The {\tt Handle} is the handle name used for the
connection. The {\tt Driver} is the driver being used for the connection.
The {\tt User} and {\tt Password} are the user name and password being used
for the connection. The user is responsible for giving the name to the
handle.  To connect to the data source mydb using the user name xsb and
password xsb with the odbc driver, the call is as follows:

\begin{verbatim}
| ?- db_connect(ha, odbc, mydb, xsb, xsb).
\end{verbatim}

\noindent
where {\tt ha} is the user-chosen handle name (a Prolog atom) for the
connection.

The {\tt db\_connect/6} predicate is used for drivers other than ODBC.  The
arguments {\tt Handle}, {\tt Driver}, {\tt User}, and {\tt Password} are
the same as for {\tt db\_connect/5}. The {\tt Server} and {\tt Database}
arguments specify the server and database to connect to.  For example, for
a connection to a database called {\tt test} located on the server {\tt
  wolfe} with the user name {\tt xsb}, the password {\tt foo}, and using
the {\tt mysql} driver, the call is:

\begin{verbatim}
| ?- db_connect(ha, mysql, wolfe, test, xsb, foo).
\end{verbatim}

\noindent
where {\tt ha} is the handle name the user chose for the connection.

If the connection is successfully made, the predicate invocation will
succeed.  This step is necessary before anything can be done with the
data sources since it gives XSB the opportunity to initialize system
resources for the session.

To close a database connection use:

\begin{verbatim}
| ?- db_disconnect(Handle).
\end{verbatim}

\noindent
where handle is the handle name for the connection. For example, 
to close the connection to above mysql database call:

\begin{verbatim}
| ?- db_disconnect(ha).
\end{verbatim}

and XSB will give all the resources it allocated for this session back
to the system.


\subsection{Querying Databases}
%=====================================

The interface supports two types of querying. In direct querying, the
query statement is not prepared while in prepared querying the query
statement is prepared before being executed. The results from
both types of querying are retrieved tuple at a time.
Direct querying is done by the predicate:

\begin{verbatim}
| ?- db_query(ConnectionHandle, QueryHandle, SQLQueryList, ReturnList).
\end{verbatim}

ConnectionHandle is the name of the handle used for the database connection.
QueryHandle is the name of the handle for this particular query. 
Currently, the query handle is being used only for prepared queries. 
However, in future versions, the query handle can be used in direct
queries to retrieve arbitrary tuples from a result set using cursors.
Also, it will be possible to combine arbitrary tuples from different 
queries to the same database using the query handle.
The SQLQueryList is a list of terms which is used to build the SQL query.
The terms in this list are ground atoms. ReturnList is a list of 
variables each of which correspond to a return value in the query.
It is upto the user to specify the correct number of return variables 
corresponding to the query. Also, as in the case of a connection handle, 
the user is responsible for giving the name to the query handle.
For example, a query on the student database to select all the students
for a given advisor is accomplished by the call:

\begin{verbatim}
| ?- X = adv,
     db_query(ha, qa, ['select T.name from student T where T.advisor = ', X], [P]),
     fail.
\end{verbatim}

\noindent
where {\tt ha} and {\tt qa} are respectively the connection handle and query
handle name the user chose.

Observe that the query list is composed of the SQL string and a ground value
for the advisor. The return list is made of one variable corresponding to
the student name. The failure drive loop retrieves all the tuples.

Preparing a  query is done by the call to the predicate:

\begin{verbatim}
| ?- db_prepare(ConnectionHandle, QueryHandle, SQLQueryList).
\end{verbatim}

As before, ConnectionHandle and QueryHandle specify the handles for
the connection and the query. The SQLQueryList is a list of terms which
build up the query string. The placeholder `?' is used for values which 
have to be bound during the execution of the statement.
For example, to prepare a query for selecting the advisor name for a student
name using our student database:

\begin{verbatim}
| ?- db_prepare(ha, qa, ['select T.advisor from student T where T.name = ?']).
\end{verbatim}

A prepared statement is executed using the predicate:

\begin{verbatim}
| ?- db_prepare_execute(QueryHandle, BindList, ReturnList).
\end{verbatim}

The BindList contains the ground values corresponding to the `?' in
the prepared statement. The ReturnList is a list of variables for
each argument in a tuple of the result set.

For direct querying, the statement handle is closed automatically when
all the tuples in the result set have been retrieved. In order to explicitly
close a statement handle, and free all the resources associated with
the handle, a call is made to the predicate:

\begin{verbatim}
| ?- db_statement_close(QueryHandle).
\end{verbatim}

\noindent
where QueryHandle is the query handle for the statement to be closed.

The interface is also able to transparently handle Prolog terms. 
Users can both save and retrieve terms without any special processing.


\section{Error Handling}
%============================

Each predicate in the XSB-DB interface throws an exception with the functor 
%%
\begin{verbatim}
   xsb_error(database(Number), Message) 
\end{verbatim}
%%
where Number is a string with the 
error number and Message is a string with a slightly detailed error message.
It is upto the user to catch this exception
and proceed with error handling. This is done by the throw-catch error
handling mechanism in XSB. For example, in order to catch the error which
will be thrown when the user attempts to close a database connection for 
a handle {\tt (ha}) which does not exist:

\begin{verbatim}
| ?- catch(db_disconnect(ha),
     xsb_error(database(Number), Message), handler(Number, Message)).
\end{verbatim}

It is the user's responsibility to define the handler predicate which can be as simple
as printing out the error number and message or may involve more 
complicated  processing.

A list of error numbers and messages that are thrown by the XSB-DB interface
is given below:

\begin{itemize}

\item {\bf XSB\_DBI\_001: XSB\_DBI ERROR: Driver already registered}\\
This error is thrown when the user tries to load a driver, using the {\tt load\_driver} 
predicate, which has already been loaded previously.

\item {\bf XSB\_DBI\_002: XSB\_DBI ERROR: Driver does not exist}\\
This error is thrown when the user tries to connect to a database, using
{\tt db\_connect}, with a driver which has not been loaded.

\item {\bf XSB\_DBI\_003: XSB\_DBI ERROR: Function does not exist in this driver}\\
This error is thrown when the user tries to use a function support 
for which does not exist in the corresponding driver. For example, 
this error is generated if the user
tries to use {\tt db\_prepare} for a connection established with the
mysql driver.

\item {\bf XSB\_DBI\_004: XSB\_DBI ERROR: No such connection handle}\\
This error is thrown when the user tries to use a connection handle
which has not been created.

\item {\bf XSB\_DBI\_005: XSB\_DBI ERROR: No such query handle}\\
This error is thrown when the user tries to use a query handle which has 
not been created.

\item {\bf XSB\_DBI\_006: XSB\_DBI ERROR: Connection handle already exists}\\
This error is thrown when the user tries to create a connection handle in
{\tt db\_connect} using a name which already exists as a connection handle.

\item {\bf XSB\_DBI\_007: XSB\_DBI ERROR: Query handle already exists}\\
This error is thrown when the user tries to create a query handle, in
{\tt db\_query} or {\tt db\_prepare}, using a name which already exists as
a query handle for a different query.

\item {\bf XSB\_DBI\_008: XSB\_DBI ERROR: Not all parameters supplied}\\
This error is thrown when the user tries to execute a prepared statement,
using {\tt db\_prepare\_execute}, without supplying values for all the
parameters in the statement.

\item {\bf XSB\_DBI\_009: XSB\_DBI ERROR: Unbound variable in parameter list}\\
This error is thrown when the user tries to execute a prepared statement,
using {\tt db\_prepare\_execute}, without binding all the parameters of the
statement.

\end{itemize}

\section{Notes on specific drivers}
%============================

\subsubsection{ODBC Driver}

The ODBC driver has been tested in Linux using the {\tt unixodbc} driver
manager.  It currently supports the following functionality: (a)
connecting to a database using a DSN, (b) direct querying of the database,
(c) using prepared statements to query the database, (d) closing a
statement handle, and (d) disconnecting from the database.
The ODBC driver has also been tested under Windows and Cygwin.


\subsubsection{MySQL Driver}

The MySQL driver provides access to the native MySQL C API.
Currently, it has support for the following functionality: (a) connecting
to a database using {\tt db\_connect}, (b) direct querying of
the database, (c) using prepared statements to query the database, (d) closing a
statement handle, and (e) disconnecting from the database.

The MySQL driver has been tested under Linux.


\subsubsection{Driver for the Embedded MySQL Server}

This driver provides access to the Embedded MySQL Server Library {\tt libmysqld}. Currently, it has support for the following functionality: (a) connecting
to a database {\tt db\_connect}, (b) direct querying of
the database, (c) using prepared statements to query the database, (d) closing a
statement handle, and (e) disconnecting from the database.

The MySQL driver for Embedded MySQL Server has been tested under Linux.

\noindent
In order to use this driver, you will need:
\begin{itemize}
 \item MySQL with Embedded Server installed on your machine. If your don't
 have a precompiled binary distribution of MySQL, which was configured with
 libmysqld support (the embedded server library), you will need to build
 MySQL from sources and configure it with the {\tt --with-embedded-server} option.

 \item append to {\tt /etc/my.cnf} (or {\tt /etc/mysql/my.cnf} -- whichever
   is used on your machine) or $\sim${\tt/.my.cnf}:
       \begin{verbatim}
               [mysqlembedded_driver_SERVER]
               language = /usr/local/mysql/share/mysql/english
               datadir = ..... 
       \end{verbatim}
       You will probably need to replace
       {\tt /usr/local/mysql/share/mysql/english} with a directory
       appropriate for your MySQL installation.

       You might also need to set the {\tt datadir} option to specify
       the directory where the databases managed by the embedded server are
       to be kept. This has to be done if there is a possibility of running
       the embedded MySQL server alongside the regular MySQL server. In
       that case, the {\tt datadir} directory of the embedded server must
       be different from the {\tt datadir} directory of the regular server
       (which is likely to be specified using the {\tt datadir} option in
       {\tt /etc/my.cnf} or {\tt /etc/mysql/my.cnf}. This is because
       specifying the same directory might lead to a corruption of your
       databases. See {\tt
         http://dev.mysql.com/doc/refman/5.1/en/multiple-servers.html} for
       further details on running multiple servers.
\end{itemize}

Please note that loading the embedded MySQL driver increases the memory
footprint of XSB. This additional memory is released automatically when XSB
exits. If you need to release the memory before exiting XSB, you can call {\tt
  driverMySQLEmbedded\_lib\_end} after disconnecting from MySQL.  Note
that once {\tt driverMySQLEmbedded\_lib\_end} is called, no further
connections to MySQL are allowed from the currently running session of XSB
(or else XSB will exit abnormally).



%%% Local Variables: 
%%% mode: latex
%%% TeX-master: "manual2"
%%% End: 
