\chapter{XSB's POSIX Regular Expression and Wildcard Matching Packages}

\begin{center}
{\Large {\bf By Michael Kifer}}
\end{center}

XSB has an efficient interface to POSIX pattern regular expression and
wildcard matching functions.  To take advantage of these features, you must
build XSB using a C compiler that supports POSIX 1.0 (for regular
expression matching) and the forthcoming POSIX 2.0 (for wildcard matching).
The recent versions of GCC and SunPro compiler will do, as probably will
many other compilers. This also works under Windows, provided you install
Cygnus' CygWin and use GCC to compile.

\section{Regular Expression Matching and Substitution}

The following discussion assumes that you are familiar with the syntax of
regular expressions and have a reasonably good idea about their
capabilities. One easily accessible description of POSIX regular
expressions is found in the on-line Emacs manual.

The regular expression matching functionality is provided by the package
called {\tt Regmatch}. To use it, type:
%%
\begin{verbatim}
   :- [regmatch].
\end{verbatim}
%%

If you are planning to use pattern matching from within an XSB program,
then you also need to add the following directive:
%%
\begin{verbatim}
    :- import re_match/7, re_subst/6 from regmatch.
\end{verbatim}
%%

The \verb|re_match/7| predicate performs regular expression matching.
The predicate \verb|re_subst/6| matches parts of a regular expression
against a string, substitutes the matched part with another string, and
then returns that new string.

The \verb|re_match/7| predicate has the following calling sequence:
%%
\begin{verbatim}
 re_match(+Regexp, +InputStr, ?IgnoreCase, -Match, -Prematch, -Postmatch, -ParenMatches)
\end{verbatim}
%%
{\tt Regexp} is a regular expression, {\it e.g.},
``\verb|abc([^;,]*); (dd|ee)*;|''. It can be a Prolog atom or string ({\it i.e.}, a list of
characters). The above expression matches any substring that has ``abc''
followed by a sequence of characters none of which is a ``;'' or a ``,'',
followed by a ``; '', followed by a sequence that consists of zero or more
of ``dd'' or ``ee'' segments, followed by a ``;''. An example of a
where such a match can be found is ``\verb|123abc&*^; ddeedd;poi|''.

{\tt InputStr} is the string to be matched. It can be a Prolog atom or a string
(list of characters). {\tt IgnoreCase} indicates whether
the case of the letters is to be ignored. If this argument is an
uninstantiated variable, then the case is \emph{not} ignored. If this
argument is bound to a non-variable, then the case \emph{is} ignored.

The rest of the arguments are used to return the results. {\tt Match} 
is bound to the maximal substring of {\tt InputStr} that matches the regular
expression. In our case, it will be ``\verb|abc&*^; ddeedd|''.
{\tt Prematch} is bound to the part of {\tt InputStr} that
\emph{precedes} the string in {\tt Match}. In our case, it is ``{\tt 123}''.
{\tt Postmatch} is
instantiated with the suffix of {\tt InputStr} that follows the matching
substring (``{\tt poi}'' in our case).

The most powerful feature of POSIX pattern matching is the ability to
remember and return substrings matched by parenthesized subexpressions.
When the predicate succeeds, the list of such matches is bound to {\tt
  ParenMatch}. For instance, our earlier regular expression 
  ``\verb|abc([^;,]*); (dd|ee)*;|'' has two parenthetical subexpressions, which
match ``\verb|&*^|'' and ``{\tt dd}, respectively. So, {\tt ParenMatch} will
be bound to ``\verb|[&*^, dd]|''.

The maximal number of parenthetical expressions supported by the Regmatch
package is 30. Partial matches to parenthetical expressions 31 and over are
discarded.

Here is one more example of the power of POSIX regular expression matching:
%%
\begin{verbatim}
   ?- re_match("a(b*|e*)cd\\1",'qabbbcdbbbbbo',_,M,P,K,L).  
\end{verbatim}
%%
Here the result is:
%%
\begin{verbatim}
M = abbbcdbbb
P = q
K = bbo
L = [bbb]  
\end{verbatim}
%%
The interesting features here are the positional parameter
$\backslash\backslash 1$ and the alternating parenthetical expression {\tt
a(b*|e*)}. The alternating parenthetical expression here can match any
sequence of b's \emph{or} any sequence of e's. Note that if the string to
be matched is not known when we write the program, we will not know a
priori which sequence will be matched: a sequence of b's or a sequence of e's.
Moreover, we do not even know the length of that sequence.

Now, suppose, we want to make sure that the matching substrings look like this:
%%
\begin{verbatim}
abbbcdbbb
aeeeecdeeee
abbbbbbcdbbbbbb
\end{verbatim}
%%
How can we make sure that the suffix that follows ``cd'' is exactly the same
string that is stuck between ``a'' and ``cd''? This is what
$\backslash\backslash 1$ precisely does: it represents the substring
matched by the first parenthetical expression. Similarly, you can use
$\backslash\backslash 2$, etc., if the regular expression contains more
than one parenthetical expression.

{\bf Additional features.} So far, we did not discuss the type of the
result returned in the 7th argument of \verb|re_match/7| (apart from saying
that this is a list). The precise type depends on the type of the second
argument, {\tt InputStr}: if {\tt InputStr} is an atom, them {\tt ParenMatch} is a
list of atoms. If {\tt InputStr} is a Prolog string, then {\tt ParenMatch} is a
list of Prolog strings.

The \verb|re_match/7| predicate fails if {\tt Regexp} does not match {\tt InputStr}.
Otherwise, it succeeds.

We should also note that parenthetical expressions can also be represented
using the \verb|\(...\)| notation. What if you want to match a ``('' then?
You must escape it with a ``\verb|\\|'' then:
%%
\begin{verbatim}
| ?- re_match("a(b*)cd\\(",'abbbcd(bbo',_,M,P,K,L).

M = abbbcd(
P = 123
K = bbo
L = [bbb]
\end{verbatim}
%%
Now, what about matching the backslash itself? Try harder: you need four
backslashes: 
%%
\begin{verbatim}
| ?- re_match("a(b*)cd\\\\",'abbbcd\bbo',_,M,P,K,L).

M = abbbcd\
P = 123
K = bbo
L = [bbb]  
\end{verbatim}
%%

The predicate \verb|re_subst/6| has the following invocation:
%%
\begin{verbatim}
    re_subst(+Regexp, +InputStr, ?IgnoreCase, +WhichMatch, +SubstString, -OutStr)  
\end{verbatim}
%%
The parameters {\tt Regexp}, {\tt InputStr}, and {\tt IgnoreCase} have the
same meaning as in the case of \verb|re_match/7|. {\tt SubstString} must be
an atom or a character list. {\tt OutStr} must be a variable.
It is instantiated to an atom or a character list depending on whether
{\tt InputStr} is an atom or a character list.

This predicate works as follows: the regular expression {\tt Regexp} is
matched againts {\tt InputStr}. If {\tt IgnoreCase} is bound to a
non-variable, then case of the letters is ignored during the match.
Then a new string is constructed out of {\tt InputStr} and {\tt
SubstString} and bound to {\tt OutStr} as follows: 
%%
\begin{itemize}
\item if {\tt WhichMatch} is a `+', then the suffix of {\tt InputStr}
  adjacent to the end of the matched substring of {\tt InputStr} is
  replaced with {\tt SubstString}.
\item if {\tt WhichMatch} is a `--', then the prefix preceeding the matched
  substring of {\tt InputStr} is replaced with {\tt SubstString}.
\item if {\tt WhichMatch} is 0, then the matched substring itself is
  replaced with {\tt SubstString}.
\item if {\tt WhichMatch} is 1, 2, 3, etc., then the match corresponding
  to the parenthesized expression \#1, \#2, \#3, etc., is replaced with
  {\tt SubstString}. It is an error to specify a number that is higher than
  the number of parenthesized subexpressions in {\tt Regexp}.
\end{itemize}
%%
Here are some examples:
%%
\begin{verbatim}
| ?- re_subst('a(d)d','qaddf',_,-,'123',L).

L = 123addf

| ?- re_subst('a(d)d','qaddf',_,+,'123',L).

L = qadd123

| ?- re_subst('a(d)d','qaddf',_,0,'123',L).

L = q123f

| ?- re_subst('a(d)d','qaddf',_,1,'123',L).

L = qa123df

| ?- re_subst('a(d)d','qaddf',_,2,'123',L).

++Warning: RE_SUBST: Arg 4 (=2): invalid parenthetical subexpressions number

no
\end{verbatim}
%%

\section{Wildcard Matching and Globing}

These interfaces are implemented using the {\tt Wildmatch} package of XSB.
This package provides the following functionality: 
%%
\begin{enumerate}
\item Telling whether a wildcard, like the ones used in Unix shells match
  against a given string. Wildcards supported are of the kind available in
  tcsh or bash. Alternating characters ({\it e.g.}, ``\verb|[abc]|'' or
  ``\verb|[^abc]|'') are supported.
\item Finding the list of all file names in a given directory that match a
  given wildcard. This facility generalizes {\tt directory/2} (in module {\tt
    directory}), and it is much more efficient.
\end{enumerate}
%%

To use this package, you need to type:
%%
\begin{verbatim}
| ?- [wildmatch].  
\end{verbatim}
%%
If you are planning to use it in an XSB program, you will also need to add
these directive:
%%
\begin{verbatim}
:- import glob_directory/4, wildmatch/3 from wildmatch.
\end{verbatim}
%%

The calling sequence for \verb|glob_directory| is:
%%
\begin{verbatim}
   glob_directory(+Wildcard, +Directory, ?MarkDirs, -FileList)  
\end{verbatim}
%%
The parameter {\tt Wildcard} can be either a Prolog atom or a Prolog
string. {\tt Directory} is also an atom or a string; it specifies the
directory to be globbed. {\tt MarkDirs} indicates whether directory names
should be decorated with a trailing slash: if {\tt MarkDirs} is bound, then
directories will be so decorated. If MarkDirs is an unbound variable, then
trailing slashes will not be added.

{\tt FileList} gets the list of files in {\tt Directory} that match {\tt
  Wildcard}.  If {\tt Directory} is bound to an atom, then {\tt FileList}
gets bound to a list of atoms; if {\tt Directory} is a Prolog string, then
{\tt FileList} will be bound to a list of strings as well.

This predicate succeeds is at least one match is found. If no matches are
found or if {\tt Directory} does not exist or cannot be read, then the
predicate fails.

The calling sequence for wildmatch/3 is as follows:
%%
\begin{verbatim}
    wildmatch(+Wildcard, +String, ?IgnoreCase)  
\end{verbatim}
%%
{\tt Wildcard} is the same as before. {\tt String} represents the string to
be matched against {\tt Wildcard}. Like {\tt Wildcard}, {\tt String} can be
an atom or a string. {\tt IgnoreCase} indicates whether case of letters
should be ignored: if this argument is bound to a non-variable, then case
is ignored. Otherwise, if {\tt IgnoreCase} is a variable, then the case of
letters is preserved.

This predicate succeeds when {\tt Wildcard} matches {\tt String} and fails
otherwise.




%%% Local Variables: 
%%% mode: latex
%%% TeX-master: "manual2"
%%% End: 
