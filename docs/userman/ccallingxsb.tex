\chapter{Calling \ourprolog\ from C}
%===================================
\label{ccallingxsb}

There are many situations in which it may be desirable to use
\ourprolog\ as a rule-processing subcomponent of a larger system,
which is written in another language.  To do this, one wants to be
able to {\em call}\ \ourprolog\ from the host language, often C,
providing queries for \ourprolog\ to evaluate and retrieving back the
answers.  An interface for calling \ourprolog\ from C (or other
language) is provided for this purpose and is described in this
chapter.  Simple examples of the use of this interface are given in
the {\tt XSB/examples/c\_calling\_XSB} subdirectory, in files {\tt
cmain.c}, {\tt cmain2.c}, {\tt ctest.P}, and {\tt Makefile}.

\section{C Functions for Calling XSB}

XSB provides several C functions (declared in {\tt XSB/emu/cinterf.h}
and defined in \\ {\tt XSB/emu/cinterf.c}), which can be called from C
to interact with \ourprolog\ as a subroutine. These functions allow a
C program to initialize \ourprolog\ (most easily with a call to {\tt
xsb\_init\_string(options)}) and then to interact with XSB to have it
execute {\em commands} or {\em queries}. A command is a deterministic
query which simply succeeds or fails (without returning any
interesting data value.)  A non-deterministic query can be evaluated
so that its answers are retrieved one at a time, as they are
produced. There are several levels of interface provided.  The highest
level interface uses the XSB-specific C-type definition for
variable-length strings (Section \ref{sec-varstring}), to communicate
queries to \ourprolog\ and to get answers back.  The {\tt
xsb\_command\_string(cmd)} function allows you to pass a command as a
(period-terminated) string to \ourprolog .  The {\tt
xsb\_query\_string\_string(query,buff,sep)} function allows you to
pass a query to XSB as a string, and have its (first) answer returned
as a string.  Subsequent answers can be calculated and retrieved using
{\tt xsb\_next\_string(buff,sep)}.

The second level provides routines that return answers with an
interface that does not require variable-length strings.  The routines
at this level are: 
\begin{itemize}
\item {\tt xsb\_query\_string\_string\_b(query,buff,bufflen,anslen,sep)}, 
\item {\tt xsb\_next\_string\_b(buff,bufflen,anslen,sep)}, and
\item  {\tt xsb\_get\_last\_answer(buff,bufflen,anslen)}.  
\end{itemize}
They are normally intended to be used with the initialization routines
above.

There are lower-level interfaces that allow you to manipulate directly
the XSB data structures (both to construct queries and to retrieve
answers) and thus avoid the overhead of converting to and from
strings.  See the detailed descriptions of the routines below to see
how to use the lower level interface.

Currently, only one query can be active at a time.
I.e., one must completely finish processing one query (either by
retrieving all the answers for it, or by issuing a call to {\tt
xsb\_close\_query()}, before trying to evaluate another.  The
routines to perform all these functions are described below:

\begin{description}
\ouritem{int xsb\_init(int argc, char *argv[])} \index{{\tt xsb\_init}}
This is a C-callable function that initializes \ourprolog . It must be
called before any other calls can be made.  {\tt argc} is the count of
the number of arguments in the {\tt argv} vector.  The {\tt argv}
vector is exactly as would be passed from the command line to
\ourprolog .  It must contain at least the following two things:
%%
\begin{itemize}
\item $\tt argv[0]$ must be an absolute or relative path name of the XSB
  installation directory ({\it i.e.}, {\tt \$XSB\_DIR}.  Here is an
  example, which assumes that we invoke the C program from the XSB
  installation directory.
    %%
    \begin{verbatim}
int main(int argc, char *argv[])
{ 
  int myargc = 2;
  char *myargv[2];

  /* XSB_init relies on the calling program to pass the addr of the XSB
     installation directory. From here, it will find all the libraries */
  myargv[0] = ".";
  myargv[1] = "-n";

  /* Initialize xsb */
  xsb_init(myargc,myargv);
    \end{verbatim}
    %%
  \item $\tt argv[1]$ must be the {\tt -n} flag.  This flag tells
    \ourprolog\ not to start the read-eval-print top loop, but to act as a
    subroutine to a calling C routine.
\end{itemize}
%%
Other flags are optional, but can be used to modify sizes of the various
spaces used in \ourprolog.  {\tt xsb\_init} returns 0 if initialization is
completed, and 1 if an error is encountered.

\ouritem{int xsb\_init\_string(char *options)} \index{{\tt xsb\_init\_string}}
This is a variant of {\tt xsb\_init} which takes the command line as a
string argument (rather than as a argc/argv pair.)  For example, a call
could be
%%
\begin{verbatim}
   xsb_init_string(". -n");
\end{verbatim}
%%
Note that just as with {\tt xsb\_init}, you must pass the path name of the
XSB installation directory. In the above, we pass ``.'', assuming that we
are invoking the C program from the XSB installation directory.  The
parameters following the file name are just as those that could appear on a
command line.  The function of this subroutine is exactly the same as {\tt
  xsb\_init}, and its return codes are the same.

\ouritem{int xsb\_command()} \index{{\tt xsb\_command}}
This function passes a command to \ourprolog .  No query can be active
when this command is called.  Before calling {\tt xsb\_command}, the
calling program must construct the \ourprolog\ term representing the
command in register 1 in \ourprolog 's space.  This can be done by
using the {\tt c2p\_*} (and {\tt p2p\_*}) routines, which are
described in Section \ref{c2p_p2p_p2c} below.  Register 2 may also be
set before the call to {\tt xsb\_query} (using {\tt
xsb\_make\_vars(int)} and {\tt xsb\_set\_var\_*()}) in which case any
variables set to values in the {\tt ret/n} term will be so bound in
the call to the command goal.  {\tt xsb\_command} invokes the command
represented in register 1 and returns 0 if the command succeeds and 1
if it fails. In either case it resets register 1 back to a free
variable. If there is an error, it returns 2.

\ouritem{int xsb\_command\_string(char *cmd)} \index{{\tt xsb\_command\_string}}
This function passes a command to \ourprolog .  The command is a
string consisting of a term that can be read by the \ourprolog\
reader.  The string must be terminated by a period (.).  Any previous
query must have already been closed.  In all other respects, {\tt
xsb\_command\_string} is similar to {\tt xsb\_command}.

\ouritem{int xsb\_query()} \index{{\tt xsb\_query}}
This function passes a query to \ourprolog .  Any previous query must
have already been closed.  A query is expected to return possibly
multiple data answers.  The first is found and made available to the
caller as a result of this call.  To get subsequent answers, {\tt
xsb\_next} must be called.  Before calling {\tt xsb\_query} the caller
must construct the term representing the query in \ourprolog 's
register 1 (using routines described in Section \ref{c2p_p2p_p2c}
below.)  If the query has no answers (i.e., just fails), register 1 is
set back to a free variable and {\tt xsb\_query} returns 1.  If the
query has at least one answer, the variables in the query term in
register 1 are bound to those answers and {\tt xsb\_query} returns 0.
In addition, register 2 is bound to a term whose main functor symbol
is {\tt ret/n}, where n is the number of variables in the query. The
main subfields of this term are set to the variable values for the
first answer. (These fields can be accessed by the functions {\tt
p2c\_*}, or the functions {\tt xsb\_var\_*}, described in Section
\ref{c2p_p2p_p2c} below.)  Thus there are two places the answers are
returned. Register 2 is used to make it easier to access them.  To get
subsequent answers, {\tt xsb\_next} must be called.  Register 2 may
also be set before the call to {\tt xsb\_query} (using {\tt
xsb\_make\_vars(int)} and {\tt xsb\_set\_var\_*()}) in which case any
variables set to values in the {\tt ret/n} term will be so bound in
the call to the goal.

\ouritem{int xsb\_query\_string(char *query)} \index{{\tt xsb\_query\_string}}
This function passes a query to \ourprolog .  The query is a string
consisting of a term that can be read by the \ourprolog\ reader.  The
string must be terminated with a period (.).  Any previous query must
have already been closed.  In all other respects, xsb\_query\_string
is similar to xsb\_query, except the only way to retrieve answers is
through Register 2.  The ability to create the return structure and
bind variables in it is particularly useful in this function.

\ouritem{int xsb\_query\_string\_string(char *query, VarString *buff,
char *sep)} \index{{\tt xsb\_query\_string\_string}} This function
is a variant of {\tt xsb\_query\_string} that returns its answer (if
there is one) as a string.  An example call is:
\begin{verbatim}
rc = xsb_query_string_string("append(X,Y,[a,b,c]).",buff,";");
\end{verbatim}
The first argument is the period-terminated query string.  The second
argument is a variable string buffer in which the subroutine returns the answer
(if any.) The variable string data type {\tt VarString}  is explained in
Section~\ref{sec-varstring}. (Use the following function if you cannot 
declare a parameter of this type in your programming language.)  
The last argument is a string
provided by the caller, which is used to separate fields in the returned
answer.  For the example query, buff would be set to the string:
\begin{verbatim}
        [];[a,b,c]
\end{verbatim}
which is the first answer to the append query.  There are two fields of
this answer, corresponding to the two variables in the query, \verb|X|
and \verb|Y|.  The bindings of those variables make up the answer and
the individual fields are separated by the \verb|sep| string, here the
semicolon (\verb|;|).  Its returns are
just as for {\tt xsb\_query\_string}.  In the answer string, \ourprolog\
atoms are printed in their in their standard print form (without
quotes).  Complex terms are printed in a canonical form, with atoms
quoted if necessary, and lists produced in the normal list notation.

\ouritem{int xsb\_query\_string\_string\_b(char *query, char *buff,
int bufflen, int *anslen, char *sep)} \index{{\tt
xsb\_query\_string\_string\_b}} This function provides a lower-level
interface to {\tt xsb\_query\_string\_string} (not using the {\tt
VarString} type), which makes it easier for non-C callers (such as
Visual Basic or Delphi) to access XSB functionality.  The first and
last arguments are the same as in {\tt xsb\_query\_string\_string}.
The \verb|buff|, \verb|bufflen|, and \verb|anslen| parameters are used
to pass the answer (if any) back to the caller.  \verb|buff| is a
buffer provided by the caller in which the answer is returned.
\verb|bufflen| is the length of the buffer (\verb|buff|) and is
provided by the caller.  \verb|anslen| is returned by this routine and
is the length of the computed answer.  If that length is less than
\verb|bufflen|, then the answer is put in \verb|buff| (and
null-terminated).  If the answer is longer than will fit in the buffer
(with the null terminator), then the answer is not copied to the
buffer and 3 is returned.  In this case the caller can retrieve the
answer by providing a bigger buffer (of size greater than the returned
\verb|anslen|) in a call to {\tt xsb\_get\_last\_answer\_string}.

\ouritem{int xsb\_get\_last\_answer\_string(char *buff, int bufflen,
int *anslen)} \index{{\tt xsb\_get\_last\_answer\_string\_b}} This
function is used only when a call to {\tt
xsb\_query\_string\_string\_b} or to {\tt xsb\_next\_string\_b}
returns a 3, indicating that the buffer provided was not big enough to
contain the computed answer.  In that case the user may allocate a
larger buffer and then call this routine to retrieve the answer (that
had been saved.)  Only one answer is saved, so this routine must
called immediately after the failing call in order to get the right
answer.  The parameters are the same as the 2nd through 4th parameters
of {\tt xsb\_query\_string\_string\_b}.

\ouritem{int xsb\_next()} \index{{\tt xsb\_next}}
This routine is called after {\tt xsb\_query} (which must have
returned 0) to retrieve more answers.  It rebinds the query variables
in the term in register 1 and rebinds the argument fields of the {\tt
ret/n} answer term in register 2 to reflect the next answer to the
query.  It returns 0 if an answer is found, and returns 1 if there are
no more answers and no answer is returned. On a return of 1, the query
has been closed.  After a query is closed, another {\tt xsb\_command}
or {\tt xsb\_query} invocation can be made.

\ouritem{int xsb\_next\_string(VarString *buff,char *sep)} \index{{\tt
xsb\_next\_string}}  This routine is a variant of {\tt xsb\_next} that
returns its answer (if there is one) as a string.  Its treatment of
answers is just as {\tt xsb\_query\_string\_string}.  For example after
the example call 
\begin{verbatim}
rc = xsb_query_string_string("append(X,Y,[a,b,c]).",buff,";");
\end{verbatim}
which returns with buff set to 
\begin{verbatim}
        [];[a,b,c]
\end{verbatim}
Then a call:
\begin{verbatim}
rc = xsb_next_string(buff,";");
\end{verbatim}
returns with buff set to 
\begin{verbatim}
        [a];[b,c]
\end{verbatim}
the second answer to the indicated query. {\tt xsb\_next\_string}
returns codes just as {\tt xsb\_next}.

\ouritem{int xsb\_next\_string\_b(char *buff, int bufflen, int
*anslen, char *sep)} \index{{\tt xsb\_next\_string}} This routine is a
variant of {\tt xsb\_next\_string} that does not use the {\tt
VarString} type.  Its parameters are the same as the 2nd through 5th
parameters of {\tt xsb\_query\_string\_string\_b}.  The next answer to
the current query is returned in \verb|buff|, if there is enough
space.  If the buffer would overflow, this routine returns 3, and the
answer can be retrieved by providing a larger buffer in a call to {\tt
xsb\_get\_last\_answer\_string\_b}.  In any case, the length of the
answer is returned in \verb|anslen|.

\ouritem{int xsb\_close\_query()} \index{{\tt xsb\_close\_query}}
This routine closes a query, before all its answers have been
retrieved.  Since \ourprolog\ is (usually) a tuple-at-a-time system,
answers that are not retrieved are not computed.
It is an error to call {\tt xsb\_query} again without
first either retrieving all the answers to the previous query or
calling {\tt xsb\_close\_query} to close it.

\ouritem{int xsb\_close()} \index{{\tt xsb\_close}}
This routine closes the entire connection to \ourprolog .  After this,
no more calls can be made (including calls to {\tt xsb\_init}.)
\end{description}

\section{The Variable-length String Data Type}\label{sec-varstring}

\index{VarString} XSB uses variable-length strings to communicate with
certain C subroutines when the size of the output that needs to be passed
from the Prolog side to the C side is not known. Variable-length strings
adjust themselves depending on the size of the data they must hold and are
ideal for this situation. For instance, as we have seem the two subroutines
{\tt xsb\_query\_string\_string(query,buff,sep)} and {\tt
  xsb\_next\_string(buff,sep)} use the variable string data type, {\tt
  VarString}, for their second argument.  To use this data type, make sure
that
%%
\begin{verbatim}
#include "cinterf.h"  
\end{verbatim}
%%
appears at the top of the program file.  Variables of the {\tt VarString}
type are declared using a macro that must appear in the declaration section
of the program:
%%
\begin{verbatim}
XSB_StrDefine(buf);  
\end{verbatim}
%%
There is one important consideration concerning VarString with the
\emph{automatic} storage class: they must be
\emph{destroyed} on exit (see {\tt XSB\_StrDestroy}, below) from the procedure
that defines them, or else there will be a memory leak. 
It is not necessary to destroy static {\tt VarString}'s.

The public attributes of the type are {\tt int length} and {\tt char *string}.
Thus, {\tt buf.string} represents the actual contents of the buffer and
{\tt buf.length} is the length of that data. Although the length and the
contents of a {\tt VarString} string is readily accessible, the user {\bf must
not} modify these items directly. Instead, he should use the macros
provided for that purpose:
%%
\begin{itemize}
  \item {\tt XSB\_StrSet(VarString *vstr, char *str)}:~
    Assign the value of the regular null-terminated C string to the
    {\tt VarString} {\tt vstr}. The size of {\tt vstr} is adjusted
    automatically.
  \item {\tt XSB\_StrSetV(VarString *vstr1, VarString *vstr2)}:~
    Like {\tt XSB\_StrSet}, but the second argument is a variable-length
    string, not a regular C string.
  \item {\tt XSB\_StrAppend(VarString *vstr, char *str)}:~
    Append the null-terminated string {\tt str} to the {\tt VarString} {\tt
      vstr}. The size of {\tt vstr} is adjusted.
  \item {\tt XSB\_StrPrepend(VarString *vstr, char *str)}:~
    Like {\tt XSB\_StrAppend}, except that {\tt str} is prepended.
  \item {\tt XSB\_StrAppendV(VarString *vstr1, VarString *vstr2)}:~
    Like {\tt XSB\_StrAppend}, except that the second string is also a
    {\tt VarString}.
  \item {\tt XSB\_StrPrependV(VarString *vstr1, VarString *vstr2)}:~
    Like {\tt XSB\_StrAppendV}, except that the second string is prepended.
  \item {\tt XSB\_StrCompare(VarString *vstr1, VarString *vstr2)}:~
    Compares two {\tt VarString}. If the first one is lexicographically
    larger, then the result is positive; if the first string is smaller,
    than the result is negative; if the two strings have the same content
    ({\it i.e.}, {\tt vstr1->string} equals {\tt vstr2->string} then the
    result is zero.
  \item {\tt XSB\_StrCmp(VarString *vstr, char *str)}:~
    Like {\tt XSB\_StrCompare} but the second argument is a regular,
    null-terminated string.
  \item {\tt XSB\_StrAppendBlk(VarString *vstr, char *blk, int size)}:~
    This is like {\tt XSB\_StrAppend}, but the second argument is not assumed
    to be null-terminated. Instead, {\tt size} characters pointed to by
    {\tt blk} are appended to {\tt vstr}. The size of {\tt vstr} is
    adjusted, but the content is \emph{not} null terminated.
  \item {\tt XSB\_StrPrependBlk(VarString *vstr, char *blk, int size)}:~
    Like {\tt XSB\_StrPrepend}, but {\tt blk} is not assumed to point to a
    null-terminated string. Instead, {\tt size} characters from the region
    pointed to by {\tt blk} are prepended to {\tt vstr}.
  \item {\tt XSB\_StrNullTerminate(VarString *vstr)}:~
    Null-terminates the {\tt VarString}  string {\tt vstr}. This is used in
    conjunction with {\tt XSB\_StrAppendBlk}, because the latter does not
    null-terminate variable-length strings.
  \item {\tt XSB\_StrEnsureSize(VarString *vstr, int minsize)}:~
    Ensure that the string has room for at least {\tt minsize} bytes.
    This is a low-level routine, which is used to interface to procedures
    that do not use {\tt VarString} internally. If the string is larger
    than {\tt minsize}, the size might actually shrink to the nearest
    increment that is larger {\tt minsize}.
  \item {\tt XSB\_StrShrink(VarString *vstr, int increment)}:~ Shrink the
    size of {\tt vstr} to the minimum necessary to hold the data. {\tt
      increment} becomes the new increment by which {\tt vstr} is adjusted.
    Since {\tt VarString} is automatically shrunk by {\tt XSB\_StrSet}, it
    is rarely necessary to shrink a {\tt VarString} explicitly.  However,
    one might want to change the adustment increment using this macro (the
    default increment is 128).
  \item {\tt XSB\_StrDestroy(VarString *vstr)}:~
    Destroys a {\tt VarString}.  Explicit destruction is necessary for
    {\tt VarString}'s with the automatic storage class. Otherwise, memory
    leak is possible.
\end{itemize}
%%


\section{Passing Data into an XSB Module}

The previous chapter described the low-level XSB/C interface that supports
passing the data of arbitrary complexity between XSB and C. However, in
cases when data needs to be passed into an executable XSB module by the
main C program, the following higher-level interface should suffice.  (This
interface is actually implemented using macros that call the lower level
functions.)  These routines can be used to construct commands and queries
into \ourprolog 's register 1, which is necessary before calling {\tt
  xsb\_query()} or {\tt xsb\_command()}.


\begin{description}
\ouritem{void xsb\_make\_vars((int) N)} \index{{\tt xsb\_make\_vars}}
    {\tt xsb\_make\_vars} creates a return structure of arity {\tt N}
in Register 2.  So this routine may called before calling any of {\tt
xsb\_query}, {\tt xsb\_query\_string}, {\tt xsb\_command}, or {\tt
xsb\_command\_string} if parameters are to be set to be sent to the
goal.  It must be called before calling one of the {\tt
xsb\_set\_var\_*} routines can be called. {\tt N} must be the number
of variables in the query that is to be evaluated.

\ouritem{void xsb\_set\_var\_int((int) Val, (int) N)} 
\index{{\tt xsb\_set\_var\_int}}
    {\tt set\_and\_int} sets the {\tt N}$^{th}$ field in the return
structure to the integer value {\tt Val}.  It is used to set the value of
the {\tt N}$^{th}$ variable in a query before calling {\tt xsb\_query} or
{\tt xsb\_query\_string}.  When called in \ourprolog, the query will
have the {\tt N}$^{th}$ variable set to this value.

\ouritem{void xsb\_set\_var\_string((char *) Val, (int) N)} 
\index{{\tt xsb\_set\_var\_string}}
    {\tt set\_and\_string} sets the {\tt N}$^{th}$ field in the return
structure to the atom with name {\tt Val}.  It is used to set the
value of the {\tt N}$^{th}$ variable in a query before calling {\tt
xsb\_query} or {\tt xsb\_query\_string}.  When called in \ourprolog,
the query will have the {\tt N}$^{th}$ variable set to this value.

\ouritem{void xsb\_set\_var\_float((float) Val, (int) N)} 
\index{{\tt xsb\_set\_var\_float}}
    {\tt set\_and\_float} sets the {\tt N}$^{th}$ field in the return
structure to the floating point number with value {\tt Val}.  It is
used to set the value of the {\tt N}$^{th}$ variable in a query before
calling {\tt xsb\_query} or {\tt xsb\_query\_string}.  When called in
\ourprolog, the query will have the {\tt N}$^{th}$ variable set to this
value.

\ouritem{void xsb\_var\_int((int) N)} \index{{\tt xsb\_var\_int}}
    {\tt xsb\_var\_int} is called after {\tt xsb\_query} or {\tt
xsb\_query\_string} returns an answer.  It returns the value of the
{\tt N}$^{th}$ variable in the query as set in the returned answer.
This variable must have an integer value.

\ouritem{void xsb\_var\_string((int) N)} \index{{\tt xsb\_var\_string}}
    {\tt xsb\_var\_string} is called after {\tt xsb\_query} or {\tt
xsb\_query\_string} returns an answer.  It returns the value of the
{\tt N}$^{th}$ variable in the query as set in the returned answer.
This variable must have an atom value.

\ouritem{void xsb\_var\_float((int) N)} \index{{\tt xsb\_var\_float}}
    {\tt xsb\_var\_float} is called after {\tt xsb\_query} or {\tt
xsb\_query\_string} returns an answer.  It returns the value of the
{\tt N}$^{th}$ variable in the query as set in the returned answer.
This variable must have an floating point value.


\end{description}


\section{Creating an XSB Module that Can be Called from C}

To create an executable that includes calls to the above C functions,
these routines, and the \ourprolog\ routines that they call, must be
included in the link ({\tt ld}) step.

\paragraph{Unix instructions:}
You must link your C program, which should include the main procedure, with
the XSB object file located in
%%
\begin{verbatim}
 $XSB_DIR/config/<your-system-architecture>/saved.o/xsb.o  
\end{verbatim}
%%$
Your program should include the file {\tt cinterf.h} located in the {\tt
  XSB/emu} subdirectory, which defines the routines described earlier,
which you will need to use in order to talk to XSB.  It is therefore
recommended to compile your program with the option
\verb|-I$XSB_DIR/XSB/emu|.
%%$

The file {\tt \$XSB\_DIR/config/your-system-architecture/modMakefile} is a
makefile you can use to build your programs and link them with XSB.  It is
generated automatically and contains all the right settings for your
architecture, but you will have to fill in the name of your program, etc.

It is also possible to compile and link your program with XSB using XSB
itself as follows:
%%
\begin{verbatim}
:- xsb_configuration(compiler_flags,CFLAGS),
        xsb_configuration(loader_flags,LDFLAGS),
        xsb_configuration(config_dir,CONFDIR),
        xsb_configuration(emudir,EMUDIR),
        xsb_configuration(compiler,Compiler),
        str_cat(CONFDIR, '/saved.o/', ObjDir),
        write('Compiling myprog.c ... '),
        shell([Compiler, ' -I', EMUDIR, ' -c ', CFLAGS, ' myprog.c ']),
        shell([Compiler, ' -o ', './myprog ',
               ObjDir, 'xsb.o ', ' myprog.o ', LDFLAGS]),
        writeln(done).  
\end{verbatim}
%%
This works for every architecture and is often more convenient than using
the make files.

There are simple examples of C programs calling \ourprolog\ in the
{\tt \$XSB\_DIR/examples/c\_calling\_XSB} directory, in files {\tt cmain.c},
{\tt ctest.P}, {\tt cmain2.c}.

\paragraph{Windows instructions:}
To call XSB from C, you must build it as a DLL, which is done as follows:
%%
\begin{verbatim}
  cd $XSB_DIR\XSB\build
  makexsb_wind DLL="yes"
\end{verbatim}
%%$
The DLL, which you can call dynamically from your program is then found in 
%%
\[
 \tt
 \$XSB\_DIR\backslash config\backslash \mbox{\tt x86-pc-windows}\backslash
 bin \backslash xsb.dll
\]
%%
Since your program must include the file {\tt cinterf.h}, it is recommended
to compile it with the option \verb|/I$XSB_DIR\XSB\emu|.
%%$


%%% Local Variables: 
%%% mode: latex
%%% TeX-master: "manual2"
%%% End: 
