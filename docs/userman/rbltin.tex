%========================================================

\section{Asserting, Retracting, and Other Database Modifications} \label{sec:assert}

XSB provides an array of features for modifying the dynamic database.
As a default, using {\tt assert/1}, clauses can be asserted using
first-argument indexing in a manner that is now standard to Prolog
implementations.  However, a variety of other behaviors can be
specified using the (executable) directives {\tt index/3} and {\tt
  index/2}.  For instance, dynamic clauses can be declared to have
multiple or joint indexes, and this indexing can be either hash-based
as is typical in Prolog systems or based on {\em tries}.  No matter
what kind of indexing is used, space is dynamically allocated when a
new clause is asserted and, unless specified otherwise, released after
it is retracted.  Furthermore, the size of any index table expands
dynamically as clauses are asserted.

All dynamic predicates are compiled into SLG-WAM code, however the
manner of their compilation may differ, and the differences in
compilation affect the semantics for the predicate.  If a dynamic
predicate $P/n$ is given an indexing directive of {\tt trie}, clauses
for $P/n$ will be compiled using trie instructions; otherwise clauses
for $P/n$ will be compiled into SLG-WAM instructions along the lines
of static predicates. 

Consider first dynamic predicates that use any indexing other than
{\tt trie} -- including multiple or joint indices and star indexing.
XSB asserts WAM code for such clauses so that that the execution time
of dynamic code is similar to compiled code for unit and binary
clauses.  Furthermore, tabling can be used by explicitly declaring a
predicate to be both dynamic and tabled.  In \version{}, when the
clause of a dynamic predicatw is asserted as WAM code, the {\em
  ``immediate semantics''} rather than the ISO Semantics of
assert/retract~\cite{LiOk87}.  The immediate semantics allows assert
and retract to be fast and spatially efficient, but requires that
significant care must be taken when modifying the definition of a
predicate which is currently being executed.

\comment{
| This means that significant care must be taken when
| modifying the definition of a predicate which is currently being
| executed. Notice that this makes some operations difficult. For
| example, one might try to retract from dynamically asserted
| predicates, {\tt p/1} and {\tt q/1}, exactly their intersection, by
| issuing the following query:
| \begin{center} 
| {\tt :- p(X), q(X), retract(p(X)), retract(q(X)), fail.}
| \end{center}
| Neither {\tt retract/1} nor {\tt retractall/1} support this behavior,
| due to their techniques for space reclamation.  One alternative is to
| use {\tt findall/3} to collect the intersection first, before retracting.
| Another is to use the predicates {\tt retract\_nr/1} and {\tt
| reclaim\_space/1}, described below.  
}

If a dynamic predicate is given an indexing directive of {\tt trie},
clauses of the predicate are compiled (upon a call {\tt assert/1})
using trie instructions as described in \cite{RRSSW98}.  Creation of
trie-based dynamic code is significantly faster than creation of other
dynamic code, and execution time may also be faster.  However,
trie-based predicates can only be used for unit clauses where a
relation is viewed as a set, and where the order of the facts is not
important.

XSB does not at this time fully support dynamic predicates defined
within compiled code.  The only way to generate dynamic code is by
explicitly asserting it, or by using the standard predicate {\tt
  load\_dyn/1} to read clauses from a file and assert them (see the
section {\it Asserting Dynamic Code} in Volume 2).  There is a {\tt
  dynamic/1} predicate (see page~\pageref{dynamic/1}) that declares a
predicate within the system so that if the predicate is called when no
clauses are presently defining it, the call will quietly fail instead
of issuing an {\sf ``Undefined predicate''} error message.

\begin{description}

\ouritem{asserta(+Clause)}\index{\texttt{asserta/1}} 
%
If the index specification for the preicate is not {\tt trie}, this
predicate adds a dynamic clause, {\tt Clause}, to the database {\em
before} any other clauses for the same predicate currently in the
database.  If the index specification for the predicate is {\tt trie},
the clause is asserted arbitrarily within the trie, and a warning
message sent to {\tt stderr}.

Note that because of the precedence of {\tt :-/2}, asserting a clause
containing this operator requires an extra set of parentheses: {\tt
  assert((Head :- Body))}.

{\bf Error Cases}
\bi
\item 	{\tt Clause} is not instantiated
\bi
\item 	{\tt instantiation\_error}
\ei
%
\item 	{\tt Clause} is not a callable clause.
\bi
\item 	{\tt domain\_error(callable,Clause)}
\ei
%
\item 	{\tt Clause} has a head that is a static builtin
\bi
\item 	{\tt permission\_error(modify,builtin,Clause)}
\ei
\item 	{\tt Clause} has a head that is a static user predicate
\bi
\item 	{\tt permission\_error(modify,static,Clause)}
\ei
%
\ei

\ouritem{assertz(+Clause)}\index{\texttt{assertz/1}} 
%
If the index specification for the predicate is not {\tt trie}, this
predicate adds a dynamic clause, {\tt Clause}, to the database {\em
  after} any other clauses for the same predicate currently in the
database.  If the index specification for the predicate is {\tt trie},
the clause is asserted arbitrarily within the trie, and a warning
message sent to {\tt stderr}.  Error cases are as with {\tt
  asserta/1}.

Note that because of the precedence of {\tt :-/2}, asserting a clause
containing this operator requires an extra set of parentheses: {\tt
  assert((Head :- Body))}.

\ouritem{assert(+Clause)}\index{\texttt{assert/1}}
%
If the index specification for the predicate is not {\tt trie}, this
predicate adds a dynamic clause, {\tt Clause}, to the database {\em
  after} any other clauses for the same predicate currently in the
database (acting as {\tt assertz/1}).  If the index specification for
the predicate is {\tt trie}, the clause is asserted arbitrarily within
the trie.  Error cases are as with {\tt assertz/1}.

Note that because of the precedence of {\tt :-/2}, asserting a clause
containing this operator requires an extra set of parentheses: {\tt
  assert((Head :- Body))}.

\ouritem{assert(+Clause,+AorZandVar,+Index)}\index{\texttt{assert/3}}
This is a lower-level interface to (non-trie-indexed) assert.  It is
normally not needed except in one particular situation, when assert
aborts because it needs too many registers.  In this case, this
loswer-level assert may allow the offending clause to be correctly
asserted.

The default implementation of non-trie-indexed assert generates code
with a single pass through the asserted term.  Because of this, it
cannot know when it has encountered the final occurrence of a
variable, and thus it can never release (and thus re-use) registers
that are used to refer to variables.  Since there is a limit of 255
registers in the XSB virtual machine, asserting a clause with more
than this many distinct variables results in an error.  There is an
alternative implementation of assert that initially traverses the
clause to determine the number of occurrences of each variable and
thus allows better use of registers during code generation.

{\tt Clause} is the clause to assert. {\tt AorZandVar} is an integer
whose lower 2 bits are used: The low-order bit is 0 if the clause is
to be added as the first clause, and 1 if it is to be added as the
last clause. If the second bit (2) is on, then the clause is traversed
to count variable occurrences and so improve register allocation for
variables; if it is 0, the default one-pass code-generation is done.
So, for example, if {\tt AorZandVar} is 3, then the clause will be
asserted as the last one in the predicate and the better register
allocation will be used.  {\tt Index} indicates the argument(s) on
which to index.

\ouritem{retract(+Clause)}\index{\texttt{retract/1}} 
%
Removes through backtracking all clauses in the database that match
with {\tt Clause}.  {\tt Clause} must be of one of the forms: {\tt
  Head} or {\tt Head :- Body}.  Note, that because of the precedence
of {\tt :-/2}, using the second form requires an extra set of
parentheses: {\tt retract((Head :- Body))}.  

The technical details on space reclamation are as follows.  When
retract is called, a check is made to determine whether it is safe to
reclaim space for that clause.  Safety is ensured when:
\begin{itemize}
\item A check is made of the choice point stack indicating that no
  choice point will backtack into space that is being reclaimed; AND
\begin{itemize}
\item The predicate is thread-private; OR 
\item there is a single active thread
\end{itemize}
\item AND if the predicate is tabled, there is no incomplete table
     for that predicate.
\end{itemize}
If it is safe to reclaim space for the clause, space is reclaimed
immediately.  Otherwise the clause is marked so that its space may
later be reclaimed through garbage collection.  (See {\tt
  gc\_dynamic/1}).

{\bf Error Cases}
\bi
\item 	{\tt Clause} is not instantiated
\bi
\item 	{\tt instantiation\_error}
\ei
%
\item 	{\tt Clause} is not a callable clause.
\bi
\item 	{\tt domain\_error(callable,Clause)}
\ei
%
\item 	{\tt Clause} has a head that is a static builtin
\bi
\item 	{\tt permission\_error(modify,builtin,Clause)}
\ei
\item 	{\tt Clause} has a head that is a static user predicate
\bi
\item 	{\tt permission\_error(modify,static,Clause)}
\ei
%
\ei

\ouritem{retractall(+Head)}\index{\texttt{retractall/1}} removes every
clause in the database whose head matches with {\tt Head}.  The
predicate whose clauses have been retracted retains the {\tt dynamic}
property (contrast this behavior with that of predicates {\tt
  abolish/[1,2]} below).  Predicate {\tt retractall/1} is determinate
and always succeeds.  The term {\tt Head} is not further instantiated
by this call.  Conditions for space reclamation and error cases are as
with {\tt retract/1}.

\ouritem{abolish(+PredSpec)}\index{\texttt{abolish/1}} Removes all
information about the specified predicate. {\tt PredSpec} is of the
form {\tt Pred/Arity}.  Everything about the abolished predicate is
completely forgotten by the system (including the {\tt dynamic or
  static} property, whether the predicate is tabled, and whether the
predicate is thread-shared or thread-private)~\footnote{For
  compatability with older Prologs, there is also an {\tt abolish/2}
  which takes {\tt Pred} and {\tt Arity} as its two arguments.}.  Any
completed tables for the predicate are also removed.

It is an error to abolish a predicate when there is more than 1 active
thread, regardless of whether the predicate is thread-private or
thread-shared.  The reason for this is that, even if {\tt PredInd}
denotes a thread-private predicate, one thread may be making use of
{\tt PredInd} as another thread abolishes it.  {\tt abolish/1} throws
an error in such a case to prevent such a semantic inconsistency.
Similarly, if there is a non-completed table for {\tt PredInd}, an
error is thrown to prevent incompleteness in the tabled computation.

\compatability \version{} of XSB allows static predicates to be
abolished and their space reclaimed.  Such space is reclaimed
immediately, and unlike the case for abolished static code,  no check
is made to ensure that XSB's choice point stack is free of choice
points for the abolished static predicate.  Abolishing static code is
thus dangerous and should be avoided unless a user is certain it is
safe to use.

{\bf Error Cases}
\bi
\item 	{\tt PredInd}, {\tt Pred} or {\tt Arity} is not instantiated
\bi
\item 	{\tt instantiation\_error}
\ei
%
\item 	{\tt Arity} is not in the range 0..255 ({\tt max\_arity})
\bi
\item 	{\tt domain\_error(arity\_indicator,Arity)}
\ei
%
\item 	{\tt PredInd} indicates a static builtin
\bi
\item 	{\tt permission\_error(modify,builtin,Predind)}
\ei
%
\item {\tt abolish/1} is called when there is more than 1 active thread.
\bi
\item {\tt misc\_error}
\ei
%
\item {\tt PredInd} has a non-completed table in the current thread.
\bi
\item {\tt table\_error}
\ei
%
\item There are active backtrack points to a (dynamic) clause for {\tt
  PredInd}~\footnote{XSB throws an error in this case because garbage
  collection for abolished predicates has not been implemented
  (unlike for retract(all) and various table abolishes).  Besides, you
  shouldn't be abolishing a predicate that you could backtrack into.
  What were you thinking?}. 
\bi
\item {\tt misc\_error}
\ei
%
\ei

\ouritem{clause(+Head,?Body)}\index{\texttt{clause/2}}
Returns through backtracking all dynamic clauses in the database whose head
matches {\tt Head} and Body matches {\tt Body}.  For facts the {\tt Body} is
{\tt true}.  

{\bf Error Cases}
\bi
\item 	{\tt Head} or {\tt Body} is not instantiated
\bi
\item 	{\tt instantiation\_error}
\ei
%
\item 	{\tt Head} (or {\tt Body}) is not a callable clause.
\bi
\item 	{\tt domain\_error(callable,Head)}
\ei
%
\item 	{\tt Head} is a static builtin
\bi
\item 	{\tt permission\_error(access,builtin,Head)}
\ei
\item 	{\tt Head} is a static user predicate
\bi
\item 	{\tt permission\_error(access,static,Clause)}
\ei
%
\ei

\ouritem{gc\_dynamic(-N)}\index{\texttt{gc\_dynamic/1}} 
%
Invokes the garbage collector for dynamic clauses that have been
retracted, or whose predicate has been abolished.  When called with
more than 1 active thread, {\tt gc\_dynamic/1} will always perform
garbage collection for that thread's private retracted clauses;
however in \version{}, it will only perform garbage collection for
retracted thread-shared clauses if there is a single active thread.
{\tt N} is the number or shared and/or private frames left to be
collected -- if {\tt N} is unified to 0, then all possible garbage
collecting has been performed.  {\tt N} is unified to -1 garbage
collection was not attempted (due to multiple active threads).

By default, {\tt gc\_dynamic/1} is called automatically at the top
level of the XSB interpreter, when abolishing a predicate, and when
calling retractall for an ``open'' term containing no variable
bindings.

\comment{
Now obsolete, what with GC.

| \ouritem{retract\_nr(+Clause)}\index{\texttt{retract\_nr/1}}
| Performs just as {\tt retract/1} does, except that it does not reclaim the
| space used by the retracted clause. This is provided to allow programmers
| to modify dynamic clauses while executing them (a practice that is 
| discouraged.) For example, to retract an intersection, as described above,
| one could do:
| \begin{center}
| {\tt :- p(X), q(X), retract\_nr(p(X)), retract\_nr(q(X)), fail.}
| \end{center}
| In order to reclaim space after using {\tt retract\_nr/1}, see 
| {\tt reclaim\_space/1} below.  Predicate {\tt retract\_nr/1}
| is not a standard predicate and must be imported from module {\tt assert}.
| {\tt retract\_nr/1} is provided for (partial)
| compatibility with the {\tt retract/1} predicate of SB-Prolog.

%In this case, the {\tt retract\_nr/1} deletes the clauses, but will not
%drastically modify the clause data structure,
%and this code will execute. Of course, space is not reclaimed for further
%use (but see {\tt reclaim\_space/1} below).  Predicate {\tt retract\_nr/1}
%is not a standard predicate but must be imported from module {\tt assert}.
%The use of this predicate is discouraged; it is provided for (partial)
%compatibility with the {\tt retract/1} predicate provided by SB-Prolog that
%did not reclaim space.

| \ouritem{reclaim\_space(+Head)}\index{\texttt{reclaim\_space/1}} Runs
| through the dynamic code for the predicate indicated by {\tt Head},
| and reclaims space for any clauses that have been deleted from that
| predicate by {\tt retract\_nr/1}.  This cannot safely be used when
| execution is still within some invocation of the specified predicate,
| or will backtrack into such a scope.  To complete our example of
| retracting the intersection of dynamic predicates:
| \begin{center}
| {\tt :- p(X), q(X), retract\_nr(p(X)), retract\_nr(q(X)), fail\ ;\\
|      reclaim\_space(p(\_)), reclaim\_space(q(\_)).}
| \end{center}
| would do the trick. Notice that the {\tt reclaim\_space} calls 
| must be made after execution has completely failed
| out of choice points for {\tt q(X)} and {\tt p(X)}.  Predicate 
| {\tt reclaim\_space/1} is not
| standard but must be imported from module {\tt assert}.
| As with {\tt retract\_nr}, the use of this predicate is discouraged; 
| it is provided for (partial) compatibility with SB-Prolog.
}
\comment{ 
TLS: I dont think we need hashtable size given that we use dynamic
hashing.  We still are supporting it, but we dont need to encourage it
\vspace{-.35in}
\ouritem{index(+PredSpec, +IndexSpec, +HashTableSize)}\index{\texttt{index/3}}

If {\tt index/3} is used, then the predicate indicated by {\tt
PredSpec} is declared to be indexed according to {\tt IndexSpec} using
initial hash table of sizes of {\tt HashTableSize}.  After this
directive is given, all clauses asserted to {\tt PredSpec} will be so
indexed.  

For dynamic predicates,
{\tt index/2} is an executable directive that can be used to specify
the indexing of a predicate before clauses to that predicate have been
asserted.  


As an example, one could specify: {\tt index(p/5,[1+2,1,4],300)}.
After clauses are asserted to it, a call to {\tt p/5} would first
check to see if both the first and second arguments are nonvariable
and if so, use an index based on both those values. Otherwise, it
would see if the second argument is nonvariable and if so, use an
index based on it. Otherwise, it would see if the fourth argument is
nonvariable and if so use an index based on it. As a last resort, it would
use no index but backtrack through all the clauses in the predicate.
(Notice that it may well make sense to include an argument that 
appears in a joint specification later alone, as 1 in this example,
but it never makes sense foring the single argument to appear earlier. In
that case the joint index would never be used.)

}

\ouritem{index(+PredSpec, +IndexSpec)}\index{\texttt{index/2}}
\label{index_dynamic} \index{indexing!dynamic predicates}

In general, XSB supports hash-based indexing on various arguments or
combinations of arguments, along with trie-based indexing.  The
availability of various kinds of indexing depends on whether code is
static (e.g. compiled) or dynamic (e.g. asserted or loaded with {\tt
load\_dyn/1}).  The executable directive {\tt index/2} does {\em
not\/} re-index an already existing predicate but takes effect only
for clauses asserted after the directive has been given.  Index
directives can be given to the compiler as part of source code or
executed during program execution (analogously to {\tt op/3}).

\begin{itemize}
\item {\em Hash-based Indexing} 
\begin{itemize}
\item {\em Static Predicates}\ 
In this case {\tt IndexSpec} must be a non-negative integer which
indicates the argument on which an index is to be constructed.  If
{\tt IndexSpec} is~0, then no index is kept (possibly an efficient
strategy for predicates with only one or two clauses.)
\item {\em Dynamic Predicates}
For a dynamic predicate, (to which no clauses have yet been asserted),
{\tt IndexSpec} is either an {\tt IndexElt} or a list of {\tt
IndexElt}s.  Each {\tt IndexElt} defines an index and specifies an argument or group of
arguments that make up the search key of the index.  An argument is
indicated by a small integer ({\tt ArgNo}) indicating the argument number (starting
from 1) to use in the index.  An argument indicator may optionally be annotated
as {\tt *(ArgNo)}.  The argument number alone indicates that only the
main functor symbol of the indicated argument will participate in the
index.  When annotated with the asterisk, the first 5 fields in the
corresponding term (in a depth-first traversal of the term) will be
used in the index.  If there are fewer than 5, they all will be used.
If any of the first 5 is a variable, then the index cannot be used.

An index is usually on a single argument, in which case the {\tt
IndexElt} consists of a single argument indicator.  However, sometimes
one wants an index on multiple arguments, meaning that the values of
several arguments are to be concatenated to create the search key of
the index.  Such a multi-argument (or joint) index is indicated by
using an {\tt IndexElt} that has up to 3 argument indicators separated
by the {\tt +} (plus) operator, e.g., {\tt 1+2+3}.

For example, {\tt index(p/3,[2,1])} indicates that clauses asserted to
the predicate {\tt p/3} should be indexed on both the second and the
first argument.  Subsequent calls to {\tt p/3} will first check to see
if the second argument is nonvariable, and if so use that index, using
the main functor symbol of that argument. If the second argument is
variable, it will next check to see if the first argument is
nonvariable and if so, use that index, built on the main functor
symbol of the first argument.  

{\tt index(p/3,[*(2),1])} would result in similar behavior as the
previous example, but the first index to be tried (on the second
argument) would be built using more of the term value in that second
argument position (not just the main functor symbol.)

As another example, one could specify: {\tt index(p/5,[1+2,1,4])}.
After clauses are asserted to it, a call to {\tt p/5} would first
check to see if both the first and second arguments are nonvariable
and if so, use an index based on both those values. Otherwise, it
would see if the first argument is nonvariable and if so, use an index
based on it. Otherwise, it would see if the fourth argument is
nonvariable and if so use an index based on it. As a last resort, it
would use no index but backtrack through all the clauses in the
predicate.  In all these cases, the indexes are built using only the
main functor symbol in the indicated argument position. (Notice that
it may well make sense to include an argument that appears in a joint
specification later alone, as 1 in this example, but it never makes
sense forcing the single argument to appear earlier. In that case the
joint index would never be used.)

If we want to use similar indexing on {\tt p/5} of the previous
example, except say argument 1 takes on complex term values and we
want to index on more of those terms, we might specify the index as
{\tt index(p/5,[*(1)+2,*(1),4])}.

\end{itemize}

\item {\em Trie-based Indexing}
The executable declaration {\tt index(Predspec,trie)} causes clauses
for {\tt Predspec} to be asserted using tries (see \cite{RRSSW98},
which is available through the XSB web page).  The name trie indexing
is something of a misnomer since the trie itself both indexes the term
and represents it.  In XSB, the above trie index is formed using a
left-to-right traversal of the unit clauses.  These indexes can be
very effective if discriminating information lies deep within a term,
and if there is sharing of left-prefixes of a term, can reduce the
space needed to represent terms.  Furthermore, asserting a unit clause
as a trie is much faster than asserting it using default WAM code.
\comment{
Trie indexing can be used with alternative or joint indexes.  For the
directive {\tt index(p/3,[2,1],trie)}, two trie indices would be
formed for {\tt p/3}: one that traversed arguments in order {\em
2,1,3} and another that traversed arguments in order {\em 1,2,3}.  The
actual implementation seeks to reduce redundant storage of code for
alternative indices.
}

Despite these advantages, representing terms as tries leads to
semantic differences from asserted code, of which the user should be
aware.  First, the order of clauses within a trie is arbitrary: using
{\tt asserta/1} or {\tt assertz} for a predicate currently using trie
indexing will give the same behavior as using {\tt assert}.  Also, the
current version of XSB only allows trie indexing for unit clauses.

Trie-based indexing is available only for dynamic predicates.
\end{itemize}

\ouritem{dynamic(+PredSpec)}\index{\texttt{dynamic/1}}\label{dynamic/1}
is an executable predicate which converts a predicate specified as
(Predicate/Arity) to a dynamic predicate. If {\tt Predicate} is not
previously defined, it will be initialized to empty (so that calls to
it quietly fail, instead of issuing {\sf ``Undefined predicate''}
error messages.) If the predicate is previously defined and dynamic,
{\tt dynamic/1} is a noop. If previously defined as compiled, {\tt
Predicate} will be converted to dynamic, which means that clauses can
be added, although the compiled portion cannot be manipulated.  Note
that {\tt dynamic/1} can be used like a compiler directive, since it
will be passed through to be executed when the module is loaded. Note,
however, that the semantics is different from that of the standard
\cite{ISO-Prolog} when the file contains clauses defining the
so-specified predicate.

\ouritem{table(+PredSpec)}\index{\texttt{table/1}}
is an executable predicate, where PredSpec is a predicate
specification for a dynamic predicate. (This is also a compiler
directive when {\tt PredSpec} specifies a compiled predicate. See the
section of this manual on compiler directives.) This predicate
declares a dynamic predicate to be tabled. It simply saves information
to be used at the time of assert and so it must be called before any
clauses are asserted into the specified predicate in order for the
predicate to be tabled.

\end{description}

\subsection{Reading Dynamic Code from Files} \label{sec:LoadDyn}
%===============================================

Several builtin predicates are available that can assert the contents
of a file into XSB's databse.  These predicates are useful when code
needs to be dynamic, or when the they contain a large number of
clauses or facts.  Configured properly, files containing millions of
facts can be read and asserted into memory in under a minute, making
XSB suitable for certain kinds of in-memory database
operations~\footnote{In \version{}, loading code dynamically can also
be useful when the clauses contain atoms whose length is more than 255
that cannot be handled by the XSB compiler.}.  

Each of the predicates in this section allow loading from files with
proper prolog extensions, and makes use of the XSB library paths.  See
Sections~\ref{LibPath} and~\ref{sec:filenames} for details.

\begin{description}
\ouritem{load\_dyn(+FileName)}
\index{\texttt{load\_dyn/1}}\label{load_dyn/1}
%\predindex{load\_dyn/1~(L)}
    Asserts the contents of file {\tt FileName} into the database.
    All existing clauses of the predicates in the file that already
    appear in the database, are retracted, unless there is a {\tt
    multifile/1} declaration for them.  An indexing declaration of a
    predcate {\tt p/n} in {\tt FileName} will be observed as long as
    the declarations occur before the first clause of {\tt p/n}.  file
    will be observed as Clauses in {\tt FileName} must be in a format
    that {\tt read/1} will process.  So, for example, operators are
    permitted.  As usual, clauses of predicates are not retracted if
    they are compiled instead of dynamically asserted.  All predicates
    are loaded into {\tt usermod}.  Module declarations such as {\tt
    :- export} are ignored and a warning is issued.

    Dynamically loaded files can be filtered through the XSB preprocessor.
    To do this, put the following in the source file: 
    %%
    \begin{verbatim}
    :- compiler_options([xpp_on]).      
    \end{verbatim}

    %%
    Of course, the name \verb|compiler_options| might seem like a misnomer
    here (since the file is not being compiled), but it is convenient to
    use the same directive both for compiling and loading, in case the same
    source file is used both ways.

{\bf Error Cases}
\bi
\item 	{\tt FileName} is a variable
 \bi
 \item 	{\tt instantiation\_error}
 \ei
\item 	{\tt FileName} is not an atom.
 \bi
 \item 	{\tt type\_error(atom,Filename)}
 \ei
\item {\tt FileName} has been loaded previously in the session {\em and}
  there is more than one active thread.  
 \bi
 \item 	{\tt misc\_error}
 \ei
%
\ei

\ouritem{load\_dyn(+FileName,+Dir)}
\index{\texttt{load\_dyn/2}}\label{load_dyn/2}
    Asserts the contents of file {\tt FileName} into the database.
    {\tt Dir} indicates whether {\tt assertz} or {\tt asserta} is to
    be used.  If {\tt Dir} is {\tt z}, then {\tt assertz} is used and
    the behavior of {\tt load\_dyn(FileName)} is obtained.  If {\tt
    Dir} is {\tt a}, then {\tt asserta} is used to add the clauses to
    the database, and clauses will be in the reverse order of their
    appearance in the input file.  {\tt asserta} is faster than {\tt
    assertz} for predicates such that their indexing and data result
    in many hash collisions.  {\tt Dir} is ignored for facts in {\tt
    FileName} that are trie-indexed.

{\bf Error Cases}
\bi
\item 	{\tt FileName} is a variable
\bi
\item 	{\tt instantiation\_error}
\ei
\item 	{\tt FileName} is not an atom:
\bi
\item 	{\tt type\_error(atom,FileName)}
\ei
%
\item 	{\tt Dir} is not equal to {\tt a} or {\tt z}~\footnote{For
backward compatability, {\tt 0} and {\tt 1} are also allowed.}: 
\bi
\item 	{\tt domain\_error(a\_or\_z,Dir)}
\ei
%
\item {\tt FileName} has been loaded previously in the session {\em and}
  there is more than one active thread.  
\bi
\item 	{\tt misc\_error}
\ei
\ei

\ouritem{load\_dync(+FileName)}\index{\texttt{load\_dync/1}}
\index{canonical forma}
    Acts as {\tt load\_dyn/1}, but assumes that facts are in
    ``canonical'' format and is much faster as a result.  In XSB, a
    term is in canonical format if it does not use any operators other
    than list notation and comma-list notation.  This is the format
    produced by the predicate {\tt write\_canonical/1}. (See {\tt
    cvt\_canonical/2} to convert a file from the usual {\tt read/1}
    format to {\tt read\_canonical} format.)  As usual, clauses of
    predicates are not retracted if they are compiled instead of
    dynamically asserted. All predicates are loaded into {\tt
    usermod}.  {\tt :- export} declarations are ignored and a warning
    is issued.

    Notice that this predicate can be used to load files of Datalog
    facts (since they will be in canonical format).  This predicate is
    significantly faster than {\tt load\_dyn/1} and should be used
    when speed is important.  (See {\tt load\_dync/2} below for
    further efficiency considerations.)  A file that is to be
    dynamically loaded often but not often modified by hand should be
    loaded with this predicate. 

    As with \verb|load_dyn/1|, the source file can be filtered through the C
    preprocessor. However, since all clauses in such a file must be in
    canonical form, the \verb|compiler_options/1| directive should look as
    follows:
    %%
    \begin{verbatim}
     :-(compiler_options('.'(xpp_on,[]))).      
    \end{verbatim}
    %%

{\bf Error Cases}
\bi
\item 	{\tt FileName} is a variable
\bi
\item 	{\tt instantiation\_error}
\ei
\item 	{\tt FileName} is not an atom.
\bi
\item 	{\tt type\_error(atom,FileName)}
\ei
%
\item {\tt FileName} has been loaded previously in the session {\em and}
  there is more than one active thread.  
\bi
\item 	{\tt misc\_error}
\ei
\ei

\ournewitem{load\_dync(+FileName,+Dir)}{consult}
\index{\texttt{load\_dync/2}}\label{load_dync/2}
    Acts as {\tt load\_dyn/2}, but assumes that facts are in
    ``canonical'' format.  {\tt Dir} is ignored for trie-asserted
    code, but otherwise indicates whether {\tt assertz} or {\tt
    asserta} is to be used.  If {\tt Dir} is {\tt z}, then {\tt
    assertz} is used and the exact behavior of {\tt
    load\_dync(FileName)} is obtained.  If {\tt Dir} is {\tt a}, then
    {\tt asserta} is used to add the clauses to the database, and
    clauses will end up in the reverse order of their appearance in
    the input file.  

    Setting {\tt Dir} to {\tt a} for non trie-asserted code can
    sometimes be {\em much} faster than the default of {\tt z}.  The
    reason has to do with how indexes on dynamic code are represented.
    Indexes use hash tables with bucket chains.  No pointers are kept
    to the ends of bucket chains, so when adding a new clause to the
    end of a bucket (as in {\tt assertz}), the entire chain must be
    run.  Notice that in the limiting case of only one populated
    bucket (e.g., when all clauses have the same index term), this
    makes assertz-ing a sequence of clauses quadratic.  However, when
    using {\tt asserta}, the new clause is added to the beginning of
    its hash bucket, and this can be done in constant time, resulting
    in linear behavior for asserta-ing a sequence of clauses.

{\bf Error Cases}
\bi
\item 	{\tt FileName} is a variable
\bi
\item 	{\tt instantiation\_error}
\ei
\item 	{\tt FileName} is not an atom:
\bi
\item 	{\tt type\_error(atom,FileName)}
\ei
%
\item 	{\tt Dir} is not instantiated to {\tt a} or {\tt z}~\footnote{For
backward compatability, {\tt 0} and {\tt 1} are also allowed.}: 
\bi
\item 	{\tt domain\_error(a\_or\_z,Dir)}
\ei
%
\item {\tt FileName} has been loaded previously in the session {\em and}
  there is more than one active thread.  
\bi
\item 	{\tt misc\_error}
\ei
\ei

\ouritem{ensure\_loaded(+FileName,+Action)}
\index{\texttt{ensure\_loaded/2}}
%
This predicate does nothing if {\tt FileName} has been loaded or
consulted into XSB, and has not changed since it was loaded or
consulted.  Otherwise
%
\bi
\item If {\tt Action} is instantiated to {\tt dyn} the behavior is as
{\tt load\_dyn/1} (or {\tt load\_dyn(FileName,z)}).

\item If {\tt Action} is instantiated to {\tt dyna} the behavior is as
{\tt load\_dyn(FileName,a)}.

\item If {\tt Action} is instantiated to {\tt dync} the behavior is as
{\tt load\_dync/1} (or {\tt load\_dync(FileName,z)}).

\item If {\tt Action} is instantiated to {\tt dynca} the behavior is as
{\tt load\_dync(FileName,a)}.

\item If {\tt Action} is instantiated to {\tt consult}, {\tt FileName}
is consulted.

\ei

{\bf Error Cases}
\bi
\item 	{\tt FileName} is not instantiated:
\bi
\item 	{\tt instantiation\_error}
\ei
\item 	{\tt FileName} is not an atom:
\bi
\item 	{\tt type\_error(atom,FileName)}
\ei
%
\item 	{\tt Action} is not a valid load action as described above
\bi
\item 	{\tt domain\_error(loadAction,Action)}
\ei
%
\ei

\ournewitem{cvt\_canonical(+FileName1,+FileName2)}{consult}
\index{\texttt{cvt\_canonical/2}}
    Converts a file from standard term format to ``canonical'' format.
    The input file name is {\tt FileName1}; the converted file is put in
    {\tt FileName2}.  This predicate can be used to convert a file in
    standard Prolog format to one loadable by {\tt load\_dync/1}.
\end{description}

%----------------------------------------------------------------------

\index{transaction logic}
\subsection{The {\tt storage} Module: Associative Arrays and Backtrackable Updates}
\label{storage module}

XSB provides a high-level interface that allows the creation of
``objects'' that efficiently manage the storage of facts or of
associations between keys and values.  Of course, facts and
associative arrays can be easily managed in Prolog itself, but the
{\tt storage} module is highly efficient and supports the semantics of
backtrackable updates as defined by Transaction logic \cite{BoKi94} in
addition to immediate updates.  The semantics of backtrackable updates
means that an update made by the storage module may is provisional
until the update is committed.  Otherwise, if a subgoal calling the
update fails, the change is undone. The commit itself may be made
either by the predicate {\tt storage\_commit/1}, or less cleanly by
cutting over the update itself.

A storage object $O$ is referred to by a name, which must be a
Prolog atom.  $O$ can be associated either with a set of facts or a
set of \emph{key-value pairs}.  Within a given storage object each key
is associated with a unique value: however since keys and values can
be arbitrary Prolog terms, this constraint need not be a practical
restriction.  A storage object $O$ is created on demand, simply by
calling (a backtrackable or non-backtrackable) update predicate that
refers to $O$.  However to reclaim $O$'s space within a running
thread, the predicate {\tt storage\_reclaim\_space/1} must be called.
Backtackable non-backtrackable updates can be made to the same storage
object, although doing so may not always be a good programming practice.

If multiple threads are used, each storage object is private to a
thread, and space for a storage object is reclaimed upon a thread's
exit.  Thread-shared storage objects may be supported in future
versions.

All the predicates described in this section must be imported from
module {\tt storage}.

\subsubsection{Non-backtrackable Storage}

\begin{description}
\ouritem{storage\_insert\_keypair(+StorageName,+Key, +Value, ?Inserted)}\index{\texttt{storage\_insert\_keypair/4}}
%%
Insert the given Key-Value pair into {\tt StorageName}.  If the pair
is new, then {\tt Inserted} unifies with {\tt 1}. If the pair is
already in {\tt StorageName}, then {\tt Inserted} unifies with {\tt
  0}. If {\tt StorageName} already contains a pair with the given key
that is associated with a \emph{different} value, then {\tt Inserted}
unifies with {\tt -1}.  The first argument, {\tt StorageName}, must be
an atom naming the storage to be used. Different names denote
different storages.  In all cases the predicate succeeds.

\ouritem{storage\_delete\_keypair(+StorageName, +Key, ?Deleted)}
\index{\texttt{storage\_delete\_keypair/3}}
%%
Delete the key-value pair with the given key from {\tt
  StorageName}. If the pair was in {\tt StorageName} then {\tt
  Deleted} unifies with {\tt 1}.  If it was \emph{not} in {\tt
  StorageName}s then {\tt Deleted} unifies with {\tt 0}.  The first
argument, {\tt StorageName}, must be an atom naming the storage object
to be used. Different names denote different storages.  In both cases
the predicate succeeds.

\ouritem{storage\_find\_keypair(+StorageName, +Key, ?Value)}
\index{\texttt{storage\_find\_keypair/3}}
%%
If {\tt StorageName} has a key pair with the given key, then {\tt Value} unifies
with the value stored in {\tt StorageName}. If no such pair exists in the
database, then the goal fails.

Note that this predicate works with non-backtrackable associative arrays
described above as well as with the backtrackable ones, described below.

\ouritem{storage\_insert\_fact(+StorageName, +Fact, ?Inserted)}\index{\texttt{storage\_insert\_fact/3}}
Similar to keypair insertion, but this primitive inserts facts rather than
key pairs.

\ouritem{storage\_delete\_fact(+StorageName, +Fact, ?Inserted)}\index{\texttt{storage\_delete\_fact/3}}
Similar to keypair deletion, but this primitive deletes facts rather than
key pairs.

\ouritem{storage\_find\_fact(+StorageName, +Fact)}\index{\texttt{storage\_find\_fact/2}}
Similar to keypair finding, but this primitive finds facts facts rather than
key pairs.
%%
\end{description}

\subsubsection{Backtrackable Updates}
\label{backtrackable update}

\begin{description}
\ouritem{storage\_insert\_keypair\_bt(+StorageName, +Key, +Value, ?Inserted)}
\index{\texttt{storage\_insert\_keypair\_bt/4}} 
%%
A call to this predicate inserts a key pair into {\tt StorageName} as
does {\tt storage\_insert\_keypair/4}, and the key-value pair may be
queried via {\tt storage\_find\_keypair/3}, just as with the
non-backtrackable updates described above.  In addition, the key-value
pair can be removed from {\tt StorageName} by explicit deletion.
However, the key pair will be removed from {\tt StorageName} upon
failing over the insertion goal {\em unless} a commit is made to {\tt
  StorageName} through the goal {\tt storage\_commit(StorageName)}.
The exact semantics is defined by Transaction Logic \cite{BoKi94}.

Note it is the update itself that is backtrackable, not the key-value
pair.  Hence, a key-pair may be (provisionally) inserted by a
backtrackable update and deleted by a non-backtrackable update, or
inserted by a non-backtrackable update and (provisionally) deleted by
a backtrackable update.  Of course, whether such a mixture makes sense
would depend on a given application.

\ouritem{storage\_delete\_keypair\_bt(+StorageName, +Key, ?Deleted)}
\index{\texttt{storage\_delete\_keypair\_bt/3}}
%%
Like {\tt storage\_delete\_keypair/3}, but backtrackable as described
for {\tt storage\_insert\_keypair\_bt/4}.

\ouritem{storage\_insert\_fact\_bt(+StorageName, +Goal)} \index{\texttt{storage\_insert\_fact\_bt/2}}
%%
Like {\tt storage\_insert\_fact/2}, but backtrackable.

\ouritem{storage\_delete\_fact\_bt(+StorageName, +Goal)} 
\index{\texttt{storage\_delete\_fact\_bt/2}}
%%
This is a backtrackable version of {\tt storage\_delete\_fact/2}.

\ouritem{storage\_commit(+StorageName)} \index{\texttt{storage\_commit/1}}
%%
Commits to {\tt StorageName} any backtrackable updates since the last
commit, or since initialization if no commit has been made to {\tt
  StorageName}.  If {\tt StorageName} does not exist, the predicate
silently fails.
%%
\end{description}
%%
\subsubsection{Reclaiming Space}
%%
\begin{description}
\ouritem{storage\_reclaim\_space(+StorageName)} \index{\texttt{storage\_reclaim\_space/1}}
%%
This is similar to {\tt reclaim\_space/1} for {\tt assert} and {\tt
  retract}, but it is used for storage managed by the primitives defined in
the {\tt storage} module. As with {\tt reclaim\_space/1}, this goal is
typically called just before returning to the top level.
%%
\end{description}



%%% Local Variables: 
%%% mode: latex
%%% TeX-master: "manual1"
%%% End: 
