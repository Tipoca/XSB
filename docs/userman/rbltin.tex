%========================================================

\section{Asserting, Retracting, and Other Database Modifications} \label{sec:assert}

XSB provides an array of features for modifying the dynamic database.
Using {\tt assert/1}, clauses can be asserted using first-argument
indexing in a manner that is now standard to Prolog implementations.
While this is the default behavior for XSB, other behavior can be
specified using the (executable) directives {\tt index/3} and {\tt
index/2}.  For instance, dynamic clauses can be declared to have
multiple or joint indexes, while this indexing can be either
hash-based as is typical in Prolog systems or based on {\em tries}.
No matter what kind of indexing is used, space is dynamically
allocated when a new clause is asserted and, unless specified
otherwise, released when it is retracted.  Furthermore, the size of
any index table expands dynamically as clauses are asserted.

Consider first dynamic predicates that use traditional hash-based
indexing.  XSB asserts WAM code for such clauses, leading to execution
times similar to compiled code for unit and binary clauses.
Furthermore, tabling can be used with a dynamic predicate by
explicitly declaring a predicate to be both dynamic and tabled.  For
clauses that are asserted as WAM code, the {\em ``immediate
semantics''} of dynamic predicates is used, not the so-called {\em
``logical semantics''} of assert/retract \cite{LiOk87}. This means
that significant care must be taken when modifying the definition of a
predicate which is currently being executed. Notice that this makes
some operations difficult. For example, one might try to retract from
dynamically asserted predicates, {\tt p/1} and {\tt q/1}, exactly
their intersection, by issuing the following query:
\begin{center} 
{\tt :- p(X), q(X), retract(p(X)), retract(q(X)), fail.}
\end{center}
Neither {\tt retract/1} nor {\tt retractall/1} support this behavior,
due to their techniques for space reclamation.  One alternative is to
use {\tt findall/3} to collect the intersection first, before retracting.
Another is to use the predicates {\tt retract\_nr/1} and {\tt
reclaim\_space/1}, described below.  

Asserting clauses as WAM code might be considerably slow for some
applications.  To remedy this, XSB provides an alternative to {\tt
assert/1} which implements assert's functionality using the trie-based
tabling data structures \cite{RRSSW98}.  Though trie-based dynamic
code can be created (and usually executed) significantly faster than
using {\tt assert/1}, users of the following predicates should be
aware that trie-based assert can be used only for unit clauses where a
relation is viewed as a set, and where the order of the facts is not
important.

XSB does not at this time fully support dynamic predicates defined
within compiled code.  The only way to generate dynamic code is by
explicitly asserting it, or by using the standard predicate {\tt
load\_dyn/1} to read clauses from a file and assert them (see
the section {\it Asserting Dynamic Code} in Volume 2).  There is a
{\tt dynamic/1} predicate (see 
page~\pageref{dynamic/1}) that declares a predicate within the system
so that if the predicate is called when no clauses are presently
defining it, the call will quietly fail instead of issuing an {\sf
``Undefined predicate''} error message.

\begin{description}

\ouritem{assert(+Clause)}\index{\texttt{assert/1}}
adds a dynamic clause, {\tt Clause}, to the database. {\tt Clause}
must be of one of the forms: {\tt Head} or {\tt Head :- Body}. Note
that because of the precedence of {\tt :-/2}, using the second form
requires an extra set of parentheses: {\tt assert((Head :- Body))}.
Default: first-argument indexing.

\ouritem{asserta(+Clause)}\index{\texttt{asserta/1}}
If the index specification for the preicate is not {\tt trie}, this
predicate adds a dynamic clause, {\tt Clause}, to the database {\em
before} any other clauses for the same predicate currently in the
database.  If the index specification for the predicate is {\tt trie},
the clause is asserted arbitrarily within the trie, and a warning
message sent to {\tt stderr}.


\ouritem{assertz(+Clause)}\index{\texttt{assertz/1}}
If the index specification for the predicate is not {\tt trie}, this
predicate adds a dynamic clause, {\tt Clause}, to the database {\em
after} any other clauses for the same predicate currently in the
database.  If the index specification for the predicate is {\tt trie},
the clause is asserted arbitrarily within the trie, and a warning
message sent to {\tt stderr}.

\ouritem{retract(+Clause)}\index{\texttt{retract/1}}
removes through backtracking all clauses in the database that match with
{\tt Clause}.  {\tt Clause} must be of one of the forms:  {\tt Head} or
{\tt Head :- Body}.  Note, that because of the precedence of {\tt :-/2},
using the second form requires an extra set of parentheses:
{\tt retract((Head :- Body))}.  Space is reclaimed when a 
clause is retracted.

\ouritem{retractall(+Head)}\index{\texttt{retractall/1}}
removes every clause in the database whose head matches with {\tt Head}.
The predicate whose clauses have been retracted retains the {\tt dynamic}
property (contrast this behavior with that of predicates 
{\tt abolish/[1,2]} below).
Predicate {\tt retractall/1} is determinate and always succeeds.
The term {\tt Head} is not further instantiated by this call.

\ouritem{abolish(+PredSpec)}\index{\texttt{abolish/1}}
Removes the definition of the specified predicate.  {\tt PredSpec}
is of the form {\tt Pred/Arity}.  Everything about the abolished
predicate is completely forgotten by the system (including the
{\tt dynamic} property).  There is also an {\tt abolish/2} which
takes {\tt Pred} and {\tt Arity} as its two arguments.

\ouritem{clause(+Head,?Body)}\index{\texttt{clause/2}}
Returns through backtracking all dynamic clauses in the database whose head
matches {\tt Head} and Body matches {\tt Body}.  For facts the {\tt Body} is
{\tt true}.  

\ouritem{retract\_nr(+Clause)}\index{\texttt{retract\_nr/1}}
Performs just as {\tt retract/1} does, except that it does not reclaim the
space used by the retracted clause. This is provided to allow programmers
to modify dynamic clauses while executing them (a practice that is 
discouraged.) For example, to retract an intersection, as described above,
one could do:
\begin{center}
{\tt :- p(X), q(X), retract\_nr(p(X)), retract\_nr(q(X)), fail.}
\end{center}
In order to reclaim space after using {\tt retract\_nr/1}, see 
{\tt reclaim\_space/1} below.  Predicate {\tt retract\_nr/1}
is not a standard predicate and must be imported from module {\tt assert}.
{\tt retract\_nr/1} is provided for (partial)
compatibility with the {\tt retract/1} predicate of SB-Prolog.

%In this case, the {\tt retract\_nr/1} deletes the clauses, but will not
%drastically modify the clause data structure,
%and this code will execute. Of course, space is not reclaimed for further
%use (but see {\tt reclaim\_space/1} below).  Predicate {\tt retract\_nr/1}
%is not a standard predicate but must be imported from module {\tt assert}.
%The use of this predicate is discouraged; it is provided for (partial)
%compatibility with the {\tt retract/1} predicate provided by SB-Prolog that
%did not reclaim space.

\ouritem{reclaim\_space(+Head)}\index{\texttt{reclaim\_space/1}}
Runs through the dynamic code for the predicate indicated by {\tt Head}, and
reclaims space for any clauses that have been deleted from that predicate by
{\tt retract\_nr/1}.  This cannot safely be used when execution is still
within some invocation of the specified predicate, or will backtrack into
such a scope.  To complete our example of retracting the intersection of
dynamic predicates:
\begin{center}
{\tt :- p(X), q(X), retract\_nr(p(X)), retract\_nr(q(X)), fail\ ;\\
     reclaim\_space(p(\_)), reclaim\_space(q(\_)).}
\end{center}
would do the trick. Notice that the {\tt reclaim\_space} calls 
must be made after execution has completely failed
out of choice points for {\tt q(X)} and {\tt p(X)}.  Predicate 
{\tt reclaim\_space/1} is not
standard but must be imported from module {\tt assert}.
As with {\tt retract\_nr}, the use of this predicate is discouraged; 
it is provided for (partial) compatibility with SB-Prolog.

\comment{ 
TLS: I dont think we need hashtable size given that we use dynamic
hashing.  We still are supporting it, but we dont need to encourage it
\vspace{-.35in}
\ouritem{index(+PredSpec, +IndexSpec, +HashTableSize)}\index{\texttt{index/3}}

If {\tt index/3} is used, then the predicate indicated by {\tt
PredSpec} is declared to be indexed according to {\tt IndexSpec} using
initial hash table of sizes of {\tt HashTableSize}.  After this
directive is given, all clauses asserted to {\tt PredSpec} will be so
indexed.  

For dynamic predicates,
{\tt index/2} is an executable directive that can be used to specify
the indexing of a predicate before clauses to that predicate have been
asserted.  


As an example, one could specify: {\tt index(p/5,[1+2,1,4],300)}.
After clauses are asserted to it, a call to {\tt p/5} would first
check to see if both the first and second arguments are nonvariable
and if so, use an index based on both those values. Otherwise, it
would see if the second argument is nonvariable and if so, use an
index based on it. Otherwise, it would see if the fourth argument is
nonvariable and if so use an index based on it. As a last resort, it would
use no index but backtrack through all the clauses in the predicate.
(Notice that it may well make sense to include an argument that 
appears in a joint specification later alone, as 1 in this example,
but it never makes sense foring the single argument to appear earlier. In
that case the joint index would never be used.)

}

\ouritem{index(+PredSpec, +IndexSpec)}\index{\texttt{index/2}}
\label{index_dynamic} \index{indexing!dynamic predicates}

In general, XSB supports hash-based indexing on various arguments or
combinations of arguments, along with trie-based indexing.  The
availability of various kinds of indexing depends on whether code is
static (e.g. compiled) or dynamic (e.g. asserted or loaded with {\tt
load\_dyn/1}).  The executable directive {\tt index/2} does {\em
not\/} re-index an already existing predicate but takes effect only
for clauses asserted after the directive has been given.  Index
directives can be given to the compiler as part of source code or
executed during program execution (analogously to {\tt op/3}).

\begin{itemize}
\item {\em Hash-based Indexing} 
\begin{itemize}
\item {\em Static Predicates}\ 
In this case {\tt IndexSpec} must be a non-negative integer which
indicates the argument on which an index is to be constructed.  If
{\tt IndexSpec} is~0, then no index is kept (possibly an efficient
strategy for predicates with only one or two clauses.)
\item {\em Dynamic Predicates}
For a dynamic predicate, (to which no clauses have yet been asserted),
{\tt IndexSpec} is either an {\tt IndexElt} or a list of {\tt
IndexElt}s.  Each {\tt IndexElt} defines an index and specifies an argument or group of
arguments that make up the search key of the index.  An argument is
indicated by a small integer ({\tt ArgNo}) indicating the argument number (starting
from 1) to use in the index.  An argument indicator may optionally be annotated
as {\tt *(ArgNo)}.  The argument number alone indicates that only the
main functor symbol of the indicated argument will participate in the
index.  When annotated with the asterisk, the first 5 fields in the
corresponding term (in a depth-first traversal of the term) will be
used in the index.  If there are fewer than 5, they all will be used.
If any of the first 5 is a variable, then the index cannot be used.

An index is usually on a single argument, in which case the {\tt
IndexElt} consists of a single argument indicator.  However, sometimes
one wants an index on multiple arguments, meaning that the values of
several arguments are to be concatenated to create the search key of
the index.  Such a multi-argument (or joint) index is indicated by
using an {\tt IndexElt} that has up to 3 argument indicators separated
by the {\tt +} (plus) operator, e.g., {\tt 1+2+3}.

For example, {\tt index(p/3,[2,1])} indicates that clauses asserted to
the predicate {\tt p/3} should be indexed on both the second and the
first argument.  Subsequent calls to {\tt p/3} will first check to see
if the second argument is nonvariable, and if so use that index, using
the main functor symbol of that argument. If the second argument is
variable, it will next check to see if the first argument is
nonvariable and if so, use that index, built on the main functor
symbol of the first argument.  

{\tt index(p/3,[*(2),1])} would result in similar behavior as the
previous example, but the first index to be tried (on the second
argument) would be built using more of the term value in that second
argument position (not just the main functor symbol.)

As another example, one could specify: {\tt index(p/5,[1+2,1,4])}.
After clauses are asserted to it, a call to {\tt p/5} would first
check to see if both the first and second arguments are nonvariable
and if so, use an index based on both those values. Otherwise, it
would see if the first argument is nonvariable and if so, use an index
based on it. Otherwise, it would see if the fourth argument is
nonvariable and if so use an index based on it. As a last resort, it
would use no index but backtrack through all the clauses in the
predicate.  In all these cases, the indexes are built using only the
main functor symbol in the indicated argument position. (Notice that
it may well make sense to include an argument that appears in a joint
specification later alone, as 1 in this example, but it never makes
sense forcing the single argument to appear earlier. In that case the
joint index would never be used.)

If we want to use similar indexing on {\tt p/5} of the previous
example, except say argument 1 takes on complex term values and we
want to index on more of those terms, we might specify the index as
{\tt index(p/5,[*(1)+2,*(1),4])}.

\end{itemize}

\item {\em Trie-based Indexing}
The executable declaration {\tt index(Predspec,trie)} causes clauses
for {\tt Predspec} to be asserted using tries (see \cite{RRSSW98},
which is available through the XSB web page).  The name trie indexing
is something of a misnomer since the trie itself both indexes the term
and represents it.  In XSB, the above trie index is formed using a
left-to-right traversal of the unit clauses.  These indexes can be
very effective if discriminating information lies deep within a term,
and if there is sharing of left-prefixes of a term, can reduce the
space needed to represent terms.  Furthermore, asserting a unit clause
as a trie is much faster than asserting it using default WAM code.
\comment{
Trie indexing can be used with alternative or joint indexes.  For the
directive {\tt index(p/3,[2,1],trie)}, two trie indices would be
formed for {\tt p/3}: one that traversed arguments in order {\em
2,1,3} and another that traversed arguments in order {\em 1,2,3}.  The
actual implementation seeks to reduce redundant storage of code for
alternative indices.
}

Despite these advantages, representing terms as tries leads to
semantic differences from asserted code, of which the user should be
aware.  First, the order of clauses within a trie is arbitrary: using
{\tt asserta/1} or {\tt assertz} for a predicate currently using trie
indexing will give the same behavior as using {\tt assert}.  Also, the
current version of XSB only allows trie indexing for unit clauses.

Trie-based indexing is available only for dynamic predicates.
\end{itemize}

\ouritem{dynamic(+PredSpec)}\index{\texttt{dynamic/1}}\label{dynamic/1}
is an executable predicate which converts a predicate specified as
(Predicate/Arity) to a dynamic predicate. If {\tt Predicate} is not
previously defined, it will be initialized to empty (so that calls to
it quietly fail, instead of issuing {\sf ``Undefined predicate''}
error messages.) If the predicate is previously defined and dynamic,
{\tt dynamic/1} is a noop. If previously defined as compiled, {\tt
Predicate} will be converted to dynamic, which means that clauses can
be added, although the compiled portion cannot be manipulated.  Note
that {\tt dynamic/1} can be used like a compiler directive, since it
will be passed through to be executed when the module is loaded. Note,
however, that the semantics is different from that of the standard
\cite{ISO-Prolog} when the file contains clauses defining the
so-specified predicate.

\ouritem{table(+PredSpec)}\index{\texttt{table/1}}
is an executable predicate, where PredSpec is a predicate
specification for a dynamic predicate. (This is also a compiler
directive when {\tt PredSpec} specifies a compiled predicate. See the
section of this manual on compiler directives.) This predicate
declares a dynamic predicate to be tabled. It simply saves information
to be used at the time of assert and so it must be called before any
clauses are asserted into the specified predicate in order for the
predicate to be tabled.

\end{description}

%----------------------------------------------------------------------
%\input{tr_assert}      % To be uncommented when they become standard.
%----------------------------------------------------------------------

\subsection{The {\tt storage} Module: Associative Arrays and Backtrackable Updates}

\label{storage module}
XSB provides a high-level interface that supports efficient storage and
querying of key-value pairs. A \emph{key-value pair} is an association
between keys and the corresponding values. There can be at most one value
associated with a given key. A key-value pair can be stored, deleted or
queried. XSB provides two sets of predicates for handling such pairs:
backtrackable and non-backtrackable. The backtrackable primitives for
insertion and deletion of key-value pairs commit their changes to the
database only if the goal succeeds. Otherwise, if the goal fails, the
change is undone. Similarly, XSB provides primitive for backtrackable
updates analogous to {\tt assert} and {\tt retract}. The semantics of
backtrackable updates is defined using Transaction logic \cite{BoKi94}.

All the predicates described in this section must be imported from
module {\tt storage}.

\subsubsection{Non-backtrackable Storage}

\begin{description}
\ouritem{storage\_insert\_keypair(+StorageName,+Key, +Value, ?Inserted)}\index{\texttt{storage\_insert\_keypair/4}}
%%
Insert the given Key-Value pair into the database. If the pair is new, then
{\tt Inserted} unifies with 1. If the pair is already in the database, then
{\tt Inserted} unifies with 0. If the database already contains a pair with
the given key that is associated with a \emph{different} value, then
{\tt Inserted} unifies with -1.
The first argument, {\tt Storage}, must be an atom naming the storage to be
used. Different names denote different storages.
In both cases the predicate succeeds.

\ouritem{storage\_delete\_keypair(+StorageName, +Key, ?Deleted)}
\index{\texttt{storage\_delete\_keypair/3}}
%%
Delete the key-value pair with the given key from the databases. If
the pair was in the database then {\tt Deleted} unifies with 1.
If it was \emph{not} in the databases then {\tt Deleted} unifies with 0.
The first argument, {\tt Storage}, must be an atom naming the storage to be
used. Different names denote different storages.
In both cases the predicate succeeds.

\ouritem{storage\_find\_keypair(+StorageName, +Key, ?Value)}
\index{\texttt{storage\_find\_keypair/3}}
%%
If the database has a key pair with the given key, then {\tt Value} unifies
with the value stored in the database. If no such pair exists in the
database, then the goal fails.

Note that this predicate works with non-backtrackable associative arrays
described above as well as with the backtrackable ones, described below.

\ouritem{storage\_insert\_fact(+StorageName, +Fact, ?Inserted)}\index{\texttt{storage\_insert\_fact/3}}
Similar to keypair insertion, but this primitive inserts facts rather than
key pairs.

\ouritem{storage\_delete\_fact(+StorageName, +Fact, ?Inserted)}\index{\texttt{storage\_delete\_fact/3}}
Similar to keypair deletion, but this primitive deletes facts rather than
key pairs.

\ouritem{storage\_find\_fact(+StorageName, +Fact)}\index{\texttt{storage\_find\_fact/2}}
Similar to keypair finding, but this primitive finds facts facts rather than
key pairs.
%%
\end{description}

\subsubsection{Backtrackable Updates}
\label{backtrackable update}

\begin{description}
\ouritem{storage\_insert\_keypair\_bt(+StorageName, +Key, +Value, ?Inserted)}
\index{\texttt{storage\_insert\_keypair\_bt/4}}
This predicate works exactly as its non-backtrackable counterpart,
{\tt storage\_insert\_keypair/4}, when the top-level goal succeeds.
However, if the top-level goal fails, then the result of the insertion is
undone. In other words, the pair remains in the database until it is
explicitly deleted or until the top-level query fails. The exact semantics
is defined by Transaction Logic \cite{BoKi94}.

Backtrackable key-value pairs are kept in the same database as
non-backtrackable pairs and are queried through the same predicate
{\tt keypair\_find/2}.

\ouritem{storage\_delete\_keypair\_bt(+StorageName, +Key, ?Deleted)}
\index{\texttt{storage\_delete\_keypair\_bt/3}}
%%
Like {\tt storage\_delete\_keypair/3}, but backtrackable.


\ouritem{storage\_insert\_fact\_bt(+StorageName, +Goal)} \index{\texttt{storage\_insert\_fact\_bt/2}}
%%
Like {\tt storage\_insert\_fact/2}, but backtrackable.

\ouritem{storage\_delete\_fact\_bt(+StorageName, +Goal)} 
\index{\texttt{storage\_delete\_fact\_bt/2}}
%%
This is a backtrackable version of {\tt storage\_delete\_fact/2}.

\ouritem{storage\_reclaim\_space(+StorageName)} \index{\texttt{storage\_reclaim\_space/1}}
%%
This is similar to {\tt reclaim\_space/1} for {\tt assert} and {\tt
  retract}, but it is used for storage managed by the primitives defined in
the {\tt storage} module. As with {\tt reclaim\_space/1}, this goal is
typically called just before returning to the top level.
%%
\end{description}



%%% Local Variables: 
%%% mode: latex
%%% TeX-master: "manual1"
%%% End: 
