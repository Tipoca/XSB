\chapter{Libwww: The XSB Internet Access Package}
\label{chap-libwww}

\begin{center}
{\Large {\bf By Michael Kifer}}
\end{center}


\section{Features and Configuration}
This package was inspired by the PiLLoW project.
The XSB Libwww package offers much better performance and a superset of the
PiLLoW functionality as related to the HTTP protocol, but this package does
not implement the part of PiLLoW that deals with construction of Web pages.

The XSB Libwww is implemented in C and relies on the basic HTTP functions
provided by the Libwww library developed by the WWW Consortium
(\verb|http://www.w3c.org/Library|). Therefore, this library must be
installed in order for the XSB Libwww package to work. In addition, XSB
must be configured to work with the Libwww library as follows:
%%
\begin{verbatim}
    configure --with-libwww=directory-where-Libwww-is-installed  
\end{verbatim}
%%

One of the most important aspects of the Libwww package is that it allows
XSB to dispatch multiple HTTP requests, which interleave their Web access
phases. This can be a significant performance boost. Furthermore, the HTNL
and the XML parsers begin their work as the fragments of pages arrive, so
by the time the page is fully accepted, it is also parsed. Here is a list
of features provided by the XSB Libwww package:
%%
\begin{itemize}
  \item  HTML-4 parser.
  \item  XML parser (non-validating).
  \item  Page fetching (without parsing).
  \item  Form handling.
  \item  HTTP header information.
  \item  Multiple, interleaved HTTP requests.
  \item  Basic and digest authentication.
  \item  Redirection and proxies.
\end{itemize}

\section{Accessing Internet with Libwww}

%%
To start using the package, you must load it first:
%%
\begin{verbatim}
    :- [libwww].  
\end{verbatim}
%%

The general form of a Web call is as follows:

%%
\begin{verbatim}
    :- libwww_request([request1, request2, ..., request_n]).  
\end{verbatim}
%%
Each request has the following syntax:
%%
\begin{verbatim}
    request_type(+URL, +RequestParams, -ResponseParams, -Result, -Status)
\end{verbatim}
%%
The request type functor must be either {\tt htmlparse}, {\tt xmlparse},
{\tt fetch}, or {\tt header}. The first two are requests to parse HTML/XML
pages, respectively. {\tt Fetch} is a request to bring in a page without
parsing, nd {\tt header} is a request to retrieve only the header
information (which is returned in the {\tt ResponseParams} argument---see
below).
%%
The URL must be an atom or a string (list of characters).\footnote{
  %%
  The string feature will be deprecated when XSB will have working atom
  garbage collection. When URL is a list of characters, then Result is also
  a list of characters, which eases the burden on the atom table and allows
  XSB to work longer before memory is exhausted.
  %%
  }
%%
Request parameters must be either a variable (in which case the request is
considered to not have special parameters) or a list. The following terms
are allowed in that list:
%%
\begin{itemize}
  \item {\tt timeout(+Secs)} -- request timeout. If it is not specified, a
    default value (5 seconds) is used. Only the first request in a list
    should have the timeout value set. Timeouts that appear in subsequent
    requests are ignored.
  \item {\tt authentication([c(+Realm,+Username,+Pasword),...])} -- If the
    site requires authentication, you should specify it in a list as an
    argument to the {\tt authentication/1} functor. {\tt Realm} is a string
    that the servers return to let applications know which
    username/password pair to send (in case the application works with
    several pages that require different authentication). They are used as
    page identifiers for the authentication purposes. If {\tt Realm} is an
    atom ({\it e.g.}, {\tt authentication('FooSite', boo, moo)}), then when
    a Web server requests authentication for the {\tt FooSite} realm, the
    Libwww package will send the foo/moo user-password pair.

    If {\tt Realm} is a variable, then it is considered to match every
    realm. The Libwww package searches for matching authentication triples
    in the order they appear in the authentication list. Thus, the triple
    where {\tt Realm} is a variable should appear last.
  \item {\tt formdata([attval\_pair1, attval\_pair2, ...])} --- list of
    attribute/value pairs to fill out a form (in case {\tt URL} is a CGI
    script). Each attribute/value pair must be an atom of the form
    {\tt attr=val}.
  \item {\tt selection(Taglist1,Taglist2,Taglist3)} --- if the request is
    {\tt htmlparse} or {\tt xmlparse}, then this term provides control over
    which tags to parse. {\tt Taglist1} is a list of tags that specifies
    inside which tags to parse. For instance, if it is {\tt [ul,form]}
    than parsing will be done only inside these elements. Other
    elements will be ignored. {\tt Taglist2} tells the system to stop
    parsing inside the corresponding elements. For instance, {\tt
      [table]} means that parsing should be done only inside {\tt ul} and
    {\tt form} elements. However, if we hit a table during parsing,
    then parsing should stop unless we hit {\tt ul} or {\tt form} inside
    the {\tt table} element. This switching of parsing on and off can
    continue to arbitrary depth. {\tt Taglist3} is a list of tags that are
    to be ignored completely. That is, the parsing process will simply
    strip these tags (but not the text inside them). For instance, if
    {\tt Taglist3} is {\tt [p,i]} and the page contains ``{\tt <p>foo
    <i>moo</i>}'' then parsing will be done as if the page contained just
  ``{\tt foo moo}''.
\end{itemize}
%%
The {\tt ResponseParams} argument is a list of terms returned by the {\tt
  libwww\_request} call. It contains two kinds of information: header
information and sub-request information. The header information consists of
terms like: {\tt header('Content-Type', 'text/html')}, {\tt
  header('Server', 'Netscape-Enterprise/3.6 SP2')}, etc., as defined by the
HTTP protocol ({\tt header/2} is a functor and its arguments are atoms).
The sub-request information consists of terms of the form: {\tt
  subrequest('http://www.foo.org/test/file.html',-401)}. It indicates that
during processing of the current request, it was necessary to access
another page, {\tt http://www.foo.org/test/file.html}, but the server
responded with the error code -401 (authentication error).
Such sub-requests might be spawned during XML parsing.

The {\tt Result} of a {\tt libwww\_request} call depends on the request type.
In case of {\tt fetch} it is an atom or a list of characters (depending on
whether {\tt URL} was specified as an atom or a list of characters), or it
might be an unbound variable in case of an error. For {\tt header}
requests, {\tt Result} is always an unbound variable.

For {\tt htmlparse} and {\tt xmlparse}, Result is a variable in case of an
error and a complex term otherwise. In the latter case, it is a list of the
form {\tt [elt1,...,elt\_n]}, where each {\tt elt\_i} is of the form:
%%
\begin{verbatim}
    elt(tag, [attval(attrname,value),...], [elt1',...,elt'_m])
\end{verbatim}
%%
The second argument here represents the list of attribute-value pairs.  In
HTML, some attributes, like {\tt checked}, can be binary, in which case the
corresponding value will be unbound. The third argument represents
HTML or XML elements that are within the scope of {\tt tag}. These elements
have the same syntax as the parent element:
\verb|elt(tag',attrs,sub-elements)|. If a tag has no attributes or if
it does not have sub-elements, the corresponding lists will be empty.
One special tag, {\tt pcdata}, is introduced to represent pieces of text
that appear in the document. This tag is our own creation---neither HTML
nor XML use tags to represent text. One important difference between {\tt
  pcdata} and other tags is that the third argument in {\tt
  elt(pcdata,...,...)} is an atom or a list of characters, not a list
(unlike other tags). If URL was specified as an atom, then the third
argument of the {\tt pcdata}-element is an atom as well. If URL is a
character list, then so is the corresponding argument in the
{\tt pcdata}-element.

Finally, {\tt Status} is bound to an integer that represents the return
code from the HTTP request. A complete list of return codes is given in
{\tt XSB/prolog\_includes/http\_errors.h}. If you need to refer to error
codes in your Prolog application, it is advisable to use symbolic
notation. To make this happen, put the following lines at the top of your
program: 
%%
\begin{verbatim}
    :- compiler_options([xpp_on]).
    #include "http_errors.h"
\end{verbatim}
%%
The Libwww package also includes a predicate that is convenient for
providing English language explanations to the errors:
%%
\begin{verbatim}
    :- import http_liberr/3 from usermod.  
\end{verbatim}
%%
The first argument of this predicate is the error code, the second is an
explanation in English, and the last is the class of the error ({\it e.g.},
internal, server error, client error, etc.). For full details see
{\tt XSB/packages/libwww/http\_liberr.P}. Note that the code for a successful 
call is HT\_LOADED (=200), not zero or one!

\section{Example}

Here is a complete example:
%%
\begin{verbatim}
    | ?- libwww_request([xmlparse('http://public.org/test/simple1.xml',[timeout(4)],
                                  P,Y,Z)]),
          http_liberr(Z,Explanation,Class).

    P = [header(Content-Type,text/html),
         subrequest('http://public.org/secret/001.ent',-401)]
    Y = [elt(doc,[],[elt(pcdata,[],' '),
                     elt(foo,[attval(att1,123),attval(att2,ppp)],
                             [elt(pcdata,[],'Test1 '),elt(pcdata,[],' Test2 '),
                              elt(pcdata,[],adsdd),elt(pcdata,[], )]),
                     elt(pcdata,[],  ),
                     elt(a,[],[elt(pcdata,[], aaaaaaaaaaa),
                     elt(pcdata,[],' '),
                     elt(b,[attval(att,1)],[]),
                     elt(pcdata,[],' '),
                     elt(c,[attval(att,2)],[]),
                     elt(pcdata,[],' '),
                     elt(d,[],[elt(pcdata,[],dddddddd),
                     elt(f,[],[elt(pcdata,[],kkkkkkk)]),
                     elt(pcdata,[],abc)]),elt(pcdata,[],'  ')])])]
    Z = 200
    Explanation = 'OK'
    Class = 'success'
\end{verbatim}
%%
The above is a successful (because of the return code 200) request to parse
an XML page. This page apparently had a reference to an external entity
that was located in a protected domain. Since we did not supply
authentication information, the call returned authentication failure for
that subrequest (as indicated by the term 
{\tt subrequest('http://public.org/secret/001.ent',-401)} in the fourth
argument).




%%% Local Variables: 
%%% mode: latex
%%% TeX-master: "manual2"
%%% End: 
