\chapter{{\tt clpr}: The CPL(R) package}
%===========================

\begin{center}
{\Large  {\bf By Leslie De Koninck based on code by Christian Holzbaur}}~\footnote{The
  CLP(R) files are based on those written by De Konink for SWI Prolog,
  and were ported to XSB by Terrance Swift.}
\end{center}

The CLP(R) library supports solutions of linear equations and
inequalities over the real numbers and the lazy treatment of
nonlinear equations.  In displaying sets of equations and
disequations, the library removes redundancies, performs projections,
and provides for linear optimization.  The goal of the XSB port is to
provide the same CLP(R) functionality as in other platforms, but also
to allow constraints to be used by tabled predicates.  This section
provides a general introduction to the CLP(R) functionality available
in XSB, for further information see {\tt http://www.ai.univie.ac.at/clpqr}.

The {\tt clpr} package may be loaded by the command {\tt [clpr]}.
Loading the package imports exported predicates from the various files
in the {\tt clpr} package into {\tt usermod} (see Volume 1, Section
3.3) so that they may be used in the interpreter.  Modules that use
the exported predicates need to explicitly import them from the files
in which they are defined (e.g. {\tt bv}, as shown below).

XSB's tabling engine supports the use of attributed variables
(Section~\ref{sec:attributed-variables}), which in turn have been used
to port real constraints to XSB under the CLP(R) library of Christian
Holzbauer \cite{Holz95}.  Constraint equations are represented using
the Prolog syntax for evaluable functions (Volume 1, Section 6.2.1).
Formally:

{\it
\begin{tabbing}
12 \= 12345678901234567890 \= 12345678901234567890 \=	\kill
\> ConstraintSet --$>$   \>  C     \\
\> \>                      $|$ C {\tt ,} C   \\
\\
\> C --$>$ \> Expr {\tt =:=} Expr \> {\rm equation} \\
\>    \> $|$ Expr {\tt =} Expr \> {\rm equation} \\
\>    \> $|$ Expr {\tt $<$} Expr \> {\rm strict inequation} \\
\>    \> $|$ Expr {\tt $>$} Expr \> {\rm nonstrict inequation} \\
\>    \> $|$ Expr {\tt =$<$} Expr \> {\rm nonstrict inequation} \\
\>    \> $|$ Expr {\tt $>$=} Expr \> {\rm disequation} \\
\\
\> Expr --$>$  \> variable \> {\rm Prolog variable} \\
\> \> $|$ number \> {\rm floating point number} \\
\> \> $|$ {\tt +} Expr \\
\> \> $|$ {\tt -} Expr \\
\> \> $|$ Expr {\tt +}  Expr \\
\> \> $|$ Expr {\tt -} Expr \\
\> \> $|$ Expr {\tt *} Expr \\
\> \> $|$ Expr {\tt /} Expr \\
\> \> $|$ {\tt abs($Expr$)} \\
\> \> $|$ {\tt sin($Expr$)} \\
\> \> $|$ {\tt cos($Expr$)} \\
\> \> $|$ {\tt tan($Expr$)} \\
\> \> $|$ {\tt pow($Expr$,$Expr$)} \> {\rm raise to the power} \\
\> \> $|$ {\tt exp($Expr$,$Expr$)} \> {\rm raise to the power} \\
\> \> $|$ {\tt min($Expr$,$Expr$)} \> {\rm minimum of two expressions} \\
\> \> $|$ {\tt max($Expr$,$Expr$)} \> {\rm maximum of two expressions} \\
\> \> $|$ \verb|#|(Expr) \> {\rm symbolic numerical constants} 
\end{tabbing}
}

\begin{description}
\ournewitem{\{+Constraints\}} {} \index{{\tt \{\}/1}}

When the CLP(R) package is loaded, inclusion of equations in braces
({\tt \{\}}) adds the equations to the constraint store where they are
checked for satisfiability.

{\small
\begin{verbatim}

     | ?- [clpr].
     [clpr loaded]
     [itf3 loaded]
     [nf loaded]

     yes

     | ?- {X = Y+1, Y = 3*X}.

     X = -0.5000
     Y = -1.5000;

     yes
\end{verbatim}
}

\ournewitem{entailed(Constraint)} {} \index{{\tt entailed/1}}

Succeeds if {\tt Constraint} is logically implied by the current
constraint store.  {\tt entailed/1} does not change the constraint
store.

{\small
\begin{verbatim}

| ?- {A =< 4},entailed(A =\= 5).

yes   
\end{verbatim}
}

\ournewitem{inf(+Expr,-Val)} {bv} \index{{\tt inf/2}}

\ournewitem{sup(+Expr,-Val)} {bv} \index{{\tt sup/2}}

\ournewitem{minimize(Expr)} {bv} \index{{\tt minimize/1}}

\ournewitem{maximize(Expr)} {bv} \index{{\tt maximize/1}}

These four related predicates provide various mechanisms to compute
the maximum and minimum of expressions over variables in a constraint
store.  In the case where the expression is not bounded from above
over the reals {\tt sup/2} and {\tt maximize/1} will fail; similarly
if the expression is not bounded from below {\tt inf/2} and {\tt
minimize/1} will fail.

{\small
\begin{verbatim}

| ?- {X = 2*Y,Y >= 7},inf(X,F).

X = _h8841
Y = _h9506
F = 14.0000

| ?- {X = 2*Y,Y >= 7},minimize(X).

X = 14.0000
Y = 7.0000

| ?- {X = 2*Y,Y =< 7},maximize(X-2).

X = 14.0000
Y = 7.0000

| ?- {X = 2*Y,Y =< 7},sup(X-2,Z).

X = _h8975
Y = _h9640
Z = 12.0000

yes
| ?- {X = 2*Y,Y =< 7},maximize(X-2).

X = 14.0000
Y = 7.0000

yes
\end{verbatim}
}

\ouritem{bb\_inf(+IntegerList,+Expr,-Inf,-Vertex, +Eps)}  

Works like {\tt inf/2} in {\tt Expr} but assumes that all the
variables in {\tt IntegerList} have integral values.  {\tt Eps} is a
positive number between $0$ and $0.5$ that specifies how close an
element of {\tt IntegerList} must be to an integer to be considered
integral -- i.e. for such an {\tt X}, {\tt abs(round(X) - X) < Eps}.
Upon success, {\tt Vertex} is instantiated to the integral values of
all variables in {\tt IntegerList}.  {\tt bb\_inf/5} works properly for
non-strict inequalities only.

\begin{verbatim}
| ?- {X > Y + Z,Y > 1, Z > 1},bb_inf([Y,Z],X,Inf,Vertex,0).

X = _h14286
Y = _h10914
Z = _h13553
Inf = 4.0000
Vertex = [2.0000,2.0000]

yes
\end{verbatim}

\ouritem{bb\_inf(+IntegerList,+Expr,-Inf)}  

Works like {\tt bb\_inf/5}, but with the neighborhood, {\tt Eps}, set
to {\tt 0.001}.

\begin{verbatim}
|?- {X >= Y+Z, Y > 1, Z > 1}, bb_inf([Y,Z],X,Inf)

X = _h14289
Y = _h10913
Z = _h13556
Inf = 4.

yes
\end{verbatim}

\ournewitem{dump(+Variables,+NewVars,-CodedVars} {clpr} 
\index{{\tt dump/3}}
%
For a list of variables {\tt Variables} and a list of variable names
{\tt NewVars}, returns in {\tt CodedVars} the constraints on the
variables, without affecting the constraint store.

Example:
\begin{verbatim}
| ?- {X > Y+1, Y > 2},
     dump([X,Y], [x,y], CS).

X = _h17748
Y = _h17139
CS = [y > 2.0000,x - y > 1.0000];
\end{verbatim}

\end{description}


