\chapter{Getting Started with XSB} \label{quick_start}
%============================================================

This section describes the steps needed to install XSB under UNIX and
under DOS/Windows.

\section{Installing XSB under UNIX}
%==========================================
\label{installation_options}

If you are installing on a UNIX platform, the version of XSB that you
received may not include all the object code files so that an
installation will be necessary.  The easiest way to install XSB is to
use the following procedure.

\begin{enumerate}
\item   Make sure that after you have obtained XSB by anonymous ftp 
	(using the {\tt bin}ary option) or from the web, you have
	uncompressed it by following the instructions found in the
	file {\tt README}.

\item	Decide in which directory in your file system you want to
	install XSB and copy or move XSB there.

	In the rest of this manual, let us use \verb'$XSB_DIR' to
	refer to this directory.  Note the original subdirectory
	structure must be maintained, namely, the directory
	\verb'$XSB_DIR' should contain all the subdirectories and files
	that came with the distribution. In particular, the following
	directories are required for XSB to work:
	\verb'emu', \verb'syslib', \verb'cmplib', 
	\verb'lib', \verb'packages', \verb'build', and \verb'etc'.

\item Change directory to \verb'$XSB_DIR/build' and then run these commands:
  %%
  \begin{quote}
    \tt
    configure\\
    \tt
    makexsb
  \end{quote}
  %%
  %%$
  This is it!
  
  In addition, it is now possible to install XSB in a shared directory
  ({\it e.g.}, {\tt /usr/local}) for everyone to use.  In this situation,
  you should use the following sequence of commands:
  %%
  \begin{quote}
    \tt
    configure --prefix=\verb'$SHARED_XSB'\\
    \tt
    makexsb\\
    \tt
    makexsb install
  \end{quote}
  %%$
  where \verb'$SHARED_XSB' denotes the shared directory where XSB is
  installed.  In all cases, XSB can be run using the script
  %%$
  \begin{quote}
    \verb'$XSB_DIR/bin/xsb'
  \end{quote}
  %%$
  However, if XSB is installed in a central location, the script for
  general use is:
  \begin{quote}
    \verb'<central-installation-directory>/<xsb-version>/bin/xsb'
  \end{quote}
\end{enumerate}
  %%

{\bf Important:} The XSB executable determines the location of the
libraries it needs based on the full path name by which it was invoked.
The ``smart script'' \verb|bin/xsb| also uses its full path name to
determine the location of the various scripts that it needs in order to
figure out the configuration of your machine.  Therefore, there are certain
limitations on how XSB can be invoked.

Here are some legal ways to invoke XSB:
%%
\begin{enumerate}
\item  invoking the smart script \verb|bin/xsb| or the XSB executable using
  their absolute or relative path name.
\item using an alias for \verb|bin/xsb| or the executable.
\item creating a new shell script that invokes either \verb|bin/xsb| or the
  XSB executable using their {\em full\/} path names. 
\end{enumerate}
%%

Here are some ways that are guaranteed to not work in some or all cases:
%%
\begin{enumerate}
\item  creating a hard link to either \verb|bin/xsb| or the executable.
\item creating a symbolic link to either \verb|bin/xsb| or the
  executable.
\item changing the relative position of either \verb|bin/xsb| or the
  XSB executable with respect to the rest of the XSB directory tree.
\end{enumerate}
%%

%%
\begin{description}
\item[Type of Machine.]
  Configure automatically detects your machine and OS type and builds XSB
  accordingly. Moreover, you can build XSB for different architectures
  while using the same tree and the same installation directory provided,
  of course, that these machines are sharing this directory, say using NFS
  or Samba. All you will have to do is to login to a different machine with
  a different architecture or OS type, and repeat the above sequence of
  comands. 
  
  The configuration files for different architectures reside in different
  directories, and there is no danger of an architecture conflict.
  Moreover, you can keep using the same {\tt ./bin/xsb} script regardless
  of the architecture. It will detect your configuration and will use the
  right files for the right architecture! 
  
\item[Choice of the C Compiler and Other options] \label{cc} The {\tt
    configure} script will attempt to use {\tt gcc}, if it is available.
  Otherwise, it will revert to {\tt cc} or {\tt acc}.  Some versions of
  {\tt gcc} are broken, in which case you would have to give {\tt
    configure} an additional directive {\tt --with-cc}.  You can also {\tt
    --disable-optimization} (to change the default), {\tt --enable-debug},
  and there are many other options.  Type {\tt configure --help} to see
  them all. Also see the file \verb'$XSB_DIR/INSTALL' for more details.
    %%$
\end{description}

\subsection{Possible Installation Problems}

\paragraph*{Lack of Space for Optimized Compilation of C Code}
When making the optimized version of the emulator, the temporary space
available to the C compiler for intermediate files is sometimes not
sufficient. For example on one of our SPARCstations that had very
little {\tt /tmp} space the {\tt "-O4"} option could not be used for
the compilation of files {\tt emuloop.c}, and {\tt tries.c}, without
changing the default {\tt tmp} directory and increasing the swap
space.  Depending on your C compiler, the amount and nature of {\tt
/tmp} and swap space of your machine you may or may not encounter
problems.  If you are using the SUN C compiler, and have disk space in
one of your directories, say {\tt dir}, add the following option to
the entries of any files that cannot be compiled:

\demo{       -temp=dir}

\noindent
If you are using the Gnu C compiler, consult its manual pages
to find out how you can change the default {\tt tmp} directory or how you
can use pipes to avoid the use of temporary space during compiling.
Usually changing the default directory can be done by declaring/modifying
the {\tt TMPDIR} environment variable as follows:

\demo{       setenv TMPDIR dir}

\paragraph*{Missing XSB Object Files}
When an object (*.O) file is missing from the {\tt lib} directories you can
normally run the {\tt make} command  in that directory to restore it
(instructions for doing so are given in Chapter
\ref{quick_start}).  However, to restore an object file in the
directories {\tt syslib} and {\tt cmplib}, one needs to have a
separate Prolog compiler accessible (such as a separate copy of
XSB), because the XSB compiler uses most of the files in these
two directories and hence will not function when some of them are
missing.  For this reason, distributed versions normally include all
the object files in {\tt syslib} and {\tt cmplib}.



\section{Installing XSB under DOS} \label{quick:DOS}
%==========================================

***This got to be changed!!!

\begin{enumerate}
\item 
   XSB will unpack into a subdirectory named {\tt XSB}.
   Assuming that you have {\tt XSB???.ZIP} in the top-level directory,
   you can issue the command
\begin{verbatim}
   unzip386 xsb???.zip
\end{verbatim}
   which will install XSB in the directory {\tt $\backslash$XSB}.

\item 
   If you have no other source of DPMI support (Windows is one source),
   unpack {\tt csdpmi1b.zip} somewhere (unpacking in a directory named
   DJGPP is suggested), and be sure that the exe files are somewhere
   in your path.  Likewise for {\tt wmemu2b.zip}, if you don't have a
   floating-point unit.

\item 
   If you unpacked XSB in the top-level directory, go to step 4.
   Otherwise, you must ensure that XSB can find its system files.
   This can be accomplished in any of the following ways:
\begin{enumerate}
\item   Set the environment variable XSBHOME to point to XSB's
       home directory.  For example, if you unpacked XSB in
       {\tt c:$\backslash$mystuff}, you would add the following line to
       {\tt AUTOEXEC.BAT}:
\begin{verbatim}
       SET XSBHOME=c:\mystuff\xsb
\end{verbatim}
\begin{center}
   -or-
\end{center}
\item  Specify XSB's home directory on the command line when
       starting XSB (using the -D option)
\begin{center}
   -or-
\end{center}
\item  Modify the Makefile the emu directory to reflect the
       home directory of XSB, and recompile.
\end{enumerate}
   Most users will probably prefer the first option.

\item 
   If you modified {\tt AUTOEXEC.BAT}, either reboot or set the variables
   by hand.  Start XSB as follows:
\begin{verbatim}
   \xsb\emu\xsb -i
\end{verbatim}
   (appropriately modified to reflect where you installed XSB).  You
   may move the executable {\tt
   $\backslash$xsb$\backslash$emu$\backslash$xsb.exe to} some directory in
   your search path, if you like, or add {\tt $\backslash$xsb$\backslash$emu}
   to your path.  Don't move XSB's system files (in cmplib, lib, and
   syslib) unless you also ensure that XSB can find them in their new
   location (see Step 3).

   The XSB executable in COFF format
   ({\tt $\backslash$xsb$\backslash$emu$\backslash$xsb}) is needed for the C
   interface.  If you don't plan to use the C interface, you can
   delete this file.
\end{enumerate}


\section{Building XSB using MS Visual C++}
%=========================================

***This will need to be changed!!!

You can compile XSB under Microsoft Visual C++ compiler in development
environment to create a console-supported top loop or a DLL by following
these steps (based on MS Visual C++ 4.1):

\begin{enumerate}

\item
   Open the MS Visual C++ Studio.

\item
   Click on {\bf File}, then click on {\bf New}.  Select {\bf Project
   Workspace} from the list, and press {\bf OK}.  Then a window will pop
   up.  Select {\bf Console Application} (to create a console-supported
   top loop) or {\bf Dynamic-Link Library} (to create a DLL), and set
   the {\bf Name}, {\bf Platform} and {\bf Location} for the project.
   Then click on {\bf Create} to close the window.

\item
   Click on {\bf Insert}, then click on {\bf Files into Project ...}.
   Select all files from the {\tt emu} directory (highlight the {\tt*.[chi]}
   files) and add them to the project (click {\bf Add}) (if {\tt xddemain.c}
   is included, delete it!).

\item
   Click on {\bf Build}, then {\bf Set Default Configurations}.  Set to
   a ``Release'' or ``Debug'' version.

\item
   Click on {\bf Build}, then {\bf Settings} to open the {\bf Project
   Settings} window.  For both ``Debug'' and ``Release'' versions
   (highlight them both --- should already be that way when window pops
   up):

	\begin{enumerate}

	\item Click on the {\bf C/C++} tab, add the following
	definitions in the {\bf Preprocessor definitions} input box
	(WIN32, and \_CONSOLE are already there): {\tt WIN\_NT}
	(required), {\tt DIR=$\backslash$"C:$\backslash
	\backslash$xsbsys$\backslash \backslash$xsb$\backslash$"} (with
	your xsb directory, required), {\tt SOCKET\_IO} (for socket
	support, if desired), {\tt ORACLE} (for Oracle interface, if
	desired), {\tt XSB\_ODBC} (for ODBC interface, if desired), {\tt
	XSB\_DLL} (if you are compiling XSB as a DLL) and {\tt
	EXECUTABLE=xsbdll} (if compiling a DLL, how it will name .dll
	file).

	\item Click on the {\bf Link} tab, add the following items in
	the {\bf Object/Library Modules} input box ({\tt kernel32.lib,
	user32.lib, gdi32.lib, winspool.lib, comdlg32.lib, advapi32.lib,
	shell32.lib, ole32.lib, oleaut32.lib, uuid.lib, odbc32.lib,
	odbccp32.lib} are already there): {\tt wsock32.lib} (for socket
	support) and the full path to an SQL lib (for Oracle support,
	e.g., {\tt
	d:$\backslash$orant$\backslash$pro22$\backslash$lib$\backslash$msvc$\backslash$sqllib18.lib})

	\end{enumerate}

Then, click on {\bf OK} in the {\bf Project Settings} window.

\item 
   Click on {\bf Build}, then {\bf Update All Dependencies ...}.

\item 
   Click on {\bf Build}, then {\bf Rebuild All}.

\end{enumerate}

If you chose {\bf Console Application}, then you can run the XSB executable
either from a DOS prompt or by double clicking the executable.

If you chose {\bf Dynamic-Link Library}, this will create a dynamic link
library for XSB that can then be used and called as described in the XSB
manual under the C-calling-XSB interface.  The calling conventions used
in the exported DLL routines are the standard calling conventions.  If
you want the C calling conventions, replace the XSB\_DLL flag with
XSB\_DLL\_C.


\section{Invoking XSB}
%============================

XSB can be invoked by the command:
\begin{verbatim}
       $XSB_DIR/bin/xsb -i
\end{verbatim}
%%$
if you have installed XSB in your private directory.
If XSB is instaled in a shared directory ({\it e.g.}, \verb'$SHARED_XSB'
for the entire site, then you should use
\begin{verbatim}
       $SHARED_XSB/bin/xsb -i
\end{verbatim}
%%
In both cases, you will find yourself in the top level interpreter.  
As mentioned above, this script automatically detects the system
configuration you are running on and will use the right files and
executables. (Of course, XSB should have been built for that architecture
earlier.)

You may want to make an alias such as {\tt \smallourprolog} to the command
for convenience. However, you should {\bf not} make hard links to this
script or to the XSB executable. If you invoke XSB via such a hard link,
XSB wil likely be confused and will not find its libraries.
That said, you {\bf can} create other scripts and cal the above script from
there.  

Most of the ``standard'' Prolog predicates are supported by XSB, 
so those of you who consider yourselves champion entomologists, can try
to test them for bugs now.  Details are in Chapter~\ref{standard}.


\section{Compiling XSB programs}
%=======================================

All source programs should be in files whose names have the 
suffix {\tt .P}.  One of the ways to compile a program from a file in 
the current directory and load it into memory, is to type the query:
\begin{verbatim}
     [my_file].
\end{verbatim}
where \verb'my_file' is the name of the file, or preferably, the name
of the module (obtained from the file name by deleting the suffix {\tt .P}).
To find more about the module system of XSB see Section~\ref{Modules}.

If you are eccentric (or you don't know how to use an editor) you can also 
compile and load predicates input directly from the terminal by using the
command:
\begin{verbatim}
     [user].
\end{verbatim}
A {\tt CTRL-d} or the atom \verb'end_of_file' followed by a period 
terminates the input stream.


\section{Sample XSB Programs}
%====================================

If for some reason you don't feel like writing your own XSB programs, 
there are several sample XSB programs in the directory: 
\verb'$XSB_DIR/examples'.  All contain source code.

The entry predicates of all the programs in that directory are given
the names {\tt demo/0} (which prints out results) and {\tt go/0}
(which does not print results). Hence, a sample session might look like
(the actual times shown below may vary and some extra information is given
using comments after the \% character):

{\footnotesize
 \begin{verbatim}
     my_favourite_prompt> cd $XSB_DIR/examples
     my_favourite_prompt> ../emu/xsb -i -s
     XSB Version 1.3.0 (93/9/13)
     [sequential, single word, optimal mode]
     | ?- [prof_lib].      % loads into the system predicate measure/1
     [prof_lib loaded]

     yes
     | ?- [queens].
     [queens loaded]

     yes.
     | ?- demo.
     ..........          % output of 8-queens problem here

     yes.
     | ?- measure(demo).
     ..........          % same junk printed here
     Time used: 2.409 sec
     == Statistics ===      
     Permanent space: 131752 in use, 384K allocated.
     Stack space allocated: 64K+64K+768K+768K+8K
     Stacks in use : global 164, local 88, trail 48, cp 220
     SLG Stacks in use : table 0, tab_heap 0 opentables 0
     num unwinds: 731
     Max stack used: global 66152, local 500, trail 420, cp 2780
     TIME: cputime: 2.41*1 sec, elapsetime 8.02 sec

     yes
     | ?- [map].         % This will generate some warnings but don't worry
     ++Warning: demo/0 (type 2) was defined in another module!
     ++Warning: go/0 (type 2) was defined in another module!
     [map loaded]

     yes.
     | ?- demo.
     ..........          % output of map coloring problem here

     yes.
     | ?- measure(go).   % Don't give me all this junk again!
     Statistics is reset.
     Time used: 0.629  sec
     == Statistics ===      
     Permanent space: 132328 in use, 384K allocated.
     Stack space allocated: 64K+64K+768K+768K+8K
     Stacks in use : global 164, local 88, trail 48, cp 220
     SLG Stacks in use : table 0, tab_heap 0 opentables 0
     num unwinds: 708
     Max stack used: global 66432, local 576, trail 612, cp 2216
     TIME: cputime: 0.63*1 sec, elapsetime 0.68 sec

     yes
     | ?- halt.          % I had enough !!!

     End XSB (cputime 7.52s, elapsetime 125.69s)
     my_favourite_prompt>
 \end{verbatim}
}


\section{Exiting XSB}
%===========================

If you want to exit XSB, issue the command \verb'halt.' or
simply type \verb'CTRL-d' at the XSB prompt. To exit XSB while it is
executing queries, strike \verb'CTRL-c' a number of times.


%%% Local Variables: 
%%% mode: latex
%%% TeX-master: "manual"
%%% End: 
