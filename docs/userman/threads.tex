\chapter{Multi-Threaded Programming in XSB} \label{chap:threads}

Starting with Version 3.0, XSB supports the use of Posix threads to
perform separable computations, and in certain cases to parallelize
them.  Posix threads have a simple and clear API, and are available on
all Unixes as well as on Windows via ???.  This chapter introduces how
to program with threads in XSB through a series of examples.  Later
sections discuss performance aspects of our implementation as well as
describing relevant predicates.  A general knowledge of multi-threaded
programming is assumed, such as can be found in \cite{LewB98,Bute97}.

\section{Getting Started with Multi-Threading}
%
In Version 3.0 the default configuration of XSB does not include
multi-threading.  This is partly because multi-threading is new, and
despite our efforts, the multi-threaded engine may contain bugs not
present in the single-
threaded engine.  In addition, in Version 3.0,
not all packages have yet been made thread-safe so that not all
configurations are supported with multi-threading.  Currently ODBC is
supported with the multi-threaded configuration, but {\tt dbdrivers},
{\tt xpath}, {\tt iodbc}, {\tt interprolog} and {\tt smodels} have not
been. {\sc What about others?}

With this in mind, making the multi-threaded engine is simple:
configure and make XSB as in Chapter~\ref{chap:system}, but include
the command {\tt --enable-mt}.  When you invoke the newly made
configuration of XSB you should see {\tt engine: multi-threading} in
the configuration list below the banner rather than {\tt engine:
  slg-wam}.

\paragraph{Hello World for Beginners}
%
We naturally start with a program to print ``hello world''.  Within
the multi-threaded engine, import {\tt xsb\_thread\_create/2} from the
module {\tt thread}, and type the command
\begin{center} 
{\tt ?- xsb\_thread\_create(writeln('hello world'),Id)} 
\end{center} 
you should see something like 
\begin{center} 
{\tt Id = 1hello world} 
\end{center} 
%
while the output is a little ugly, the simple ``hello world'' program
does illustrate simple multi-threading at work.  The calling thread
(i.e. the thread controlling the command-line interpreter which we
call $T_{prompt}$) executes the predicate {\tt xsb\_thread\_create/2}
which creates a thread $T_{child}$ and immediately returns with the
{\em XSB thread id} of the created thread.  Meanwhile, $T_{child}$
initializes its stacks and other memory areas and executes the goal
{\tt writeln('hello world')}.  $T_{child}$ and $T_{prompt}$ share most
of their process-level information: in particular they share a common
I/O stream for standard output, leading to the output above.  What is
happening may be seen a little more easily by executing the command
%
\begin{center} 
{\tt ?- xsb\_thread\_create((sleep(1),writeln('hello world')),Id)} 
\end{center} 
%
In this case the interpreter reports that {\tt F} is bound to a thread
id, then about a second later {\tt writeln/1} is executed.  

The simple ``hello world'' program illustrates a couple of points.
First, it is easy to create a thread in XSB and have that thread do
work.  Second, it can be tricky to coordinate actions among threads.
We'll explore these two themes in more detail, but first suppose we
are determined to extend out multi-threaded program so that it
produces good output.  One way to do this is to {\em join}
$T_{prompt}$ and $T_{child}$ as follows 
%
\begin{verbatim}
?- xsb_thread_create(writeln('hello world'),Id),
   xsb_thread_join(Id,ExitCode).
hello world

Id = 1
ExitCode = no_error_ball - 0
\end{verbatim}
%
In this case, as soon as $T_{prompt}$ has issued a command to create
$T_{child}$, it executes {\tt xsb\_thread\_join/2}.  This later
predicate makes a system call to the underlying operating system to
suspend $T_{prompt}$ until $T_{child}$ has exited.  {\tt
  xsb\_thread\_join/2} returns both the error ball (if any) and the
exit code of $T_{child}$.  As usual, a value of {\tt 0} indicates
normal termination, as does the fact that no error ball was found.

So far, we've introduced a few concepts that have not been fully
discussed.  First is the concept of an {\em XSB thread id}: XSB
manages up to {\tt MAXTHREAD} active threads using XSB thread
ids~\footnote{This number can be reset via a command line option to
  XSB, see Section~\ref{sec:EmuOptions}}.  It should be noted that the
XSB thread id of a thread is different from the identifier of the
underlying pthread.  An XSB thread id is a Prolog term, and unlike
Posix thread ids, XSB thread ids can be compared for equality using
unification.  The actual form of an XSB thread id, however, is subject
to change between versions, so programs should not make use of the
exact form of an XSB thread id.  In the multi-threaded engine, the XSB
thread id of any thread can be queried using the predicate {\tt
  xsb\_thread\_self/1}.

\section{Communication among Threads}

\begin{example}
Consider the program fragment
%
\begin{verbatim}
:- dynamic p/1.

test:- xsb_thread_create(assert(p(1)),_X).
\end{verbatim}
If you type the goal {\tt ?- test} and then the goal {\tt ?- p(X)},
the call {\tt p(X)} will fail.  
\end{example}

This illustrates an important point about dynamic and tabled
predicates in the multi-threaded engine: by default such predicate
{\tt p/n} are private to the thread that asserts clauses to {\tt p/n}
or that creates the table for {\tt p/n}.  This contrasts to static
code is always shared between threads.  To allow {\tt p(1)} to be
visible to various threads, it must be declared to be shared with the
following declaration.
%
\begin{verbatim}
n:- thread_shared(p(_)).
\end{verbatim}

The ability to share dynamic code between predicates provides an
extremely powerful mechanism for threads to communicate.  So why does
XSB make dynamic predicates thread-private by default?  The main
reason for this is that if dozens or hundreds of threads are running
concurrently, shared dynamic code becomes an expensive synchronization
point.  Code for shared predicates must be more heavily mutexed than
code for private predicates.  In particular, XSB may not immediately
reclaim the space of retracted clause, to avoid the possibility of
some computation backtracking into a clause that has been reclaimed.
Rather XSB may garbage collect the space of the retracted clauses at a
later time.  While clause garbage collection is simple enough to
implement for a single thread, garbage collecting clauses for shared
dynamic predicates is nearly impossible to do when multiple threads
are active.  Thus one set of reasons for making dynamic predicates
private by default are based on efficiency~\footnote{\version{} of
  XSB allows garbage collection of retracted clauses only when there
  is a single active thread.  Future versions will likely offer more
  powerful garbage collectors.}.  {\sc Clause GC is not yet
  implemented; also need to talk about tries}.

The second reason for making dynamic predicates thread-private by
default is semantic.  Suppose thread $T_1$ starts a tabled computation
that depends on the dynamic shared predicate {\tt p/1}.  While $T_1$
is computing the table, thread $T_2$ asserts a clause to {\tt p/1}.
$T_1$'s table is likely to be inconsistent, leading to the problem of
{\em read consistency} of any table that depends on thread-shared
dynamic predicates.  In \version , users are responsible for ensuring
read consistency of any tables that depend on shared dynamic data.
Future versions of XSB are intended to allow more sophisticated
mechanisms for read consistency.

Not only can tables depend on thread-shared or thread-private dynamic
data, but 
%in the default multi-threaded configuration of XSB, which
%uses local scheduling for tabled evaluation, 
the tables themselves may be thread-shared or thread-private.  Like
dynamic code, tables are private by default; the declaration {\tt
  thread\_shared/1} allows completed tables for a predicate evaluated
with call-variance to be shared among threads~\footnote{In \version,
  tabled predicates using call-subsumption are always private; an
  attempt to make such a predicate thread-shared throws an
  exception.}.
%
To some extent, tabling considerations for making a predicate
thread-shared or thread-private are like those of dynamic code.
Thread-private tables require fewer synchonization points overall, and
garbage collection of space for abolished thread-private tables is far
easier to perform.  However the semantics of how tabling information
is shared depends on whether local or batched evaluation is used with
the multi-threaded engine.  Recall from
Chapter~\ref{chap:TablingOverview} that local evaluation is so-named
because computation always takes place in the SCC most recently
created, and no answer is returned outside of an SCC until the SCC has
been completely evaluated.  Within this scheduling strategy it is not
often useful to share answers between tables that have not been
completed -- as local evaluation would allow these answers to be
returned only if the tables were in the same SCC.  This leads to a
concurrency semantics called {\em Shared Completed Tables}.  Shared
Completed Tables can in fact be supported by a relatively simple
algorithm for optimistic concurrency control.  If goals to two
mutually dependent tables $Table_a$ and $Table_b$ are called
concurrently by two different threads, $Thread_a$ and $Thread_b$,
nothing is done until it is detected that $Table_a$ and $Table_b$ are
both incomplete and are contained in a SCC of the table dependency
grahp.  At that time, one of the threads (e.g. $Thread_a$) takes over
recomputation of all tables in the SCC, and when the SCC is completed,
any remaining answers are returned to other threads that had invoked
goals in the SCC.  While $Thread_a$ is completing this computation,
$Thread_b$ suspends until the SCC is complete.  Thus the semantics of
Shared Completed Tables supports concurrency, but does not support any
but the most coarse-grained parallelism.

Batched evaluation, on the other hand, allows answers to be returned
outside of an SCC before that SCC has been completed.  Concurrency
control for batched evaluation is similar to that for local
evaluation, except in the following case.  Assume as before that
$Table_a$, first called by $Thread_a$, and $Table_b$ first called by
$Thread_b$ are determined to be in the same SCC, and that $Thread_a$
takes over computation of subgoals in the SCC.  Now, $Thread_b$,
rather than suspending, may continue work.  In particular, $Thread_b$
can return any answers in $Table_b$ that it finds, regardless of
whether they have been produced by $Thread_b$ (before $Thread_a$ took
over the SCC) or by $Thread_a$ (afterwards).  We call this type of
concurrency semantics, {\em Table Parallelism}.  Table Parallelism can
be used to program producer-consumer examples, as well as to implement
Or- and And- parallelism.  Table Parallelism was first introduced
in~\cite{FHSW95}, but the mechanism now used for implementing Table
Parallelism differs significantly from what was described there.  In
\version{} of XSB, the implementation of Table Parallelism is
experimental: in particular, it does not yet support tabled negation.

For either semantics of shared tables, in \version{}, users of
thread-shared tables are responsible for ensuring read consistency.
Note that, in principle, thread-shared tables may depend on
thread-private tables and vice-versa.  Either type of table may depend
on thread-private or thread-shared dynamic code.  In addition, a
predicate may be {\em both} dynamic and tabled, and its clauses and
tables may be either thread-private or thread-shared.

\section{Error Conditions: Joinable and Detached Threads}

So far we have assumed that the goal called in {\tt
  xsb\_thread\_create/2} terminates normally --- by success or
failure.  But what if a thread throws an error while executing a goal?
How long should error information for a thread persist, and how can it
be checked?

Our approach relies on the semantics of pthreads, which can be either
{\em joinable} or {\em detached}.  After a joinable pthread $T_{dead}$
has terminated, status information about $T_{dead}$ persists until
some other thread joins it -- at which time the information is
removed.  On the other hand, if $T_{dead}$ is detached, status
information is removed as soon as $T_{dead}$ terminates.  A thread is
valid if it has not yet terminated, or if it is joinable and has not
yet been joined.  Reclamation of thread status information differs
from that of thread-specific data structures: upon normal or
exceptional termination of $T_{dead}$, any mutexes held by $T_{dead}$,
are released, as well as any memory automatically allocated in the
process of initializing $T_{dead}$'s, or executing its goal --
including stacks, private dynamic code, private tables, etc.
XSB-specific status information about threads follows the pthread
model: by default, error information is available when joining a
joinable thread, but not otherwise~\footnote{This behavior can, of
  course, be overridden by embedding goals within {\tt catch/3} and
  handling errors separately, or simply by adding a default user error
  handler: see Chapter~\ref{chap:exception} for details.}.

\begin{example}
Suppose the goal 
\begin{verbatim}
 ?- xsb_thread_create(functor(X,Y,Z),F).
\end{verbatim}
%
is executed.  By default, this will produce the result
%
\begin{verbatim}
X = _h113
Y = _h127
Z = _h141
F = 1++Error[XSB/Runtime/P]: [Instantiation]  in arg 2 of predicate functor/3
\end{verbatim}
%
In fact, the variable bindings are output to {\tt STDOUT}, while the
error message
%
\begin{verbatim}
++Error[XSB/Runtime/P]: [Instantiation]  in arg 2 of predicate functor/3
\end{verbatim}
%
is output to {\tt STDERR}, and may be redirected.  The call
%
\begin{verbatim}
?- xsb_thread_join(2,Error).
\end{verbatim}
%
returns
%
\begin{verbatim}
Error = error(instantiation_error, in arg 2 of predicate functor/3,
              [[Forward Continuation...,... standard:call/1,... standard:catch/3],
                Backward Continuation...]) - 0
\end{verbatim}
%
In other words, {\tt Error} is instantiated to a {\tt -/2} structure,
the first argument of which is a standard XSB error term, and the
second argument of which is the exit code of the thread.  
\end{example}
%
In general, error terms in XSB should be more useful than C exit
codes, but C exit codes may conceivably be useful if XSB is called by
one of the interfaces for C calling XSB~\footnote{C calling XSB
  interfaces are not yet supported in \version.}.  By default,
however, normal termination of an xsb thread will result in an exit
code of {\tt 0}, a cancelled thread will have an exit code of {\tt 1}
and a thread terminated with any other error will have an exit code of
{\tt 2}.

As with pthreads, XSB threads are created as joinable by default, but
can be created as detached using an option in {\tt
  xsb\_thread\_create/3}.  Alternatively, a thread created as joinable
can be made detached by {\tt xsb\_set\_detached/1}.  Both of these
predicates are described in Section~\ref{sec:mt-threading}.

\section{Thread Cancellation}
%
There may be a number of situations in which it is useful to give one
thread the ability to cancel the execution of another thread.  Within
the semantics of pthreads, this is called {\em thread cancellation}.
At the C level, thread cancellation can be tricky, as mutexes must be
released, allocated memory freed, and so on.  Accordingly, the
predicate {\tt xsb\_thread\_cancel/1} cancels XSB threads by acting
purely within the SLG-WAM engine.  When thread $T_1$ interrupts thread
$T_2$, $T_1$ writes to the XSB interrupt vector in $T_2$.  Later, when
$T_2$ checks its interrupt vector, it throws a cancellation error,
which causes it to clean up its mutexes, memory, private tables and
dynamic code, and then exit.

Any thread can cancel any other thread, with the exception that the
{\em main} thread, which controls the console (or interface to C or
interprolog) cannot be cancelled.  The main thread always has XSB
thread id {\tt 0} in both the single-threaded and multi-threaded
systems.

\section{Performance and other Considerations}
%
For running programs that do not use multiple threads, the
multi-threaded engine has a minimal overhead compared to the
single-threaded engine.  Times for single-threaded execution of Prolog
or tabled programs range from about 10-20\% slower to 10-20\% {\em
  faster} for the multi-threaded engine compared to the
single-threaded engine.  Speedups for running multiple threads on
multiple processors depends heavily on the applications run and on the
underlying operating system.

The size of a given thread may be a consideration for multi-threaded
applications.  Each thread has an area of thread-private variables
that are ``global'' to its own virtual machine, called the {\em thread
  context}, which accounts for about 2-4Kbytes of space.  Much larger
are the various stacks used by the threads for tabled and Prolog
execution.  Almost all of XSB's stacks are fully expandable, and the
initial size of several of these stacks may be set explicitly as
options in {\tt xsb\_thread\_create/3}.  Explicitly setting a default
thread stack size for an XSB thread to be smaller than the default
process stack size may be useful for applications that have a large
number of concurrently running threads.  

Other performance considerations involve the contention by threads for
shared resources.  As discussed above, contention may arise when
creating or abolishing tables, or when asserting or retracting dynamic
code --- however in either case thread-private predicates give rise to
less contention than thread-shared predicates.  In terms of I/O, each
XSB stream up to the maximum number of file descriptors has its own
mutex; as a result threads writing to different streams will not
contend for I/O.  Thus, in multi-threaded applications, it may be more
efficient to open and close streams and access these streams
explicitly, than to redirect standard input or standard output through
{\tt see/1} and {\tt tell/1}.

\section{Predicates for Multi-Threading} \label{sec:mt-threading}

\begin{description}

\ournewitem{xsb\_thread\_create(+Goal,ThreadId,,+OptionsList)}{thread}
\index{\texttt{xsb\_thread\_create/3}}
%
When called from thread $T$, this predicate creates a new XSB thread
$T_{new}$ to execute {\tt Goal}.  When goal either succeeds, throws an
unhandled error, or fails, $T_{new}$ exits, but {\tt
  xsb\_thread\_create/2} will succeed immediately, binding {\tt
  ThreadId} to the XSB thread id of $T_{new}$.  {\tt Goal} must be
callable, but need not be fully instantiated.  No bindings from {\tt
  Goal} are passed back from $T$ to $T_{new}$, so communication
between $T_{new}$ and $T$ must be through tables, asserted code, or
other side effects.

{\tt OptionList} allows optional parameters in the configuration for
the initial size of XSB stacks and to indicate whether $T_{new}$ is to
be created as detached.  Note that XSB threads allow automatic stack
allocation, so that the size options may be most useful for
applications with very large numbers of threads.  In this case,
setting initial stack sizes to be small may allow more threads to be
created on a given hardware platform.  Also note that only XSB stacks
are affected, the stack size of the underlying pthread remains
unaltered.
%
\bi
\item {\tt glsize(N)}: create thread with global (heap) plus local
  stack size initially set to {\tt N} kbytes.  If not specified, the
  default size is used (cf. Section~\ref{sec:EmuOptions}).
%
\item {\tt tcsize(N)}: create thread with trail plus choice point 
  stack size initially set to {\tt N} kbytes.  If not specified, the
  default size is used (cf. Section~\ref{sec:EmuOptions}).
%
\item {\tt complsize(N)}: create thread with completion
  stack size initially set to {\tt N} kbytes. If not specified, the
  default size is used (cf. Section~\ref{sec:EmuOptions}).
%
\item {\tt pdlsize(N)}: create thread with {\tt N} kbytes of
  unification stack.  If not specified, the
  default size is used (cf. Section~\ref{sec:EmuOptions}).
%
\item {\tt detached}: creates detached thread.  If not specified, the
  thread created will be joinable.
\ei

{\bf Error Cases}
\bi
\item 	{\tt Goal} is not callable.
\bi
\item 	{\tt type\_error(callable\_term,Goal)}
\ei
\item 	{\tt ThreadId} is not a variable
\bi
\item 	{\tt instantiation\_error}
\ei
\item   No more system threads are available (EAGAIN)
\bi
\item {\tt resource\_error('system threads')}
\ei
\item 	{\tt OptionList} is not instantiated
\bi
\item 	{\tt instantiation\_error}
\ei
\item 	{\tt OptionList} contains an option, {\tt Option} not described above
\bi
\item 	{\tt domain\_error(xsb\_thread\_create\_option,Option)}
\ei
\ei

\ournewitem{xsb\_thread\_create(+Goal,-ThreadId)}{thread}
\index{\texttt{xsb\_thread\_create/2}}
%
Acts as {\tt xsb\_thread\_create(Goal,ThreadId,[])}.

\ournewitem{xsb\_thread\_join(+ThreadIds,-ExitDesignators)}{thread}
\index{\texttt{xsb\_thread\_join/2}} 
% 
When {\tt xsb\_thread\_join/2} is called by thread $T$, {\tt
  ThreadIds} must be instantiated to either an XSB thread id, or a
list of XSB thread ids; in either case, {\tt ExitDesignators} must be
uninstantiated.  The action of the predicate is to suspend $T$ until
all of the threads denoted by {\tt ThreadIds} have exited.  At this
time, any remaining resources for the threads in {\tt ThreadIds} is
reclaimed, and {\tt ExitCode} is either an element of the form:
\begin{center}
{\tt ExitBall -  ExitCode} 
\end{center}
%
or a list of such elements.  For each XSB thread id $T_1$ in
{ThreadIds}, {\tt ExitBall} is either
\begin{itemize}
%
\item {\tt no\_error\_ball} if $T_1$ exited normally (through success or
  failure), without throwing an error or being cancelled.
%
\item An error ball if $T_1$ threw an error that was not explicitly
  handled by a catch or user error handler.
%
\item A special error ball of the form {\tt
  error(thead\_cancel,$T_1$,Backtrace)} if $T_1$ was cancelled by some
  other thread.  In this case, {\tt Backtrace} is an empty backtrace,
  added for conformability with other error forms.
\end{itemize}
%
ExitCode is either {\tt 0} for normal termination, or the value
specified by an explicit call to {\tt xsb\_thread\_exit/1}.

{\bf Error Cases}
\bi
\item 	{\tt ThreadIds} is not instantiated
\bi
\item 	{\tt instantiation\_error}
\ei
%
\item 	{\tt ThreadIds} is not a list of XSB thread ids
\bi
\item 	{\tt domain\_error(listof\_threadids,ThreadIds)}
\ei
%
\item 	{\tt ExitCodes} is not a variable
\bi
\item 	{\tt instantiation\_error}
\ei
%
\item   {\tt ThreadId} does not correspond to a valid thread
\bi
\item   {\tt existence\_error(valid\_thread,ThreadId)}
\ei
\item   {\tt ThreadId} does not correspond to a joinable thread
  (i.e. {\tt ThreadId} is detached).
\bi
\item   {\tt permission\_error(join,non\_joinable\_thread,ThreadId)}
\ei
\ei

\ournewitem{xsb\_thread\_exit(+ExitCode)}{thread}
\index{\texttt{xsb\_thread\_exit/1}} 
%
Exits a thread $T$ with {\tt ExitCode} after releasing any mutexes
held by $T$, freeing any thread-specific memory allocated for $T$ (we
hope).  {\tt ExitCode} will be used if the caller of $T$ joins to $T$,
but will be ignored in other cases.  There is no need to call this
routine on normal termination of a thread as it is called implicitly
on success or (final) failure of a thread's goal.

{\bf Error Cases}
\bi
\item 	{\tt ThreadId} is not an integer
\bi
\item 	{\tt type\_error(integer,ExitCode)}
\ei
\ei

\ournewitem{xsb\_thread\_self(-ThreadId)}{thread}
\index{\texttt{xsb\_thread\_self/1}} 
%
Returns the XSB thread id of the calling thread.

{\bf Error Cases}
\bi
\item 	{\tt ThreadId} is not a variable
\bi
\item 	{\tt instantiation\_error}
\ei
\ei

\ournewitem{xsb\_thread\_detach(+ThreadId)}{thread}
\index{\texttt{xsb\_thread\_detach/1}} 
%
Detaches a joinable thread denoted by {\tt ThreadId} so that all
resources will be reclaimed upon its exit.  The thread denoted by {\tt
  ThreadId} will no longer be joinable, once it is detached.

Note, if it is known at thread creation time that a thread should be
detached, it is better to create the thread as detached via {\tt
  xsb\_thread\_create/3}.

{\bf Error Cases}
\bi
\item 	{\tt ThreadId} is not an integer
\bi
\item 	{\tt type\_error(integer,ExitCode)}
\ei
\item   {\tt ThreadId} does not correspond to a valid thread
\bi
\item   {\tt existence\_error(valid\_thread,ThreadId)}
\ei
\item 	{\tt ThreadId} is active but not joinable
\bi
\item 	{\tt permission\_error(thread\_detach,non\_joinable\_thread,ThreadId)}
\ei
\ei

\ournewitem{xsb\_thread\_cancel(+ThreadId)}{thread}
\index{\texttt{xsb\_thread\_cancel/1}} 
%
Cancels the XSB thread denoted by {\tt ThreadId}.  The cancellation
does not use pthread cancellation mechanisms, rather it uses XSB's
interrupt mechanism to set {\tt ThreadId}'s interrupt vector.  When
this interrupt vector is checked, {\tt ThreadId} will throw a thread
cancellation error, which can be caught within {\tt ThreadId} like any
other error.  Usually, this means that {\tt ThreadId} will abort and
exit.  Note that the high-level upon which thread cancellation is
implemented allows for simplicity and portability of thread
cancellation.  \footnote{While interrupt vector is not checked upon
  every SLG-WAM instruction, it is checked on every {\sf call} and
  {\sf execute} instruction, and should should be suitable for almost
  any need for cancellation.}.

The main XSB thread cannot be cancelled; apart from that any thread
can cancel any other thread.

{\bf Error Cases}
\bi
\item 	{\tt ThreadId} is not instantiated
\bi
\item 	{\tt instantiation\_error}
\ei
%
\item   {\tt ThreadId} does not correspond to valid thread
\bi
\item   {\tt existence\_error(valid\_thread,ThreadId)}
\ei
\item   {\tt ThreadId} denotes the main thread.
\bi
\item   {\tt permission\_error(cancel,main\_thread,ThreadId)}
\ei
\ei

\ournewitem{xsb\_thread\_yield}{thread}
\index{\texttt{xsb\_thread\_yield/0}} 
% 
Make the calling thread ready to be run {\em after} other threads of
the same priority.  This predicate relies on the real-time extensions
to pthreads specified in Posix 1b, and may not be available on all
platforms.

{\bf Error Cases}
\bi
\item 	The current platform does not support Posix real-time extensions
\bi
\item 	{\tt misc\_error}
\ei
\ei

\comment{
\ournewitem{xsb\_user\_mutex\_lock(MutexId)}{thread}
\index{\texttt{xsb\_user\_mutex\_lock/1}} 
%
Locks a user-mutex, currently numbered 0 through 9, with 0
corresponding to the mutex that controls I/O to the console.  These
mutexes are initialized automatically at process startup, and in
addition each of the mutexes have a {\tt pthread\_mutexattr\_t} type
of {\tt NORMAL}.

{\bf Error Cases}
\bi
\item 	{\tt MutexId} is not an integer
\bi
\item 	{\tt type\_error(integer,MutexId)}
\ei
\item   {\tt MutexId} does is not the index of a user mutex
\bi
\item 	{\tt domain\_error(mutex\_id,MutexId)}
\ei
\item 	Locking {\tt MutexId} would lead to a deadlock
\bi
\item 	{\tt permission\_error('mutex lock','deadlocking mutex',MutexId)}
\ei
\ei

\ournewitem{xsb\_user\_mutex\_trylock(MutexId)}{thread}
\index{\texttt{xsb\_user_mutex_trylock/1}} 
%
If the user-mutex denoted by {\tt MutexId} is not busy, this predicate
locks {\tt MutexId} and succeeds; if {\tt MutexId} is busy the
predicate fails, rather than waiting for the lock as with {\tt
  xsb\_user\_mutex\_trylock/1}

{\bf Error Cases}
\bi
\item 	{\tt MutexId} is not an integer
\bi
\item 	{\tt type\_error(integer,MutexId)}
\ei
\item   {\tt MutexId} does is not the index of a user mutex
\bi
\item 	{\tt domain\_error(mutex\_id,MutexId)}
\ei
\ei

\ournewitem{xsb\_user\_mutex\_unlock/0}{thread}
\index{\texttt{xsb\_user\_mutex\_unlock/1}} 
%
Unlocks a user mutex (user mutexes are described above).

{\bf Error Cases}
\bi
\item 	{\tt MutexId} is not an integer
\bi
\item 	{\tt type\_error(integer,MutexId)}
\ei
\item   {\tt MutexId} does is not the index of a user mutex
\bi
\item 	{\tt domain\_error(mutex\_id,MutexId)}
\ei
\item 	The thread calling {\tt xsb\_user\_mutex\_unlock/1} does not
  own the lock for {\tt MutexId} 
\bi
\item 	{\tt permission\_error('mutex unlock',mutex,MutexId)}
\ei
\ei
}
\end{description}
