\chapter{Multi-Threaded Programming in XSB} \label{chap:threads}

Starting from Version 3.0, XSB supports the use of Posix threads to
perform separable computations, and in certain cases to parellelize
them.  Posix threads have a simple and clear API, and are available on
all Unixes as well as on Windows via ???.  This chapter introduces how
to program with threads in XSB through a series of examples.  Later
sections discuss performance aspects of our implementation as well as
describing relevant predicates.  A general knowledge of multi-threaded
programming is assumed, such as can be found in \cite{LewB98}.

\comment{
\section{Getting Started with Multi-Threading}
%
In Version 3.0 the default configuration of XSB does not include
multi-threading.  This is partly because multi-threading is new, and
despite our efforts, the multi-threaded engine may contain bugs not
present in the single-threaded engine.  In addition, the
multi-threaded engine is about 5-20\% slower than the single-threaded
engine due to its use of mutexes and its greater use of dynamic,
rather than static, C variables.  Finally, in Version 3.0, not all
packages have yet been made thread-safe so that not all configurations
are supported with multi-threading.  Currently ODBC is supported with
the multi-threaded configuration, but {\tt dbdrivers}, {\tt xpath},
{\tt iodbc}, {\tt interprolog} and {\tt smodels} have not been. (???)

With this in mind, making the multi-threaded engine is simple:
configure and make XSB as in Chapter~\ref{}, but include the command
{\tt --enable-mt}.  When you invoke the newly made configuration of
XSB you should see {\tt engine: multi-threading} in the configuration
list below the banner.

\paragraph{Hello World for Beginners}
%
We naturally start with a hello world program.  Within the
multi-threaded engine, import {\tt xsb\_thread\_create/2} from the
module thread, and type the command 
\begin{center} 
{\tt ?- xsb\_thread\_create(writeln('hello world'),Id)} 
\end{center} 
you should see something like 
\begin{center} 
{\tt Id = 1hello world} 
\end{center} 
%
while the output is a little ugly, the simple ``hello world'' program
does illustrate simple multi-threading at work.  The calling thread
(i.e. the thread controlling the command-line interpreter which we
call $T_{prompt}$) executes the predicate {\tt xsb\_thread\_create/2}
which creates a thread $T_{child}$ and immediately returns with the
{\em XSB thread id} of the created thread.  Meanwhile, $T_{child}$
initializes its stacks and other memory areas and executes the goal
{\tt writeln('hello world')}.  $T_{child}$ and $T_{prompt}$ share most
of their process-level information: in particular they share a common
I/O stream for standard output, leading to the output above.  What is
happening may be seen a little more easily by executing the command
%
\begin{center} 
{\tt ?- xsb\_thread\_create((sleep(1),writeln('hello world')),Id)} 
\end{center} 
%
In this case the interpreter reports that {\tt F} is bound to a thread
id, then about a second later {\tt writeln/1} is executed.  

The simple ``hello world'' program illustrates a couple of points.
First, it is easy to create a thread in XSB and have that thread do
work.  Second, it can be tricky to coordinate actions among threads.
We'll explore these two themes in more detail, but first suppose we
are determined to extend out multi-threaded program so that it
produces good output.  One way to do this is to {\em join}
$T_{prompt}$ and $T_{child}$ as follows 
%
\begin{verbatim}
?- xsb_thread_create(writeln('hello world'),Id),
   xsb_thread_join(Id,ExitCode).
hello world

Id = 1
ExitCode = 0
\end{verbatim}
%
In this case, as soon as $T_{prompt}$ has issued a command to create
$T_{child}$, it executes {\tt xsb\_thread\_join/2}.  This later
predicate makes a system call to the underlying operating system to
suspend $T_{prompt}$ until $T_{child}$ has exited.  {\tt
  xsb\_thread\_join/2} returns the exit code of $T_{child}$: as usual,
a value of {\tt 0} indicates normal termination.  

\paragraph{Multi-threading Concepts}

The previous section introduced some new concepts that were not fully
discussed.  First was the concept of an {\em XSB thread id}: XSB
manages up to {\tt MAXTHREAD} active threads using XSB thread ids.  It
should be noted that the XSB thread id of a thread is different from
the identifier of the underlying pthread.  An XSB thread id is a
Prolog term, and unlike Posix thread ids, XSB thread ids can be
compared for equality using unification.  The actual form of an XSB
thread id, however, is subject to change between versions, so programs
should not make use of the exact form of an XSB thread id.  In the
multi-threaded engine, the XSB thread id of any thread can be queried
using the predicate {\tt xsb\_thread\_self/1}

{\sc talk at some point about exit codes and error terms}
}

\section{Predicates for Multi-Threading}

\begin{description}

\ournewitem{xsb\_thread\_create(+Goal,ThreadId,,+OptionsList)}{thread}
\index{\texttt{xsb\_thread\_create/3}}
%
Called from thread $T$, creates a new XSB thread $T_{new}$ to execute
{\tt Goal}: when goal either succeeds or fails, $T_{new}$ exits, but
{\tt xsb\_thread\_create/2} will succeed immediately, binding {\tt
  ThreadId} to the XSB thread id of $T_{new}$.  {\tt Goal} must be
callable, but need not be fully instantiated.  No bindings from {\tt
  Goal} are passed back from $T$ to $T_{new}$, so communication
between $T_{new}$ and $T$ must be through tables, asserted code, or
other side effects.

{\tt OptionList} allows optional parameters in the configuration for
the initial size of XSB stacks and to indicate whether the thread is
to be created as detached.  Note that XSB threads allow automatic
stack allocation, so that the size options may be most useful for
applications with very large numbers of threads.  In this case,
setting initial stack sizes to be small may allow more threads to be
created on a given hardware platform.  Also note that only XSB stacks
are affected, the stack size of the underlying pthread remains
unaltered.
%
\bi
\item {\tt glsize(N)}: create thread with global (heap) plus local
  stack size initially set ot {\tt N} kbytes.  If not specified, the
  default size is used (cf. Section~\ref{sec:EmuOptions}).
%
\item {\tt tcsize(N)}: create thread with trail plus choice point 
  stack size initially set ot {\tt N} kbytes.  If not specified, the
  default size is used (cf. Section~\ref{sec:EmuOptions}).
%
\item {\tt complsize(N)}: create thread with completion
  stack size initially set ot {\tt N} kbytes. If not specified, the
  default size is used (cf. Section~\ref{sec:EmuOptions}).
%
\item {\tt pdlsize(N)}: create thread with {\tt N} kbytes of
  unification stack.  If not specified, the
  default size is used (cf. Section~\ref{sec:EmuOptions}).
%
\item {\tt detached}: creates detached thread.  If not specified, the
  thread created will be joinable (and can later be set as detached by
  {\tt xsb\_thread\_detach/1}.
\ei

{\bf Error Cases}
\bi
\item 	{\tt Goal} is not callable.
\bi
\item 	{\tt type\_error(callable\_term,Goal)}
\ei
\item 	{\tt ThreadId} is not a variable
\bi
\item 	{\tt instantiation\_error}
\ei
\item   No more system threads are available (EAGAIN)
\bi
\item {\tt resource\_error('system threads')}
\ei
\item 	{\tt OptionList} is not instantiated
\bi
\item 	{\tt instantiation\_error}
\ei
\item 	{\tt OptionList} contains an option, {\tt Option} not described above
\bi
\item 	{\tt domain\_error(xsb\_thread\_create\_optoin,Option)}
\ei
\ei

\ournewitem{xsb\_thread\_create(+Goal,-ThreadId)}{thread}
\index{\texttt{xsb\_thread\_create/2}}
%
Acts as {\tt xsb\_thread\_create(Goal,ThreadId,[])}.

\ournewitem{xsb\_thread\_join(+ThreadIds,-ExitCodes)}{thread}
\index{\texttt{xsb\_thread\_join/2}} 
% 
When {\tt xsb\_thread\_join/2} is called by thread $T$, {\tt
  ThreadIds} must be instantiated to either an XSB thread id, or a
list of XSB thread ids and {\tt ExitCodes} must be uninstantiated.
The action of the predicate is to suspend $T$ until all of the threads
denoted by {\tt ThreadIds} have exited: the exact semantics relies on
the {\sc pthread} join operation.

{\bf Error Cases}
\bi
\item 	{\tt ThreadIds} is not instantiated
\bi
\item 	{\tt instantiation\_error}
\ei
%
\item 	{\tt ThreadIds} is not a list of XSB thread ids
\bi
\item 	{\tt domain\_error(listof\_threadids,ThreadIds)}
\ei
%
\item 	{\tt ExitCodes} is not a variable
\bi
\item 	{\tt instantiation\_error}
\ei
%
\item   {\tt ThreadId} does not correspond to a live thread
\bi
\item   {\tt existence\_error(live\_thread,ThreadId)}
\ei
\item   {\tt ThreadId} does not correspond to a joinable thread
  (i.e. {\tt ThreadId} is detached).
\bi
\item   {\tt permission\_error(join,non\_joinable\_thread,ThreadId)}
\ei
\ei

\ournewitem{xsb\_thread\_exit(+ExitCode)}{thread}
\index{\texttt{xsb\_thread\_exit/1}} 
%
Exits a thread $T$ after releasing any mutexes held by $T$, freeing
any thread-specific memory allocated for $T$ (we hope) and exits with
{\tt ExitCode}.  This exit code will be used if the caller of $T$
joins to $T$, but will be ignored in other cases (???).  There is no
need to call this routine on normal termination of a thread as it is
called implicitly on success or (final) failure of a thread's goal.

{\bf Error Cases}
\bi
\item 	{\tt ThreadId} is not an integer
\bi
\item 	{\tt type\_error(integer,ExitCode)}
\ei
\ei

\ournewitem{xsb\_thread\_self(-ThreadId)}{thread}
\index{\texttt{xsb\_thread\_self/1}} 
%
Returns the XSB thread id of the calling thread.

{\bf Error Cases}
\bi
\item 	{\tt ThreadId} is not a variable
\bi
\item 	{\tt instantiation\_error}
\ei
\ei

\ournewitem{xsb\_thread\_detach(+ThreadId)}{thread}
\index{\texttt{xsb\_thread\_detach/1}} 
%
Detaches the threadd denoted by {\tt ThreadId} so that all resources
will be reclaimed upon its exit.  The thread denoted by {\tt ThreadId}
will no longer be joinable, once it is detached.

{\bf Error Cases}
\bi
\item 	{\tt ThreadId} is not an integer
\bi
\item 	{\tt type\_error(integer,ExitCode)}
\ei
\item   {\tt ThreadId} does not correspond to a live thread
\bi
\item   {\tt existence\_error(live\_thread,ThreadId)}
\ei
\item 	{\tt ThreadId} is active but not joinable
\bi
\item 	{\tt permission\_error(thread\_detach,non\_joinable\_thread,ThreadId)}
\ei
\ei

\ournewitem{xsb\_thread\_yield}{thread}
\index{\texttt{xsb\_thread\_yield/0}} 
% 
Make the calling thread ready to be run {\em after} other threads of
the same priority.  This predicate relies on the real-time extensions
to pthreads specified in Posix 1b, and may not be available on all
platforms.

{\bf Error Cases}
\bi
\item 	The current platform does not support Posix real-time extensions
\bi
\item 	{\tt misc\_error}
\ei
\ei

\ournewitem{xsb\_user\_mutex\_lock(MutexId)}{thread}
\index{\texttt{xsb\_user\_mutex\_lock/1}} 
%
Locks a user-mutex, currently numbered 0 through 9, with 0
corresponding to the mutex that controls I/O to the console.  These
mutexes are initalized automatically at process startup, and In
addition each of the mutexes have a {\tt pthread\_mutexattr\_t} type
of {\tt NORMAL}.

{\bf Error Cases}
\bi
\item 	{\tt MutexId} is not an integer
\bi
\item 	{\tt type\_error(integer,MutexId)}
\ei
\item   {\tt MutexId} does is not the index of a user mutex
\bi
\item 	{\tt domain\_error(mutex\_id,MutexId)}
\ei
\item 	Locking {\tt MutexId} would lead to a deadlock
\bi
\item 	{\tt permission\_error('mutex lock','deadlocking mutex',MutexId)}
\ei
\ei

\ournewitem{xsb\_user\_mutex\_trylock(MutexId)}{thread}
\index{\texttt{xsb\_user_mutex_trylock/1}} 
%
If the user-mutex denoted by {\tt MutexId} is not busy, this predicate
locks {\tt MutexId} and succeeds; if {\tt MutexId} is busy the
predicate fails, rather than waiting for the lock as with {\tt
  xsb\_user\_mutex\_trylock/1}

{\bf Error Cases}
\bi
\item 	{\tt MutexId} is not an integer
\bi
\item 	{\tt type\_error(integer,MutexId)}
\ei
\item   {\tt MutexId} does is not the index of a user mutex
\bi
\item 	{\tt domain\_error(mutex\_id,MutexId)}
\ei
\ei

\ournewitem{xsb\_user\_mutex\_unlock/0}{thread}
\index{\texttt{xsb\_user\_mutex\_unlock/1}} 
%
Unlocks a user mutex (user mutexes are described above).

{\bf Error Cases}
\bi
\item 	{\tt MutexId} is not an integer
\bi
\item 	{\tt type\_error(integer,MutexId)}
\ei
\item   {\tt MutexId} does is not the index of a user mutex
\bi
\item 	{\tt domain\_error(mutex\_id,MutexId)}
\ei
\item 	The thread calling {\tt xsb\_user\_mutex\_unlock/1} does not
  own the lock for {\tt MutexId} 
\bi
\item 	{\tt permission\_error('mutex unlock',mutex,MutexId)}
\ei
\ei

\end{description}
