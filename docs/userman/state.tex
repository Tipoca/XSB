\section{Information about the System State} \label{State}
%========================================================
\index{state of the system} \index{system, state of}

Various aspects of the state of an instance of XSB --- information
about what predicates, modules, or dynamic clauses have been loaded,
their object files, along with other information can be inspected in
ways similar to many Prolog systems.  However, because the atom-based
module system of XSB may associate structures with particular modules,
predicates are provided to inspect these elements as well.  The
following descriptions of {\em state} predicates use the terms {\em
predicate indicator}, {\em term indicator} and {\em current module} to
mean the following:
\begin{itemize}
\item By {\em predicate indicator} \index{predicate indicator} we mean a
      {\em compound term} of the form {\tt M:F/A} or simply {\tt F/A}.
      When the predicate indicator is fully instantiated, {\tt M} and {\tt F}
      are atoms representing the {\em module name} and the {\em functor} 
      of the predicate respectively and {\tt A} is a non negative integer 
      representing its {\em arity}.

      Example: {\tt usermod:append/3}
\item By {\em term indicator} \index{term indicator} we mean a predicate or
      function symbol of arity N followed by a sequence of N variables
      (enclosed in parentheses if N is greater than zero).  A term indicator
      may optionally be prefixed by the module name, thus it can be of the
      form {\tt M:Term}.

      Example: {\tt usermod:append(\_,\_,\_)}
\item A module {\tt M} becomes a {\em current (i.e. ``known'') module} as
      soon as it is loaded in the system or when another module that is
      loaded in the system imports some predicates from module {\tt M}.

      Note that due to the dynamic loading of \ourprolog, a module can be 
      current even if it has not been loaded, and that some predicates of 
      that module may not be defined. In fact, a module can be current even
      if it does not exist.  This situation occurs when a predicate is 
      improperly imported from a non-existent module.  Despite this, 
      a module can never lose the property of being {\em current}.
\end{itemize}

\begin{description}

\ournewitem{current\_input(?Stream)}{ISO}\index{{\tt current\_input/1}}
    Succeeds iff stream {\tt Stream} is the current input stream, or 
    procedurally unifies {\tt Stream} with the current input stream.
    There are no error conditions for this predicate.

\ournewitem{current\_output(?Stream)}{ISO}\index{{\tt current\_output/1}}
    Succeeds iff stream {\tt Stream} is the current output stream, or 
    procedurally unifies {\tt Stream} with the current output stream.
    There are no error conditions for this predicate.

\ouritem{current\_module(?Module)}\index{{\tt current\_module/1}}
    The standard predicate {\tt current\_module/1} allows the user to check 
    whether a given module is {\em current} or to generate (through 
    backtracking) all currently known modules.
    Succeeds iff {\tt Module} is one of the modules in the database. This
    includes both user modules and system modules.  

    Note that predicate {\tt current\_module/1} succeeds for a given module 
    even if that module is not a real module (in the sense taht it does not
    export any predicates). There are no error conditions associated 
    with this predicate; if its argument does not unify with one of the 
    current modules, {\tt current\_module/1} simply fails.

\ouritem{current\_module(?Module, ?ObjectFile)}\index{{\tt current\_module/2}}
    Predicate {\tt current\_module/2} gives the relationship between the
    modules and their associated object file names. The file name 
    {\tt ObjectFile} must be absolute and always ends with {\tt '.O'}.

    It is possible for a current module to have no associated file name 
    (as is the case for modules like {\tt "usermod"} and {\tt "global"}), 
    or for the system to be unable to determine the file name of a current
    module. In both cases, predicate {\tt current\_module/1} will succeed
    for this module, while {\tt current\_module/2} will fail. The system is
    unable to determine the file name of a given module if that module is not
    in one of the directories of the search path (see Section~\ref{LibPath}).
    Once again, there are no error conditions associated with this predicate;
    if the arguments of {\tt current\_module/2} are not correct, or 
    {\tt Module} has no associated {\tt File}, the predicate will simply fail.

\ouritem{current\_atom(?Atom\_Indicator)}\index{{\tt current\_atom/1}}
    Generates (through backtracking) all currently known atoms, and unifies
    each in turn with {\tt Atom\_Indicator}. 

%\ouritem{current\_atom(?Atom)}\index{{\tt current\_atom/2}}
%\predindex{\$current\_atom/2~(B)}
%    However, atoms considered ``internal'' symbols,
%    i.e.\ those whose names begin with {\bf \$} or {\bf \_\$} are not returned.
%    The intrepid user who wishes to access such internal atoms as
%    well can use the goal
%    \begin{center}
%    {\tt ?- \$current\_atom({\tt Atom}, 1).}
%    \end{center}

\ouritem{current\_functor(?Predicate\_Indicator)}
\index{{\tt current\_functor/1}}
    The standard predicate {\tt current\_functor/1} can be used to find all
    the currently known terms appearing in a particular current module. 
    It succeeds iff {\tt Predicate\_Indicator} is a predicate 
    indicator for any term that appears in the database. Note that 
    this includes terms both in system and in user defined modules, even 
    terms that may be not yet loaded in the system. The behaviour of
    {\tt current\_functor/1} may be contrasted with that of
    {\tt current\_predicate/1}, which reports only those predicates which 
    have been loaded in the system (both Prolog and foreign predicates) or
    are dynamic predicates.
    There are no error conditions associated with this predicate; if its 
    argument is not a predicate indicator the predicate simply fails.

    Predicate {\tt current\_functor/1} comes in two flavours depending on 
    the form of its argument ({\tt Predicate\_Indicator}):
    \begin{enumerate}
    \item If {\tt Predicate\_Indicator} is of the form 
	  {\tt Module:Functor/Arity}, then the execution of
	  {\tt current\_functor/1} will backtrack through all
	  the current modules of the system (user defined, system
	  defined and global modules). 
    \item If, however, {\tt Predicate\_Indicator} is uninstantiated 
	  or has the form {\tt Functor/Arity}, then predicate 
	  {\tt current\_functor/1} backtracks only through the terms appearing 
	  in the global modules of the system (in other words searches only 
	  modules {\tt "usermod"} and {\tt "global"}). This flavour is only 
	  for convenience, since this is the common use of predicate 
	  {\tt current\_functor/1}. Note that all the following are equivalent:

	  \stuff{
          \>	\>	| ?- current\_functor(Functor/Arity).     \\
          \>	\>	| ?- current\_functor(Predicate).         \\
          \>	\>	| ?- current\_functor(usermod:Predicate). \\
          \>	\>	| ?- current\_functor(global:Predicate).  
          }
    \end{enumerate}

    So, to backtrack through all of the functors of positive arity (function
    and predicate symbols) that appear in the global modules of the system
    regardless of whether they are system or a user defined, use:

    \stuff{
    \> \>	| ?- current\_functor(Functor/Arity), Arity > 0.
    }

\ouritem{current\_functor(?Name, ?Term\_Indicator)}
\index{{\tt current\_functor/2}}
    Succeeds iff {\tt Term\_Indicator} is the most general term 
    corresponding to one of the currently known terms having {\tt Name} 
    as their functor appearing in a current module.  (Both system and user 
    defined modules are checked).  Or procedurally, 
    {\tt current\_functor/2} unifies {\tt Name} with the name of a functor 
    known to the database, and {\tt Term\_Indicator} with the most 
    general term corresponding to that functor. The flavours of this predicate 
    are analogous to the ones of {\tt current\_functor/1} according to 
    whether {\tt Term\_Indicator} has one of the following two forms:
    \begin{enumerate}
    \item{\tt Module:Term.}
    \item{\tt Term} (for global modules).
    \end{enumerate}
    If {\tt Term\_Indicator} is uninstantiated, then this predicate succeeds
    only for global modules.  
    As in {\tt current\_functor/1} even unloaded predicates are reported
    (if they have been imported and are are known to the database).

    For example, if a predicate {\tt foo/2} and and a function symbol 
    {\tt foo/1} are defined into module {\tt blah}, the following query 
    will return:

    \stuff{
    \>	\>	| ?- current\_functor(foo, blah:Term).	\\
    \>  \>						\\
    \>	\>      Term = foo(\_638788,\_638792);		\\
    \>  \>						\\
    \>	\>	Term = foo(\_638788);			\\
    \>  \>						\\
    \>	\>	no
    }

    If a module is specified, {\tt current\_functor/2} succeeds only for
    those functors (function and predicate symbols) which are defined in 
    that module. Unless the module is one of the global modules, 
    {\tt current\_functor/2} fails for the predicates which are imported 
    into that module.

    On the other hand, the goal:

    \stuff{
    \>	\>	| ?- current\_functor(Name, Term).
    }

    can be used to backtrack through every known term {\tt Term} in the global
    modules of \ourprolog's database that has {\tt Name} as its functor.

    Note that the order of term generation is undetermined. Once again, 
    there are no error conditions associated with this predicate; if its
    arguments are inappropriate, the predicate simply fails.

\ouritem{current\_predicate(?Predicate\_Indicator)}
\index{{\tt current\_predicate/1}}
    The predicate {\tt current\_predicate/1} can be used to find all
    the predicates that are defined and loaded in a particular current module. 
    The module can be either a Prolog or a foreign module (see the
Chapter {\it Foreign Language Interface} in Volume 2.
    This predicate succeeds iff {\tt Predicate\_Indicator} is a 
    predicate indicator for one of the procedures (both Prolog and foreign 
    language ones) that are loaded in the database or that are dynamic. Note
    that this includes procedures both in system and in user defined modules. 
    Unlike {\tt current\_functor/1} which reports all predicates that are
    somehow known to the database, {\tt current\_predicate/1} reports only
    those predicates that are either created dynamically (for example using 
    {\tt assert/1}) or loaded in the system.  (I.e. it excludes those
    predicates which have been imported, but not loaded).
    There are no error conditions associated with this predicate; if its 
    argument is not what it should be, the predicate simply fails.

    Like {\tt current\_functor/1}, predicate {\tt current\_predicate/1}
    comes in two flavours depending on  the form of its argument 
    ({\tt Predicate\_Indicator}).
    \begin{enumerate}
    \item If {\tt Predicate\_Indicator} has the form 
	  {\tt Module:Functor/Arity}, then the execution of
	  {\tt current\_predicate/1} unifies the predicate indicator
	  with predicates in all current modules (user defined, system
	  defined and global modules). 
    \item If, however, {\tt Predicate\_Indicator} is uninstantiated 
	  or has the form {\tt Functor/Arity}, then {\tt current\_predicate/1} 
	  backtracks only through the predicates loaded in the global modules 
	  of the system (in other words searches only modules {\tt "usermod"}
	  and {\tt "global"}).
	  Since this is the common use of predicate {\tt current\_predicate/1}, 
	  this flavour is only for convenience.
	  Note that all the following are equivalent:

	  \stuff{
          \>	\>	| ?- current\_predicate(Functor/Arity).     \\
          \>	\>	| ?- current\_predicate(Predicate).         \\
          \>	\>	| ?- current\_predicate(usermod:Predicate). \\
          \>	\>	| ?- current\_predicate(global:Predicate).  
          }
    \end{enumerate}

    So, to backtrack through all of the predicates defined and loaded in module 
    {\tt blah}, regardless of whether {\tt blah} is a system or a user 
    defined module
    \footnote{The only limitation is that {\tt blah} must indeed 
	      be a module in the sense that it exports at least one symbol.}
    , use:

    \stuff{
    \> \>	| ?- current\_predicate(blah:Predicate).
    }

    In this case {\tt Predicate} will have the form: {\tt Functor/Arity}.

    To backtrack through all predicates defined and loaded in any current 
    module, use:

    \stuff{
    \> \>	| ?- current\_predicate(Module:Functor/Arity).
    }

    This succeeds once for every predicate that is loaded in \ourprolog's 
    database.

    To find the predicates having arity 3 that are loaded in the global 
    modules of the system, use: 

    \stuff{
    \> \>	| ?- current\_predicate(Functor/3).
    }

    while to find all predicates loaded in the global modules of the system
    regardless of their arity, use:

    \stuff{
    \> \>	| ?- current\_predicate(Predicate). \\
    }


\ouritem{current\_predicate(?Name, ?Term\_Indicator)}
\index{{\tt current\_predicate/2}}
    Succeeds iff {\tt Term\_Indicator} is the most general term 
    corresponding to one of the predicates having functor {\tt Name} that are 
    defined and loaded in a particular module in the database. 
    (The module can be either system or user defined).
    Or procedurally, {\tt current\_predicate/2}
    unifies {\tt Name} with the name of a loaded predicate, and 
    {\tt Term\_Indicator} with the most general term corresponding to that
    predicate.  The flavours of this predicate are analogous to those of 
    {\tt current\_predicate/1} and behave according to whether 
    {\tt Term\_Indicator} has one of the following two forms:
    \begin{enumerate}
    \item{\tt Module:Term.}
    \item{\tt Term} (module is assumed to be global or usermod).
    \end{enumerate}
    If {\tt Term\_Indicator} is uninstantiated, then this predicate succeeds
    only for global modules. Like {\tt current\_predicate/1} only 
    predicates that have a property in the following set:
    \begin{center}
    {\tt $\{$ loaded, dynamic, foreign $\}$ }
    \end{center}
    (see {\tt predicate\_property/2} below) are reported.

    For example, if predicates {\tt foo/1} and {\tt foo/3} are defined and
    loaded into module {\tt blah}, the following query will return:

    \stuff{
    \>	\>	| ?- current\_predicate(foo, blah:Term).\\
    \>  \>						\\
    \>	\>      Term = foo(\_638788,\_638792,\_638796);	\\
    \>  \>						\\
    \>	\>	Term = foo(\_638788);			\\
    \>  \>						\\
    \>	\>	no
    }

    If a module is specified, {\tt current\_predicate/2} succeeds only for
    those predicates which are defined and loaded in that module. Unless 
    the module is one of the global modules, {\tt current\_predicate/2} fails
    for those predicates which are imported into that module.

    On the other hand, the goal:

    \stuff{
    \>	\>	| ?- current\_predicate(Name, Term).
    }

    can be used to backtrack through every predicate that is loaded in the
    global modules of \ourprolog's database.

    Note that the order of term generation is undetermined. Once again, 
    there are no error conditions associated with this predicate; if its
    argument is not what it should be, the predicate simply fails.

%??? need to define visible.
\label{PredProp}
\ouritem{predicate\_property(?Term\_Indicator, ?Property)}
\index{{\tt predicate\_property/2}}
    The standard predicate {\tt predicate\_property/2} can be used to find 
    the properties of any predicate that is visible to a particular module.
    Succeeds iff {\tt Term\_Indicator} is a term indicator for a current 
    predicate whose principal functor is a predicate having {\tt Property} 
    as one of its properties. Or procedurally, {\tt Property} is unified 
    with the currently known properties of the predicate having 
    {\tt Term\_Indicator} as its skeletal specification.
    
    A brief description of {\tt predicate\_property/2} is as follows:
    \begin{itemize} 
\item If {\tt Term\_Indicator} is not a variable, and is a structure
	or atom, then {\tt Property} is successively unified with the
	various properties associated with {\tt Term\_Indicator}.  If
	{\tt Term\_Indicator} is not a known to the system, the call
	succeeds with {\tt Property} successively unified to {\tt
	exported} and {\tt unclassified}.  These properties can be
	considered as a default for any structure or atom.
\item If {\tt Property} is bound to a valid predicate property, then {\tt
	predicate\_property/2} successively unifies {\tt
	Term\_Indicator} with the skeletal specifications of all
	predicates known to the system having the specified {\tt
	Property}.  
\item If {\tt Term\_Indicator} is a variable, then
	it is unified (successively through backtracking) with the
	most general term for a predicate whose known properties are
	unified with {\tt Property}.  
\item If {\tt Term\_Indicator}
	is not a term indicator, or if {\tt Property} is not a valid
	predicate property, the call fails.  
\end{itemize} 
\noindent
For example, all the loaded predicate skeletal specifications in
module {\tt "usermod"} may be enumerated using:

    \stuff{
    \>   \>	| ?- predicate\_property(Pred, loaded).
    }

    Also the following query finds all predicate skeletal specifications that 
    are exported by module {\tt blah}:

    \stuff{
    \>   \>	| ?- predicate\_property(blah:Pred, exported).
    }

    Currently, the following properties are associated with predicates 
    either implicitly or by declaration.  Double lines show property
    categories, and a predicate can have at most one property of each
    category.

    \begin{center}
    \begin{tabular}{||l|l||}               \hline 
	{\em Property}		& {\em Explanation} \\ \hline \hline
	{\tt unclassified}	& 
		The predicate symbol is not yet classified according  \\
	&	to this category. This property has various meanings. \\ 
	&	Usually for exported predicate symbols in system or   \\
	&	user defined modules it means that the predicate is   \\
	&	yet unloaded (because it has not been used).	      \\
	&	In global modules it usually means that the predicate \\
	&	is either a function symbol, or an unloaded predicate \\
	&	symbol (including constants). \\ \hline
	{\tt dynamic}	& 
		The predicate is dynamic. \\ \hline
	{\tt loaded}		& 
		The predicate (including internal predicates) is a \\ 
	&	Prolog predicate loaded into the module in question; \\
	&	this is always the case for predicates in global modules.\\ 
		\hline
	{\tt unloaded}		& 
		The predicate is yet unloaded into the module \\
	&	in question.\\ \hline
	{\tt foreign}	& 
		The predicate is a foreign predicate. This implies that  \\ 
	&	the predicate is already loaded in the system, because	 \\
	&	currently there is no way for XSB  to know that a \\
	&	predicate is a foreign predicate until it is loaded in	 \\
	&	the system.\\ \hline
%	{\tt function}	& 
%		The predicate symbol is declared as a function. \\ \hline
	\hline
	{\tt exported}	&
		The predicate symbol is exported by the module in \\ 
	&	question; in other words the predicate symbol is \\
	&	visible to any other module in the system. \\ \hline
	{\tt local}	& 
		The predicate symbol is local to the module \\
	&	in question. \\ \hline
	{\tt imported\_from(Mod)}	& 
		The predicate symbol is imported into the module in \\
	&	question from module {\tt Mod}. \\ \hline
	\hline
	{\tt spied}		&
		The predicate symbol has been declared spied \\
	&	(either conditionally or unconditionally). \\ \hline
	\hline
	{\tt tabled}	& 
		The predicate has been declared tabled. \\ \hline
	\hline
	{\tt xsb\_standard\_pred}	& 
		The predicate symbol has the same Functor and Arity as  \\
	&	one of XSB's standard predicates, and is available to \\
	&	the user without needing to load a file or import the \\
	&	predicate from a module. \\ \hline
    \end{tabular}
    \end{center}

    Finally, since {\tt dynamic} is usually declared as an operator with 
    precedence greater than 999, writing the following:

    \stuff{
    \>   \>	| ?- predicate\_property(X, dynamic).
    }

    will cause a syntax error. The way to achieve the desired result is to
    parenthesize the operator like in:

    \stuff{
    \>   \>	| ?- predicate\_property(X, (dynamic)).
    }


\ouritem{module\_property(?Module, ?Property)}\index{{\tt module\_property/2}}
    The standard predicate {\tt module\_property/2} can be used to find the
    properties of any current module.
    Succeeds iff {\tt Module} is the name of a current module having 
    {\tt Property} as one of its properties. Or procedurally, {\tt Property}
    is unified with the currently known properties of the module having 
    {\tt Module} as its name.

    Currently, the following properties are associated with modules 
    implicitly 

    \begin{center}
    \begin{tabular}{||l|l||}               \hline 
	{\em Property}		& {\em Explanation} \\ \hline \hline
	{\tt unloaded}		& 
		The module (including system modules) though it is \\
	&	current, is yet unloaded in the system. \\ \hline
	{\tt loaded}		& 
		The module (including system modules) is loaded in the \\
	&	system; this is always the case for global modules.\\ \hline
    \end{tabular}
    \end{center}

\ouritem{listing}\index{{\tt listing/0}}
    Lists in the current output stream the clauses for all dynamic
    predicates found in module {\tt usermod}.  Note that {\tt listing/0}
    does not list any compiled predicates unless they have the
    {\tt dynamic} property (see {\tt predicate\_property/2}).  A
    predicate gets the {\tt dynamic} property when it is explicitly
    declared as {\tt dynamic}, or automatically acquires it when some
    clauses for that predicate are asserted in the database.  In
    cases where a predicate was compiled but converted to {\tt dynamic}
    by asserting additional clauses for that predicate, {\tt listing/0}
    will just display an indication that there exist compiled clauses
    for that predicate and only the dynamically created clauses of the
    predicate will be listed.  For example:

    \stuff{ 
    \>   \>     | ?- [user]. \\ 
    \>   \>     [Compiling user] \\
    \>   \>     a(X) :- b(X). \\
    \>   \>     a(1). \\
    \>   \>     [user compiled, cpu time used: 0.3 seconds] \\
    \>   \>     [user loaded] \\
    \>   \>     \\
    \>   \>     yes \\
    \>   \>     | ?- assert(a(3)). \\
    \>   \>     \\
    \>   \>     yes \\
    \>   \>     | ?- listing. \\
    \>   \>     \\
    \>   \>     a(A) :- \\
    \>   \>  \>    \$compiled. \\
    \>   \>     a(3). \\
    \>   \>     \\
    \>   \>     yes \\
    }

    Predicate {\tt listing/0} always succeeds.  The query:

    \stuff{
    \>   \>     | ?- listing.
    }

    \noindent
    is just a notational shorthand for the query:

    \stuff{
    \>   \>     | ?- listing(X).
    }

\ouritem{listing(+Predicate\_Indicator)}\index{{\tt listing/1}}
    If {\tt Predicate\_Indicator} is a variable then {\tt listing/1} is
    equivalent to {\tt listing/0}.
    If {\tt Predicate\_Indicator} is an atom, then {\tt listing/1} 
    lists the dynamic clauses for all predicates of that name found in 
    module {\tt usermod} of the database.
    The argument {\tt Predicate\_Indicator} can also be a predicate 
    indicator of the form {\tt Name/Arity} in which case only the 
    clauses for the specified predicate are listed.
    Finally, it is possible for {\tt Predicate\_Indicator}
    to be a list of predicate indicators and/or atoms; e.g.

    \stuff{
    \>   \>     | ?- listing([foo/2, bar, blah/4]).
    }

    If {\tt Predicate\_Indicator} is not a variable, an atom or a predicate 
    indicator (or list of predicate indicators) of the form {\tt Name/Arity}, 
    predicate {\tt listing/1} will simply fail.

    In future releases of \ourprolog, we intend to allow the user to
    specify a predicate indicator of the form {\tt Module:Name/Arity} 
    as argument of {\tt listing/1}.

\ouritem{xsb\_configuration(Feature\_Name, ?Value)}
\index{{\tt xsb\_configuration/2}}
    Succeeds iff the current value of the XSB  feature {\tt
    Feature\_Name} is {\tt Value}.

    This predicate provides information on a wide variety of features
    related to how XSB was built, including the compiler used, the compiler
    and loader flags, the machine and OS on which XSB was built, the
    release number, the various directories that XSB uses to find its
    libraries, etc.

    To find all features and their values, ask the following query:

    \stuff{
    \>   \>	| ?- xsb\_configuration(FeatureName, Value), fail.
    }

    Here is how {\tt xsb\_configuration} might look like:

\begin{verbatim}
    xsb_configuration(architecture, 'i686-pc-linux-gnu').
    %% configuration is usualy the same as architecture, but it can also
    %% contain special tags, {\it e.g.}, i686-pc-linux-gnu-dbg, for a verion
    %% built with debugging enabled.
    xsb_configuration(configuration, 'i686-pc-linux-gnu-dbg').
    xsb_configuration(host_os, 'linux-gnu').
    xsb_configuration(os_version, '2.34').
    xsb_configuration(os_type, 'linux-gnu').
    xsb_configuration(host_vendor, 'pc').
    xsb_configuration(host_cpu,  'i686').
    xsb_configuration(compiler, 'gcc').
    xsb_configuration(compiler_flags, '  -ansi -pedantic -Wall -g').
    xsb_configuration(loader_flags, ' -lm -ldl -Wl,-export-dynamic').
    xsb_configuration(compile_mode, 'debug').
    %% The following is XSB release information
    xsb_configuration(major_version, '1').
    xsb_configuration(minor_version, '9').
    xsb_configuration(beta_version, '3').
    xsb_configuration(version, '1.9-b3').
    xsb_configuration(codename, 'Code Breaker').
    xsb_configuration(release_date, date(1998, 10, 17)).
    %% XSB query evaluation directive
    xsb_configuration(scheduling_strategy, '(batched)').
    %% Support for other languages
    xsb_configuration(perl_support, 'yes').
    xsb_configuration(perl_archlib, '/usr/lib/perl5/i386-linux/5.00404').
    xsb_configuration(perl_cc_compiler, 'cc').
    xsb_configuration(perl_ccflags, '-Dbool=char -DHAS_BOOL -I/usr/local/include').
    xsb_configuration(perl_libs, '-lnsl -lndbm -lgdbm -ldb -ldl -lm -lc -lposix -lcrypt').
    xsb_configuration(javac, '/usr/bin/javac').
    /* Tells where XSB is currently residing; can be moved */
    xsb_configuration(install_dir, InstallDir) :- ...
    /* User home directory. Usually HOME. If that is null, then it would
       be the directory where XSB is currently residing.
       This is where we expect to find the .xsb directory */
    xsb_configuration(user_home, Home) :- ...
    /* Where XSB invocation script is residing */
    xsb_configuration(scriptdir, ScriptDir) :- ...
    /* where are cmplib, syslib, lib, packages, etc live */
    xsb_configuration(cmplibdir, CmplibDir) :- ...
    xsb_configuration(libdir, LibDir) :- ...
    xsb_configuration(syslibdir, SyslibDir) :- ...
    xsb_configuration(packagesdir, PackDir) :-  ...
    xsb_configuration(etcdir, EtcDir) :- ...
    /* architecture and configuration specific directories */
    xsb_configuration(config_dir, ConfigDir) :- ...
    xsb_configuration(config_libdir, ConfigLibdir) :- ...
    /* site-specific directories */
    xsb_configuration(site_dir, '/usr/local/XSB/site').
    xsb_configuration(site_libdir, SiteLibdir) :- ...
    /* site and configuration-specific directories */
    xsb_configuration(site_config_dir, SiteConfigDir) :- ...
    xsb_configuration(site_config_libdir, SiteConfigLibdir) :- ...
    /* Where user's arch-specific libraries are found by default. */
    xsb_configuration(user_config_libdir, UserConfigLibdir) :- ...
\end{verbatim}

\ouritem{xsb\_flag(?Flag\_Name, ?Value)}\index{{\tt xsb\_flag/2}}
    Succeeds iff the current value of the XSB flag {\tt Flag\_Name} is 
    {\tt Value}. So, one can enumerate all the flag names which the system
    currently understands, together with their current values by using the
    following query:

    \stuff{
    \>   \>	| ?- xsb\_flag(FlagName, Value), fail.
    }

    The flag names currently supported are:

    \begin{center}
    \begin{tabular}{||l|p{11cm}||}               \hline
	{\em Flag Name}		& {\em Purpose} \\ \hline \hline
	{\tt debugging}		& 
		{\tt "on"} iff debug mode is on; {\tt "off"} otherwise.\\
		\hline
	{\tt tracing}		& 
		{\tt "on"} iff trace mode is on; {\tt "off"} otherwise.\\
		\hline
	{\tt goal}		& 
		the goal passed to XSB on command line with the `-e'
                switch; `{\tt true.}' if nothing is passed.
                \\ \hline
        {\tt dcg\_style}        &
                the DCG style currently used; {\tt xsb} or {\tt standard}
                (standard is used in Quintus, SICSTUS, etc.).
                See Section~\ref{sec-dcg-differences} for more details.
                \\ \hline
	{\tt garbage\_collection} &
		{\tt "none", "sliding"}, or {\tt "copying"} depending
		on the garbage collection strategy that is currently
		being employed (see also Section~\ref{sec:EmuOptions}).
                \\ \hline
    \end{tabular}
    \end{center}
    
    Note that {\tt xsb\_flag} is used only for dynamic XSB settings, {\it
      i.e.}, settings that might change between sessions or within the same
    session. For static configuration information, the predicate {\tt
      xsb\_configuration/2} is used. \index{{\tt xsb\_configuration/2}} 

\ouritem{hilog\_symbol(?Symbol)}\index{{\tt hilog\_symbol/1}}
    Succeeds iff {\tt Symbol} has been declared as a HiLog symbol, or 
    procedurally unifies {\tt Symbol} with one of the currently known 
    (because of a prior declaration) HiLog symbols. The HiLog symbols
    are always atoms, but if the argument of {\tt hilog\_symbol},
    though instantiated, is not an atom the predicate simply fails.
    So, one can enumerate all the HiLog symbols by using the following
    query:

    \stuff{
    \>   \>	| ?- hilog\_symbol(X).
    }

\ouritem{current\_op(?Precedence, ?Type, ?Name)}\index{{\tt current\_op/3}}
    This predicate is used to examine the set of operators currently in force.
    It succeeds when the atom {\tt Name} is currently an operator of type 
    {\tt Type} and precedence {\tt Precedence}.  None of the arguments of
    {\tt current\_op/3} need to be instantiated at the time of the call, 
    but if they are, they must be of the following types:
    \begin{description}
    \item[{\tt Precedence}] 
	it must be an integer in the range from 1 to 1200.
    \item[{\tt Type}]
	it must be one of the atoms:
	\begin{verbatim}
	        xfx  xfy  yfx  fx  fy  hx  hy  xf  yf
	\end{verbatim}
    \item[{\tt Name}]
	it must be either an atom or a list of atoms.
    \end{description}

    Exceptions (not yet implemented):
    \begin{description}
    \item[{\tt domain\_error}]
	{\tt Precedence} is not between 1--1200, or {\tt Type} is not one
	of the listed atoms.
    \item[{\tt type\_error}]
	{\tt Name} is not an atom.
    \end{description}

\ouritem{hilog\_op(?Precedence, ?Type, ?Name)}\index{{\tt hilog\_op/3}}
    This predicate has exactly the same behaviour as {\tt current\_op/3}
    with the only difference that {\tt Type} can only have the values
    {\tt hx} and {\tt hy}.
\end{description}


%%% Local Variables: 
%%% mode: latex
%%% TeX-master: "manual1"
%%% End: 
