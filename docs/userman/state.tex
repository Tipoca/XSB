\section{Information about the System State} \label{State}
%========================================================
\index{state of the system} \index{system, state of}

Various aspects of the state of an instance of XSB --- information
about what predicates, modules, or dynamic clauses have been loaded,
their object files, along with other information can be inspected in
ways similar to many Prolog systems.  However, because the atom-based
module system of XSB may associate structures with particular modules,
predicates are provided to inspect these elements as well.  The
following descriptions of {\em state} predicates use the terms {\em
predicate indicator}, {\em term indicator} and {\em current module} to
mean the following:
\index{predicate indicator}
\index{term indicator}
\begin{itemize}
\item By {\em predicate indicator} \index{predicate indicator} we mean a
      {\em compound term} of the form {\tt M:F/A} or simply {\tt F/A}.
      When the predicate indicator is fully instantiated, {\tt M} and {\tt F}
      are atoms representing the {\em module name} and the {\em functor} 
      of the predicate respectively and {\tt A} is a non negative integer 
      representing its {\em arity}.

      Example: {\tt usermod:append/3}
\item By {\em term indicator} \index{term indicator} we mean a predicate or
      function symbol of arity N followed by a sequence of N variables
      (enclosed in parentheses if N is greater than zero).  A term indicator
      may optionally be prefixed by the module name, thus it can be of the
      form {\tt M:Term}.

      Example: {\tt usermod:append(\_,\_,\_)}
\item A module {\tt M} becomes a {\em current (i.e. ``known'') module} as
      soon as it is loaded in the system or when another module that is
      loaded in the system imports some predicates from module {\tt M}.

      Note that due to the dynamic loading of XSB, a module can be
      current even if it has not been loaded, and that some predicates
      of that module may not be defined. In fact, a module can be
      current even if it does not exist.  This situation occurs when a
      predicate is improperly imported from a non-existent module.
      Despite this, a module can never lose the property of being {\em
        current}.
\end{itemize}

\begin{description}

\isoitem{current\_input(?Stream)}{current\_input/1}
    Succeeds iff stream {\tt Stream} is the current input stream, or 
    procedurally unifies {\tt Stream} with the current input stream.

{\bf Error Cases}
\bi
\item 	{\tt Stream} is neither a variable nor a stream identifier
\bi
\item 	{\tt domain\_error(stream\_or\_variable,Stream))}
\ei
\ei
%
%\compatability 
%
%In XSB {\tt current\_input/1} does not throw an error if {\tt Stream}
%is not a current input stream, but quietly fails instead.

\isoitem{current\_output(?Stream)}{ISO}{current\_output/1}
%
    Succeeds iff stream {\tt Stream} is the current output stream, or 
    procedurally unifies {\tt Stream} with the current output stream.

{\bf Error Cases}
\bi
\item 	{\tt Stream} is neither a variable nor a stream identifier
\bi
\item 	{\tt domain\_error(stream\_or\_variable,Stream))}
\ei
\ei
%
\compatability 
%
In XSB {\tt current\_input/1} does not throw an error if {\tt Stream}
is not a current input stream, but quietly fails instead.

\isoitem{current\_prolog\_flag(?Flag\_Name, ?Value)}{current\_prolog\_flag/2}
%
{\tt current\_prolog\_flag/2} allows the user to examine both dynamic
aspects of XSB along with certain non-changeable ISO flags.  Calls to
{\tt current\_prolog\_flag/2} will unify against not only standard
Prolog flags, but also values of {\tt xsb\_flag/2}.

{\bf Error Cases}
\bi
\item 	{\tt Flag\_Name} is neither a variable nor an atom.
\bi
\item 	{\tt domain\_error(atom\_or\_variable,Flag\_Name)}
\ei
\ei

\compatability The ISO flags {\tt char\_conversion} is not available
-- XSB does not use character conversion.  XSB reads double quoted
strings as lists of character codes, so that the value of the flag
{\tt double\_quotes} is always {\tt codes}, and this flag is not
settable.

\isoitem{set\_prolog\_flag(?Flag\_Name, ?Value)}{set\_prolog\_flag/2}
%
{\tt set\_prolog\_flag/2} allows the user to change settable prolog
flags.  Currently the only settable flag is the {\tt unknown} flag.
Setting the flag {\tt unknown} to {\tt fail} results in calls to
undefined predicates to quietly fail.  Setting it to {\tt warning}
causes calls to undefined predicates to generate a warning (to {\tt
  STDWARN}) and then fail.  Setting it to {\tt error} (the default)
causes calls to undefined predicates to throw an existence error.

{\bf Error Cases}
\bi
\item 	{\tt Flag\_Name} is neither a variable nor an atom.
\bi
\item 	{\tt domain\_error(settable\_flag,Flag\_Name)}
\ei
\ei

\isoitem{current\_predicate(?Predicate\_Indicator)}{current\_predicate/1}
%
{\tt current\_predicate/1} can be used to backtrack through indicators
for loaded user or system predicates.  If {\tt Predicate\_Indicator}
unifies with {\tt Module:F/A} all loaded predicates unifying with this
indicator is returned.  If {\tt Predicate\_indicator} is {\tt F/A},
{\tt current\_predicate/1} behaves as if it were called with the form
{\tt usermod:F/A}.  Unlike {\tt current\_functor/1} {\tt
current\_predicate/1} does not return indicators for predicates that
have been imported but not actually loaded into code space.  For more
detailed analysis of predicate properties, the predicate {\tt
predicate\_property/2} can be used.

As an example to backtrack through all of the predicates defined and loaded in
module {\tt blah}, regardless of whether {\tt blah} is a system or a
user defined module, use:

    \stuff{
    \> \>	| ?- current\_predicate(blah:Predicate).
    }

    In this case {\tt Predicate} will have the form: {\tt Functor/Arity}.

    To backtrack through all predicates defined and loaded in any current 
    module, use:

    \stuff{
    \> \>	| ?- current\_predicate(Module:Functor/Arity).
    }

    This succeeds once for every predicate that is loaded in XSB's
    database.

    To find the predicates having arity 3 that are loaded in {\tt
    usermod}, use:

    \stuff{
    \> \>	| ?- current\_predicate(usermod:Functor/3).
    }

    while to find all predicates loaded in the global modules of the system
    regardless of their arity, use:

    \stuff{
    \> \>	| ?- current\_predicate(usermod:Predicate). \\
    }
%
{\bf Error Cases}
\bi
\item 	{\tt Predicate\_indicator} is neither a variable nor a predicate indicator
\bi
\item 	{\tt type\_error(predicate\_indicator,Predicate\_indicator))}
\ei
\ei
%
\compatability
%
In XSB, {\tt current\_predicate} will backtrack through system
predicates as well as user predicates.


%-----------------------------------------------------------------------------------------------
\comment{
%TLS: I don't see that this gives us anything useful over
%current_predicate/1, so why confuse the user?
%\ouritem{current\_predicate(?Name, ?Term\_Indicator)}
%{current\_predicate/2}}
%    Succeeds iff {\tt Term\_Indicator} is the most general term 
%    corresponding to one of the predicates having functor {\tt Name} that are 
%    defined and loaded in a particular module in the database. 
%    (The module can be either system or user defined).
%    Or procedurally, {\tt current\_predicate/2}
%    unifies {\tt Name} with the name of a loaded predicate, and 
%    {\tt Term\_Indicator} with the most general term corresponding to that
%   predicate.  The flavours of this predicate are analogous to those of 
%    {\tt current\_predicate/1} and behave according to whether 
%    {\tt Term\_Indicator} has one of the following two forms:
%    \begin{enumerate}
%    \item{\tt Module:Term.}
%    \item{\tt Term} (module is assumed to be usermod).
%    \end{enumerate}
%    If {\tt Term\_Indicator} is uninstantiated, then this predicate succeeds
%    only for {\tt usermod}. Like {\tt current\_predicate/1} only 
%    predicates that have a property in the following set:
%    \begin{center}
%    {\tt $\{$ loaded, dynamic, foreign $\}$ }
%    \end{center}
%    (see {\tt predicate\_property/2} below) are reported.
%
%    For example, if predicates {\tt foo/1} and {\tt foo/3} are defined and
%    loaded into module {\tt blah}, the following query will return:
%
%    \stuff{
%    \>	\>	| ?- current\_predicate(foo, blah:Term).\\
%    \>  \>						\\
%    \>	\>      Term = foo(\_638788,\_638792,\_638796);	\\
%    \>  \>						\\
%    \>	\>	Term = foo(\_638788);			\\
%    \>  \>						\\
%    \>	\>	no
%    }
%
%    If a module is specified, {\tt current\_predicate/2} succeeds only for
%    those predicates which are defined and loaded in that module. Unless 
%    the module is one of the global modules, {\tt current\_predicate/2} fails
%    for those predicates which are imported into that module.
%
%    On the other hand, the goal:
%
%    \stuff{
%    \>	\>	| ?- current\_predicate(Name, Term).
%    }
%
%    can be used to backtrack through every predicate that is loaded in the
%    global modules of \ourprolog's database.
%
%    Note that the order of term generation is undetermined. Once again, 
%    there are no error conditions associated with this predicate; if its
%    argument is not what it should be, the predicate simply fails.
%
%\ouritem{current\_functor(?Name, ?Term\_Indicator)}
%{current\_functor/2}}
%    Succeeds iff {\tt Term\_Indicator} is the most general term 
%    corresponding to one of the currently known terms having {\tt Name} 
%    as their functor appearing in a current module.  (Both system and user 
%    defined modules are checked).  Or procedurally, 
%    {\tt current\_functor/2} unifies {\tt Name} with the name of a functor 
%    known to the database, and {\tt Term\_Indicator} with the most 
%    general term corresponding to that functor. The flavours of this predicate 
%    are analogous to the ones of {\tt current\_functor/1} according to 
%    whether {\tt Term\_Indicator} has one of the following two forms:
%    \begin{enumerate}
%    \item{\tt Module:Term.}
%    \item{\tt Term} (for global modules).
%    \end{enumerate}
%    If {\tt Term\_Indicator} is uninstantiated, then this predicate succeeds
%    only for global modules.  
%    As in {\tt current\_functor/1} even unloaded predicates are reported
%    (if they have been imported and are are known to the database).
%
%    For example, if a predicate {\tt foo/2} and and a function symbol 
%    {\tt foo/1} are defined into module {\tt blah}, the following query 
%    will return:
%
%    \stuff{
%    \>	\>	| ?- current\_functor(foo, blah:Term).	\\
%    \>  \>						\\
%    \>	\>      Term = foo(\_638788,\_638792);		\\
%    \>  \>						\\
%    \>	\>	Term = foo(\_638788);			\\
%    \>  \>						\\
%    \>	\>	no
%    }
%
%    If a module is specified, {\tt current\_functor/2} succeeds only for
%    those functors (function and predicate symbols) which are defined in 
%    that module. Unless the module is one of the global modules, 
%    {\tt current\_functor/2} fails for the predicates which are imported 
%    into that module.
%
%    On the other hand, the goal:
%
%    \stuff{
%    \>	\>	| ?- current\_functor(Name, Term).
%    }
%
%    can be used to backtrack through every known term {\tt Term} in the global
%    modules of \ourprolog's database that has {\tt Name} as its functor.
%
%    Note that the order of term generation is undetermined. Once again, 
%    there are no error conditions associated with this predicate; if its
%    arguments are inappropriate, the predicate simply fails.
}
%--------------------------------------------------------------------------------------------------

\standarditem{current\_module(?Module)}{current\_module/1}
    The standard predicate {\tt current\_module/1} allows the user to
    check whether a given module is {\em current} or to generate
    (through backtracking) all currently known modules.  Succeeds iff
    {\tt Module} is one of the modules in the database. This includes
    both user modules and system modules.  For more detailed analysis
    of module properties, the predicate {\tt module\_property/2}
    can be used.

    Note that predicate {\tt current\_module/1} succeeds for a given
    module even if that module does not export any predicates. There
    are no error conditions associated with this predicate; if its
    argument does not unify with one of the current modules, {\tt
    current\_module/1} simply fails.

\standarditem{current\_module(?Module, ?ObjectFile)}{current\_module/2}
    Predicate {\tt current\_module/2} gives the relationship between
    the modules and their associated object file names. The file name
    {\tt ObjectFile} must be absolute and end with the object file
    extension for the system (by default, {\tt .xwam}).
%
    It is possible for a current module to have no associated file
    name (as is the case for {\tt "usermod"}), or for the system to be
    unable to determine the file name of a current module. In both
    cases, predicate {\tt current\_module/1} will succeed for this
    module, while {\tt current\_module/2} will fail. The system is
    unable to determine the file name of a given module if that module
    is not in one of the directories of the search path (see
    Section~\ref{LibPath}).  Once again, there are no error conditions
    associated with this predicate; if the arguments of {\tt
    current\_module/2} are not correct, or {\tt Module} has no
    associated {\tt File}, the predicate will simply fail.

\standarditem{current\_functor(?Predicate\_Indicator)}
{current\_functor/1}
{\tt current\_predicate/1} can be used to backtrack through indicators
for all non-atomic terms occurring in loaded modules.  If {\tt
Predicate\_Indicator} unifies with {\tt Module:F/A} all term
indicators unifying with {\tt F/A} in a module unifying with {\tt
Module} are returned.  If {\tt Predicate\_indicator} is {\tt F/A},
{\tt current\_predicate/1} behaves as if it were called with the form
{\tt usermod:F/A}.  Unlike {\tt current\_predicate/1} {\tt
current\_functor/1} returns not only structures occurring in
predicates but predicates that are imported into loaded modules but
are not yet themselves loaded.

As an example, to backtrack through all of the functors of positive
arity (function and predicate symbols) that appear in the global
modules of the system regardless of whether they are system or a user
defined, use:

    \stuff{
    \> \>	| ?- current\_functor(Functor/Arity), Arity > 0.
    }

    There are no error conditions associated with this predicate; if its 
    argument is not a predicate indicator the predicate simply fails.

\standarditem{current\_index(Functor/Arity,IndexSpec)}
{current\_index/2}
%
XSB has a variety of ways to index dynamic predicate including
alternate argument indexing, multiple argument indexing,
star-indexing, and tries, as discussed in Section~\ref{sec:assert}.
In addition XSB allows a choice of which argument to index for
compiled predicates as well.  {\tt current\_index/2} returns the index
specification for each functor/arity pair unifying with {\tt
Functor/Arity} and visible from the calling context of  {\tt current\_index/2}.

\standarditem{current\_atom(?Atom\_Indicator)}{current\_atom/1}
    Generates (through backtracking) all currently known atoms, and unifies
    each in turn with {\tt Atom\_Indicator}. 

%??? need to define visible.
\label{PredProp}
\standarditem{predicate\_property(?Term\_Indicator, ?Property)}
{predicate\_property/2}
    The standard predicate {\tt predicate\_property/2} can be used to find 
    the properties of any predicate that is visible to a particular module.
    Succeeds iff {\tt Term\_Indicator} is a term indicator for a current 
    predicate whose principal functor is a predicate having {\tt Property} 
    as one of its properties. Or procedurally, {\tt Property} is unified 
    with the currently known properties of the predicate having 
    {\tt Term\_Indicator} as its skeletal specification.
    
    A brief description of {\tt predicate\_property/2} is as follows:
    \begin{itemize} 
\item If {\tt Term\_Indicator} is not a variable, and is a structure
	or atom, then {\tt Property} is successively unified with the
	various properties associated with {\tt Term\_Indicator}.  If
	{\tt Term\_Indicator} is not a known to the system, the call
	succeeds with {\tt Property} successively unified to {\tt
	exported} and {\tt unclassified}.  These properties can be
	considered as a default for any structure or atom.
\item If {\tt Property} is bound to a valid predicate property, then {\tt
	predicate\_property/2} successively unifies {\tt
	Term\_Indicator} with the skeletal specifications of all
	predicates known to the system having the specified {\tt
	Property}.  
\item If {\tt Term\_Indicator} is a variable, then
	it is unified (successively through backtracking) with the
	most general term for a predicate whose known properties are
	unified with {\tt Property}.  
\item If {\tt Term\_Indicator}
	is not a term indicator, or if {\tt Property} is not a valid
	predicate property, the call fails.  
\end{itemize} 
\noindent
For example, all the loaded predicate skeletal specifications in
module {\tt "usermod"} may be enumerated using:

    \stuff{
    \>   \>	| ?- predicate\_property(Pred, loaded).
    }

    Also the following query finds all predicate skeletal specifications that 
    are exported by module {\tt blah}:

    \stuff{
    \>   \>	| ?- predicate\_property(blah:Pred, exported).
    }

    Currently, the following properties are associated with predicates 
    either implicitly or by declaration.  Double lines show property
    categories, and a predicate can have at most one property of each
    category.

    \begin{center}
    \begin{tabular}{||l|l||}               \hline 
	{\em Property}		& {\em Explanation} \\ \hline \hline
	{\tt unclassified}	& 
		The predicate symbol is not yet classified according  \\
	&	to this category. This property has various meanings. \\ 
	&	Usually for exported predicate symbols in system or   \\
	&	user defined modules it means that the predicate is   \\
	&	yet unloaded (because it has not been used).	      \\
	&	In {\tt usermod} it usually means that the predicate \\
	&	is either a function symbol, or an unloaded predicate \\
	&	symbol (including constants). \\ \hline
	{\tt dynamic}	& 
		The predicate is dynamic. \\ \hline
	{\tt loaded}		& 
		The predicate (including internal predicates) is a \\ 
	&	Prolog predicate loaded into the module in question; \\
	&	this is always the case for predicates in {\tt usermod}.\\ 
		\hline
	{\tt unloaded}		& 
		The predicate is yet unloaded into the module \\
	&	in question.\\ \hline
	{\tt foreign}	& 
		The predicate is a foreign predicate. This implies that  \\ 
	&	the predicate is already loaded in the system, because	 \\
	&	currently there is no way for XSB  to know that a \\
	&	predicate is a foreign predicate until it is loaded in	 \\
	&	the system.\\ \hline
%	{\tt function}	& 
%		The predicate symbol is declared as a function. \\ \hline
	\hline
	{\tt exported}	&
		The predicate symbol is exported by the module in \\ 
	&	question; in other words the predicate symbol is \\
	&	visible to any other module in the system. \\ \hline
	{\tt local}	& 
		The predicate symbol is local to the module \\
	&	in question. \\ \hline
	{\tt imported\_from(Mod)}	& 
		The predicate symbol is imported into the module in \\
	&	question from module {\tt Mod}. \\ \hline
	\hline
	{\tt spied}		&
		The predicate symbol has been declared spied \\
	&	(either conditionally or unconditionally). \\ \hline
	\hline
	{\tt tabled}	& 
		The predicate has been declared tabled. \\ \hline
	\hline
	{\tt shared}	& 
		The predicate has been declared shared in the \\
        &       multi-threaded engine.  This means that any dynamic \\
        &       code or tables for this predicate will be shared among \\
        &       threads, but it does not affect static, non-tabled code. \\ \hline
	\hline
	{\tt xsb\_standard\_pred}	& 
		The predicate symbol has the same Functor and Arity as  \\
	&	one of XSB's standard predicates, and is available to \\
	&	the user without needing to load a file or import the \\
	&	predicate from a module. \\ \hline
    \end{tabular}
    \end{center}

    Finally, since {\tt dynamic} is usually declared as an operator with 
    precedence greater than 999, writing the following:

    \stuff{
    \>   \>	| ?- predicate\_property(X, dynamic).
    }

    will cause a syntax error. The way to achieve the desired result is to
    parenthesize the operator like in:

    \stuff{
    \>   \>	| ?- predicate\_property(X, (dynamic)).
    }


\standarditem{module\_property(?Module, ?Property)}{module\_property/2}
    The standard predicate {\tt module\_property/2} can be used to find the
    properties of any current module.
    Succeeds iff {\tt Module} is the name of a current module having 
    {\tt Property} as one of its properties. Or procedurally, {\tt Property}
    is unified with the currently known properties of the module having 
    {\tt Module} as its name.

    Currently, the following properties are associated with modules 
    implicitly 

    \begin{center}
    \begin{tabular}{||l|l||}               \hline 
	{\em Property}		& {\em Explanation} \\ \hline \hline
	{\tt unloaded}		& 
		The module (including system modules) though it is \\
	&	current, is yet unloaded in the system. \\ \hline
	{\tt loaded}		& 
		The module (including system modules) is loaded in the \\
	&	system; this is always the case for {\tt usermod}.\\ \hline
    \end{tabular}
    \end{center}

\standarditem{listing}{listing/0}
    Lists in the current output stream the clauses for all dynamic
    predicates found in module {\tt usermod}.  Note that {\tt listing/0}
    does not list any compiled predicates unless they have the
    {\tt dynamic} property (see {\tt predicate\_property/2}).  A
    predicate gets the {\tt dynamic} property when it is explicitly
    declared as {\tt dynamic}, or automatically acquires it when some
    clauses for that predicate are asserted in the database.  In
    cases where a predicate was compiled but converted to {\tt dynamic}
    by asserting additional clauses for that predicate, {\tt listing/0}
    will just display an indication that there exist compiled clauses
    for that predicate and only the dynamically created clauses of the
    predicate will be listed.  For example:

    \stuff{ 
    \>   \>     | ?- [user]. \\ 
    \>   \>     [Compiling user] \\
    \>   \>     a(X) :- b(X). \\
    \>   \>     a(1). \\
    \>   \>     [user compiled, cpu time used: 0.3 seconds] \\
    \>   \>     [user loaded] \\
    \>   \>     \\
    \>   \>     yes \\
    \>   \>     | ?- assert(a(3)). \\
    \>   \>     \\
    \>   \>     yes \\
    \>   \>     | ?- listing. \\
    \>   \>     \\
    \>   \>     a(A) :- \\
    \>   \>  \>    \$compiled. \\
    \>   \>     a(3). \\
    \>   \>     \\
    \>   \>     yes \\
    }

    Predicate {\tt listing/0} always succeeds.  The query:

    \stuff{
    \>   \>     | ?- listing.
    }

    \noindent
    is just a notational shorthand for the query:

    \stuff{
    \>   \>     | ?- listing(X).
    }

\standarditem{listing(+Predicate\_Indicator)}{listing/1}
    If {\tt Predicate\_Indicator} is a variable then {\tt listing/1} is
    equivalent to {\tt listing/0}.
    If {\tt Predicate\_Indicator} is an atom, then {\tt listing/1} 
    lists the dynamic clauses for all predicates of that name found in 
    module {\tt usermod} of the database.
    The argument {\tt Predicate\_Indicator} can also be a predicate 
    indicator of the form {\tt Name/Arity} in which case only the 
    clauses for the specified predicate are listed.
    Finally, it is possible for {\tt Predicate\_Indicator}
    to be a list of predicate indicators and/or atoms; e.g.

    \stuff{
    \>   \>     | ?- listing([foo/2, bar, blah/4]).
    }

    If {\tt Predicate\_Indicator} is not a variable, an atom or a predicate 
    indicator (or list of predicate indicators) of the form {\tt Name/Arity}, 
    predicate {\tt listing/1} will simply fail.

    In future releases of \ourprolog, we intend to allow the user to
    specify a predicate indicator of the form {\tt Module:Name/Arity} 
    as argument of {\tt listing/1}.

\standarditem{xsb\_configuration(Feature\_Name, ?Value)}
{xsb\_configuration/2}
    Succeeds iff the current value of the XSB  feature {\tt
    Feature\_Name} is {\tt Value}.

    This predicate provides information on a wide variety of features
    related to how XSB was built, including the compiler used, the compiler
    and loader flags, the machine and OS on which XSB was built, the
    release number, the various directories that XSB uses to find its
    libraries, etc.

    To find all features and their values, ask the following query:

    \stuff{
    \>   \>	| ?- xsb\_configuration(FeatureName, Value), fail.
    }

    Here is how {\tt xsb\_configuration} might look like:

{\small
\begin{verbatim}
    xsb_configuration(architecture, 'i386-apple-darwin8.9.1').
    %% configuration is usualy the same as architecture, but it can also
    %% contain special tags, {\it e.g.}, i386-apple-darwin8.9.1-dbg, for a verion
    %% built with debugging enabled.
    xsb_configuration(configuration, 'i386-apple-darwin8.9.1-dbg').
    xsb_configuration(host_os, 'darwin8.9.1').
    xsb_configuration(os_version, '8.9.1').
    xsb_configuration(os_type, 'darwin').
    xsb_configuration(host_vendor, 'apple').
    xsb_configuration(host_cpu,  'i386').
    xsb_configuration(compiler, 'gcc').
    xsb_configuration(compiler_flags, '-faltivec -fPOC -Wall -pipe -g').
    xsb_configuration(loader_flags, '-g -lm ').
    xsb_configuration(compile_mode, 'debug').
    %% The type of XSB engine configured.
    xsb_configuration(scheduling_strategy, '(local)').
    xsb_configuration(engine_mode, 'slg-wam').
    xsb_configuration(word_size, '32').
    %% The following is XSB release information
    xsb_configuration(major_version, '3').
    xsb_configuration(minor_version, '1').
    xsb_configuration(beta_version, '').
    xsb_configuration(version, '3.1').
    xsb_configuration(codename, 'Incognito').
    xsb_configuration(release_date, date(1998, 10, 17)).
    %% Support for other languages
    xsb_configuration(perl_support, 'yes').v
    xsb_configuration(perl_archlib, '/usr/lib/perl5/i386-linux/5.00404').
    xsb_configuration(perl_cc_compiler, 'cc').
    xsb_configuration(perl_ccflags, '-Dbool=char -DHAS_BOOL -I/usr/local/include').
    xsb_configuration(perl_libs, '-lnsl -lndbm -lgdbm -ldb -ldl -lm -lc -lposix -lcrypt').
    xsb_configuration(javac, '/usr/bin/javac').
    /* Tells where XSB is currently residing; can be moved */
    xsb_configuration(install_dir, InstallDir) :- ...
    /* User home directory. Usually HOME. If that is null, then it would
       be the directory where XSB is currently residing.
       This is where we expect to find the .xsb directory */
    xsb_configuration(user_home, Home) :- ...
    /* Where XSB invocation script is residing */
    xsb_configuration(scriptdir, ScriptDir) :- ...
    /* where are cmplib, syslib, lib, packages, etc live */
    xsb_configuration(cmplibdir, CmplibDir) :- ...
    xsb_configuration(libdir, LibDir) :- ...
    xsb_configuration(syslibdir, SyslibDir) :- ...
    xsb_configuration(packagesdir, PackDir) :-  ...
    xsb_configuration(etcdir, EtcDir) :- ...
    /* architecture and configuration specific directories */
    xsb_configuration(config_dir, ConfigDir) :- ...
    xsb_configuration(config_libdir, ConfigLibdir) :- ...
    /* site-specific directories */
    xsb_configuration(site_dir, '/usr/local/XSB/site').
    xsb_configuration(site_libdir, SiteLibdir) :- ...
    /* site and configuration-specific directories */
    xsb_configuration(site_config_dir, SiteConfigDir) :- ...
    xsb_configuration(site_config_libdir, SiteConfigLibdir) :- ...
    /* Where user's arch-specific libraries are found by default. */
    xsb_configuration(user_config_libdir, UserConfigLibdir) :- ...
\end{verbatim}
}

\standarditem{xsb\_flag(?Flag\_Name, ?Value)}{xsb\_flag/2}
    Succeeds iff the current value of the XSB flag {\tt Flag\_Name} is 
    {\tt Value}. So, one can enumerate all the flag names which the system
    currently understands, together with their current values by using the
    following query:

    \stuff{
    \>   \>	| ?- xsb\_flag(FlagName, Value), fail.
    }

    The flag names currently supported are:
\begin{itemize}
%
\item {\tt backtrace\_on\_error} {\tt on} iff system-handled errors
automatically print out the trace of the execution stack where the
error arose, {\tt off} otherwise. Default is {\tt on}.  In the
multi-threaded engine, this flag is thread-specific and controls
whether the backtrace for a current execution will be printed to {\tt
  STDERR}.
%
\item {\tt dcg\_style}  the DCG style currently used; {\tt xsb} or {\tt
  standard} (standard is used in Quintus, SICSTUS, etc.).  See
Section~\ref{sec-dcg-differences} for more details. Default is {\tt
  xsb}. This flag affects all threads in the process.
%
\item {\tt debugging}  {\tt on} iff debug mode is on; {\tt off} otherwise.
This flag affects all threads in the process. 
%
\item {\tt heap\_garbage\_collection}  {\tt indirection}, {\tt none}, {\tt
  sliding}, or {\tt copying} depending on the heap garbage collection
strategy that is currently being employed (see also
Section~\ref{sec:EmuOptions}).  Default is {\tt indirection}.  This
flag is private to each thread.  
%
\item {\tt clause\_garbage\_collection}  {\tt on} if garbage collection for
retracted clauses is allowed, and off otherwise. Default is {\tt
  on}.  This flag is private to each thread.  
%
\item {\tt atom\_garbage\_collection}  {\tt on} if garbage collection for
atomic constants is allowed, and off otherwise. Default is {\tt on}.
This flag is global for all threads (currently, string garbage
collection will only be invoked if there is a single active thread.)

%
\item {\tt table\_gc\_action}  {\tt abolish\_tables\_transitively} if
  predicates or subgoals that depend on a conditional answer of an
  abolished table are to be abolished automatically, and {\tt
    abolish\_tables\_singly} if not.  Default is {\tt
    abolish\_tables\_transitively}.  This flag affects all threads in
  the process. 
%
\item {\tt goal}  the goal passed to XSB on command line with the `-e'
switch; `{\tt true.}' if nothing is passed.  This flag may be
examined, but not set. 
%
\item {\tt tracing}  {\tt on} iff trace mode is on; {\tt off}
otherwise. This flag affects all threads in the process. 
%
\item {\tt write\_depth}  The depth to which a term is written by {\tt
  write}-like predicates.  Default is 64.  This flag affects all
threads in the process.  

\item {\tt warning\_action} The action to take on warnings: the
  default value {\tt print\_warning} prints a warning message to the
  XSB {\tt STDWARN} stream when {\tt warning/1} is called; {\tt
    silent\_warning} silently succeeds when {\tt warning/1} is called;
  and {\tt error\_warning/1} throws a miscellaneous exception.

\item {\tt thread\_glsize} In the multi-threaded engine, the initial
  size, in kbytes, of the global and local stack area of a newly
  created thread if no such option is explicitly passed.  By default
  this is 768 (or 1536 for 64-bit configurations), or whatever was
  passed in if the command-line option {\tt -m} was used, but that
  value may be modified at any time by resetting the flag.  This flag
  affects a thread created by any thread in the process.

\item {\tt thread\_tcpsize} In the multi-threaded engine, the initial
  size, in kbytes, of the trail and choice point area of a newly
  created thread if no such option is explicitly passed.  By default
  this is 768 (or 1536 for 64-bit configurations), or whatever was
  passed in if the command-line option {\tt -c} was used, but that
  value may be modified at any time by resetting the flag.  This flag
  affects a thread created by any thread in the process.

\item {\tt thread\_complsize} In the multi-threaded engine, the
  initial size, in kbytes, of the completion stack area of a newly
  created thread if no such option is explicitly passed.  By default
  this is 64 (or 128 for 64-bit configurations), or whatever was
  passed in if the command-line option {\tt -0} was used, but that
  value may be modified at any time by resetting the flag.  This flag
  affects a thread created by any thread in the process.

\item {\tt thread\_pdlsize} In the multi-threaded engine, the initial
  size, in kbytes, of the unification stack area of a newly created
  thread if no such option is explicitly passed.  By default this is
  64 (or 128 for 64-bit configurations), or whatever was passed in if
  the command-line option {\tt -m} was used, but that value may be
  modified at any time by resetting the flag.  This flag affects a
  thread created by any thread in the process.

\item {\tt thread\_detached} In the multi-threaded engine, this
  specifies whether threads are to be created as detached or joinable
  if no explicit option is passed.  A value of {\tt true} indicates
  that threads are to be created as detached, and {\tt false} as
  joinable.  If this flag is not set, its default is {\tt false}.

\item {\tt max\_threads} In the multi-threaded engine, the maximum
  number of valid threads.  By default this is 1024 and this value may
  not be reset at runtime, but it may be set by the command-line
  option {\tt --max\_threads}.  This option is settable only by a
  command-line argument, and has no effect in the single-threaded
  engine.

\item {\tt max\_queue\_size} In the multi-threaded engine, the default
  maximum number of terms a message queue contains before writes to
  the message queue block.  By default this is 1000.  If set to 0,
  queues by default will be unbounded.  This option has no effect in
  the single-threaded engine.

\index{shared\_predicates}
\item {\tt shared\_predicates} In the multi-threaded engine, indicates
  whether predicates are considered thread-shared by default -- that
  is, whether tables or dynamic predicates are shared among threads.
  By default this is false, and predicates are considered
  thread-private by default.  This option is settable only by a
  command-line argument, and has no effect in the
  single-threaded engine.

\end{itemize}
    
Note that {\tt xsb\_flag} is used only for dynamic XSB settings, {\it
  i.e.}, settings that might change between sessions (via command line
arguments) or within the same session (via modifiable flags).  For
static configuration information, the predicate {\tt
  xsb\_configuration/2} is used. {xsb\_configuration/2}


\standarditem{set\_xsb\_flag(+Flag\_Name, +Value)}{set\_xsb\_flag/2}
%
Changes one of the dynamic XSB settings described for {\tt
  xsb\_flag/2}.

{\bf Error Cases}
\bi
\item 	{\tt Flag\_Name} or {\tt Value} is a variable.
\bi
\item 	{\tt instantiation\_error}
\ei
%
\item 	{\tt Flag\_Name} is not the name of a recognized XSB flag.
\bi
\item 	{\tt domain\_error(xsb\_flag,Flag\_Name)}
\ei

\ei


\standarditem{hilog\_symbol(?Symbol)}{hilog\_symbol/1}
    Succeeds iff {\tt Symbol} has been declared as a HiLog symbol, or 
    procedurally unifies {\tt Symbol} with one of the currently known 
    (because of a prior declaration) HiLog symbols. The HiLog symbols
    are always atoms, but if the argument of {\tt hilog\_symbol},
    though instantiated, is not an atom the predicate simply fails.
    So, one can enumerate all the HiLog symbols by using the following
    query:

    \stuff{
    \>   \>	| ?- hilog\_symbol(X).
    }

\isoitem{current\_op(?Precedence, ?Specifier, ?Name)}{current\_op/3}
%
    This predicate is used to examine the set of operators currently
    in force.  It succeeds when the atom {\tt Name} is currently an
    operator of type {\tt Specifier} and precedence {\tt Precedence}.  None
    of the arguments of {\tt current\_op/3} need to be instantiated at
    the time of the call, but if they are, they must be of the
    following types: 
\begin{description}
\item[{\tt Precedence}] must be an integer in the range from 1 to 1200.  
\item[{\tt Specifier}] must be one of the atoms: 
\begin{verbatim} 
xfx xfy yfx fx fy hx hy xf yf 
\end{verbatim} 
\item[{\tt Name}] it must be an atom.
\end{description}

{\bf Error Cases}
\bi
\item 	{\tt Precedence} is neither a variable nor an integer in the
range from 1 to 1200. 
\bi
\item 	{\tt domain\_error(operator\_priority,Precedence)}
\ei
\item 	{\tt Specifier} is neither a variable nor an operator
specifier of the types above.
\bi
\item 	{\tt domain\_error(operator\_specifier,Specifier)}
\ei
\item 	{\tt Name} is neither a variable nor an atom.
\bi
\item 	{\tt domain\_error(atom\_or\_variable,Name)}
\ei
\ei

\standarditem{hilog\_op(?Precedence, ?Type, ?Name)}{hilog\_op/3}
    This predicate has exactly the same behaviour as {\tt current\_op/3}
    with the only difference that {\tt Type} can only have the values
    {\tt hx} and {\tt hy}.
\end{description}


%%% Local Variables: 
%%% mode: latex
%%% TeX-master: "manual1"
%%% End: 

%======================================================

\section{Execution State}\label{environmental}

\begin{description}

\standarditem{break}{break/0}
    Causes the current execution to be suspended at the beginning of the next 
    call.  The interpreter then enters break level 1 and is ready to accept
    input as if it were at top level.  If another call to {\tt break/0} is 
    encountered, it moves up to break level 2, and so on.  While execution 
    is done at break level $n>0$ the prompt changes to {\tt $n$: ?-}.

    To close a break level and resume the suspended execution, the user can 
    type the the atom {\tt end\_of\_file} or the end-of-file character 
    applicable on the system (usually {\tt CTRL-d} on UNIX systems).  
    Predicate {\tt break/0} 
    then succeeds (note in the following example that the calls to {\tt break/0}
    do not succeed), and the execution of the interrupted program is resumed.  
    Alternatively, the suspended execution can be abandoned by calling the 
    standard predicate {\tt abort/0}, which causes a return to the top level.

    An example of {\tt break/0} 's use is the following:

    \stuff{
        \>   \>     | ?- break. \\
        \>   \>     [ Break (level 1) ] \\
        \>   \>     1: ?- break. \\
        \>   \>     [ Break (level 2) ] \\
        \>   \>     2: ?- end\_of\_file. \\
        \>   \>     [ End break (level 2) ] \\
        \\
        \>   \>     yes \\
        \>   \>     1: ?-
    }

    Entering a break closes all incomplete tables (those which may not have a 
    complete set of answers).  Closed tables are unaffected, even if 
    the tables were created during the computation for which the break was
    entered.

\isoitem{halt}{ISO}{halt/0}
%
    Exits the XSB session regardless of the break level.  On exiting
    the system cpu and elapsed time information is displayed.

\isoitem{halt(Code)}{ISO}{halt/1}
%
Exits the XSB session regardless of the break level, sending the
integer {\tt Code} to the parent process.  Normally {\tt 0} is
considered to indicate normal termination, while other exit codes are
used to report various degrees of abnormality.

{\bf Error Cases}
\bi
\item 	{\tt Code} is not an integer
\bi
\item 	{\tt type\_error(Integer,Code)}
\ei
\ei

\standarditem{prompt(+NewPrompt, ?OldPrompt)}{prompt/2}
    Sets the prompt of the top level interpreter to {\tt NewPrompt} and 
    returns the old prompt in {\tt OldPrompt}.

    An example of {\tt prompt/2} 's use is the following:

    \stuff{
        \>   \>     | ?- prompt('Yes master > ', P). \\
        \\
        \>   \>     P = | ?- ; \\
        \\
        \>   \>     no \\
        \>   \>     Yes master > fail. \\
        \\
        \>   \>     no \\
        \>   \>     Yes master >
    }

\ourmoditem{trimcore}{trimcore/0}{machine}
%
A call to {\tt trimcore/0} reallocates an XSB thread's execution
stacks (and some tabling stacks) to their initial allocation size, the
action affecting only the memory areas for the calling thread.  When
XSB is called in standalone or server mode, {\tt trimcore/0} is
automatically called when the top interpreter level is reached.  When
XSB is embedded in a process, {\tt trimcore/0} is called at the top
interpreter level for any thread created through
{\ xsb\_ccall\_thread\_create()} (see Volume 2, Chapter 3 {\em
  Embedding XSB in a Process}).

\standarditem{gc\_heap}{gc\_heap/0}
\index{garbage collection!heap}
%
Explicitly invokes the garbage collector for a thread's heap. By
default, heap garbage collection is called automatically for each
thread upon stack expansion, unless the {\tt xsb\_flag} {\tt
  heap\_garbage\_collection} is set to {\tt none}.  Automatic heap
garbage collection should rarely need to be turned off, and should
rarely need to be invoked manually.

\comment{
\standarditem{garbage\_collection(+Option)}{garbage\_collection/1}
Sets the system so that subsequent heap garbage collecting will be
done according to the specified {\tt Option}.  {\tt Option} may be the
atom \verb|none| indicating that heap garbage collection is turned
off; it may be the atom \verb|sliding| indicating that sliding garbage
collection will be done; the atom \verb|copying| indicating that the
copying garbage collector will be used; or it may be the atom
\verb|indirection| indicating that the indirect-sliding garbage
collector will be used.
}
 
\standarditem{cputime(-CPU\_Time)}{cputime/1}
%
Returns the (process-level) {\tt CPU\_Time} at the time of the call in
seconds.  The difference between results of successive calls to this
predicate can measure the time spent in specific predicates.  Note
that in the multi-threaded engine, {cputime/1} measures the time for
all threads.

\standarditem{walltime(-Time)}{cputime/1}
%
Returns the \texttt{Time}, in seconds, since execution started, or
since the last call to \texttt{statistics(0)} by any thread.

\standarditem{time(+Goal)}{time/1}
\index{Prologs!SWI}
\index{Prologs!YAP}
%
Prints both the CPU time and wall time taken by the execution of \texttt{Goal}.
Any choice-points of \texttt{Goal} are discarded. The definition of predicate is based 
on the SWI-Prolog definition (minus reporting the number of inferences, which XSB does 
not currently support). This predicate is also found on other Prolog compilers such as YAP.

\standarditem{statistics}{statistics/0}
%
Displays usage information on the current output stream, including: 
\begin{itemize} 
\item Process-level information about allocated memory excluding
  execution stacks but including: 
\begin{itemize}
\item {\tt atom} Space used to maintain global information about
  predicates and structures. 
%
\item {\tt string} Space used to maintain information about atomic
  constants in XSB.
%
\item {\tt asserted} Space allocated for dynamic code.
%
\item {\tt asserted} Space allocated for static code.
%
\item {\tt foreign} Space allocated for foreign predicates.
%
\item {\tt table} Space allocated for XSB's tables.
%
\item {\tt findall} Space allocated for buffers to support {\tt
  findall/3} and similar predicates.
%
\item {\tt mt-private} Private space used by threads.
%
\item {\tt profiling} Space used to maintain profiling information, if
  XSB is called with profiling on.
%
\item {\tt gc temp} Temporary space for used for heap garbage
  collector.
%
\item {\tt interprolog} space allocated for the Interprolog XSB/Java
  interface.
%
\item {\tt thread} space allocated for the thread table
%
\item the space occupied by subgoal and answer tables (in the form of
  tries) \cite{RRSSW98,CuSW99b,TST99}.  In the multi-threaded
  configuration process level table space includes shared tables but
  not private tables.
\end{itemize}

\item Thread-specific information about allocation of memory
  for the calling thread including the  
\begin{itemize} 
\item Global stack (heap) and local (environment) stack (see e.g.
  \cite{AitK90}) for the calling thread. Memory for these two WAM
  stacks is allocated as a single unit so that each stack grows
  together; information is provided on the current allocation for the
  stacks as well as on the stack sizes themselves.  (See
  Section~\ref{sec:EmuOptions} for the memory re-allocation
  algorithm).  
%
\item Trail and choice point stack (see e.g. \cite{AitK90}) for the
  calling thread.  Memory for these two WAM stacks is allocated as a
  single unit so that each stack grows together; information is
  provided on the current allocation for the stacks as well as on the
  stack sizes themselves.  The (re-)allocation follows the algorithm
  sketched in Section~\ref{sec:EmuOptions}).  (See
  Section~\ref{sec:EmuOptions} for the memory re-allocation
  algorithm).  
%
\item SLG unification stack for the calling thread This stack is used
  as a space to copy terms from the execution stacks into table space,
  or back out.  This stack will not be reallocated unless extremely
  large terms are tabled.
%
\item SLG completion stack for the calling thread.  The completion
  stack is used to perform incremental completion for sets of mutually
  dependent tabled subgoals.  One completion stack frame is allocated
  per tabled subgoal \cite{SaSw98} but the size of these frames is
  version-dependent.
%
\item the space occupied by private subgoal and answer tables for the
  calling thread. 

\end{itemize}
In XSB's single-threaded configuration, maximum space used by each of
will be output if the {\tt '-s'} command-line option is used

\item Information about the number of tabling operations performed in
  the session by any thread.  Note that the statistics are divided up
  between calls to predicates that use variant tabling and those that
  use (call) subsumptive tabling (see
  Section~\ref{sec:SimilarityMeasures} and \cite{TST99}).
\begin{itemize}
\item Call Subsumption Subgoal Operations.  For predicates that use
subsumptive tabling, the total number of subsumptive subgoal calls is
given, as is the number of new calls ({\tt producers}) and the number
of repeated calls to non-completed tables ({\tt variants}).
Furthermore, the number of properly subsumed calls to incomplete
tables is given, along with the number of subsumed calls to completed
tables.  Finally, the total number of subsumptive table entries
overall is given, including all producer and consumer calls.
%
\item Call Subsumption Answer Operations.  In call subsumptive
tabling, answer lists are copied from producer subgoals to subsumed
consumer subgoals (this operation is not required in variant tabling).
The number of {\tt answer ident} operations represents the number of
times this copy is done.  In addition, the number of consumptions
performed by all consuming subsumptive table entries is also given.
%
\item Call Variance Subgoal Operations.  For call variance the number
of subgoal check/insert operations is given along with the unique
number of subgoals encountered ({\tt generator}) and the number of
redundant consumer encountered ({\tt consumer}).
%
\item Total Answer Operations.  For both variant and subsumptive
tables, the number of answer check insert operations is given along
with the number of answers actually inserted into the table and the
number of redundant answers derived.
\end{itemize}
%
\item Garbage Collection Information.  Time spent garbage collecting
  by the calling thread and number of heap cells collected.

\item Information about process CPU and clock time, as well as the
  number of active threads.
    \end{itemize}

As mentioned above, if XSB is configured with the single-threaded
engine and is invoked with the {\tt '-s'} option (see
Section~\ref{sec:EmuOptions}), additional information is printed out
about maximum use of each execution stack and table space.  However,
the {\tt '-s'} option can substantially slow down the emulator so
benchmarks of time should be performed separately from benchmarks of
space.

{\bf Example}: The following printout shows how the {\tt statistics/0}
output looks if it is invoked with the {\tt '-s'} option (without it
the {\tt Maximum stack used}, and {\tt Maximum table space used} lines
are not shown).  Information about the allocation size is provided
since the sizes can be changed through emulator options (see
Section~\ref{sec:EmuOptions}).

    {\footnotesize
     \begin{verbatim}
| ?- statistics.
Memory (total)         2429504 bytes:       726696 in use,      1702808 free
  permanent space       645520 bytes:       645520 in use,            0 free
    atom                                    120328
    string                                  156872
    asserted                                  3184
    compiled                                358216
    other                                     6920
  glob/loc space        786432 bytes:          652 in use,       785780 free
    global                                     456 bytes
    local                                      196 bytes
  trail/cp space        786432 bytes:          476 in use,       785956 free
    trail                                       88 bytes
    choice point                               388 bytes
  SLG unific. space      65536 bytes:            0 in use,        65536 free
  SLG completion         65536 bytes:            0 in use,        65536 free
  SLG table space        80048 bytes:        80048 in use,            0 free

  Maximum stack used: global 436724, local 14780, trail 27304, cp 20292,
                      SLG completion 0 (0 subgoals)
  Maximum table space used:  0 bytes

Tabling Operations
  0 subsumptive call check/insert ops: 0 producers, 0 variants,
  0 properly subsumed (0 table entries), 0 used completed table.
  0 relevant answer ident ops.  0 consumptions via answer list.
  0 variant call check/insert ops: 0 producers, 0 variants.
  0 answer check/insert ops: 0 unique inserts, 0 redundant.


   0 heap (  0 string) garbage collections by copying: collected 0 cells in 0.000000 secs

Time: 0.190 sec. cputime,  13.921 sec. elapsetime
\end{verbatim}} 

%===================

\standarditem{statistics(+Type)}{statistics/1}
%
{\tt statistics/1} allows the user to output detailed statistical
information about the atom and symbol tables, as well as about table
space.  The following calls to {\tt statistics/1} are supported:
%
\begin{itemize}
\item {\tt statistics(reset)}  Resets the CPU time as well as counts
for various tabling operations. 
%
\item {\tt statistics(atom)} Outputs statistics about both the atom
and symbol tables.  An example is: 
%
{\footnotesize
\begin{verbatim}
| ?- statistics(atom).

Symbol table statistics:
------------------------
Table Size:	8191
Total Symbols:	1188
            used buckets:              1088  (range: [0, 8174])
            unused buckets:            7103
            maximum bucket size:       3  (#: 18)

String table statistics:
------------------------
Table Size:	16381
Total Strings:	1702
            used buckets:              1598  (range: [0, 16373])
            unused buckets:            14783
            maximum bucket size:       3  (#: 2318)
\end{verbatim}}
%
\item {\tt statistics(table)} Outputs {\em very} detailed statistics
  about table space, including breakdowns into variant and subsumptive
  call- and answer- trie nodes and hash tables; answer return list
  nodes, and structures for conditional answers (cf. \cite{SaSw98,
    RRSSW98, TST99, CuSW99a}).  In the multi-threaded engine, these
  data structures are reported both for shared tables and for private
  tables of the calling thread.

  While this option is intended primarily for developers, it can also
  provide valuable information for the serious user of tabling.
\end{itemize}

\end{description}

%need op (just so we dont forget ???


%=====================================================================


