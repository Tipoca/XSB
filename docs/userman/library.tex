\chapter{Library Utilities} \label{library_utilities}
%====================================================

In this chapter we introduce some useful predicates that are supplied
with the system. These predicates are available only when imported
them from (or explicitly consult) the corresponding modules.


\section{List Processing}
%========================
The XSB library contains various list utilities, some of which 
are listed below.  These predicates should be explicitly imported from
the module specified after the skeletal specification of each predicate.
There are a lot more useful list processing predicates in various modules
of the XSB system, and the interested user can find them by 
looking at the sources.

\begin{description}
\ournewitem{append(?List1, ?List2, ?List3)}{basics}\index{{\tt append/3}}
%\predindex{append/3~(L)}
    Succeeds if list {\tt List3} is the concatenation of lists 
    {\tt List1} and {\tt List2}.

\ournewitem{member(?Element, ?List)}{basics}\index{{\tt member/2}}
%\predindex{member/2~(L)}
    Checks whether {\tt Element} unifies with any element of list 
    {\tt List}, succeeding more than once if there are multiple 
    such elements.

\ournewitem{memberchk(?Element, ?List)}{basics}\index{{\tt memberchk/2}}
%\predindex{memberchk/2~(L)}
    Similar to {\tt member}/2, except that {\tt memberchk}/2 is
    deterministic, i.e.\ does not succeed more than once for any call.

\ournewitem{ith(?Index, ?List, ?Element)}{basics}\index{{\tt ith/3}}
%\predindex{ith/3~(L)}
    Succeeds if the ${\tt Index^{th}}$ element of the list {\tt List} 
    unifies with {\tt Element}.  Fails if {\tt Index} is not a positive
    integer or greater than the length of {\tt List}.
    Either {\tt Index} and {\tt List}, or {\tt List} and {\tt Element}, 
    should be instantiated (but not necessarily ground) at the time of 
    the call.

\ournewitem{length(?List, ?Length)}{basics}\index{{\tt length/2}}
%\predindex{length/2~(L)}
    Succeeds if the length of the list {\tt List} is {\tt Length}.
    This predicate is deterministic if {\tt List} is instantiated 
    to a list of definite length, but is nondeterministic if 
    {\tt List} is a variable or has a variable tail.  If {\tt List}
    is uninstantiated, it is unified with a list of length {\tt Length}
    that contains variables.

\ournewitem{same\_length(?List1, ?List2)}{basics}\index{{\tt same\_length/2}}
%\predindex{same\_length/2~(L)}
    Succeeds if list {\tt List1} and {\tt List2} are both lists of
    the same number of elements.  No relation between the types or
    values of their elements is implied.  This predicate may be used
    to generate either list (containing variables as elements) given
    the other, or to generate two lists of the same length, in which
    case the arguments will be bound to lists of length $0,1,2,\ldots$.

\ournewitem{select(?Element, ?L1, ?L2)}{basics}\index{{\tt select/3}}
%\predindex{select/3~(L)}
    {\tt List2} derives from {\tt List1} by selecting (removing) an 
    {\tt Element} non-deterministically.

\ournewitem{reverse(+List, ?ReversedList)}{basics}\index{{\tt reverse/2}}
%\predindex{reverse/2~(L)}
    Succeeds if {\tt ReversedList} is the reverse of list {\tt List}.
    If {\tt List} is not a proper list, {\tt reverse/2} can succeed
    arbitrarily many times.  It works only one way.

\ournewitem{perm(+List, ?Perm)}{basics}\index{{\tt perm/2}}
%\predindex{perm/2~(L)}
    Succeeds when {\tt List} and {\tt Perm} are permutations of each
    other.  The main use of {\tt perm/2} is to generate permutations
    of a given list.  {\tt List} must be a proper list.
    {\tt Perm} may be partly instantiated.

\ournewitem{subseq(?Sequence, ?SubSequence, ?Complement)}{basics}
\index{{\tt subseq/3}}
%\predindex{subseq/3(L)}
    Succeeds when {\tt SubSequence} and {\tt Complement} are both
    subsequences of the list {\tt Sequence} (the order of corresponding
    elements being preserved) and every element of {\tt Sequence} which
    is not in {\tt SubSequence} is in the {\tt Complement} and vice
    versa.  That is,
    \[ length({\tt Sequence}) =
		length({\tt SubSequence})+length({\tt Complement}) \]
    for example, {\tt subseq([1,2,3,4], [1,3], [2,4]).}
    The main use of {\tt subseq/3} is to generate subsets and their
    complements together, but can also be used to interleave two lists
    in all possible ways.

\ournewitem{merge(+List1, +List2, ?List3)}{listutil}\index{{\tt merge/3}}
%\predindex{merge/3~(L)}
    Succeeds if {\tt List3} is the list resulting from ``merging'' lists 
    {\tt List1} and {\tt List2},
    i.e.\ the elements of {\tt List1} together with any element of 
    {\tt List2} not occurring in {\tt List1}.
    If~{\tt List1} or~{\tt List2} contain duplicates, {\tt List3} may 
    also contain duplicates.

\ournewitem{absmerge(+List1, +List2, ?List3)}{listutil}\index{{\tt absmerge/3}}
%\predindex{absmerge/3~(L)}
    Predicate {\tt absmerge/3} is similar to {\tt merge/3}, except that 
    it uses predicate {\tt absmember/2} described below rather than 
    {\tt member/2}.

\ournewitem{absmember(+Element, +List)}{listutil}\index{{\tt absmember/2}}
%\predindex{absmember/2~(L)}
    Similar to {\tt member}/2, except that it checks for identity
    (through the use of predicate {\tt '=='/2}) rather than unifiability 
    (through {\tt '='/2}) of {\tt Element} with elements of {\tt List}.

\ournewitem{member2(?Element, ?List)}{listutil}\index{{\tt member2/2}}
%\predindex{member2/2~(L)}
    Checks whether {\tt Element} unifies with any of the actual elements 
    of {\tt List}.  The only difference between this predicate and 
    predicate {\tt member/2} is on lists having a variable tail, 
    e.g.\ \verb'[a, b, c | _ ]': while {\tt member/2} would insert 
    {\tt Element} at the end of such a list if it did not find it, 
    Predicate {\tt member2/2} only checks for membership but does not 
    insert the {\tt Element} into the list if it is not there.

\ournewitem{delete\_ith(+Index, +List, ?Element, ?RestList)}{listutil}
\index{{\tt delete\_ith/4}}
%\predindex{delete\_ith/4~(L)}
    Succeeds if the ${\tt Index^{th}}$ element of the list {\tt List}
    unifies with {\tt Element}, and {\tt RestList} is {\tt List} with
    {\tt Element} removed.  Fails if {\tt Index} is not a positive
    integer or greater than the length of {\tt List}.

\ournewitem{closetail(?List)}{listutil}\index{{\tt closetail/1}}
%\predindex{closetail/1~(L)}
    Predicate {\tt closetail/1} closes the tail of an open-ended list.
    It succeeds only once.

\end{description}



\section{Asserting Dynamic Code} \label{LoadDyn}
%===============================================

The module {\tt consult} in directory {\tt lib} provides several handy
library predicates that can assert the contents of a file into
XSB's database.  The use of these predicates may be necessary
when the code needs to be {\tt dynamic} (so that it is retractable),
or when it contains atoms whose length is more than 255 that cannot be
handled by the XSB compiler.

\begin{description}
\ournewitem{load\_dyn(+FileName)}{consult}
\index{{\tt load\_dyn/1}}\label{load_dyn/1}
%\predindex{load\_dyn/1~(L)}
    Asserts the contents of file {\tt FileName} into the database.
    All existing clauses of the predicates in the file that alreay
    appear in the database, are retracted unless there is a {\tt
    multifile/1} declaration for them.  Clauses in the file must be
    in a format that {\tt read/1} will process.  So, for example,
    operators are permitted.  As usual, clauses of predicates are not
    retracted if they are compiled instead of dynamically asserted.
    All predicates are loaded into {\tt usermod}.  Module declarations
    such as {\tt :- export} are ignored and a warning is issued.
    Starting with Version 1.5.0, this predicate is a standard predicate
    of XSB.

\ournewitem{ensure\_dyn\_loaded(+FileName)}{consult}
\index{{\tt ensure\_dyn\_loaded/1}}
    Is similar to {\tt load\_dyn/1} except that it does nothing if the
    file has previously been loaded and the file has not been changed
    since.  However the file will be reloaded if the index declaration of
    any predicate in that file has changed to require more indexing, or a
    larger hash table.

\ournewitem{load\_dync(+FileName)}{consult}\index{{\tt load\_dync/1}}
%\predindex{load\_dync/1~(L)}
    Asserts the contents of file {\tt FileName} into the database.
    All existing clauses of the predicates in the file that already appear
    in the database, are retracted unless there is a {\tt multifile/1}
    directive for them.  The terms in the file {\tt FileName} must be in
    ``canonical'' format; that is, they must not use any operators (or
    list notation.) This is the format produced by the predicate {\tt
    write\_canonical/1}. (See {\tt cvt\_canonical/2} to convert a file from
    the usual {\tt read/1} format to {\tt read\_canonical} format.)  As
    usual, clauses of predicates are not retracted if they are compiled
    instead of dynamically asserted. All predicates are loaded into {\tt
    usermod}.  {\tt :- export} declarations are ignored and a warning is
    issued.

    Notice that this predicate can be used to load files of Datalog facts
    (since they will be in canonical format).  This predicate is
    significantly faster than {\tt load\_dyn/1} and should be used when
    speed is important.  A file that is to be dynamically loaded often but
    not often modified by hand should be loaded with this predicate.  Use
    predicate {\tt cvt\_canonical/2} (see below) to convert a usual file
    to a format readable by this predicate.

\ournewitem{ensure\_dync\_loaded(+FileName)}{consult}
\index{{\tt ensure\_dync\_loaded/1}}
    Is similar to {\tt load\_dync/1} except that it does nothing if the
    file has previously been loaded and the file has not been changed
    since.  However the file will be reloaded if the index declaration of
    any predicate in that file has changed to require more indexing, or a
    larger hash table.

\ournewitem{cvt\_canonical(+FileName1,+FileName2)}{consult}
\index{{\tt cvt\_canonical/2}}
    Converts a file from standard term format to ``canonical'' format.
    The input file name is {\tt FileName1}; the converted file is put in
    {\tt FileName2}.  This predicate can be used to convert a file in
    standard Prolog format to one loadable by {\tt load\_dync/1}.
\end{description}

%----------------------------------------------------------------------


\section{Ground, Numbervars, Subsumption, Variant} \label{NumberVars}
%====================================================================

\begin{description}
\ournewitem{ground(+X)}{basics}\index{{\tt ground/1}}
%\predindex{ground/1~(L)}
    Succeeds if {\tt X} is currently instantiated to a term that is 
    completely bound (has no uninstantiated variables in it); 
    otherwise it fails.  Predicate {\tt ground/1} has no associated 
    error conditions.

\ournewitem{numbervars(+Term, +FirstN, ?LastN)}{num\_vars}
\index{{\tt numbervars/3}}
%\predindex{numbervars/3~(L)}
    This predicate provides a mechanism for grounding a (HiLog) term
    so that it may be analyzed.  Each variable in the (HiLog) term
    {\tt Term} is instantiated to a term of the form \verb|'$VAR'(N)|,
    where {\tt N} is an integer starting from {\tt FirstN}.  
    {\tt FirstN} is used as the value of {\tt N} for the first
    variable in {\tt Term} (starting from the left). The second distinct
    variable in {\tt Term} is given a value of {\tt N} satisfying
    {\tt "N is FirstN + 1"} and so on.  The last variable in {\tt Term}
    has the value {\tt LastN-1}.

\ournewitem{numbervars(+Term)}{num\_vars}\index{{\tt numbervars/1}}
%\predindex{numbervars/1~(L)}
    This predicate is defined as:
    \begin{center}
    {\tt   numbervars(Term, 0, \_)}.
    \end{center}
    It is included solely for convenience.

%\ournewitem{varnumbers(Term, FirstN, Copy)}{num\_vars}
%\index{{\tt varnumbers/3}}
%%\predindex{varnumbers/3~(B)}
%    This predicate is a partial inverse of predicate {\tt numbervars/3}.
%    It unifies {\tt Copy} with a copy of {\tt Term} in which subterms of
%    the form \verb|'$VAR'(N)| where {\tt N} is an integer not less than
%    {\tt FirstN} have been systematically replaced by fresh variables. 
%    Since 0 is the usual second argument of numbervars/3,
%    \begin{center}
%    {\tt   varnumbers(Term, Copy)}
%    \end{center}
%    is also provided.

\ournewitem{subsumes(?Term1, +Term2)}{subsumes}\index{{\tt subsumes/2}}
%\predindex{subsumes/2~(L)}
    Term subsumption is a sort of one-way unification.  Term {\tt Term1}
    and {\tt Term2} unify if they have a common instance, and unification
    in Prolog instantiates both terms to that (most general) common instance.
    {\tt Term1} subsumes {\tt Term2} if {\tt Term2} is already an instance of
    {\tt Term1}.  For our purposes, {\tt Term2} is an instance of {\tt Term1}
    if there is a substitution that leaves {\tt Term2} unchanged and makes
    {\tt Term1} identical to {\tt Term2}.  Predicate {\tt subsumes/2} does
    not work as described if {\tt Term1} and {\tt Term2} share common
    variables.

\ournewitem{subsumes\_chk(+Term1, +Term2)}{subsumes}
\index{{\tt subsumes\_chk/2}}
%\predindex{subsumes\_chk/2~(L)}
    The {\tt subsumes\_chk/2} predicate is true when {\tt Term1} subsumes 
    {\tt Term2}; that is, when {\tt Term2} is already an instance of
    {\tt Term1}.  This predicate simply checks for subsumption and 
    does not bind any variables either in {\tt Term1} or in {\tt Term2}.

    Examples:
    {\footnotesize
     \begin{verbatim}
            | ?- subsumes_chk(a(X,f,Y,X),a(U,X,b,S)).

            no
            | ?- subsumes_chk(a(X,Y,X),a(b,b,b)).

            X = _595884
            Y = _595624
     \end{verbatim}}

\ournewitem{variant(?Term1, ?Term2)}{subsumes}\index{{\tt variant/2}}
%\predindex{variant/2~(L)}
    This predicate is true when {\tt Term1} and {\tt Term2} are 
    alphabetic variants.  That is, you could imagine that {\tt variant/2}
    as being defined like:
    \begin{center}
    \begin{minipage}{3.5in}
    \begin{verbatim}
	variant(Term1, Term2) :-
	     subsumes_chk(Term1, Term2),
	     subsumes_chk(Term2, Term1).
    \end{verbatim}
    \end{minipage}
    \end{center}
    but the actual implementation of {\tt variant/2} is considerably more
    efficient.  However, in general, it does not work for terms that share
    variables; an assumption that holds for most (reasonable) uses of
    {\tt variant/2}.  Whenever the terms may share variables, the above
    definition of variant (via {\tt subsumes\_chk/2}) should be used.
\end{description}


\section{Lower-Level I/O}
%======================

XSB  has various low-level routines that support input and output, 
at both the term level and the character level.  The supporting
predicates are described here.

\begin{description}

\ournewitem{fmt\_read(+Fmt,-Term,-Ret)}{file\_io}\index{{\tt fmt\_read/3}}
%\predindex{fmt\_read/4~(L)}
    This predicate provides a low-level routine for reading data from
    the current input file (which must have been already opened by using
    {\tt see/1}) according to a C format, as used in the C function
    {\tt scanf}. To use it, it must be imported from the module {\tt
    file\_io}.  {\tt Fmt} must be a string of characters (enclosed in "")
    representing the format that 
    will be passed to the C call to {\tt scanf}.  See the C
    documentation for {\tt scanf} for the meaning of this string.
    The usual alphabetical C escape characters ({\it e.g.}, $\backslash n$)
    are recognized, but not the octal or the hexadecimal ones.
    Another difference with C is that, unlike most C compilers, XSB insists
    that a single {\tt \%} in the format string signifies format conversion
    specification. (Some C compilers might output {\tt \%} if it is not
    folowed by a valid type conversion spec.) So, to output {\tt \%}
    you must type {\tt \%\%}.
    Format can also be an atom enclosed in single quotes. However, in that
    case, escape sequences are not recognized and are printed as is.
    {\tt Term} is a term ({\it e.g.}, {\tt args(X,Y,Z)})  whose arguments
    will be unified with 
    the field values read in.  (The functor symbol of {\tt
    Term} is ignored.)  
    {\tt Ret} must be a variable and it will be assigned a return value by
    the predicate: a negative integer if end-of-file is encountered;
    otherwise the number of fields read (as returned by {\tt scanf}.)

\ournewitem{fmt\_write(+Fmt,+Term)}{file\_io}\index{{\tt fmt\_write/2}}
    This predicate provides a low-level routine for writing data to
    the current output file (which must have been already opened by using
    {\tt tell/1}) according to a C format, as used in the C function
    {\tt printf}.
    To use it, it must be imported from the module {\tt file\_io}.
    {\tt Fmt} must be a string of characters (enclosed in "")
    representing the format that 
    will be passed to the C call to {\tt scanf}.  See the C
    documentation for {\tt scanf} for the meaning of this string.
    The usual alphabetical C escape characters ({\it e.g.}, $\backslash n$)
    are recognized, but not the octal or the hexadecimal ones.
    Another difference with C is that, unlike most C compilers, XSB insists
    that a single {\tt \%} in the format string signifies format conversion
    specification. (Some C compilers might output {\tt \%} if it is not
    folowed by a valid type conversion spec.) So, to output {\tt \%}
    you must type {\tt \%\%}. Format can also be an atom, but then escape
    sequences are not recognized.
    {\tt Term} is a term ({\it e.g.}, {\tt args(X,Y,Z)}) whose arguments
    will be output. 
    The functor symbol of {\tt Term} is ignored.

\ournewitem{fmt\_write\_string(-String,+Fmt,+Term)}{file\_io}\index{{\tt fmt\_write\_string/2}}
    This predicate works like the C function {\tt sprintf}. It takes the
    format string and substitutes the values from the arguments of {\tt
      Term} ({\it e.g.}, {\tt args(X,Y,Z)}) for the formatting instructions
    \%s, \%d, etc. The result is available in {\tt String}. {\tt Fmt} is a
    string or an atom that represents the format, as described earlier.

\ournewitem{file\_open(+FileName,+Mode,-FileDes)}{file\_io}
\index{{\tt file\_open/3}}
    Opens a file with name {\tt FileName} to be accessed in mode {\tt
    Mode} and returns a file-descriptor in {\tt FileDes} that can be
    used to access the file.  If {\tt Mode} is 0, the the file is
    opened for reading; if it is 1, the file is opened for writing; if
    it is 2, the file is opened for appending.  If {\tt Mode} is 3,
    then the string making up the atom {\tt FileName} is treated as
    the contents of the file, and a descriptor is returned that allows
    ``file'' access to that string.  This is how one can use
    XSB's term I/O routines to build terms from atoms.

\ournewitem{file\_close(+FileDes)}{file\_io}\index{{\tt file\_close/1}}
    Closes the file (or string) for descriptor {\tt FileDes}.

\ournewitem{file\_flush(+FileDes, -Return)}{file\_io}
\index{{\tt file\_flush/2}}
    Any buffered data gets delivered. If the call is successful, {\tt Return}
    is zero; otherwise {\tt EOF} is returned.

\ournewitem{file\_seek(+FileDes, +Offset, +Place, -Return)}{file\_io}
\index{{\tt file\_seek/4}}
    Sets the file position indicator for the next input or output
    operation. The position is {\tt Offset} bytes from {\tt Place}.
    The value of {\tt Place} can be 0, 1, or 2, which correspond to
    the beginning of the file, the current position in the file, or
    the end of the file, respectively. If the call is successful,
    {\tt Return} is set to zero.

\ournewitem{file\_truncate(+FileDes, +Length, -Return)}{file\_io}
\index{{\tt file\_truncate/3}}
    The regular file  referenced by file descriptor {\tt FileDes}
    is chopped to have a size of {\tt Length} bytes. Upon successful
    completion {\tt Return} is set to zero.

\ournewitem{file\_write(+FileDes,+Term)}{xsb\_writ}
\index{{\tt file\_write/2}}
    Writes the term {\tt Term} to the file (or string) with descriptor {\tt
    FileDes}.

\ournewitem{file\_read(+FileDes,-Term)}{xsb\_read}
\index{{\tt file\_read/2}}
    Reads a term from the file (or string) with descriptor {\tt
    FileDes} into {\tt Term}.  Note that the term must be terminated
    with a period (.) (whether it appears in a file or in a string.)

\ournewitem{file\_read(+FileDes,-Term,-Vars)}{xsb\_read}
\index{{\tt file\_read/3}}
    Reads a term from the file (or string) with descriptor {\tt
    FileDes} into {\tt Term}, and returns in {\tt Vars} an open-tailed list of
    pairs of names of variables and the variables themselves that
    appear in Term.  For example, reading a term {\tt f(a,X,Y,X)}
    would result in {\tt term} being bound to {\tt
    f(a,\_25,\_26,\_25)} (for some internal variables) and {\tt Vars}
    being bound to {[vv('X',\_25),vv('Y',\_26)|\_83]}.  Note that the
    pairing functor symbol is {\tt vv/2} and it must be imported from
    {\tt xsb\_read} along with this read predicate.  Also note that 
    {\tt Vars} is not a proper list, but has a free variable instead 
    of [] at its end.

\ournewitem{file\_read\_canonical(+FileDes,-Term,-Psc)}{machine}
\index{{\tt file\_read\_canonical/3}}
    Reads a term that is in canonical format from the the file
    indicated by {\tt FileDes} (as returned by {\tt file\_open/3} or
    by {\tt stat\_flag(10,FileDes))}, and returns it in {\tt Term}.
    It also returns (in {\tt Psc}) the psc address of the main functor
    symbol of the term, if it is the same as that of the previously
    read term, and the current term is a ground (non 0-ary) fact.
    (This is used for efficiency in the implementation of {\tt
    load\_dync/1}).  Otherwise {\tt Psc} is set to 0.  To initialize
    its previous psc value to zero, this predicate can be called with
    {\tt FileDes} of -1000.

\ournewitem{file\_read\_line(+FileDes,-String,-IsFullLine)}{file\_io}
\index{{\tt file\_read\_line/3}}
    This is a low-level predicate that allows XSB to read input files
    efficiently, line by line. It returns the string read from {\tt
      FileDes} using the variable {\tt String}. The output variable {\tt
      IsFullLine} is 1, if the line read contains the newline character at
    the end. Otherwise, it is 0. This latter case arises in two situations:
    when the last line in the stream does not have a newline  character or
    when the line is very long, longer than the buffer allocated for that
    purpose. (In such a case {\tt file\_read\_line/3} will read only as
    much as the buffer allows.) 

    This predicate fails on reaching the end of file.
    
    {\bf Important}: This predicate does not intern the string it reads
    from the input, so you cannot unify or compare the value read with
    anything. Moreover, if you use this predicate twice, then the second
    call replaces the value read by the first call. In other words, this
    predicate is very low-level and should be used with great care.
    One simple use of this facility is to copy one file into another.
    See \verb|file_read_line_atom| and \verb|file_read_line_list| for
    alternatives that are easier to use.

\ournewitem{file\_read\_line\_atom(+FileDes,-String,-IsFullLine)}{file\_io}
\index{{\tt file\_read\_line\_atom/3}}
    This predicate is like \verb|file_read_line|, but the line read from
    the input is interned. Therefore, {\tt String} is instantiated
    to a normal atom.
    
    {\bf Important}: At present, XSB does not have atom table garbage
    collector. Therefore, each line read from the file using this predicate
    gets stored in the atom table. Thus, large files can cause XSB to run
    out of memory.  This problem will go away when atom table garbage
    collection is implemented.

\ournewitem{file\_read\_line\_atom(-String,-IsFullLine)}{file\_io}
\index{{\tt file\_read\_line\_atom/2}}
   Like \verb|file_read_line_atom/3|, but {\tt FileDes} is not required.
   The file being read is the one previously opened with {\tt see/1}.

\ournewitem{file\_read\_line\_list(+FileDes,-CharList,-IsFullLine)}{file\_io}
\index{{\tt file\_read\_line\list/3}}
    This predicate is like \verb|file_read_line_atom|, but the line read from
    the input is converted into a list of characters.
    This predicate is \emph{much} more efficient than {{\tt fget\_line/3}}
    (see below), and is recommended when speed is important.

\ournewitem{file\_read\_line\_list(-String,-IsFullLine)}{file\_io}
\index{{\tt file\_read\_line\_list/2}}
   Like \verb|file_read_line_list/3|, but {\tt FileDes} is not required.
   The file being read is the one previously opened with {\tt see/1}.

\ournewitem{fget\_line(+Str,-Inlist,-Next)}{scrptutl}
\index{{\tt fget\_line/3}}
    {\tt fget\_line/3} reads one line from the input stream {\tt Str} and
    unifies {\tt Inlist} to the list of ASCII integers representing the
    characters in the line, and {\tt Next} to the line terminator, either
    a newline ({\tt 10}) or EOF ({\tt-1}).

\ournewitem{file\_write\_line(+FileDes, +String, +Offset)}{file\_io}
\index{{\tt file\_write\_line/3}}
   Write {\tt String} beginning with character {\tt Offset}  to the output
   file represented by the file descriptor {\tt FileDes}. {\tt String} can
   be an atom or a list of ASCII characters. This does \emph{not} put the
   newline character at the end of the string (unless {\tt String} already
   had this character). Note that escape sequences, like \verb|\n|, are
   recognized if {\tt String} is a character list, but are output as is if
   {\tt String} is an atom.
   
\ournewitem{file\_write\_line(+String, +Offset)}{file\_io}
\index{{\tt file\_write\_line/2}}
   Like \verb|file_write_line/3|, but output goes to the currently open
   output stream.


\ournewitem{file\_getbuf(+FileDes, +ByteCount, -String, -BytesRead)}{file\_io}
\index{{\tt file\_getbuf/4}}
Read {\tt ByteCount} bytes from file represented by file descriptor {\tt
  FileDes} (which must already be open for reading) into variable {\tt
  String}. This is analogous to {\tt fread} in C.  

Note: The string read is {\bf not} interned. Please see exlanations to
\verb|file_read_line| on this matter. Because of the difficulties in using
this predicate, the predicates \verb|file_getbuf_atom| and
\verb|file_getbuf_list| are often better alternatives.

\ournewitem{file\_getbuf\_atom(+FileDes, +ByteCount, -String, -BytesRead)}{file\_io}
\index{{\tt file\_getbuf\_atom/4}}
Like \verb|file_getbuf|, but {\tt String} is instantiated to an interned atom.

Note: because XSB does not have an atom table garbage collector yet, this
predicate should not be used to read large files.

\ournewitem{file\_getbuf\_atom(+ByteCount, -String, -BytesRead)}{file\_io}
\index{{\tt file\_getbuf\_atom/3}}
Like \verb|file_getbuf_atom/4|, but reads from the currently open input stream
(using {\tt see/1}).

\ournewitem{file\_getbuf\_list(+FileDes, +ByteCount, -CharList, -BytesRead)}{file\_io}
\index{{\tt file\_getbuf\_list/4}}
Like \verb|file_getbuf_atom/4|, but {\tt CharList} is instantiated to a list
of characters that represent the string read from the input.

\ournewitem{file\_getbuf\_list(+ByteCount, -String, -BytesRead)}{file\_io}
\index{{\tt file\_getbuf\_list/3}}
Like \verb|file_getbuf_list/3|, but reads from the currently open input stream
({\it i.e.}, with {\tt see/1}).


\ournewitem{file\_putbuf(+FileDes, +ByteCount, +String, +Offset, -BytesWritten)}{file\_io}
\index{{\tt file\_putbuf/5}}
Write {\tt ByteCount} bytes into file represented by file descriptor {\tt
  FileDes} (which must already be open for writing) from variable {\tt
  String} at position {\tt Offset}. This is analogous to C {\tt fwrite}.
The value of {\tt String} can be an atom or a list of ASCII characters.

\ournewitem{file\_putbuf(+ByteCount, +String, +Offset, -BytesWritten)}{file\_io}
\index{{\tt file\_putbuf/4}}
Like \verb|file_putbuf/3|, but output goes to the currently open output stream.


\end{description}

\section{Script Writing Utilities}
%========================

Prolog, (or XSB) can be useful for writing scripts in a UNIX system.
Prolog's simple syntax and declarative semantics make it especially
suitable for scripts that involve text processing.  The following
library functions are intended to be used on UNIX-based platforms.
Furthermore, they are new to Version 1.7 and should be considered to
be in the Beta test stage.

\begin{description}
\ournewitem{date(?Date)}{scrptutl}
\index{{\tt date/1}}
	Unifies {\tt Date} to the current date, returned as a Prolog
term, suitable for term comparison.

Example:
{\footnotesize
\begin{verbatim}
                > date 
                Thu Feb 20 08:46:08 EST 1997
                > xsb -i
               XSB Version 1.7
               [sequential, single word, optimal mode]
               | ?- [scrptutl].
               [scrptutl loaded]

               yes
               | ?- date(D).
               D = date(1997,1,20,8,47,41)

               yes
\end{verbatim}}

\ournewitem{file\_time(+FileName, -time(Time1,Time2))}{file\_io}
\index{{\tt file\_time/2}}

Returns file's modification time. Because 
XSB steals 5 bits from each word, time must be returned as two words:
Time1, representing the most significant digits, and Time2, representing
the less significant digits.

\ournewitem{file\_size(+FileName, -Size)}{file\_io}
\index{{\tt file\_size/2}}

Returns file size.

\ournewitem{directory(+Path,?Directory)}{directry}
\index{{\tt directory/2}}

	Unifies {\tt Directory} with a list of files in the directory
specified by path.  Information about the files is similar to that
obtained by {\tt ls -l}, but transformed for ease of processing.


\ournewitem{expand\_filename(+FileName,-ExpandedName)}{machine}
\index{{\tt expand\_filename/2}}
Expands the file name passed as the first argument and binds the variable
in the second argument to the expanded name. This includes expanding Unix
tildas, prepending the current directory, etc. In addition, the expanded
file name is ``rectified'' so that multiple repeated slashes are replaced
with a single slash, the intervening ``./'' are removed, and ``../'' are
applied so that the preceding item in the path name is deleted. For
instance, if the current directory is {\tt /home}, then {\tt
  abc//cde/..///ff/./b} will be converted into {\tt /home/abc/ff/b}.

Under NT and Windows, this predicates does rectification (using
backslashes when appropriate), but it does not expand the tildas.

\ournewitem{tilde\_expand\_filename(+FileName,-ExpandedName)}{machine}
\index{{\tt tilde\_expand\_filename/2}}
Like {\tt expand\_filename/2}, but only expands tildas and does
rectification. This does not prepend the current working directory to
relative file names.

\ournewitem{is\_absolute\_filename(+FileName)}{machine} \index{{\tt
is\_absolute\_filename/1}} Succeeds, if file name is absolute; fails
otherwise.  This predicate works also under NT and Windows, {\it
i.e.}, it recognizes drive letters, etc.

\ournewitem{parse\_filename(+FileName,-Dir,-Base,-Extension)}{machine}
\index{{\tt parse\_filename/4}}
This predicate parses file names by separating the directory part, the base
name part, and file extension. If file extension is found, it is removed
from the base name. Also, directory names are rectified and if a directory
name starts with a tilde (in Unix), then it is expanded. Directory names
always end with a slash or a backslash, as appropriate for the OS at hand.

For instance, {\tt $\sim$john///doe/dir1//../foo.bar} will be parsed into:
{\tt /home/john/doe/}, {\tt foo}, and {\tt bar} (where we assume that    
{\tt /home/john} is what {\tt $\sim$john} expands into).  


\ournewitem{sysin(+Command,?Output)}{scrptutl}
\index{{\tt sysin/2}}

	Unifies {\tt Output} with the stdout of {\tt Command}
represented as a list of lists of tokens, where each element in the
outer list of {\tt Command} represents a line of stdout.

Example:
{\footnotesize
\begin{verbatim}
                > uname -a
                Linux swiftlap 1.2.8 #1 Sun May 7 13:10:10 CDT 1995 i486
                | ?- sysin('uname -a',Output).

                Output = [[Linux,swiftlap,1.2.8,#1,Sun,May,7,13:10:10,CDT,1995,i486]]

                yes
\end{verbatim}}

\end{description}


\comment{
The following section about socket is provided by Juliana.
- Baoqiu (1/16/1999)
}

\section{Socket I/O}

The XSB socket library defines a number of predicates manipulating
sockets. They are all rather straight-forward interfaces to the
corresponding BSD-type socket functions with the same name. For detailed
information on sockets, the reader is referred to the Unix man pages
(another good source is \emph{Unix Network Programming}, by W. Richard
Stevens).

To load the package, enter the query 
\begin{verbatim}
| ?- [sockets].
\end{verbatim}

Examples can be found in the examples directory in the XSB source tree.

\begin{description}
\ouritem{socket(-Sockfd)}\index{{\tt socket/1}}
    A socket Sockfd in the AF\_INET domain is created (AF\_UNIX is not
implemented). 

\ouritem{socket\_close(+Sockfd)}\index{{\tt socket\_close/1}}
    Sockfd is closed. Sockets used in socket\_connect/2 should not be
closed by socket\_close/1 as they will be closed when the corresponding
stream is closed.

\ouritem{socket\_bind(+Sockfd,+Port)}\index{{\tt socket\_bind/2}}
   The socket Sockfd is bound to the address (Port is the port number).

\ouritem{socket\_connect(+Sockfd,+Port,+Hostname,Sockfptr)}\index{{\tt socket\_connect/4}}
    The socket Sockfd is connected to the address (Hostname and
Port). Sockfptr is a stream on which items can be both read and written.

\ouritem{socket\_listen(+Socket, +Length)}\index{{\tt socket\_listen/2}}
    The socket Sockfd is defined to have a maximum backlog queue of
Length pending connections.

\ouritem{socket\_accept(+Sockfd,-Sockfptr)}\index{{\tt socket\_accept/2}}
    Block the caller until a connection attempt arrives. If the incoming 
    queue is not empty, the first connection request is accepted, the call
    succeeds and binds Sockfptr to a stream that can be used for this new
    connection.

\ouritem{socket\_flush(+Sockfptr)}\index{{\tt socket\_flush/1}}
    Any buffered data in stream Sockfptr gets delivered.

\ouritem{socket\_select} \index{{\tt socket\_select}} \emph{Not implemented.}

\end{description}

All streams above can be both read from and written on. All
I/O-predicates operating on streams can be used, for example
file\_read/2, file\_write/2, etc.


\comment{

TLS --- Im not quite sure that the sockets are ready for prime time.

\section{The Socket Library}

As a convenience, XSB provides access to sockets through the following
routines.  Currently only the domain {\tt AF\_INET} is supported.

\begin{description}
\ournewitem{socket(-Socket\_fd)}{socket}

\ournewitem{socket\_request(+Socket\_fd,+Port,+Hostname,+Sockfptr)}{socket}



\ournewitem{socket\_bind(+Socket\_fd,+Port)}{socket}

\ournewitem{socket\_listen(+Socket\_fd,+Length)}{socket}

\ournewitem{socket\_accept(+Socket\_fd,Sockfptr)}{socket}

\ournewitem{socket\_connect(+Socket\_fd,Port,Hostname,Sockfptr)}{socket}

\ournewitem{socket\_flush(+Socket\_fd)}{socket}

\ournewitem{socket\_close(+Socket\_fd)}{socket}

\end{description}

(dsw - This is stuff Hasan wrote, wich ought to be formated, cleaned up an dincluded: )

The following calls can be imported from the ``socket'' library.

socket(-Sockfd) :
Creates a new TCP communication end point of type AF_INET. Succeeds with
binding Sockfd to a small integer called a socket descriptor.

socket_bind(+Sockfd, +Port) :
Used for attaching a specific local (integer) Port for the socket with
descriptor Sockfd.

socket_listen(+Sockfd, +QueueLength) :
Used for announcing willingness to accept connections at the socket identified
with descriptor Sockfd. The (integer) QueueLength sets the length of the queue
for the incoming connections at this socket.

socket_accept(+Sockfd, -NewSockfd) :
Block the caller until a connection attempt arrives. If the incoming queue is
not empty, the first connection request is accepted, the call succeeds and
binds NewSockfd with a socket descriptor that will be used for this new
connection.

socket_connect(+Sockfd, +Port, +Hostname, -Sockfileptr) :
The caller identifies a server's IP address in Hostname and port in Port.
TCP will attempt to establish a connection with the server using local socket
with descriptor Sockfd. For information about Sockfileptr argument see the
note below.

socket_close(+Sockfd) :
Terminates and closes a connection identified with the descriptor Sockfd.

socket_recv(+Sockfd, -Data) :
Receives a message from the connection identified with the descriptor Sockfd,
strips off the special end_of_message symbol ```'' and binds it to variable Data.

socket_send(+Sockfd, +Data) :
Appends the end_of_message symbol ```'' to the atom Data and sends Data over
the connection identified with the descriptor Sockfd.

socket_send_eof(+Sockfd) :
Sends the special EOF symbol appended with end_of_message symbol ```'' over
the connection identified with the descriptor Sockfd.

socket_send_asci(+Sockfd, Code) :
Send the ascii character with ASCII code Code appended with end_of_message
symbol ```''  over the connection identified with the descriptor Sockfd.


Note: (UNIX only) file_write_canonical(+Sockfileptr, +Data)
		  file_read_canonical(+Sockfileptr, -Data, -Psc) 
		  socket_flush(+Sockfileptr)

The above calls can be used for sending/flushing and receiving Data over a
Unix file descriptor Sockfileptr associated with a socket. Sockfileptr can be
either obtained from the socket_accept(+Sockfd, -Sockfileptr) in server mode
or from socket_connect(+Sockfd, +Port, +Hostname, -Sockfileptr) in client mode.

}

\section{Arrays}
%===============

The module {\tt array1} in directory {\tt lib} provides a very simple 
backtrackable array implementation.  The predicates through which the 
array objects are manipulated are:

\begin{description}
\ournewitem{array\_new(-Array, +Size)}{array1}\index{{\tt array\_new/2}}
%\predindex{array\_new/2~(L)}
    Creates a one dimensional empty array of size {\tt Size}.  All the 
    elements of this array are variables.
\ournewitem{array\_elt(+Array, +Index, ?Element)}{array1}
\index{{\tt array\_elt/3}}
%\predindex{array\_elt/3~(L)}
    True iff {\tt Element} is the {\tt Index}-th element of array 
    {\tt Array}.
\ournewitem{array\_update(+Array, +Index, +Elem, -NewArray)}{array1}
\index{{\tt array\_update/4}}
%\predindex{array\_update/4~(L)}
    Updates the array {\tt Array} such that the {\tt Index}-th element
    of the new array is {\tt Elem} and returns the new array in 
    {\tt NewArray}.  The implementation is quite efficient in that it 
    avoids the copying of the entire array.
\end{description}

A small example that shows the use of these predicates is the following:
{\footnotesize
 \begin{verbatim}
           | ?- import [array_new/2, array_elt/3, array_update/4] from array1.

           yes
           | ?- array_new(A, 4), array_update(A,1,1,B), array_update(B,2,2,C),
                ( array_update(C,3,3,D), array_elt(D,3,E)
                ; array_update(C,3,6,D), array_elt(D,3,E)
                ; array_update(C,3,7,D), array_elt(D,3,E)
                ).

           A = array(1,2,3,_874600)
           B = array(1,2,3,_874600)
           C = array(1,2,3,_874600)
           D = array(1,2,3,_874600)
           E = 3;

           A = array(1,2,6,_874600)
           B = array(1,2,6,_874600)
           C = array(1,2,6,_874600)
           D = array(1,2,6,_874600)
           E = 6;

           A = array(1,2,7,_874600)
           B = array(1,2,7,_874600)
           C = array(1,2,7,_874600)
           D = array(1,2,7,_874600)
           E = 7;

           no
 \end{verbatim}
}


\section{The Profiling Library}
%==============================

These predicates are available by loading the library through
{\tt [prof\_lib]}.

\begin{description}
\ournewitem{measure(+Call)}{prof\_lib}\index{{\tt measure/1}}
%\predindex{measure/1~(L)}\label{p:measure}
     The same as {\tt once/1} (see the section {\it Meta-Predicates}
in Volume 1)
     but it also prints out statistical information about the call,
     that looks like the information printed out by predicate
     {\tt statistics/0} (see the section {\it Execution State} in Volume 1).
\end{description}

%----------------------------------------------------------------------
%\input{tr_assert}
\section{Asserts/Retracts using Tries }

\index{{\tt trie\_assert/1}}
\index{{\tt trie\_retract/1}}
\index{{\tt trie\_retractall/1}}
\index{{\tt trie\_retract\_nr/1}}
\index{{\tt abolish\_trie\_asserted/1}}
\index{{\tt trie\_dynamic/1}}

In \version, trie asserted code has been merged with standard asserted
code.  If the user wishes to use tries for dynamic code, the
recommended programming practice is as outlined in the section {\it
Modification of the Database} in Volume 1.
For compatability with previous versions, the
predicates {\tt trie\_assert/1}, {\tt trie\_retract/1}, {\tt
trie\_retractall/1}, {\tt trie\_retract\_nr/1}, {\tt
abolish\_trie\_asserted/1} and {\tt trie\_dynamic/1} can be imported
from the module {\sf tables}.  However, if the current index
specification of these predicates is {\tt trie} (again, see the
section {\it Modification of the Database} in Volume 1, the predicates
are defined as {\tt assert/1}, {\tt 
retract/1}, {\tt retractall/1}, {\tt retract\_nr/1}, {\tt abolish/1}
and {\tt dynamic/1} respectively.  If the index specification is other
than {\tt tries}, the predicate will issue a warning message and have
no effect on the database.

\section{Extended Logic Programs}  \label{library_utilities:wfsx}
\index{WFSX}
\index{negation!explicit negation}
As explained in the section {\it Using Tabling in XSB}, XSB can
compute normal logic programs according to the well-founded semantics.
In fact, XSB 
can also compute {\em Extended Logic Programs}, which contain an
operator for explicit negation (written using the symbol {\tt -} in
addition to the negation-by-failure of the well-founded semantics
(\verb|\+| or {\tt not}).  Extended logic programs can be extremely
useful when reasoning about actions, for model-based diagnosis, and
for many other uses \cite{AlPe95}.  The library, {\sf slx} provides a means
to compile programs so that they can be executed by XSB according to
the {\em well-founded semantics with explicit negation} \cite{ADP95}.
Briefly, WFSX is an extension of the well-founded semantics to include
explicit negation and which is based on the {\em coherence principle}
in which an atom is taken to be default false if it is proven to be
explicitly false, intutively:
\[
-p \Rightarrow not\ p.
\]

This section is not intended to be a primer on extended logic
programming or on WFSX semantics, but we do provide a few sample
programs to indicate the action of WFSX.  Consider the program
{\small 
\begin{verbatim}
s:- not t.

t:- r.
t.

r:- not r.
\end{verbatim}
}
If the clause {\tt -t} were not present, the atoms {\tt r, t, s} would
all be undefined in WFSX just as they would be in the well-founded
semantics.  However, when the clause {\tt t} is included, {\tt t}
becomes true in the well-founded model, while {\tt s} becomes false.
Next, consider the program
{\small 
\begin{verbatim}
s:- not t.

t:- r.
-t.

r:- not r.
\end{verbatim}
}
In this program, the explicitly false truth value for {\tt t} obtained
by the rule {\tt -t} overrides the undefined truth value for {\tt t}
obtained by the rule {\tt t:- r}.  The WFSX model for this program
will assign the truth value of {\tt t} as false, and that of {\tt s}
as true.  If the above program were contained in the file {\tt
test.P}, an XSB session ising {\tt test.P} might look like the
following:
{\small
\begin{verbatim}
              > xsb
              
              | ?- [slx].
              [slx loaded]
            
              yes
              | ?- slx_compile('test.P').
              [Compiling ./tmptest]
              [tmptest compiled, cpu time used: 0.1280 seconds]
              [tmptest loaded]
            
              | ?- s.
              
              yes
              | ?- t.

              no
              | ?- naf t.
              
              yes
              | ?- r.

              no
              | ?- naf r.
              
              no
              | ?- und r.
              
              yes
\end{verbatim}
}
In the above program, the query {\tt ?- t.} did not succeed,  because
{\tt t} is false in WFSX: accordingly the query {\tt naf t} did
succeed, because it is true that t is false via negation-as-failure,
in addition to {\tt t} being false via explicit negation.  Note that
after being processed by the SLX preprocessor, {\tt r} is undefined
but does not succeed, although {\tt und r} will succeed.

We note in passing that programs under WFSX can be paraconsistent.
For instance in the program.
{\small
\begin{verbatim}
              p:- q.

              q:- not q.
              -q.
\end{verbatim}
}
both {\tt p} and {\tt q} will be true {\em and} false in the WFSX
model.  Accordingly, under SLX preprocessing, both {\tt p} and {\tt
naf p} will succeed.

\begin{description}
\ournewitem{slx\_compile(+File)}{slx}
Preprocesses and loads the extended logic program named {\tt File}.
Default negation in {\tt File} must be represented using the operator
{\tt not} rather than using {\tt tnot} or \verb|\+|.  If {\tt L} is an
objective literal (e.g. of the form $A$ or $-A$ where $A$ is an atom),
a query {\tt ?- L} will succeed if {\tt L} is true in the WFSX model,
{\tt naf L} will succeed if {\tt L} is false in the WFSX model, and
{\tt und L} will succeed if {\tt L} is undefined in the WFSX model.
\end{description}



%%% Local Variables: 
%%% mode: latex
%%% TeX-master: "manual2"
%%% End: 
