\chapter{Introduction} \label{introduction}
%==========================================

XSB is a research-oriented Logic Programming system for Unix and
Windows-based systems.  In addition to providing all the
functionality of Prolog, XSB contains several features not usually
found in Logic Programming systems, including
\begin{itemize}
\item Evaluation according to the Well-Founded Semantics \cite{VGRS91}
  through full SLG resolution;
\item A compiled HiLog implementation;
\item A variety of indexing techniques for asserted code, along with a
  novel transformation technique called {\em unification factoring} that
  can improve program speed and indexing for compiled code;
\item A number of interfaces to other software systems, such a C, Java,
  Perl and Oracle.
\item Extensive pattern matching libraries, which are especially useful for
  Web applications.
\item Preprosessors and Interpreters so that XSB can be used to evaluate
  programs that are based on advanced formalisms, such as extended logic
  progams (according to the Well-Founded Semantics \cite{ADP94}) and
  F-Logic \cite{KLW95}.
\item Source code availability for portability and extensibility.
\end{itemize}
 
Though XSB can be used as a Prolog system\footnote{Many of the Prolog
components of XSB are based on PSB-Prolog~\cite{Xu90}, which itself is
based on version 2.0 of SB-Prolog~\cite{Debr88}.}, we avoid referring
to XSB as such, because of the availability of SLG resolution and the
handling of HiLog terms.  These facilities, while seemingly simple,
significantly extend its capabilities beyond those of a typical Prolog
system. We feel that these capabilities justify viewing XSB as a new
paradigm for Logic Programming.

To understand the implications of SLG resolution \cite{ChWa96}, recall
that Prolog is based on a depth-first search strategy (SLD) which uses
program clause resolution.  As such, Prolog is susceptible to getting
lost in an infinite branch of a search tree, where it may loop
infinitely.  SLG evaluation, available in XSB, can correctly evaluate
many such logic programs.  To take the simplest of examples, any query
to the program:
\begin{center}
\begin{minipage}{3.8in}
\begin{verbatim}
:- table ancestor/2.

ancestor(X,Y) :- parent(X,Y).
ancestor(X,Y) :- ancestor(X,Z), parent(Z,Y).
\end{verbatim}
\end{minipage}
\end{center}
will terminate in XSB, since {\tt ancestor/2} is compiled as a tabled
predicate; Prolog systems, however, would go into an infinite loop.
The user can declare that SLG resolution is to be used for a predicate
by using {\tt table} declarations, as here.  Alternately, an {\tt
auto\_table} compiler directive can be used to direct the system to
invoke a simple static analysis to decide what predicates to table
(see Section~\ref{tabling_directives}).  This power to solve recursive
queries has proven very useful in a number of areas, including
deductive databases, language processing \cite{syntactica, semantica},
program analysis \cite{DRW96, CoDS96, Boul97}, model checking
\cite{RRRSSW97} and diagnosis \cite{diagnostica}.  For efficiency, we
have implemented SLG at the abstract machine level so that tabled
predicates will be executed with the speed of compiled Prolog.  We
finally note that for definite programs SLG resolution is similar to
other tabling methods such as OLDT resolution~\cite{TaSa86} (see
Chapter
\ref{tabling_overview} for details).

The use of tabling also makes possible the evaluation of programs with
negation.  When logic programming rules have negation, paradoxes
become possible.  As an example consider one of Russel's paradoxes ---
the barber in a town shaves every person who does not shave himself ---
written as a logic program.
\begin{center}
\begin{verbatim} 
:- table shaves/2.

shaves(barber,Person):- person(Person), tnot(shaves(Person,Person)).
person(barber).
person(mayor).
\end{verbatim} 
\end{center}
Logically speaking, however it should deduce that the barber shaves
the mayor, while in a skeptical semantics it is undefined whether the
barber shaves himself or not.  The well-founded semantics, does not
treat this conclusion as either true or false, but as {\em undefined}.
Prolog, of course, would enter an infinite loop.  XSB's treatment of
negation is discussed further in Chapter \ref{tabling_overview}.

The second important extension in XSB is support of HiLog
programming \cite{ChKW93,SaWa95}.  HiLog allows a form of 
higher-order programming, in
which predicate ``symbols'' can be variable or structured.  For example,
definition and execution of {\em generic predicates} like this generic
transitive closure relation are allowed:
\begin{center}
\begin{minipage}{3.7in}
\begin{verbatim}
closure(R)(X,Y) :- R(X,Y).
closure(R)(X,Y) :- R(X,Z), closure(R)(Z,Y).
\end{verbatim}
\end{minipage}
\end{center}
where {\tt closure(R)/2} is (syntactically) a second-order predicate
which, given any relation {\tt R}, returns its transitive closure
relation {\tt closure(R)}.  Support is provided for reading and
writing HiLog terms, converting them to or from internal format as
necessary (see Section~\ref{HiLog2Prolog}).  Special meta-logical
standard predicates (see Section~\ref{MetaLogical}) are also provided
for inspection and handling of HiLog terms.  Unlike earlier versions
of XSB (prior to version 1.3.1) the current version automatically
provides {\em full compilation of HiLog predicates}.  As a result,
most uses of HiLog execute at essentially the speed of compiled
Prolog.  For more information about the compilation scheme for HiLog
employed in XSB see~\cite{SaWa95}.

HiLog can also be used with tabling, so that the program above can also be
written as:
\begin{center}
\begin{minipage}{3.7in}
\begin{verbatim}
:- table closure(_)(_,_).

closure(R)(X,Y) :- R(X,Y).
closure(R)(X,Y) :- closure(R)(X,Z), R(Z,Y).
\end{verbatim}
\end{minipage}
\end{center}

A goal of XSB is to provide in implementation engine for both logic
programming and for data-oriented applications such as in-memory
deductive database queries and data mining \cite{SaSw94}.  One
prerequisite for this functionality is the ability to load a large
amount of data very quickly.  We have taken care to code in C a
compiler for asserted clauses.  The result is that the speed of
asserting and retracting code is faster in XSB than in any
other Prolog system that we know of.  At the same time, because
asserted code is compiled into SLG-WAM code, the speed of executing
asserted code in XSB is faster than that of other Prologs as
well.  We note however, that XSB does not follow the semantics
of assert specified in \cite{LiOk87}.

Data oriented applications may also require indices other than
Prolog's first argument indexing.  XSB offers a variety of indexing
techniques for asserted code.  Clauses can be indexed on a groups of
arguments or on alternative arguments.  For instance, the executable
directive {\tt index(p/4,[3,2+1])} specifies indexes on the (outer
functor symbol of) the third argument and on a combination of (the
outer function symbol of) a combination of the second and first
arguments.  If data is expected to be structured within function
symbols and is in unit clauses, the directive {\tt index(p/4,trie)}
constructs an indexing trie of the {\tt p/4} clauses using a
left-to-right traversal through each clausennm.  Representing data in
this way allows discrimination of information nested arbitrarily deep
within clauses.  These modes of indexing can becombined: {\tt
index(p/4,[3,2+1],trie)} creates alternative trie indices beginning
with the third argument and with the second and first argument.  Using
such indexing XSB can efficiently perform intensive analyses of
in-memory knowledge bases with 1 million or so facts.  Indexing
techniques for asserted code are covered in Section
\ref{sec:assert}.

For compiled code, XSB offers {\em unification factoring}, which
extends clause indexing methods found in functional programming into
the logic programming framework.  Briefly, unification factoring can
offer not only complete indexing through non-deterministic indexing
automata, but can also $factor$ elementary unification operations.
The general technique is described in \cite{DRRSSSW95}, and the XSB
directives needed to use it are covered in Section
\ref{the_compiler}.

A number of interfaces are available to link XSB to other systems.
In UNIX systems XSB can be directly linked into C programs; in
Windows-based system XSB can be linked into C programs through a DLL
interface.  On either class of operating system, C functions can be
made callable from XSB either directly within a process, or using a
socket library.  XSB can access external data in a variety of ways:
through an Oracle interface, through an ODBC interface (documented in
a separate manual), or through a variety of mechanisms to read data
from flat files.  These interfaces are described in Volume 2 of this manual.

Another feature of XSB is its support for extensions of normal logic
programs through preprocessing libraries.  These libraries are
described in the section {\it Library Utilities} in Volume 2 of this
manual. 

Source code is provided for the whole of XSB, including the
engine, interfaces and supporting functions written in C, along with
the compiler, top-level interpreter and libraries written in Prolog.

\comment{
One reason for making everything available is to allow users to port
XSB to new platforms --- which they have done.  The system can be
compiled using most C compilers (though we recommend using the GNU C
Compiler), and currently we have tested XSB on Microsoft Windows,
Windows 95, and Windows NT.  For 32-bit UNIX architectures, XSB has
been tested on PowerMacs running MkLinux, SUN-3 workstations,
SUN-4/SPARCstations (running SunOS 4.x.x or Solaris 2.x), NeXT
machines (running Mach), 80386/80486/80586 based machines running
Linux, 386BSD, Solaris, or NeXTSTEP, Silicon Graphics machines running
IRIX, Hewlett-Packard 300/400 or 700 series machines running HP-UX,
DECstations running Ultrix, Amiga machines running AmigaDOS, IBM
RS/6000 machines running AIX, and Sony NEWS workstations running NEWS
OS 4.2.x or higher.  The ability to run on 64-bit architectures is new
to Version 1.8 and our 64-bit port has been tested on Other ports are
planned, particularly to 64 bit Silicon Graphics machines running
IRIX.  Parties interested in doing other ports should contact us at
{\tt xsb-contact@cs.sunysb.edu}.  }

\comment{
Taken out for version 1.8

The other advantage of source code availability is that it allows
users to extend the system themselves.  The user can experiment with
new builtins, implement alternate clause indexing, or even add new
emulator instructions at a fraction of the effort involved in creating
a system from scratch.  Of course, while we encourage experimentation,
users must support their own code in such cases.

Indeed, while support of SB-Prolog at SUNY Stony Brook and at University
of Arizona has ended, we intend to support XSB (to the extent we
can) in the future.  We hope that former users of SB-Prolog will switch
to XSB and notice a more robust system with faster compilation
and execution.


XSB can also be used as a Prolog system, without SLG or HiLog.  As a
Prolog system, there are several features of XSB which may be
of interest to users.
\begin{itemize}
\item Fast compilation of dynamic code.
\item Object file compilation.
\item Module system and inference of symbol usage at compilation.
\item Dynamic loading of Prolog and C code.
\end{itemize}

The module system of XSB is {\em file based} (one module per
file), {\em atom based} (where all non-predicate symbols are normally
assumed to be global to all modules), and {\em flat} (modules cannot
be nested).  Though the user is not obliged to use the module system,
the dynamic loader and the foreign language interface are both based
upon it.  Additionally, the module system premits operation of a use
inferencer which indicates at compile-time whether symbols in a module
are unused (not called by an exported predicate) or undefined.

The dynamic loader lazily loads files containing predicates when they 
are first called, which can be useful for reducing program space and
start-up time.  (Files can also be loaded explicitly by the use of 
standard predicates {\tt consult/[1,2]}).  

While these features are explained in detail below, this document does
not intend to be a tutorial for using the Prolog langage;
rather it focuses on the extensions of XSB to Prolog
systems.  For users new to Prolog, a textbook such as
\cite{Brat86,ClMe84,MaWa88,StSh86} should be used along with this manual.
}

It should be mentioned that we adopt some standard notational
conventions, such as the name/arity convention for describing
predicates and functors, {\tt +} to denote input arguments, {\tt -} to
denote output arguments, {\tt ?} for arguments that may be either
input or output and {\tt \#} for arguments that are both input and
output (can be changed by the procedure).  See Section
\ref{mode_declarations} for more details.
\index{notational conventions}  Also, the manual uses UNIX syntax for
files and directories except when it specifically addresses other
operating systems such as Windows.  

Finally, we note that XSB is under continuous development, and
this document ---intended to be the user manual--- reflects the
current status (\version) of our system.  While we have taken great
effort to create a robust and efficient system, we would like to
emphasize that XSB is also a research system and is to some
degree experimental.  When the research features of XSB ---
tabling, HiLog, and Indexing Techniques --- are discussed in this
manual, we also cite documents where they are fully explained.  All of
these documents can be found via the world-wide web or anonymous ftp
from {\tt \{www/ftp\}.cs.sunysb.edu}, the same host from which XSB can
be obtained.

While some of \version\ is subject to change in future releases, we
will try to be as upward-compatible as possible. We would also like to
hear from experienced users of our system about features they would
like us to include.  We do try to accomodate serious users of XSB
whenever we can.  Finally, we must mention that the use of
undocumented features is not supported, and at the user's own risk.


