5
\chapter{sgml and xpath: SGML/XML/HTML Parsers and XPath}

  \begin{center}
    {\Large {\bf By Rohan Shirwaikar}}
  \end{center}



\section{Introduction}
This suite of packages consists of the {\tt sgml} package, which can
parse XML, HTML, XHTML, and even SGML documents and the {\tt xpath}
package, which supports XPath queries on XML documents.  The {\tt
  sgml} package is an adaptation of a similar package in SWI Prolog
and a port of SWI's codebase with some minor changes. The {\tt xpath}
package provides an interface to the popular {\tt libxml2} library,
which supports XPath and XML parsing, and is used in Mozilla based
browsers. At present, the XML parsing capabilities of {\tt libxml2}
are not utilized explicitly in XSB, but such support might be provided
in the future. The {\tt sgml} package does not rely on {\tt
  libxml2}~\footnote{This package has not yet been tested for
  thread-safety}.

\paragraph{Installation and configuration.}
The {\tt sgml} package does not require any installation steps under
Unix-based systems or under Cygwin. Under native Windows, if you downloaded XSB
from CVS, you need to compile the package as follows:
%%
\begin{verbatim}
    cd XSB\packages\sgml\cc
    nmake /f NMakefile.mak
\end{verbatim}
%%
You need MS Visual Studio for that. If you downloaded a prebuilt version
of XSB, then the {\tt sgml}  package should have already been compiled for you
and no installation is required.

The details of the {\tt xpath} package and the corresponding configuration
instructions appear in Section~\ref{sec-xpath}.

\section{Overview of the SGML Parser}

The {\tt sgml} package accepts
input in the form of files, URLs and Prolog atoms.
To load the {\tt sgml} parser, the user should type
%%
\begin{verbatim}
 ?- [sgml].  
\end{verbatim}
%%
at the prompt.
If {\tt test.html} is a file with the following contents 
%%
\begin{verbatim}
<!DOCTYPE HTML PUBLIC "-//W3C//DTD HTML 3.2//EN">

<html>
<head>
<title>Demo</title>
</head>
<body>

<h1 align=center>This is a demo</title>

<p>Paragraphs in HTML need not be closed.

<p>This is called `omitted-tag' handling.
</body>
</html>
\end{verbatim}
%%
then the following call
%%
\begin{verbatim}
?- load_html_structure(file('test.html'), Term, Warn).
\end{verbatim}
%%
will parse the document and bind {\tt Term} to the following Prolog term: 
%%
\begin{verbatim}
[ element(html,
          [],
          [ element(head,
                    [],
                    [ element(title,
                              [],
                              [ 'Demo'
                              ])
                    ]),
            element(body,
                    [],
                    [ '\n',
                      element(h1,
                              [ align = center
                              ],
                              [ 'This is a demo'
                              ]),
                      '\n\n',
                      element(p,
                              [],
                              [ 'Paragraphs in HTML need not be closed.\n'
                              ]),
                      element(p,
                              [],
                              [ 'This is called `omitted-tag\' handling.'
                              ])
                    ])
          ])
].
\end{verbatim}
%%



\index{\texttt{CDATA}}
\noindent
The XML document is converted into a list of Prolog terms of the form {\tt
  element(\emph{Name},\emph{Attributes},\emph{Content})}.  Each term
corresponds to an XML element. \emph{Name} represents the name of the element.
\emph{Attributes} is a list of attribute-value pairs of the element.  
\emph{Content} is a list of child-elements and CDATA.
For instance, 
%%
\begin{verbatim}
    <aaa>fooo<bbb>foo1</bbb></aaa>  
\end{verbatim}
%%
will be parsed as
%%
\begin{verbatim}
    element(aaa,[],[fooo, element(bbb,[],[foo1])])  
\end{verbatim}
%%

Entities (e.g. \verb$&lt;$) are returned as part of {\tt CDATA},
unless they cannot be represented. See {\bf load\_sgml\_structure/3}
for details.

\section{Predicate Reference}
\subsection{Loading Structured Documents}

SGML, HTML, and XML documents are parsed by the predicate
{\bf load\_structure/4}, which has many options. For 
convenience, a number of commonly used shorthands are provided
to parse SGML, XML, HTML, and XHTML documents
respectively.
%%
\begin{description}
  \index{\texttt{load\_sgml\_structure/3}}
\item[load\_sgml\_structure({\it +Source, -Content, -Warn})]\mbox{}
  \index{\texttt{load\_xml\_structure/3}}
\item[load\_xml\_structure({\it +Source, -Content, -Warn})]\mbox{}
  \index{\texttt{load\_html\_structure/3}}
\item[load\_html\_structure({\it +Source, -Content, -Warn})]\mbox{}
  \index{\texttt{load\_xhtml\_structure/3}}
\item[load\_xhtml\_structure({\it +Source, -Content, -Warn})]\mbox{}
\end{description}
%%
The parameters of these predicates have the same meaning as those in {\bf
  load\_structure/4}, and are described below.

The above predicates (in fact, just {\tt load\_xml\_structure/3} and
{\tt load\_html\_structure/3}) are the most commonly used predicates of the
{\tt sgml} package. The other predicates described in this section are
needed only for advanced uses of the package. 


\begin{description}
  \index{\texttt{load\_structure/4}}
\item[{\bf load\_structure}{\bf (}{\it +Source, -Content, +Options, -Warn}{\bf )}]\mbox{}
  \\
  {\it Source} can have one of the following forms:
   {\tt url({\it {url}})}, {\tt file({\it file name})},
  {\tt string({\it 'document as a Prolog atom'})}.
  The parsed document is returned in {\it Content}.
  {\it Warn} is bound to a (possibly empty) list of warnings generated
  during the parsing process.
  {\it Options} is a list of parameters that control parsing, which are
  described later.

  The list {\it Content}  can have the following members:
%%
  \begin{description}
    \index{\texttt{CDATA}}
  \item[A Prolog atom]\mbox{}\\
    Atoms are used to represent character strings, i.e., {\tt CDATA}. 

    \index{\texttt{NAMES}}
    \index{\texttt{NUMBER}}
  \item[{\bf element}{\bf (}{\it Name, Attributes, Content}{\bf
    )}]\mbox{}\\{\it Name} is the name of the element tag. Since SGML is 
    case-insensitive, all element names are returned as lowercase atoms.
    
    {\it Attributes} is a list of pairs the form {\it Name}={\it
      Value}, where \emph{Name} is the name of an attribute and
    \emph{Value} is its value.  Values of type {\tt CDATA} are represented
    as atoms. The values of multi-valued attributes ({\tt NAMES},
    \emph{etc.}) are represented as a lists of atoms.  Handling of the
    attributes of types {\tt NUMBER} and {\tt NUMBERS} depends on the
    setting of the {\tt number(+NumberMode)} option of {\bf
      set\_sgml\_parser/2} or {\bf load\_structure/3} (see later).  By
    default the values of such attributes
    are represented as atoms, but the {\tt number(...)} option can also 
    specify that these values must be converted to
    Prolog integers.
    
    {\it Content} is a list that represents
    the content for the element.

  \item[{\bf entity}{\bf (}{\it Code}{\bf )}]\mbox{}\\
    If a character entity (e.g., \verb$&#913;$) is encountered that 
    cannot be represented in the Prolog character set, this term is 
    returned. It represents the code of the encountered character (e.g.,
    {\tt entity(913)}).

  \item[{\bf entity}{\bf (}{\it Name}{\bf )}]\mbox{}\\
    This is a special case of {\tt entity(Code)}, intended to handle 
    special symbols by their name rather than character code.
    If an entity refers to a character entity holding a single character, 
    but this character cannot be represented in the Prolog character set, 
    this term is returned. For example, if the contents of an element is
    \verb$&Alpha; &lt; &Beta;$ then it will be represented as follows:
    %%
\begin{verbatim}
    [ entity('Alpha'), ' < ', entity('Beta') ]
\end{verbatim}
    %%
    Note that entity names are case sensitive in both SGML and XML.

    \index{\texttt{SDATA}}
  \item[{\bf sdata}{\bf (}{\it Text}{\bf )}]\mbox{}\\
    If an entity with declared content-type {\tt SDATA} is encountered, this 
    term is used. The data of the entity instantiates {\it Text}.

    \index{\texttt{NDATA}}
  \item[{\bf ndata}{\bf (}{\it Text}{\bf )}]\mbox{}\\
    If an entity with declared content-type {\tt NDATA} is encountered, this 
    term is used. The data instantiates {\it Text}.
    
  \item[{\bf pi}{\bf (}{\it Text}{\bf )}]\mbox{}\\
    If a processing instruction is encountered (\verb$<?...?>$), {\it Text} holds the text of the processing instruction. Please note that the
    \verb$<?xml ...?>$ instruction is ignored and is not treated as a
    processing instruction.
  \end{description}

  The {\it Options} parameter is a list that controls parsing. Members of
  that list can be of the following form:
%%
  \begin{description}

  \item[{\bf dtd}{\bf (}{\it ?DTD}{\bf )}]\mbox{}\\
    Reference to a DTD object. If specified, the \verb$<!DOCTYPE ...>$
    declaration supplied with the document
    is ignored and the document is parsed and validated against 
    the provided DTD. If the DTD argument is a variable, then
    a the variable \emph{DTD} gets bound to
    the DTD object created out of the DTD supplied with the document.

    \index{\texttt{xml}}
    \index{\texttt{sgml}}
    \index{\texttt{xmlns}}
  \item[{\bf dialect}{\bf (}{\it +Dialect}{\bf )}]\mbox{}\\
    Specify the parsing dialect. The supported dialects are
    {\tt sgml} (default),
    {\tt xml} 
    and {\tt xmlns}.
    
  \item[{\bf space}{\bf (}{\it +SpaceMode}{\bf )}]\mbox{}\\
    Sets the space handling mode for the initial environment. This mode is
    inherited by the other environments, which can override the inherited
    value using the XML reserved attribute {\bf xml:space}. See
    Section~\ref{sec:space} for details.

    \index{\texttt{token}}
    \index{\texttt{integer}}
  \item[{\bf number}{\bf (}{\it +NumberMode}{\bf )}]\mbox{}\\
    Determines how attributes of type {\tt NUMBER} and {\tt NUMBERS} are
    handled.  If {\tt token} is specified (the default) they are passed as
    an atom.  If {\tt integer} is specified the parser attempts to convert
    the value to an integer.  If conversion is successful, the attribute is
    represented as a Prolog integer.  Otherwise the value is represented as
    an atom.  Note that SGML defines a numeric attribute to be a sequence
    of digits.  The {\tt -} (minus) sign is not allowed and {\tt 1} is
    different from {\tt 01}.  For this reason the default is to handle
    numeric attributes as tokens.  If conversion to integer is enabled,
    negative values are silently accepted and the minus sign is ignored.


    \index{\texttt{false}}
  \item[{\bf defaults}{\bf (}{\it +Bool}{\bf )}]\mbox{}\\
    Determines how default and fixed attributes from the DTD are used.  By
    default, defaults are included in the output if they do not appear in
    the source.  If {\tt false}, only the attributes occurring in the source
    are emitted.
    
    
  \item[{\bf file}{\bf (}{\it +Name}{\bf )}]\mbox{}\\
    Sets the name of the input file for error reporting.
    This is useful if the input is a stream that is not coming from 
    a file. In this case, errors and warnings will not have the file name
    in them, and this option allows one to force inclusion of a file name
    in such messages.

  \item[{\bf line}{\bf (}{\it +Line}{\bf )}]\mbox{}\\
    Sets the starting line-number for reporting errors. For instance, if
    {\tt line(10)} is specified and an error is found at line X then the
    error message will say that the error occurred at line X+10.
    This option is used when the input stream does not start with the first
    line of a file.
    
  \item[{\bf max\_errors}{\bf (}{\it +Max}{\bf )}]\mbox{}\\
Sets the maximum number of errors.  The default is 50. If this number
    is reached, the following exception is raised:
%%
    \begin{quote}
      {\tt error(limit\_exceeded(max\_errors, Max), \_)} 
    \end{quote}

  \end{description}

\end{description}

\subsection{Handling of  White Spaces}\label{sec:space}

Four modes for handling white-spaces are provided. The initial mode can be
switched using the {\tt space(SpaceMode)} option to
{\bf load\_structure/3} or {\bf set\_sgml\_parser/2}. In XML
mode, the mode is further controlled by the {\bf xml:space} attribute,
which may be specified both in the DTD and in the document. The defined
modes are:


\begin{description}
\item[{\bf space}{\bf (sgml)}]\mbox{}\\
  Newlines at the start and end of an element are removed.
    This is the default mode for the SGML dialect. 

\item[{\bf space}{\bf (preserve)}]\mbox{}\\
  White space is passed literally to the application. This mode leaves all
  white space handling to the application. This is the default mode for
  the XML dialect. 

  \index{\texttt{sgml} space mode}
\item[{\bf space}{\bf (default)}]\mbox{}\\
  In addition to {\tt sgml} space-mode, all consecutive whitespace is 
  reduced to a single space-character.

  \index{\texttt{default} space mode}
  \index{\texttt{preserve} space mode}
\item[{\bf space}{\bf (remove)}]\mbox{}\\
  In addition to {\tt default}, all leading and trailing white-space is 
  removed from {\tt CDATA} objects. If, as a result, the {\tt CDATA} 
  becomes empty, nothing is passed to the application. This mode is 
  especially handy for processing data-oriented documents, such as RDF. 
  It is not suitable for normal text documents. Consider the HTML 
  fragment below. When processed in this mode, the spaces surrounding the 
  three elements in the example below are lost.  This mode is not part of
  any standard: XML 1.0 allows only {\tt default} and {\tt preserve}.
%%
\begin{verbatim}
    Consider adjacent <b>bold</b> <ul>and</ul> <it>italic</it> words.
\end{verbatim}
  %%
  The parsed term will be
  {\tt ['Consider adjacent',element(b,[],[bold]),element(ul,[], [and]),element(it,[],[italics]),words]}.


\end{description}

\subsection{XML documents}\label{sec:xml}

\index{\texttt{sgml}}
\index{\texttt{xml}}
The parser can operate in two modes:
the {\tt sgml} mode and
the {\tt xml} mode, as 
defined by the {\tt dialect(Dialect)} option. HTML is a special case of
the SGML mode with a particular DTD. Regardless of this 
option, if the first line of the document reads as below, the parser is 
switched automatically to the XML mode.


\begin{verbatim}
	<?xml ... ?>
\end{verbatim}


Switching to XML mode implies:


\begin{itemize}
\item \emph{XML empty elements}\mbox{}\\
  The construct \verb$<element attribute ... attribute/>$ is recognized as 
  an empty element. 
  
\item \emph{Predefined entities}\mbox{}\\
  The following entities are predefined: {\tt \&lt;} (\verb$<$), {\tt \&gt;}
  (\verb$>$), {\tt \&amp;} (\verb$&$), {\tt \&apos;} (\verb$'$) 
  and {\tt \&quot;} (\verb$"$). 
  
\item \emph{Case sensitivity}\mbox{}\\
  In XML mode, names of tags and attributes
  are case-sensitive, except for the DTD 
  reserved names (i.e. {\tt ELEMENT}, \emph{etc.}). 
  
\item \emph{Character classes}\mbox{}\\
  In XML mode, underscore (\verb$_$) and colon (\verb$:$) are
  allowed in names. 

  \index{\texttt{preserve} space mode}
  \index{\texttt{remove} space mode}
\item \emph{White-space handling}\mbox{}\\
  White space mode is set to {\tt preserve}. In addition,
  the XML reserved attribute
  {\bf xml:space} is honored; it may appear both in the document and the
  DTD. The {\tt remove} extension (see {\tt space(remove)} earlier)  is allowed
  as a value of the {\bf xml:space} attribute. For
  example, the DTD statement below ensures that the {\bf pre} element
  preserves space, regardless of the default processing mode.



\begin{verbatim}
		<!ATTLIST pre xml:space nmtoken #fixed preserve>
\end{verbatim}


\end{itemize}

\subsubsection{XML Namespaces}\label{sec:xmlns}

\index{\texttt{xmlns} dialect}
Using the dialect {\tt xmlns}, the parser will recognize XML 
namespace prefixes. In this case, the names of elements are returned as a term 
of the format

\begin{quote}
  {\it URL}{\tt :}{\it LocalName} 
\end{quote}

If an identifier has no namespace prefix and there is no default namespace, it 
is returned as a simple atom. If an identifier has a namespace prefix but this 
prefix is undeclared, the namespace prefix rather than the related URL 
is returned.

Attributes declaring namespaces ({\tt xmlns:{\it ns}={\it url}})
are represented in the translation as regular attributes.


\subsection{DTD-Handling}

The DTD ({\bf D}ocument {\bf T}ype {\bf D}efinition) are internally
represented as objects
that can be created, freed, defined, and inspected. 
Like the parser itself, it is filled by opening it as a Prolog output
stream and sending data to it. This section summarizes the predicates 
for handling the DTD.


\begin{description}
  \index{\texttt{new\_dtd/2}}
\item[{\bf new\_dtd}{\bf (}{\it +DocType, -DTD, -Warn}{\bf )}]\mbox{}\\Creates an empty DTD for the named {\it DocType}. The returned 
  DTD-reference is an opaque term that can be used in the other predicates
  of this package. {\it Warn} is the list of warnings generated.

  \index{\texttt{free\_dtd/1}}
\item[{\bf free\_dtd}{\bf (}{\it +DTD, -Warn}{\bf )}]\mbox{}\\
  Deallocate all resources associated to the DTD. Further use of {\it DTD}
  is invalid. {\it Warn} is the list of warnings generated.

  \index{\texttt{open\_dtd/3}}
\item[{\bf open\_dtd}{\bf (}{\it +DTD, +Options, -Warn}{\bf )}]\mbox{}\\
   This opens and loads a DTD from a specified location (given in the
   \emph{Options} parameter (see next).
   {\it DTD} represents the created DTD object after the source is loaded.
   {\it Options} is a list options. Currently the only option supported is
   {\it source(location)}, where {\it location} can be of one of these
   forms:
   %%
   \begin{quote}
     {\tt url({\it {url}})}\\
     {\tt file({\it fileName})}\\
     {\tt string({\it 'document as a Prolog atom'})}.
   \end{quote}
   %%
 

  \index{\texttt{dtd/3}}
  \index{\texttt{doctype}}
\item[{\bf dtd}{\bf (}{\it +DocType, -DTD, -Warn}{\bf )}]\mbox{}\\
  Certain DTDs are part of the system and have known
  doctypes. Currently, 'HTML' and 'XHTML' are the only recognized built-in
  doctypes.
  Such a DTD can be used for parsing simply by
  specifying the doctype.  Thus, the {\tt dtd/3} predicate takes the
  doctype name, finds the DTD associated with the given doctype, and
  creates a dtd object for it.  {\it Warn} is the list of warnings
  generated.

  \index{\texttt{dtd/3}}
\item[{\bf dtd}{\bf (}{\it +DocType, -DTD, +DtdFile -Warn}{\bf )}]\mbox{}\\
  
  The predicate parses the DTD present at the location \emph{DtdFile} and
  creates the corresponding DTD object.  {\it DtdFile} can have one of the
  following forms: {\tt url({\it url})}, {\tt file({\it fileName})}, {\tt
    string({\it 'document as a Prolog atom'})}.

\end{description}


\subsection{Low-level Parsing Primitives}

The following primitives are used only for more complex types of parsing,
which might not be covered by the {\tt load\_structure/4} predicate. 

\begin{description}
  \index{\texttt{new\_sgml\_parser/2}}
\item[{\bf new\_sgml\_parser}{\bf (}{\it -Parser, +Options, -Warn}{\bf
    )}]\mbox{}\\Creates a new parser. {\it Warn} is the list of warnings
  generated. A parser can be used one or multiple times for parsing
  documents or parts thereof. It may be bound to a DTD or the DTD may be
  left implicit. In this case the DTD is created from the document prologue
  or (if it is not in the prologue) parsing is performed without a DTD.
  The \emph{Options} list can contain the following parameters: 

  \begin{description}
  \item[{\bf dtd}{\bf (}{\it ?DTD}{\bf )}]\mbox{}\\
    If \emph{DTD} is bound to a DTD object, this DTD is used for parsing
    the document and the document's prologue is ignored. If \emph{DTD} is a
    variable, the variable gets bound to a created DTD. This DTD may
    be created from the document prologue or build implicitly from the
    document's content.
  \end{description}



  \index{\texttt{free\_sgml\_parser/1}}
\item[{\bf free\_sgml\_parser}{\bf (}{\it +Parser, -Warn}{\bf
    )}]\mbox{}\\Destroy all resources related to the parser. This does not
  destroy the DTD if the parser was created using the {\tt dtd(\emph{DTD})}
  option. {\it Warn} is the list of warnings generated during parsing (can
    be empty).

  \index{\texttt{set\_sgml\_parser/2}}
\item[{\bf set\_sgml\_parser}{\bf (}{\it +Parser, +Option, -Warn}{\bf
    )}]\mbox{}\\Sets attributes to the parser. {\it Warn} is the list of
  warnings generated. \emph{Options} is a list that can contain the
    following members:

  \begin{description}
  \item[{\bf file}{\bf (}{\it File}{\bf )}]\mbox{}\\
    Sets the file for reporting errors and warnings. Sets the linenumber to 1. 
  \item[{\bf line}{\bf (}{\it Line}{\bf )}]\mbox{}\\
    Sets the starting line for error reporting. Useful if the stream is not
    at the start of the (file) object for generating proper line-numbers.
    This option has the same meaning as in the {\tt load\_structure/4}
    predicate. 

  \item[{\bf charpos}{\bf (}{\it Offset}{\bf )}]\mbox{}\\
    Sets the starting character location.  See also the {\tt file(File)}
    option. Used when the stream does not start from the beginning of a
    document. 

  \item[{\bf dialect}{\bf (}{\it Dialect}{\bf )}]\mbox{}\\
    Set the markup dialect. Known dialects: 


    \begin{description}
      \index{\texttt{sgml}}
    \item[{\tt sgml}]\mbox{}\\
      The default dialect. This implies markup is case-insensitive and
      standard SGML abbreviation is allowed (abbreviated attributes and
      omitted tags). 
      \index{\texttt{xml}}
    \item[{\tt xml}]\mbox{}\\
      This dialect is selected automatically if the processing instruction
      \verb$<?xml ...>$ is encountered.  \index{\texttt{xmlns}}
    \item[{\tt xmlns}]\mbox{}\\
      Process file as XML file with namespace support. 
    \end{description}

  \item[{\bf qualify\_attributes}{\bf (}{\it Boolean}{\bf )}]\mbox{}\\
    Specifies how
    to handle unqualified attributes (i.e., without an explicit namespace)
    in XML namespace ({\tt xmlns}) dialect.  By default, such attributes
    are not qualified with namespace prefixes.
    If {\tt true}, such attributes are qualified
    with the namespace of the element they appear in.

  \item[{\bf space}{\bf (}{\it SpaceMode}{\bf )}]\mbox{}\\
    Define the initial handling of white-space in {\tt PCDATA}.  This attribute is
    described in Section~\ref{sec:space}.
  \item[{\bf number}{\bf (}{\it NumberMode}{\bf )}]\mbox{}\\
    If {\tt token} is specified (the default), attributes of type number are represented as a Prolog atom.
    If {\tt integer} is specified, such attributes are translated into Prolog integers.  If
    the conversion fails (e.g., due to an overflow) a warning is issued and the
    value is represented as an atom.
  \item[{\bf doctype}{\bf (}{\it Element}{\bf )}]\mbox{}\\
    Defines the top-level element of the document. If a \verb$<!DOCTYPE ...>$
    declaration has been parsed, this declaration is used. If there is no
    {\tt DOCTYPE} declaration then the 
    parser can be instructed to use the element given in
    {\tt doctype(\_)} as the top level element. This feature is
    useful when parsing part of a document (see the {\tt parse} option to
    {\bf sgml\_parse/3}).
  \end{description}
  
  \index{\texttt{sgml\_parse/3}}
\item[{\bf sgml\_parse}{\bf (}{\it +Parser, +Options, -Warn}{\bf
    )}]\mbox{}\\Parse an XML file.  The parser can operate in two input and
  two output modes. Output is a structured term as described with {\bf
    load\_structure/4}.
  
  {\it Warn} is the list of warnings generated. A full description of
  \emph{Options} is given below.

  \begin{description}
  \item[{\bf document}{\bf (}{\it +Term}{\bf )}]\mbox{}\\
    A variable that will be unified with a list describing the content of 
    the document (see {\bf load\_structure/4}). 
  \item[{\bf source}{\bf (}{\it +Source}{\bf )}]\mbox{}\\
    {\it Source} can have one of the following forms:
   {\tt url({\it url})}, {\tt file({\it fileName})},
  {\tt string({\it 'document as a Prolog atom'})}.
  This option \emph{must} be given.
  \item[{\bf content\_length}{\bf (}{\it +Characters}{\bf )}]\mbox{}\\
    Stop parsing after the given number of
    {\it Characters}.  This option is useful for parsing
    input embedded in \emph{envelopes}, such as HTTP envelopes.
  \item[{\bf parse}{\bf (}{\it Unit}{\bf )}]\mbox{}\\
    Defines how much of the input is parsed.  This option is used to parse
    only parts of a file.
    

    \begin{description}
      \index{\texttt{file}}
    \item[{\tt file}]\mbox{}\\
      Default.  Parse everything upto the end of the input.
      \index{\texttt{element}}
    \item[{\tt element}]\mbox{}\\
      The parser stops after reading the first element. Using {\tt source(\emph{Stream})}, this implies reading is stopped as soon as the
      element is complete, and another call may be issued on the same stream
      to read the next element. 

      \index{\texttt{declaration}}
    \item[{\tt declaration}]\mbox{}\\
      This may be used to stop the parser after reading the first
      declaration.  This is useful if we want to parse only the {\tt doctype}
      declaration.
      
    \end{description}

  \item[{\bf max\_errors}{\bf (}{\it +MaxErrors}{\bf )}]\mbox{}\\
    Sets the maximum number of errors. If this number is exceeded, further 
    writes to the stream will yield an I/O error exception. Printing of 
    errors is suppressed after reaching this value. The default is 100. 
  \item[{\bf syntax\_errors}{\bf (}{\it +ErrorMode}{\bf )}]\mbox{}\\
    Defines how syntax errors are handled.

    \begin{description}
    \item[quiet]\mbox{}\\
      Suppress all messages.
    \item[print]\mbox{}\\
      Default.  Print messages.
      \index{\texttt{informational}}
      
    \end{description}
    

  \end{description}

\end{description}

\subsection{External Entities}

\index{\texttt{DOCTYPE} declaration}
While processing an SGML document the document may refer to external 
data. This occurs in three places: external parameter entities, normal 
external entities and the {\tt DOCTYPE} declaration. The current version 
of this tool deals rather primitively with external data. External 
entities can only be loaded from a file.

Two types of lines are recognized by this package:

\begin{quote}
  \obeylines \obeyspaces \tt {\tt DOCTYPE} {\it doctype} {\it file}\\
  {\tt PUBLIC} {\tt "Id "} {\it file} 
\end{quote}

The parser loads the entity from the file specified as {\it file}.
The file can be local or a URL.

\subsection{Exceptions}

Exceptions are generated by the parser in two cases. The first case is when
the user specifies wrong input.  For example when specifying
\begin{verbatim}
load_structure( string('<m></m>'), Document, [line(xyz)], Warn)
\end{verbatim}
The string {\tt xyz} is not in the domain of {\tt line}. Hence in this
case a domain error exception will be thrown.

Exceptions are generated when XML being parsed is not well formed. For
example if the input XML contains
\begin{verbatim}
'<m></m1>'
\end{verbatim}
exceptions will be thrown.

In both cases the format of the exception is
\begin{alltt}
  error( sgml({\it error term}), {\it error message})
  warning( sgml({\it warning term}), {\it warning message})
\end{alltt}
%%
where {\it error term} or {\it warning term} can be of the form

%%
\begin{itemize}
\item {\it pointer to the parser instance}, 
\item {\it line at which error occurred},
\item {\it error code}. 
\item	{\it functor(argument)}, where \emph{functor} and \emph{argument} 
  depend on the type of exception raised. For example, 
  %%
  \begin{itemize}
  \item[ ] {\tt resource-error(no-memory)}  --- if memory is unavailable
  \item[ ] {\tt permission-error(file-name)}  --- no permission to read a file
  \item[ ] A {\tt system-error(description) --- internal system error} 

  \item[ ] {\tt type-error(expected,actual)} --- data type error
  \item[ ] {\tt domain-error(functor,offending-value)}  --- the offending
    value is not in the domain of the functor. For instance, 
    in {\tt load\_structure( string('<m></m>'), Document, [line(xyz)], Warn)},
    {\tt xyz} is not in the domain of {\tt line}.  

  \item[ ] {\tt existence-error(resource)} --- resource does not exist
  \item[ ] {\tt limit-exceeded(limit,maxval)} --- value exceeds the limit.
  \end{itemize}
  %%
\end{itemize}
%%

\subsection{Unsupported features}

The current parser is rather limited. While it is able to deal with many
serious documents, it omits several less-used features of SGML and XML.
Known missing SGML features include


\begin{itemize}
\item \emph{NOTATION on entities}\mbox{}\\
  Though notation is parsed, notation attributes on external entity 
  declarations are not represented in the output.
\item \emph{NOTATION attributes}\mbox{}\\
  SGML notations may have
  attributes, declared using \verb$<!ATTLIST #NOT name attrib>$. Those data
  attributes are provided when you declare an external CDATA, NDATA, or
  SDATA entity. XML does not support external CDATA, NDATA, or SDATA
  entities, nor any of the other uses to which data attributes are put in
  SGML. 
  

\item \emph{SGML declaration}\mbox{}\\
  The `SGML declaration' is fixed, though most of the parameters are 
  handled through indirections in the implementation. 

\item \emph{The RANK feature}\mbox{}\\
  It is regarded as obsolete.
\item \emph{The LINK feature}\mbox{}\\
  It is regarded as too complicated.
\item \emph{The CONCUR feature}\mbox{}\\
  Concurrent markup allows a document to be tagged according to more than
  one DTD at the same time.  It is not supported.
\item \emph{The Catalog files}\mbox{}\\  Catalog files are not supported.
\end{itemize}



In the XML mode, the parser recognizes SGML constructs that are not allowed 
in XML. Also various extensions of XML over SGML are not yet realized.
In particular, XInclude is not implemented.

\subsection{Summary of Predicates}

\begin{longtable}[l]{ll}
  {\bf dtd/2}&Find or build a DTD for a document type\\
  {\bf free\_dtd/1}&Free a DTD object\\
  {\bf free\_sgml\_parser/1}&Destroy a parser\\
  {\bf load\_dtd/2}&Read DTD information from a file\\
  {\bf load\_structure/4}&Parse XML/SGML/HTML data into Prolog term\\
  {\bf load\_sgml\_structure/3}&Parse SGML file into Prolog term \\
  {\bf load\_html\_structure/3}&Parse HTML file into Prolog term\\
  {\bf load\_xml\_structure/3}&Parse XML file into Prolog term\\
  {\bf load\_xhtml\_structure/3}&Parse XHTML file into Prolog term\\
  {\bf new\_dtd/2}&Create a DTD object\\
  {\bf new\_sgml\_parser/2}&Create a new parser\\
  {\bf open\_dtd/3}&Open a DTD object as an output stream\\
  {\bf set\_sgml\_parser/2}&Set parser options (dialect, source, \emph{etc.})\\
  {\bf sgml\_parse/2}&Parse the input\\
  {\bf xml\_name/1}&Test atom for valid XML name\\
  {\bf xml\_quote\_attribute/2}&Quote text for use as an attribute\\
  {\bf xml\_quote\_cdata/2}&Quote text for use as PCDATA\\

\end{longtable}


\section{XPath support}\label{sec-xpath}

XPath is a query language for addressing parts of an XML document.
In XSB, this support is provided by the {\tt xpath} package.
To use this package the {\tt libxml2} XML parsing library must be installed
on the machine. It comes with most Linux distributions, since it is part of
the Gnome desktop, or one can download
it from \emph{http://xmlsoft.org/}. It is available for Linux, Solaris,
Windows, and MacOS. Note that both the library itself and the {\tt
  .h} files of that library must be installed. In some Linux distributions,
the {\tt .h} files might reside in a separate package
from the package that contains the actual library. For instance, the
library ({\tt libxml2.so})  might be in the package called {\tt libxml2}
(which is usually installed by default),
while the {\tt .h} files hight be in the package {\tt libxml2-devel}
(which is usually \emph{not} in a default installation).

On Unix-based systems (and MacOS), the package needs to be configured using the
{\tt configure} script found in the XSB's {\tt packages/xpath} directory.
The {\tt configure} script has several options such as {\tt --exec-prefix},
{\tt --prefix}, etc, which may need to be specified if the package {\tt
  libxml2} is installed in a non-standard place. In this case, {\tt
  --exec-prefix=}{\em directory-of-libxml2} must specify the directory where the
{\tt lib/libxml2.so} file is located and {\tt --prefix=}\emph{include-dir} 
where the {\tt include/libxml2} directory is 
found. The default {\tt exec-prefix} is {\tt /usr/lib} and {\tt prefix} is
{\tt /usr}. If the defaults are fine (they would be, if the Gnome desktop
is installed on your machine), then one can configure the package using
%%
\begin{verbatim}
  ./configure 
\end{verbatim}
%%
Otherwise, use
%%
\begin{verbatim}
  ./configure --prefix=/usr/share --exec-prefix=/usr/local
\end{verbatim}
%%
if, for example, {\tt libxml2.so} is in {\tt /usr/local/lib/libxml2.so} and
the included {\tt .h}  files are in {\tt /usr/share/include/libxml2}. 

On Windows and under Cygwin, the {\tt libxml2} library is already included
in the XSB distribution and does not need to be downloaded. If you are
using a prebuilt XSB distribution for Windows, then you do not need to do
anything---the package has already been built for you.

For Cygwin, you only need to run the {\tt ./configure} script without any
options. This needs to be done regardless of whether you downloaded XSB
from CVS or a released prebuilt version. 

If you downloaded XSB from CVS and want to use it under native Windows (not
Cygwin), then you would need to compile the XPath package, and you
need Microsoft's Visual Studio.  To compile the package one should do the
following:
%%
\begin{verbatim}
      cd packages\xpath\cc
      nmake /f NMakefile.mak  
\end{verbatim}
%%


The following section assumes that the reader is familiar with the syntax
of XPath and its capabilities.  To load the {\tt xpath} package, type
%%
\begin{verbatim}
   :-[xpath]. 
\end{verbatim}
%%


The program needs to include the following directive:


%%
\begin{verbatim}
   :- import parse_xpath/4 from xpath. 
\end{verbatim}
%%



XPath query evaluation is done by using the {\tt parse\_xpath}  predicate.

\begin{description}
  \index{\texttt{parse\_xpath/4}}
\item[{\bf parse\_xpath}{\bf (}{\it +Source, +XPathQuery, -Output, +NamespacePrefixList}{\bf )}]\mbox{}\\
  {\it Source} is a term of the format {\tt url({\it {url}})}, {\tt
    file({\it {filename}})} or {\tt string({\it
      'XML-document-as-a-string'})}. It specifies that the input XML
  document is contained in a file, can be fetched from a URL, or is given
  directly as a Prolog atom.

  {\it XPathQuery} is a standard XPath query which is to be evaluated on
  the XML document in {\em Source}.
  
  {\it Output} gets bound to the output term. It represents the XML element
  returned after the XPath query is evaluated on the XML document in
  \emph{Source}. The output term is of the form {\tt string({\it
  'XML-document'})}. It can then be parsed using the {\tt sgml} package
  described earlier. 
  
  {\it NamespacePrefixList} is a space separated list of pairs of the form
  \emph{prefix} = \emph{namespace}. This specifies the namespace
  prefixes that are used in the XPath query.


For example if the xpath expression is {\tt '/x:html/x:head/x:meta'}  where
{\tt x} 
is a prefix that stands for
{\tt 'http://www.w3.org/1999/xhtml'}, then {\tt x}  would have to be
defined as follows:

\begin{alltt}	
  ?- parse_xpath(url('http://w3.org'), '/x:html/x:head/x:meta', O4, 
                     'x=http://www.w3.org/1999/xhtml').
\end{alltt}
%%
In the above, the xpath query is {\tt '/x:html/x:head/x:meta'}  and the
prefix has been defined as {\tt 'x=http://www.w3.org/1999/xhtml'}. 
\end{description}



%%% Local Variables: 
%%% mode: latex
%%% TeX-master: "manual2"
%%% End: 
