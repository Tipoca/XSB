\begin{center}
{\bf {\Large 
		Credits
            %=============%
}}
\end{center}

% Apologies to anyone I left out...  It wasn't intentional! TLS.

\begin{quote}
Day-to-day care and feeding of XSB including bug fixes, ports, and
configuration management has been done by Kostis Sagonas, David
Warren, Terrance Swift, Prasad Rao, Steve Dawson, Juliana Freire,
Ernie Johnson, Baoqiu Cui, Michael Kifer, Bart Demoen and Luis F.
Castro.

In \version, the core engine development of the SLG-WAM has been
mainly implemented by Terrance Swift, Kostis Sagonas, Prasad Rao,
Juliana Freire, Ernie Johnson, Luis Castro and Rui Marques.  The
breakdown, very roughly, was that Terrance Swift wrote the initial
tabling engine, the SLG-WAM, and builtins.  Prasad Rao reimplemented
the engine's tabling subsystem to use tries for variant-based table
access and Ernie Johnson extended and refactored these routines in a
number of ways, including adding call subsumption.  Kostis Sagonas
implemented most of tabled negation.  Juliana Freire revised the table
scheduling mechanism starting from Version~1.5.0 to create the batched
and local scheduling that is currently used.  Baoqiu Cui revised the
data structures used to maintain delay lists, and added attributed
variables to the engine.  Finally, Luis Castro rewrote the emulator to
use jump tables and wrote a heap-garbage collector for the SLG-WAM.
Rui Marques was mainly responsible for making XSB multi-threaded, as
well as for the concurrency control algorithms used for shared tables.

Other engine work includes the following.  Memory expansion code for
WAM stacks was written by Ernie Johnson and Bart Demoen.  Heap garbage
collection was written by Luis de Castro, Kostis Sagonis and Bart
Demoen.  Rui Marques improved the trailing of the SLG-WAM and rewrote
much of the engine to make it compliant with 64-bit architectures.
Assert and retract code was based on code written by Jiyang Xu and
significantly revised by David S. Warren and Rui Marques.  Trie
assert/retract code, and trie interning code was written by Prasad
Rao, as was most code for reclaiming table space. The current version
of {\tt findall/3} was re-written from scratch by Bart Demoen, as was
XSB's throw and catch mechanism.  64-bit floats were added by Charles
Rojo.

In terms of Prolog code, Kostis Sagonas was responsible for HiLog
compilation and associated builtins.  Steve Dawson implemented
Unification Factoring.  The revision of XSB's I/O into ISO-compatable
streams was done by Michael Kifer and Terrance Swift.  The {\tt
  auto\_table} and {\tt suppl\_table} directives were written by
Kostis Sagonas.  The DCG expansion module was written by Kostis
Sagonas for non-tabled code and by Baoqiu Cui, Terrance Swift and
David Warren for tabled code.  The handling of the {\tt multifile}
directive was written by Baoqiu Cui. C.R. Ramakrishnan wrote the mode
analyzer for XSB.  Michael Kifer implemented the {\tt storage} module.

For configuration management, Michael Kifer rewrote parts of the XSB
code to make XSB configurable with GNU's Autoconf, implemented XSB's
package system, and integrated GPP with XSB's compiler.  GPP, the
source code preprocessor used by XSB, was written by Denis Auroux, who
also wrote the GPP manual reproduced in Appendix A.

The starting point of XSB (in 1990) was PSB-Prolog 2.0 by Jiyang Xu.
PSB-Prolog in its turn was based on SB-Prolog, primarily designed and
written by Saumya Debray, David S. Warren, and Jiyang Xu.  Thanks are
also due to Weidong Chen for his work on Prolog clause indexing for
SB-Prolog, to Richard O'Keefe, who contributed the Prolog code for the
Prolog reader and the C code for the tokenizer, and to Ciao Prolog
whose {\tt write\_term/[2,3]} we use.

... Now what did I forget this time ?

\end{quote}

%%% Local Variables: 
%%% mode: latex
%%% TeX-master: "manual1"
%%% End: 
