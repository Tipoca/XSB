\chapter{Foreign Language Interface}
%===================================
\label{foreign}

When XSB is used to build real-world systems, a foreign-language
interface may be necessary to:
\begin{itemize}
\item combine XSB with existing programs and libraries, thereby
      forming composite systems;
\item interface XSB with the operating system, graphical user 
      interfaces or other system level programs;
\item speed up certain critical operations.
\end{itemize}

XSB has both a high-level and the low-level interface to C.  The
low-level interface is much more flexible, but it requires greater
attention to details of how the data is passed between XSB and C.  To
connect XSB to a C program using the high-level interface requires
very little work, but the program must be used ``as is'' and it must
take the input and produce the output supported by this high-level
interface.  Before describing the interfaces themselves, we first
describe aspects common to both the lower- and higher- level foreign
lanauge interfaces.

The foreign language interface can also support {\tt C++} programs.
Since XSB is written in {\tt C}, the interface functions in the
foreign {\tt C++} module must have the declaration {\tt extern ``C''},
and a separate compiler option (e.g. specifying {\tt g++} rather than
{\tt gcc}) may need to be given to ensure proper linkage, inclusion of
{\tt C++} libraries, etc.  In addition, on certain platforms
compilation may need to be done externally to XSB -- see the {\tt
  xasp} package for a example of using the foreign language interface
with {\tt C++} files.  For the rest of this chapter, we restrict our
attention to foreign predicates written in {\tt C}.

\section{Foreign Language Modules}

Foreign predicates must always appear in modules, and these modules
can contain only foreign predicates.  A foreign module differs from a
Prolog module in that the foreign module's source file must appear in
a {\tt *.c} file rather than a {\tt *.P} file (or {\tt .pl} file).
This {\tt *.c} file cannot contain a {\tt main()} function.
Furthermore, a {\tt *.P} file with the same name {\em must not} be
present or else the {\tt *.c} file is ignored and the module is
compiled as a regular Prolog module.  The interface part of a foreign
module, which has the same syntax as that of a normal module, is
written in Prolog and must appear in a {\tt *.H} file.  If the
lower-level interface is used, this {\tt *.H} file contains explicit
{\tt export/1} declarations for the the foreign predicates that are to
be used by other modules; if the higher-level interface is used, the
declarations have the form {\tt foreign\_pred/1}.

The Prolog predicates attached to foreign functions are deterministic,
in the sense that they succeed at most once for a given call and are
not re-entered on backtracking.  Note that this requirement imposes no
serious limitation, since it is always possible to divide a foreign
predicate into the part to be done on the first call and the part to
be redone on backtracking.  Backtracking can then take place at the
Prolog level where it is more naturally expressed.

A foreign module can be compiled or consulted just like a normal
Prolog module.  Currently, predicates {\tt consult/[1,2]} recompile
both the {\tt *.c} and the {\tt *.H} files of a foreign module when at
least one of them has been changed from the time the corresponding
object files have been created (see the section {\it Compiling and
  Consulting} in Volume 1)~\footnote{In addition, if a C module
  compiled by the single-threaded XSB engine is loaded by the
  multi-threaded engine, it will be recompiled, and vice-versa.}.  The
C compiler used to compile the {\tt *.c} files can be set as a
compilation option or defaults to that used for the configuration of
XSB (refer to the section {\it Getting Started with XSB} in Volume 1).
Alternately, the user can control the compiler options that can be
passed to the C compiler.  To give an example, the following command
will compile file {\tt file.c} using the Gnu C Compiler with
optimization and by including {\tt /usr/local/X11/R6/include} to the
directories that will be searched for header files.
\begin{center}
{\tt  :- consult(file,
                 [cc(gcc), cc\_opts('-O2 -I/usr/local/X11/R6/include')]). }
\end{center}
If no C compiler options are specified, the compilation of the C-file
defaults to $\$CC$~{\tt -c~file.c} where $\$CC$ is the name of the C
compiler used to install XSB.  In addition, if XSB was compiled with
the `-g' debugging option, then `-g' option will be automatically
added to the C compiler options list for the foreign module.  Any
Prolog compiler options are ignored when compiling a foreign module.

Prolog-specific directives such as {\tt index}, {\tt hilog}, {\tt
  table}, {\tt auto\_table} or even {\tt import} make no sense in the
case of a foreign module and thus are ignored by the compiler.
However, another directive, namely {\tt ldoption}, is recognized in a
foreign module and is used to instruct the dynamic loading and linking
of the module.  The syntax of the {\tt ldoption} directive is simply:
\begin{center}
{\tt  :- ldoption(Option).    }
\end{center}
where {\tt Option} should either be an atom or a list of atoms.
Multiple {\tt ldoption} directives may appear in the same {\tt .H}
file of a foreign module \footnote{Mac OSX users with Panther should
  set the enviroment variable {\tt MACOSX\_DEPLOYMENT\_TARGET} to {\tt
    10.3} so that the compiler generates code that can be dynamically
  linked by XSB.}.
%(Explain more about the directive.....).
In Unix-derived systems, the foreign language interface of XSB uses
{\tt ld} command that combines object programs to create an executable
file or another object program suitable for further {\tt ld}
processing. \version\ of XSB assumes that the {\tt ld} command resides
in the file {\tt /usr/bin/ld}.

\section{Lower-Level Foreign Language Interface}

Creating a foreign predicate using the lower-level foreign language
interface is almost entirely a matter of writing C code.  Consider the
foreign module {\tt
  \$XSBDIR/examples/XSB\_calling\_c/simple\_foreign.[cH]}.  The .H file
has the form: 

\begin{verbatim}
:- export minus_one/2, my_sqrt/2, change_char/4.

:- ldoption('-lm').     % link together with the math library
\end{verbatim}

When the lower level foreign language interface is used, C functions
that implement foreign predicates must return values of type {\tt
  int}.  The return value is not used by a Prolog argument; rather if
a non-zero is returned, the foreign predicate succeeds; a zero return
value means failure.

At the C level, the function that implements the Prolog predicate must
have the same name as the Prolog predicate (that is declared in the
{\tt *.H} file), and must have a special {\em context parameter}
macro.  The context parameter macro allows C functions to be used with
both the single-threaded and multi-threaded engines, and are described
in detail in Section~\ref{sec:context-parameter}.  The Prolog level
arguments are converted to C data structures through several
predefined functions rather than through direct parameter
passing~\footnote{The inclusion of context parameters changes the
  lower-level interface for Version 3.0.  C files written for previous
  versions of XSB continue to work properly for the single-threaded
  engine in, but will not work properly for the multi-threaded
  engine.}.  The C file {\tt simple\_foreign.c} corresponding to the
above {\tt .H} file is as follows.

\begin{small}
\begin{verbatim}
/*----------------------------------------------------------------------*/

#include <math.h>
#include <stdio.h>
#include <string.h>
#include <alloca.h>

/*----- Make sure your C compiler finds the following header file.    -----
  ----- One way to do this is to include the directory XSB/emu on the -----
  ----- compiler's command line with the -I (/I in Windows) option    -----*/

#include "cinterf.h"

/*-----------------------------------------*/

int minus_one(CTXTdecl)
{
   int  i = ptoc_int(CTXTc 1);

   ctop_int(CTXTc 2, i-1);
   return TRUE;
}

/*-----------------------------------------*/

int my_sqrt(CTXTdecl)
{
   int i = ptoc_int(CTXTc 1);

   ctop_float(CTXTc 2, (float) pow((double)i, 0.5));
   return TRUE;
}

/*-----------------------------------------*/

int change_char(CTXTdecl)
{
   char *str_in;
   int  pos;
   int c;
   char *str_out;

   str_in = (char *) ptoc_string(CTXTc 1);
   str_out = (char *) alloca(strlen(str_in)+1);
   strcpy(str_out, str_in);
   pos = ptoc_int(CTXTc (2);
   c = ptoc_int(CTXTc (3);
   if (c < 0 || c > 255) /* not a character */
     return FALSE; /* this predicate will fail on the Prolog side */

   str_out[pos-1] = c;

   extern_ctop_string(CTXTc 4, str_out);  
   return TRUE;
}

/*----------------------------------------------------------------------*/
\end{verbatim}
\end{small}

Before describing the C program used, here is a sample session
illustrating the behavior of the predicates in {\tt simple\_foreign}.

\begin{small}
\begin{verbatim}
XSB Version 2.0 (Gouden Carolus) of June 26, 1999
[i686-pc-linux-gnu; mode: optimal; engine: slg-wam; scheduling: batched]
| ?- [simple_foreign].
[Compiling C file ./simple_foreign.c using gcc]
[Compiling Foreign Module ./simple_foreign]
[simple_foreign compiled, cpu time used: 0.0099993 seconds]
[simple_foreign loaded]

yes
| ?- change_char('Kostis', 2, w, TempStr), 
     change_char(TempStr, 5, h, GrkName).  

TempStr = Kwstis
GrkName = Kwsths;

no
| ?- minus_one(43, X).

X = 42;

no
| ?- minus_one(43, 42).                   % No output unification is allowed
Wrong arg in ctop_int 2a2 (Reg = 2)

yes
| ?- my_sqrt(4,X). 

X = 2

yes
| ?- my_sqrt(23,X).

X = 4.7958;

no
\end{verbatim}
\end{small}

Consider the function {\tt minus\_one()} above.  As discussed, it
takes a context parameter (explained below), and returns an integer,
and as can be seen the return values can be specified by the macros
{\tt TRUE} and {\tt FALSE}.  From the Prolog perspective the first
argument to {\tt minus\_one/2} is an (integer) input argument, while
the second is an (integer) output argument.  Input arguments for basic
C types are translated from their Prolog representation to a C
representation by functions of the form \verb|ptoc\_<type>()| -- here
{\tt ctop\_int()}.  The single parameter of such a function is the
number of the Prolog argument that is to be transformed and the
function returns the C representation.  Output arguments are converted
from C to Prolog by corresponding functions of the form
\verb|ctop_<type>()| -- here {\tt ctop\_int()}.  For coverting C back
to Prolog, the first parameter of {\tt ctop\_int()} is the number of
the Prolog argument to be transformed and the second is the C value to
be transformed.  In the session output above, if an improper argument
is given to {\tt minus\_one/2} it will emit a warning, and succeed.
Also note that the call {\tt my\_sqrt(23,X)} succeeds once, but fails
on backtracking since it is deterministic, as are all other foreign
language functions.

The above example illustrates the exchange of {\em basic} types
through the lower-level interface -- e.g. atoms, integers, and
floating-point numbers.  The lower-level interface alsoallows a user
to pass lists and terms between XSB and C as will be discussed in
Section~\ref{c2p_p2p_p2c}.

\subsection{Context Parameters} \label{sec:context-parameter}

When using the lower-level interface, {\em context parameters} must be
added to many C functions in order for the functions to be used with
XSB's multi-threaded engine.  In the multi-threaded engine, variables
for Prolog's virtual machine, as well as for thread-private data
structures are stored in a {\em context structure}.  This context
structure must be passed to any functions that need to access elements
of a thread's virtual machine -- including many of the functions that
are used to exchange data between Prolog and C.  On the other hand, in
the single-threaded engine virtual machine elements are kept in static
variables, so that context parameters are not required.

The lower-level C interface makes use of a set of macros to address
the requirements of the different engines.  The data exchange
functions discussed in this chapter, {\tt ptoc\_xxx}, {\tt
  ctop\_xxx}, {\tt c2p\_xxx}, {\tt p2c\_xxx},
and {\tt p2p\_xxx} usually, but not always, require
information about a threads virtual machine state.  
%
\comment{ The {\tt extern} ``functions'' of the lower-level API do
  not require context parameters themselves because they are actually
  macros to functions that use context parameters.  While a user of
  the lower-level C interface does not need to keep track of context
  parameters within the data exchange functions, it is import to note
  that } 
%
If a C function directly or indirectly calls a data interchange
function that requires a context parameter, the function must have a
context parameter in its declaration, calls, and prototypes in order
to be used by the multi-threaded engine.  These context parameters
have the following forms:

\begin{itemize}
\item In function {\em declarations}, use the macro {\tt CTXTdecl} in
  the code for a function that would otherwise be {\tt void}, and {\tt
    CTXTdeclc} as the first argument in the code for a function with
  parameters ({\tt CTXTdeclc} and {\tt CTXTdecl} are similar, except
  that macro expansion of {\tt CTXTdeclc} for the multi-threaded
  engine includes a comma).  The example for {\tt
    minus\_one(CTXTdecl)} shows use of this macro.

\item In function {\em calls} use the macro {\tt CTXT} in the code for
  a function that would otherwise be {\tt void}, and {\tt CTXTc} as
  the first argument in the code for a function with parameters.  As
  an example, a call to {\tt minus\_one} would have the form {\tt
    minus\_one(CTXT)}.

\item In function {\em prototypes} use the macro {\tt CTXTdecltype} in
  the code for a function that would otherwise be {\tt void}, and {\tt
    CTXTdecltypec} as the first argument in the code for a function
  with parameters.  As an example, a prototype for {\tt minus\_one}
  would have the form {\tt minus\_one(CTXTdecltype)}.

\end{itemize}

Fortunately, when compiling with the multi-threaded engine, it is easy
to determine at compile time whether context parameters are correct.
If compilation of a function {\tt foo} gives an error along the lines
of:
%
\begin{verbatim} 
  foofile.c: In function `foo':
  foofile.c:109: error: `th' undeclared (first use in this function)
\end{verbatim}
%
Then the declaration of {\tt foo} omitted a context parameter.  If
compilation gives an error along the lines of 
%
\begin{verbatim} 
  foofile.c: In function `foo_caller':
  :
  foofile.c:149: error: too few arguments to function `foo'
\end{verbatim}
Then the call to {\tt foo} may have omitted a context parameter.

Note that context parameters are {\em only} necessary if the
lower-level interface is used.  The higher-level interface
automatically generates any context parameters it needs.

\subsection{Exchanging Basic Data Types}

The basic interface assumes that correct modes ({\it i.e.}, input or
output parameters) and types are being passed between the C and Prolog
levels.  As a result, output unification should be explicitly
performed in the Prolog level.  The prototypes for the conversion
functions between Prolog and C should be declared before the
corresponding functions are used.  This is done by including the {\tt
  "cinterf.h"} header file.  Under Unix, the XSB foreign C interface
automatically finds this file in the {\tt XSB/emu} directory. Under
Windows (including Cygwin), the user must compile and create the DLL
out of the C file manually, so the compiler option
`\verb|/I...\XSB\emu|' is necessary.

The following C functions are used to convert basic types between
Prolog and C.
\begin{description}
\desc{int ptoc\_int(CTXTdeclc int N)} {\tt CTXTdeclc} is a context
parameter; {\tt N} is assumed to hold a Prolog integer corresponding
to the {\tt N}th argument of a Prolog predicate.  This function
returns the value of that argument in as a C {\tt int}.
%
\desc{double ptoc\_float(CTXTdeclc int N)} {\tt CTXTdeclc} is a
context parameter; {\tt N} is assumed to hold a Prolog integer
corresponding to the {\tt N}th argument of a Prolog predicate.  This
function returns the value of that argument as a C {\tt double}.  By
default, XSB provides double precision, but if XSB was configured with
{\tt --enable-fast-floats} less than single precision can be
provided~\footnote{The fast float configuration option does represents
  floating point values as directly tagged single precision values
  rather than as indirectly tagged double precision values. Speed
  increases in arithmatic can be gained from this optimization, in
  exchange for significant precision loss on floating point numbers.}.
%
\desc{char *ptoc\_string(CTXTdeclc int N)} {\tt CTXTdeclc} is a
context parameter; {\tt N} is assumed to hold a Prolog integer
corresponding to the {\tt N}th argument of a Prolog predicate.  This
function returns the value the C string (of type {\tt char *}) that
corresponds to this interned Prolog atom.  {\em WARNING: the string
  should be copied before being manipulated in any way: otherwise
  unexpected results may arise whenever the interned Prolog atom is
  unified}.
%
\desc{void ctop\_int(CTXTdeclc int N, int V)} {\tt CTXTdeclc} is a
context parameter; argument {\tt N} is assumed to hold a Prolog free
variable, and this function binds that variable to an integer of value
{\tt V}.
%
\desc{void ctop\_float(CTXTdeclc int N, float V)} {\tt CTXTdeclc} is a
context parameter; argument {\tt N} is assumed to hold a Prolog free
variable, and this function binds that variable to a floating point
number of value {\tt V}.
%
\desc{void extern\_ctop\_string(CTXTdeclc int N, char * V)} {\tt
  CTXTdeclc} is a context parameter; argument {\tt N} is assumed to
hold a Prolog free variable.  If needed, this function interns the
string to which {\tt V} points as a Prolog atom and then binds the
variable in argument {\tt N} to that atom.
\end{description}


\subsection{Exchanging Complex Data Types}
\label{c2p_p2p_p2c}

If the lower-level interface is used, exchanging basic data types is
sufficient for most applications.  Exchanging complex data types is
also possible, although doing so is slightly more involved than
exchanging basic types.  To exchange complex data types, the
lower-level interface uses only one C data type: {\tt prolog\_term},
which can point to any XSB term.  On the C side, the type of the term
can be checked and then processed accordingly.  For instance, if the
term turns out to be a structure, then it can be decomposed and the
functor can be extracted along with the arguments.  If the term
happens to be a list, then it can be processed in a loop and each list
member can be further decomposed into its atomic components.  The
advanced interface also provides functions to check the types of these
atomic components and for converting them into C types.

We begin by presenting the functions used to exchange complex data
types, before presenting a detailed example below.  As when exchanging
basic C types, the file {\tt emu/cinterf.h} must be included in the C
program in order to make the prototypes of the relevant functions
known to the C compiler.

The first set of functions is typically used to check the type of
Prolog terms passed into the C program. %%
\begin{description}
\ouritem{xsbBool is\_attv((prolog\_term) T)} \index{\texttt{is\_attv}}
    {\tt is\_attv(T)} returns {\tt TRUE} if {\tt T} represents an XSB
    attributed variable,  and {\tt FALSE} otherwise.

\ouritem{xsbBool is\_float((prolog\_term) T)} \index{\texttt{is\_float}}
    {\tt is\_float(T)} returns {\tt TRUE} if {\tt T} represents an XSB
    float value, and {\tt FALSE} otherwise.

\ouritem{xsbBool is\_functor((prolog\_term) T)} \index{\texttt{is\_functor}}
    {\tt is\_functor(T)} returns {\tt TRUE} if {\tt T} represents an
    XSB structure value (not a list), and {\tt FALSE} otherwise.

\ouritem{xsbBool is\_int((prolog\_term) T)} \index{\texttt{is\_int}}
    {\tt is\_int(T)} returns {\tt TRUE} if {\tt T} represents an XSB
    integer value, and {\tt FALSE} otherwise.

\ouritem{xsbBool is\_list((prolog\_term) T)} \index{\texttt{is\_list}}
    {\tt is\_list(T)} returns {\tt TRUE} if {\tt T} represents an
    XSB list value (not nil), and {\tt FALSE} otherwise.

\ouritem{xsbBool is\_nil((prolog\_term) T)} \index{\texttt{is\_nil}}
    {\tt is\_nil(T)} returns {\tt TRUE} if {\tt T} represents an XSB
    \verb|[]| (nil) value, and {\tt FALSE} otherwise.

\ouritem{xsbBool is\_string((prolog\_term) T)} \index{\texttt{is\_string}}
    {\tt is\_string(T)} returns {\tt TRUE} if {\tt T} represents an XSB
    atom value, and {\tt FALSE} otherwise.

\ouritem{xsbBool is\_var((prolog\_term) T)} \index{\texttt{is\_var}}
    {\tt is\_var(T)} returns {\tt TRUE} if {\tt T} represents an XSB
    variable, and {\tt FALSE} otherwise.

\end{description}

After checking the types of the arguments passed in from the Prolog side,
the next task usually is to convert Prolog data into the types understood
by C.  This is done with the following functions. The first three convert
between the basic types. The last two extract the functor name and the
arity.  Extraction of the components of a list and the arguments of a
structured term is explained later.

\begin{description}
\ouritem{int p2c\_int((prolog\_term) V)}
\index{\texttt{p2c\_int}} 
%
The {\tt prolog\_term} parameter must represent a Prolog integer, and
{\tt p2c\_int} returns the C representation of that integer.

\ouritem{double p2c\_float((prolog\_term) V)}
\index{\texttt{p2c\_float}} 
%
The {\tt prolog\_term} parameter must represent a Prolog floating point
number, and {\tt p2c\_float} returns the C representation of
that floating point number.

\ouritem{char *p2c\_string((prolog\_term) V)}
\index{\texttt{p2c\_string}} 
%
The {\tt prolog\_term} parameter must represent a (Prolog) atom, and
{\tt p2c\_string} returns that atom as a C string. The pointer
returned points to the actual atom name in XSB 's atom table, and thus
it must NOT be modified by the calling program.

\ouritem{char *p2c\_functor((prolog\_term) V)} 
\index{\texttt{p2c\_functor}}
%
The {\tt prolog\_term} parameter must represent a structured term (not a
list).  {\tt p2c\_functor} returns the name of the main
functor symbol of that term as a string. The pointer returned points
to the actual functor name in XSB 's space, and thus it must NOT be
modified by the calling program.

\ouritem{int p2c\_arity((prolog\_term) V)}
\index{\texttt{p2c\_arity}} 
%
The {\tt prolog\_term} parameter must represent a structured term (not
a list).  {\tt p2c\_arity} returns the arity of the main
functor symbol of that term as a C {\tt int}.
\end{description}
%%

The next batch of functions support conversion of data in the opposite
direction: from basic C types to the type {\tt prolog\_term}.  These
{\tt c2p\_*} functions all return a boolean value {\tt TRUE}
if successful and {\tt FALSE} if unsuccessful.  The XSB term argument
must always contain an XSB variable, which will be bound to the
indicated value as a side effect of the function call.

\begin{description}
\ouritem{xsbBool c2p\_int(CTXTdeclc (int) N, (prolog\_term) V)}
\index{\texttt{c2p\_int}}
%
 {\tt CTXTdeclc} is a context parameter; {\tt c2p\_int} binds the
 {\tt prolog\_term} {\tt V} (which must be a variable) to the integer
 value N, creating a Prolog integer.

\ouritem{xsbBool c2p\_float(CTXTdeclc (double) F, (prolog\_term) V)}
\index{\texttt{c2p\_float}} 
%
 {\tt CTXTdeclc} is a context parameter; {\tt c2p\_float} binds the
 {\tt prolog\_term} {\tt V} (which must be a variable) to the (double)
 float value F, creating a double Prolog float.

\ouritem{xsbBool c2p\_string(CTXTdeclc (char *) S, (prolog\_term) V)}
\index{\texttt{c2p\_string}} 
%
 {\tt CTXTdeclc} is a context parameter; {\tt c2p\_string} binds the
 {\tt prolog\_term} {\tt V} (which must be a variable) to the Prolog
 atom corresponding to the {\tt char *S}.  During this process the
 Prolog atom is interned into XSB's atom table.
\end{description}
%%

The following functions create Prolog data structures within a C
program. This is usually done in order to pass these structures back to
the Prolog side.
%%
\begin{description}
\ouritem{xsbBool c2p\_functor(CTXTdeclc (char *) S, (int) N,  (prolog\_term) V)} 
\index{\texttt{c2p\_functor}} 
%
 {\tt CTXTdeclc} is a context parameter; {\tt c2p\_functor} binds the
 {\tt prolog\_term V} (which must be a variable) to an open term whose
 main functor symbol is given by {\tt S} (of type char *) and whose
 arity is {\tt N}.  An open term is one with all arguments as new
 distinct variables.

\ouritem{xsbBool c2p\_list(CTXTdeclc (prolog\_term) V)} 
\index{\texttt{c2p\_list}}
%
 {\tt CTXTdeclc} is a context parameter; {\tt c2p\_list} binds the
 {\tt prolog\_term V} (which must be a variable) to an open list term,
 i.e., a list term with both car and cdr as new distinct
 variables. Note: to create an empty list use the function {\tt
   c2p\_nil} described below.

\ouritem{xsbBool c2p\_nil(CTXTdeclc (prolog\_term) V)} 
\index{\texttt{c2p\_nil}}
%
 {\tt CTXTdeclc} is a context parameter; {\tt c2p\_nil} binds the {\tt
   prolog\_term V} (which must be a variable) to the atom \verb|[]|
 (nil).

\ouritem{prolog\_term p2p\_new()} 
\index{\texttt{p2p\_new}} 
%
Create a new Prolog variable. This is sometimes needed when you want
to create a Prolog term on the C side and pass it to the Prolog side.
\end{description}
%%

To use the above functions, one must be able to get access to the
components of the structured Prolog terms.  This is done with the help
of the following functions:

\begin{description}
\ouritem{prolog\_term p2p\_arg((prolog\_term) T, (int) A)}
\index{\texttt{p2p\_arg}} 
%
Parameter {\tt T} must be a {\tt prolog\_term} that is a structured
term (but not a list).  A is a positive integer (no larger than the
arity of the term) that specifies an argument position of the term
{\tt T}.  {\tt p2p\_arg} returns the A$^{th}$ subfield of the
term {\tt T}.

\ouritem{prolog\_term p2p\_car((prolog\_term) T)} 
\index{\texttt{p2p\_car}}
%
Parameter {\tt T} must be a {\tt prolog\_term} that is a list (not
nil).  {\tt p2p\_car} returns the car (i.e., head of the list)
of the term T.

\ouritem{prolog\_term p2p\_cdr((prolog\_term) T)} 
\index{\texttt{p2p\_cdr}}
%
Parameter {\tt T} must be a {\tt prolog\_term} that is a list (not
nil).  {\tt p2p\_cdr} returns the cdr (i.e., tail of the list)
of the term T.
\end{description}
%%

It is important to realize that these functions return the actual
Prolog term that is, say, the head of a list or the actual argument of
a structured term. Thus, assigning a value to such a Prolog term also
modifies the head of the corresponding list or the relevant argument
of the structured term. It is precisely this feature that allows
passing structured terms and lists from the C side to the Prolog side.
For instance,
%%
\begin{verbatim}
   prolog_term plist,        /* a Prolog list           */
               structure;    /* something like f(a,b,c) */
   prolog_term tail, arg;
   ..........
   tail = p2p_cdr(plist);         /* get the list tail  */
   arg  = p2p_arg(structure, 2);  /* get the second arg */

   /* Assume that the list tail was supposed to be a prolog variable */
   if (is_var(tail))
      c2p_nil(CTXTc tail);  /* terminate the list */
   else {
      fprintf(stderr, "Something wrong with the list tail!");
      exit(1);
   }
   /* Assume that the argument was supposed to be a prolog variable */
   c2p_string(CTXTc "abcdef", arg);
\end{verbatim}
%%

In the above program fragment, we assume that both the tail of the list and
the second argument of the term were supposed to be bound to Prolog variables.
In case of the tail, we check if this is, indeed, the case. In case of the
argument, no checks are done; XSB will issue an error (which might be hard
to track down) if the second argument is not currently bound to a variable.

The last batch of functions is useful for passing data in and out of the
Prolog side of XSB. The first function is the only way to get a
{\tt prolog\_term} out of the Prolog side; the second function is
sometimes needed in order to pass complex structures from C into Prolog.
%%
\begin{description}
  \ouritem{prolog\_term reg\_term(CTXTdeclc (int) R)} 
\index{\texttt{reg\_term}}
%
  {\tt CTXTdeclc} is a context parameter.  Parameter R is an argument
  number of the Prolog predicate implemented by this C function (range
  1 to 255). The function {\tt reg\_term} returns the {\tt
    prolog\_term} in that predicate argument.
% 
\ouritem{xsbBool p2p\_unify(CTXTdeclc prolog\_term T1, prolog\_term T2)} 
\index{\texttt{p2p\_unify}} 
%
Unify the two Prolog terms. This is useful when an argument of the
Prolog predicate (implemented in C) is a structured term or a list,
which acts both as input and output parameter.  {\tt CTXTdeclc} is a
context parameter.
\end{description}
%%

For instance, consider the Prolog call {\tt test(X, f(Z))},
which is implemented by a C function with the following fragment:
%%
\begin{verbatim}
    prolog_term newterm, newvar, z_var, arg2;
    .....
    /* process argument 1 */
    c2p_functor(CTXTc "func",1,reg_term(CTXTc 1));
    c2p_string(CTXTc "str",p2p_arg(reg_term(CTXTc 1),1));
    /* process argument 2 */
    arg2 = reg_term(CTXTc 2);
    z_var = p2p_arg(arg2, 1);  /* get the var Z */
    /* bind newterm to abc(V), where V is a new var */
    c2p_functor(CTXTc "abc", 1, newterm);
    newvar = p2p_arg(newterm, 1);
    newvar = p2p_new();
    ....
    /* return TRUE (success), if unify; FALSE (failure) otherwise */
    return p2p_unify(CTXTc z_var, newterm);
\end{verbatim}
%%
On exit, the variable {\tt X} will be bound to the term {\tt
  func(str)}.  Processing argument 2 is more interesting. Here,
argument 2 is used both for input and output. If {\tt test} is called
as above, then on exit $Z$ will be bound to {\tt abc(\_h123)}, where
{\tt \_h123} is some new Prolog variable. But if the call is {\tt
  test(X,f(1))} or {\tt test(X,f(Z,V))} then this call will
\emph{fail} (fail as in Prolog, {\it i.e.}, it is not an error),
because the term passed back, {\tt abc(\_h123)}, does not unify with
{\tt f(1)} or {\tt f(Z,V)}. This effect is achieved by the use of {\tt
  p2p\_unify} above.

We conclude with two real examples of functions that pass complex data
in and out of the Prolog side of XSB. These functions are part of the
Posix regular expression matching package of XSB. The first function
uses argument 2 to accept a list of complex Prolog terms from the
Prolog side and does the processing on the C side. The second function
does the opposite: it constructs a list of complex Prolog terms on the
C side and passes it over to the Prolog side in argument 5.

%%
{\small 
\begin{verbatim}
/* XSB string substitution entry point: replace substrings specified in Arg2
   with strings in Arg3.
   In: 
       Arg1: string
       Arg2: substring specification, a list [s(B1,E1),s(B2,E2),...]
       Arg3: list of replacement string
   Out:
       Arg4: new (output) string
   Always succeeds, unless error.
*/
int do_regsubstitute__(CTXTdecl)
{
  /* Prolog args are first assigned to these, so we could examine the types
     of these objects to determine if we got strings or atoms. */
  prolog_term input_term, output_term;
  prolog_term subst_reg_term, subst_spec_list_term, subst_spec_list_term1;
  prolog_term subst_str_term=(prolog_term)0,
    subst_str_list_term, subst_str_list_term1;
  char *input_string=NULL;    /* string where matches are to be found */
  char *subst_string=NULL;
  prolog_term beg_term, end_term;
  int beg_offset=0, end_offset=0, input_len;
  int last_pos = 0; /* last scanned pos in input string */
  /* the output buffer is made large enough to include the input string and the
     substitution string. */
  char subst_buf[MAXBUFSIZE];
  char *output_ptr;
  int conversion_required=FALSE; /* from C string to Prolog char list */

  input_term = reg_term(CTXTc 1);  /* Arg1: string to find matches in */
  if (is_string(input_term)) /* check it */
    input_string = string_val(input_term);
  else if (is_list(input_term)) {
    input_string =
      p_charlist_to_c_string(input_term, input_buffer, sizeof(input_buffer),
                             "RE_SUBSTITUTE", "input string");
    conversion_required = TRUE;
  } else
    xsb_abort("RE_SUBSTITUTE: Arg 1 (the input string) must be an atom or a character list");

  input_len = strlen(input_string);

  /* arg 2: substring specification */
  subst_spec_list_term = reg_term(CTXTc 2);
  if (!is_list(subst_spec_list_term) && !is_nil(subst_spec_list_term))
    xsb_abort("RE_SUBSTITUTE: Arg 2 must be a list [s(B1,E1),s(B2,E2),...]");

  /* handle substitution string */
  subst_str_list_term = reg_term(CTXTc 3);
  if (! is_list(subst_str_list_term))
    xsb_abort("RE_SUBSTITUTE: Arg 3 must be a list of strings");

  output_term = reg_term(CTXTc 4);
  if (! is_var(output_term))
    xsb_abort("RE_SUBSTITUTE: Arg 4 (the output) must be an unbound variable");

  subst_spec_list_term1 = subst_spec_list_term;
  subst_str_list_term1 = subst_str_list_term;

  if (is_nil(subst_spec_list_term1)) {
    strncpy(output_buffer, input_string, sizeof(output_buffer));
    goto EXIT;
  }
  if (is_nil(subst_str_list_term1))
    xsb_abort("RE_SUBSTITUTE: Arg 3 must not be an empty list");

  /* initialize output buf */
  output_ptr = output_buffer;

  do {
    subst_reg_term = p2p_car(subst_spec_list_term1);
    subst_spec_list_term1 = p2p_cdr(subst_spec_list_term1);

    if (!is_nil(subst_str_list_term1)) {
      subst_str_term = p2p_car(subst_str_list_term1);
      subst_str_list_term1 = p2p_cdr(subst_str_list_term1);

      if (is_string(subst_str_term)) {
        subst_string = string_val(subst_str_term);
      } else if (is_list(subst_str_term)) {
        subst_string =
          p_charlist_to_c_string(subst_str_term, subst_buf, sizeof(subst_buf),
                                 "RE_SUBSTITUTE", "substitution string");
      } else 
        xsb_abort("RE_SUBSTITUTE: Arg 3 must be a list of strings");
    }

    beg_term = p2p_arg(subst_reg_term,1);
    end_term = p2p_arg(subst_reg_term,2);

    if (!is_int(beg_term) || !is_int(end_term))
      xsb_abort("RE_SUBSTITUTE: Non-integer in Arg 2");
    else{
      beg_offset = int_val(beg_term);
      end_offset = int_val(end_term);
    }
    /* -1 means end of string */
    if (end_offset < 0)
      end_offset = input_len;
    if ((end_offset < beg_offset) || (beg_offset < last_pos))
      xsb_abort("RE_SUBSTITUTE: Substitution regions in Arg 2 not sorted");

    /* do the actual replacement */
    strncpy(output_ptr, input_string + last_pos, beg_offset - last_pos);
    output_ptr = output_ptr + beg_offset - last_pos;
    if (sizeof(output_buffer)
        > (output_ptr - output_buffer + strlen(subst_string)))
      strcpy(output_ptr, subst_string);
    else
      xsb_abort("RE_SUBSTITUTE: Substitution result size %d > maximum %d",
                beg_offset + strlen(subst_string),
                sizeof(output_buffer));
    
    last_pos = end_offset;
    output_ptr = output_ptr + strlen(subst_string);

  } while (!is_nil(subst_spec_list_term1));

  if (sizeof(output_buffer) > (output_ptr-output_buffer+input_len-end_offset))
    strcat(output_ptr, input_string+end_offset);

 EXIT:
  /* get result out */
  if (conversion_required)
    c_string_to_p_charlist(output_buffer,output_term,"RE_SUBSTITUTE","Arg 4");
  else
    /* DO NOT intern. When atom table garbage collection is in place, then
       replace the instruction with this:
                  c2p_string(CTXTc output_buffer, output_term);
       The reason for not interning is that in Web page
       manipulation it is often necessary to process the same string many
       times. This can cause atom table overflow. Not interning allows us to
       circumvent the problem.  */
    extern_ctop_string(CTXTc 4, output_buffer);
  
  return(TRUE);
}


/* XSB regular expression matcher entry point
   In:
       Arg1: regexp
       Arg2: string
       Arg3: offset
       Arg4: ignorecase
   Out:
       Arg5: list of the form [match(bo0,eo0), match(bo1,eo1),...]
             where bo*,eo* specify the beginning and ending offsets of the
             matched substrings.
             All matched substrings are returned. Parenthesized expressions are
             ignored.
*/
int do_bulkmatch__(CTXTdecl)
{
  prolog_term listHead, listTail;
  /* Prolog args are first assigned to these, so we could examine the types
     of these objects to determine if we got strings or atoms. */
  prolog_term regexp_term, input_term, offset_term;
  prolog_term output_term = p2p_new();
  char *regexp_ptr=NULL;      /* regular expression ptr               */
  char *input_string=NULL;    /* string where matches are to be found */
  int ignorecase=FALSE;
  int return_code, paren_number, offset;
  regmatch_t *match_array;
  int last_pos=0, input_len;
  char regexp_buffer[MAXBUFSIZE];

  if (first_call)
    initialize_regexp_tbl();

  regexp_term = reg_term(CTXTc 1);  /* Arg1: regexp */
  if (is_string(regexp_term)) /* check it */
    regexp_ptr = string_val(regexp_term);
  else if (is_list(regexp_term))
    regexp_ptr =
      p_charlist_to_c_string(regexp_term, regexp_buffer, sizeof(regexp_buffer),
                             "RE_MATCH", "regular expression");
  else
    xsb_abort("RE_MATCH: Arg 1 (the regular expression) must be an atom or a character list");

  input_term = reg_term(CTXTc 2);  /* Arg2: string to find matches in */
  if (is_string(input_term)) /* check it */
    input_string = string_val(input_term);
  else if (is_list(input_term)) {
    input_string =
      p_charlist_to_c_string(input_term, input_buffer, sizeof(input_buffer),
                             "RE_MATCH", "input string");
  } else
    xsb_abort("RE_MATCH: Arg 2 (the input string) must be an atom or a character list");

  input_len = strlen(input_string);
  
  offset_term = reg_term(CTXTc 3); /* arg3: offset within the string */
  if (! is_int(offset_term))
    xsb_abort("RE_MATCH: Arg 3 (the offset) must be an integer");
  offset = int_val(offset_term);
  if (offset < 0 || offset > input_len)
    xsb_abort("RE_MATCH: Arg 3 (=%d) must be between 0 and %d", input_len);

  /* If arg 4 is bound to anything, then consider this as ignore case flag */
  if (! is_var(reg_term(CTXTc 4)))
    ignorecase = TRUE;

  last_pos = offset;
  /* returned result */
  listTail = output_term;
  while (last_pos < input_len) {
    c2p_list(CTXTc listTail); /* make it into a list */
    listHead = p2p_car(listTail); /* get head of the list */

    return_code = xsb_re_match(regexp_ptr, input_string+last_pos, ignorecase,
                               &match_array, &paren_number);
    /* exit on no match */
    if (! return_code) break;

    /* bind i-th match to listHead as match(beg,end) */
    c2p_functor(CTXTc "match", 2, listHead);
    c2p_int(CTXTc match_array[0].rm_so+last_pos, p2p_arg(listHead,1));
    c2p_int(CTXTc match_array[0].rm_eo+last_pos, p2p_arg(listHead,2));

    listTail = p2p_cdr(listTail);
    last_pos = match_array[0].rm_eo+last_pos;
  }
  c2p_nil(CTXTc listTail); /* bind tail to nil */
  return p2p_unify(CTXTc output_term, reg_term(CTXTc 5));
}
\end{verbatim}
}

\section{Foreign Modules That Link Dynamically with Other Libraries}

Sometimes a foreign module might have to link dynamically with other
(non-XSB) libraries. Typically, this happens when the foreign module
implements an interface to a large external library of utilities.
One example of this is the package {\tt libwww} in the XSB distribution,
which provides a high-level interface to the W3C's Libwww library for
accessing the Web. The library is compiled into a set of shared objects and
the {\tt libwww} module has to link with them as well as with XSB.

\index{LD\_LIBRARY\_PATH}
\index{LIBPATH}
The problem here is that the loader must know at run time where to look for
the shared objects to link with. On Unix systems, this is specified using
the environment variable {\tt LD\_LIBRARY\_PATH}; on Windows, the variable
name is {\tt LIBPATH}. For instance, 
under Bourne shell or its derivatives, the following will do:
%%
\begin{verbatim}
LD_LIBRARY_PATH=dir1:dir2:dir3
export LD_LIBRARY_PATH
\end{verbatim}
%%
One problem with this approach is that this variable must be set before
starting XSB. The other problem is that such a global setting might
interact with other foreign modules.

To alleviate the problem, XSB dynamically sets {\tt LD\_LIBRARY\_PATH}
({\tt LIBPATH} on Windows) before loading foreign modules by adding the
directories specified in the {\tt -L} option in {\tt ldoption}.
Unfortunately, this works on some systems (Linux), but not on others
(Solaris). One route around this difficulty is to build a runtime library
search path directly into the object code of the foreign module. This can
be specified using a loader flag in {\tt ldoption}.  The problem here is
that different systems use a different flag!  To circumvent this, XSB
provides a predicate that tries to guess the right flag for your system:
%%
\index{\texttt{runtime\_loader\_flag/2}}
%%
\begin{verbatim}
runtime_loader_flag(+Hint,-Flag)  
\end{verbatim}
%%
Currently it knows about a handful of the most popular systems, but this
will be expanded. The argument {\tt Hint} is not currently used.
It might be used in the future to provide {\tt runtime\_loader\_flag} with
additional information that can improve the accuracy of finding the right
runtime flags for various systems.

The above predicate can be used as follows:
%%
\begin{verbatim}
    ...,
    runtime_loader_flag(_,Flag),
    fmt_write_string(LDoptions, '%sdir1:dir2:dir2 %s', args(Flag,OldLDoption)),
    fmt_write(File, ':- ldoption(%s).', LDoptions),
    file_nl(File).
\end{verbatim}
%%

\section{Higher-Level Foreign Language Interface}

The high-level foreign predicate interface was designed to release the
programmer from the burden of having to write low-level code to
transfer data from XSB to C and vice-versa.  Instead, all the user
needs to do is to describe each C function and its corresponding
Prolog predicates in the {\tt .H} files. The interface then
automatically generates \emph{wrappers} that translate Prolog terms
and structures to proper C types, and vice-versa.  These wrappers also
check for type-correctness of arguments to the C function; in
addition, in Unix-derived systems the wrappers are automatically
compiled and loaded along with the foreign predicates in the {\tt .c}
file~\footnote{for Windows, please see special instructions in
  Section~\ref{sec:foreign-windows}.}.

As with the lower-level foreign interfaces, when predicates are
defined in a foreign module {\verb|myfile.[cH]|}, the predicates must
be explicitly imported from the module to be used~\footnote{In
  \version , a foreign module that uses the higher-level C interface
  must be explicitly consulted before it can be used.}.  For an
example of using the higher level interface, see {\tt
  \$XSBDIR/examples/XSB\_calling\_c/second\_foreign.[cH]}.

\subsection{Declaration of high level foreign predicates}

The basic format of a foreign predicate declaration is:
%%
\begin{center}
{\tt :- foreign\_pred \emph{predname}(\emph{[+-]parg1,
  [+-]parg2,...})\\
~~~~~~~~~~~~~~~~~~~from \emph{funcname}(\emph{carg1:type1, carg2:type2,
  ...}):\emph{functype}.
}
\end{center}
%%
where:

\begin{description}

\ouritem{predname} is the name of the foreign Prolog predicate. 

\ouritem{parg1, parg2, ...} are the predicate arguments. Each argument
is preceded by either '+' or '-', indicating its mode as input or
output respectively. The names of the arguments must be the same as
those used in the declaration of the corresponding C function. If a C
argument is used both for input and output, then the corresponding
Prolog argument can appear twice: once with ``+'' and once with ``-''.
In addition, a special argument \texttt{retval} is used to denote the
argument that corresponds to the return value of the C function; it
must always have the mode '-'.

\ouritem{funcname} is the name of the function in the {\tt .c} file.
At compile-time a C function with name {\tt predname} will be
generated which will translate arguments from Prolog to C, call {\tt
  funcname}, and then translate arguments back from C to Prolog.

\ouritem{carg1, carg2, ...} is the list of arguments of the C
function. The names used for the arguments must match the names used
in the Prolog declaration.

\ouritem{type1, type2, ...} are the types associated to the arguments of
the C function. This is not the set of C types, but rather a set of
descriptive types, as defined in Table~\ref{table:hltypes}.

\ouritem{functype} is the return type of the C function.

\end{description}

Table~\ref{table:hltypes} provides the correspondence between the types
allowed on the C side of a foreign module declaration and the types allowed
on the Prolog side of the declaration.

\begin{table}
\label{table:hltypes}
\scriptsize
\begin{tabular}{||l|l|l|l||}
\hline
\hline
Descriptive Type & Mode Usage & Associated C Type & Comments\\ 
\hline
\hline
int & + & int  & integer numbers \\
float & + & double & floating point numbers \\
atom & + & unsigned long & atom represented as an unsigned long\\
chars & + & char * & the textual representation of an atom is passed
to C as a string \\
chars(\emph{size}) & + & char * & the textual representation of an
atom is passed to C \\
& & & as a string in a buffer of size \emph{size} \\ 
string & + & char * & a prolog list of characters is passed to C as a
string \\
string(\emph{size}) & + & char * & a prolog list of characters is
passed to C as a string \\
term & + & {\tt prolog\_term}  & the unique representation of a term\\
intptr & + & int * & the location of a given integer\\
floatptr & + & double * & the location of a given floating point
number \\
atomptr & + & unsigned long * & the location of the unique
representation of a given atom \\
charsptr & + & char ** & the location of the textual representation of
an atom \\
stringptr & + & char ** & the location of the textual representation
of a list of characters \\
termptr & + & {\tt prolog\_term} * & the location of the unique
representation of a term \\
\hline
intptr & - & int * & the integer value returned is passed to Prolog \\
floatptr & - & double * & the floating point number is passed back to
Prolog \\
charsptr & - & char ** & the string returned is passed to Prolog as an
atom \\
stringptr & - & char ** & the string returned is passed back as a list
of characters \\
atomptr & - & unsigned long * & the number returned is passed back to
Prolog as the \\
 & & & unique representation of an atom \\
termptr & - & {\tt prolog\_term} * & the number returned is passed to Prolog
as the unique\\
 & & & representation of a term \\
\hline
chars(\emph{size}) & +- & char * & the atom is copied from Prolog to a
buffer, passed to C \\
 & & & and converted back to Prolog afterwards \\
string(\emph{size}) & +- & char * & the list of characters is copied
from Prolog to a buffer, \\
 & & & passed to C and back to Prolog afterwards \\
intptr & +- & int * & an integer is passed from Prolog to C and from C
back to Prolog \\
floatptr & +- & double * & a float number is passed from Prolog to C,
and back to Prolog \\
atomptr & +- & unsigned long * & the unique representation of an atom
is passed to C, and back to Prolog \\
charsptr & +- & char ** & the atom is passed to C as a string, and 
a string is passed to\\
 & & & Prolog as an atom \\
stringptr & +- & char ** & the list of characters is passed to C, and
a string passed to Prolog \\
 & & & as a list of characters \\
termptr & +- & {\tt prolog\_term} * & the unique representation of a term is
passed to C, \\
 & & & and back to Prolog \\
\hline
\hline
\end{tabular}
\caption{Allowed combinations of types and modes, and their meanings}
\label{tbl-types-p}
\end{table}

In all modes and types, checks are performed to ensure the types of
the arguments. Also, all arguments of type '-' are checked to be free
variables at call time.

\section{Compiling Foreign Modules on Windows and under Cygwin} 
\label{sec:foreign-windows}

Due to the complexity of creating makefiles for the different compilers
under Windows, XSB doesn't attempt to compile and build DLL's for the
Windows foreign modules automatically. However, for almost all typical
cases the user should be able to easily adapt the sample makefile for
Microsoft VC++:
%%
\begin{quote}
 {\tt XSB/examples/XSB\_calling\_c/MakefileForCreatingDLLs}
\end{quote}
%%
It is important that the C program will have the following lines near the
top of the file:
%%
\begin{verbatim}
   #include "xsb_config.h"  
   #ifdef WIN_NT
   #define XSB_DLL
   #endif
   #include "cinterf.h"
\end{verbatim}
%%

Note that these same DLLs will work under Cygwin --- XSB's C interface
under Cygwin is like that under Windows rather than Unix.


If the above makefile cannot be adapted, then the user has to create
the DLL herself.  The process is, roughly, as follows: first, compile
the module from within XSB.  This will create the XSB-specific object
file, and (if using the higher-level C interface) the
\emph{wrappers}. The \emph{wrappers} are created in a file named
\texttt{xsb\_wrap\_}\emph{modulename}\texttt{.c}.

Then, create a project, using the compiler of choice, for a
dynamically-linked library that exports symbols. In this project, the
user must include the source code of the module along with the
\emph{wrapper} created by XSB. This DLL should be linked against the
library
%%
\begin{quote}
   \verb|XSB\config\x86-pc-windows\bin\xsb.lib|
\end{quote}
%%
which is distributed with XSB. In VC++, this library should be added 
as part of the linkage specification. In addition, the following
directories for included header files must be specified as part of the
preprocessor setup:
%%
\begin{verbatim}
    XSB\config\x86-pc-windows
    XSB\prolog_includes
    XSB\emu
\end{verbatim}
%%
In VC++, make sure you check off the ``No precompiled headers'' box as part
of the ``Precompiled headers'' specification. All these options are
available through the {\tt Project>>Settings} menu item.



%%% Local Variables: 
%%% mode: latex
%%% TeX-master: "manual2"
%%% End: 
