\chapter{Foreign Language Interface}
%===================================
\label{foreign}

\section{Compiler Directives for Foreign C Modules}

When \ourprolog\ is used to build real-world systems, a foreign-language
interface may be necessary to:
\begin{itemize}
\item combine \ourprolog\ with existing programs and libraries, thereby
      forming composite systems;
\item interface \ourprolog\ with the operating system, graphical user 
      interfaces or other system level programs;
\item speed up certain critical operations.
\end{itemize}


Foreign predicates must always appear in modules, which can contain
only foreign predicates.  The main difference between a normal module
and a foreign module is the very natural one: the source file of the
module implementation, which is in C, must appear in a {\tt *.c} file
rather than a {\tt *.P} file.  This {\tt *.c} file cannot contain a
{\tt main()} function.  Furthermore, a {\tt *.P} file with the same
name {\em must not} be present or else the {\tt *.c} file is ignored
and the module is compiled as a regular Prolog module.  The interface
part of a foreign module, which has the same syntax as that of a
normal module, is written in Prolog and hence must appear in a {\tt
*.H} file.  This {\tt *.H} file contains {\tt export} declarations
for each and every one of the foreign predicates that are to be used
by other modules.  Here is an example of a {\tt .H} file for a foreign
module:

\begin{verbatim}
:- export minus_one/2, my_sqrt/2, change_char/4.

:- ldoption('-lm').     % link together with the math library
\end{verbatim}

Directives such as {\tt index}, {\tt hilog}, {\tt table}, {\tt auto\_table} 
or even {\tt import} make no sense in the case of a foreign module and thus 
are ignored by the compiler.  However, another directive, namely 
{\tt ldoption}, is recognized in a foreign module and is used to instruct 
the dynamic loading and linking of the module.  
The syntax of the {\tt ldoption} directive is simply:
\begin{center}
{\tt  :- ldoption(Option).    }
\end{center}
where {\tt Option} should either be an atom (currently of length less
or equal to 255 characters) or a list of atoms.  Multiple {\tt ldoption}
directives may appear in the same {\tt .H} file of a foreign module.

%(Explain more about the directive.....).
The foreign language interface of \ourprolog\ uses the Unix command 
{\tt ld} that combines object programs to create an executable file  
or another object program suitable for further {\tt ld} processing.
\version\ of \ourprolog\ assumes that the {\tt ld} command resides
in the file {\tt /usr/bin/ld}.

C functions that implement foreign predicates must return values of type
{\tt int}. If a non-zero is returned, the foreign predicate succeeds; a
zero return value means failure.

A well-designed foreign predicate must check that its arguments are of the
correct types and modes. However, such checks can also be done using
Prolog-side wrappers that invoke a foreign predicate.

At the C level, the procedure that implements the foreign predicate
must have the same name as the predicate (that is declared in the 
% again the example. ???
{\tt *.H} file), and it must be {\em parameterless}.  The Prolog level 
arguments are converted to C data structures through several 
predefined functions rather than through direct parameter passing.


In the current implementation, the Prolog procedures that are
attached to foreign predicates are deterministic, in the sense that
they succeed at most once for a given call and are not re-entered on
backtracking.  Note that this requirement imposes no serious
limitation, since it is always possible to divide a foreign predicate
into the part to be done on the first call and the part to be redone
on backtracking.  Backtracking can then take place at the Prolog
level where it is more naturally expressed.

A foreign module can be {\tt compile}d or {\tt consult}ed just like a
normal Prolog module.  Currently, predicates {\tt consult/[1,2]}
recompile both the {\tt *.c} and the {\tt *.H} files of a foreign
module when at least one of them has been changed from the time the
corresponding object files have been created (see the section {\it
Compiling and Consulting} in Volume 1. 
The C compiler used to compile the {\tt *.c} files can be set as a
compilation option or defaults to that used for the configuration of
XSB (refer to the section {\it Getting Started with XSB} in Volume 1.
Moreover, the user can control the compiler options that can be passed
to the C compiler.  To give an example, the following command will
compile file {\tt file.c} using the Gnu C Compiler with optimization
and by including {\tt /usr/local/X11/R6/include} to the directories that
will be searched for header files.
\begin{center}
{\tt  :- consult(file,
                 [cc(gcc), cc\_opts('-O2 -I/usr/local/X11/R6/include')]). }
\end{center}
If no C compiler options are specified, the compilation of the C-file
defaults to $CC$~{\tt -c~file.c} where $CC$ is the name of the C compiler
used to install \ourprolog.
Any Prolog compiler options are ignored when compiling a foreign module.


\section{Passing Data between XSB and C}

The XSB foreign language interface can be split in two parts.
The \emph{basic} interface supports the exchange of Prolog's 
atomic data types (atoms, integers, and floating-point numbers). 
The \emph{advanced} interface allows passing lists and terms between XSB
and C.

\subsection{Exchanging Basic Data Types}
The basic interface assumes that correct modes ({\it i.e.}, input or
output) and types are being passed between C and the Prolog level.  So,
output unification should be explicitly performed in the Prolog level.  The
function prototypes should be declared before the corresponding functions
are used.  This is done by including the {\tt "cinterf.h"} header file.
Under Unix, the XSB foreign C interface automatically finds this file in
the {\tt XSB/emu} directory. Under Windows, the user must compile and
create the DLL out of the C file manually, so the compiler option
`\verb|/I...\XSB\emu|' is necessary.

The following C functions are used to convert basic Prolog and C data types
to each other.
% again the example ???
\begin{description}
\desc{int ptoc\_int(int N)}
        Argument {\tt N} is assumed to hold a Prolog integer, and this
        function returns its integer value in C format.
\desc{float ptoc\_float(int N)}
        Argument {\tt N} is assumed to hold a Prolog floating point number,
        and this function returns its floating point value in C format.
        (Precision is less than single word floating point).
\desc{char *ptoc\_string(int N)}
        Argument {\tt N} is assumed to hold a Prolog atom, and this
        function returns the C string (of type {\tt char *}) that 
        corresponds to this Prolog atom.
\desc{void ctop\_int(int N, int V)}
        Argument {\tt N} is assumed to hold a Prolog free variable, and
        this function binds that variable to an integer of value {\tt V}.
\desc{void ctop\_float(int N, float V)}
        Argument {\tt N} is assumed to hold a Prolog free variable, and
        this function binds that variable to a floating point number of 
        value {\tt V}.
\desc{void ctop\_string(int N, char * V)}
        Argument {\tt N} is assumed to hold a Prolog free variable, and
        this function binds that variable to a Prolog atom of value {\tt V}.
        In C, {\tt V} is of type {\tt char *}.

	Note that the atom of value {\tt V} is not interned, i.e. it is 
	not inserted into the Prolog atom table.  For that reason, the
	{\tt string\_find(char *V, int Insert)} function should be used.
        Function {\tt string\_find()} searches the symbol table for
        the symbol, and if the symbol does not appear there and the
	value of {\tt Insert} is non-zero, it inserts it.  Thus, the most
	common use of this function is as follows:
	\begin{center}
	{\tt	ctop\_string(N, string\_find(V, 1))	}
	\end{center}
	Refer to the example {\tt simple\_foreign} in the {\tt examples}
	directory to see a use of this function.
\end{description}

\subsubsection*{Examples of Using the Basic C interface}

We end by a very simple example of using the foreign language
interface of \ourprolog.  The programs above and below are programs
{\tt simple\_foreign.$\{$H,c$\}$} in the {\tt examples} directory.

\begin{small}
\begin{verbatim}

#include <math.h>
#include <stdio.h>
#include <string.h>
#include <alloca.h>

/*----- Make sure your C compiler finds the following header file.    -----
  ----- The best way to do this is to include the directory XSB/emu   -----
  ----- on compiler's command line with the -I (/I in Windows) option -----*/

#include "cinterf.h"

/*----------------------------------------------------------------------*/

int minus_one(void)
{
   int  i = ptoc_int(1);

   ctop_int(2, i-1);
   return TRUE;
}

/*----------------------------------------------------------------------*/

int my_sqrt(void)
{
   int i = ptoc_int(1);

   ctop_float(2, (float) pow((double)i, 0.5));
   return TRUE;
}

/*----------------------------------------------------------------------*/

int change_char(void)
{
   char *str_in;
   int  pos;
   int c;
   char *str_out;

   str_in = (char *) ptoc_string(1);
   str_out = (char *) alloca(strlen(str_in)+1);
   strcpy(str_out, str_in);
   pos = ptoc_int(2);
   c = ptoc_int(3);
   if (c < 0 || c > 255) /* not a character */
     return FALSE; /* this predicate will fail on the Prolog side */

   str_out[pos-1] = c;

   /* Now that we have constructed a new symbol, we must ensure that it
      appears in the symbol table.  This can be done using function
      string_find() that searches the symbol table for the symbol, and
      if the symbol does not appear there, it inserts it.  If we are
      sure that the symbol already appeared in the symbol table there
      is no need to use string_find().
    */

   ctop_string(4, (char *) string_find(str_out,1));  /* 1 = INSERT */
   return TRUE;
}

/*----------------------------------------------------------------------*/
\end{verbatim}
\end{small}

Here is a sample session illustrating the use of these files.
\begin{small}
\begin{verbatim}
XSB Version 2.0 (Gouden Carolus) of June 26, 1999
[i686-pc-linux-gnu; mode: optimal; engine: slg-wam; scheduling: batched]
| ?- [simple_foreign].
[Compiling C file ./simple_foreign.c using gcc]
[Compiling Foreign Module ./simple_foreign]
[simple_foreign compiled, cpu time used: 0.0099993 seconds]
[simple_foreign loaded]

yes
| ?- change_char('Kostis', 2, 119, TempStr), % 119 is w
     change_char(TempStr, 5, 104, GrkName).  % 104 is h

TempStr = Kwstis
GrkName = Kwsths;

no
| ?- minus_one(43, X).

X = 42;

no
| ?- minus_one(43, 42).                   % No output unification is allowed
Wrong arg in ctop_int 2a2 (Reg = 2)

yes
| ?- my_sqrt(4,X). 

X = 2

yes
| ?- my_sqrt(23,X).

X = 4.7958;

no
\end{verbatim}
\end{small}

There are additional sample programs in the {\tt examples} directory that
exhibit most of the features of the foreign language interface.


\subsection{Exchanging Complex Data Types}
\label{c2p_p2p_p2c}

The advanced XSB/Prolog interface uses only one data type: {\tt
prolog\_term}.  A Prolog term (as the name suggests) can be bound to any
XSB term. On the C side, the type of the term can be checked and then
processed accordingly.
For instance, if the term turns out to be a structure, then it can be
decomposed and the functor can be extracted along with the arguments.
If the term happens to be a list, then it can be processed in a loop and
each list member can be further decomposed into its atomic components.
The advanced interface also provides functions to check the types of these
atomic components and for converting them into C types.

As with the basic C interface, the file {\tt emu/cinterf.h} must be
included in the C program in order to make the prototypes of the relevant
functions known to the C compiler.

The first set of functions is typically used to check the type of
Prolog terms passed into the C program. 
%%
\begin{description}
  \ouritem{bool is\_var((prolog\_term) T)} \index{{\tt is\_var}}
    {\tt is\_var(T)} returns TRUE if {\tt T} represents an \ourprolog\
    variable, and FALSE otherwise.

\ouritem{bool is\_int((prolog\_term) T)} \index{{\tt is\_int}}
    {\tt is\_int(T)} returns TRUE if {\tt T} represents an \ourprolog\
    integer value, and FALSE otherwise.

\ouritem{bool is\_float((prolog\_term) T)} \index{{\tt is\_float}}
    {\tt is\_float(T)} returns TRUE if {\tt T} represents an \ourprolog\
    float value, and FALSE otherwise.

\ouritem{bool is\_string((prolog\_term) T)} \index{{\tt is\_string}}
    {\tt is\_string(T)} returns TRUE if {\tt T} represents an \ourprolog\
    atom value, and FALSE otherwise.

\ouritem{bool is\_functor((prolog\_term) T)} \index{{\tt is\_functor}}
    {\tt is\_functor(T)} returns TRUE if {\tt T} represents an
    \ourprolog\ structure value (not a list), and FALSE otherwise.

\ouritem{bool is\_list((prolog\_term) T)} \index{{\tt is\_list}}
    {\tt is\_list(T)} returns TRUE if {\tt T} represents an
    \ourprolog\ list value (not nil), and FALSE otherwise.

\ouritem{bool is\_nil((prolog\_term) T)} \index{{\tt is\_nil}}
    {\tt is\_nil(T)} returns TRUE if {\tt T} represents an \ourprolog\
    \verb|[]| (nil) value, and FALSE otherwise.

\end{description}


After checking the types of the arguments passed in from the Prolog side,
the next task usually is to convert Prolog data into the types understood
by C.  This is done with the following functions. The first three convert
between the basic types. The last two extract the functor name and the
arity.  Extraction of the components of a list and the arguments of a
structured term is explained later.

\begin{description}
\ouritem{int p2c\_int((prolog\_term) V)} \index{{\tt p2c\_int}}
    The prolog\_term argument must represent an integer, and {\tt
    p2c\_int} returns the value of that integer.

\ouritem{double p2c\_float((prolog\_term) V)} \index{{\tt p2c\_float}}
    The prolog\_term argument must represent a floating point number,
    and {\tt p2c\_float} returns the value of that floating point
    number.

\ouritem{char *p2c\_string((prolog\_term) V)} \index{{\tt p2c\_string}}
    The prolog\_term argument must represent an atom, and {\tt
    p2c\_string} returns the name of that atom as a string. The
    pointer returned points to the actual atom name in \ourprolog 's
    space, and thus it must NOT be modified by the calling program.
\ouritem{char *p2c\_functor((prolog\_term) V)} \index{{\tt p2c\_functor}}
    The prolog\_term argument must represent a structured term (not a
    list).  {\tt p2c\_functor} returns the name of the main functor
    symbol of that term as a string. The pointer returned points to
    the actual functor name in \ourprolog 's space, and thus it must
    NOT be modified by the calling program.

\ouritem{int p2c\_arity((prolog\_term) V)} \index{{\tt p2c\_arity}}
    The prolog\_term argument must represent a structured term (not a
    list).  {\tt p2c\_arity} returns the arity of the main functor
    symbol of that term as an integer.
\end{description}
%%

The next batch of functions support conversion of data in the opposite
direction: from basic C types to the type {\tt prolog\_term}.  These {\tt
  c2p\_*} functions all return a boolean value TRUE if successful and FALSE
if unsuccessful.  The \ourprolog\ term argument must always contain an
\ourprolog\ variable, which will be bound to the indicated value as a side
effect of the function call.

\begin{description}
\ouritem{bool c2p\_int((int) N, (prolog\_term) V)} \index{{\tt c2p\_int}}
    {\tt c2p\_int} binds the prolog\_term V (which must be a variable)
    to the integer value N.

\ouritem{bool c2p\_float((double) F, (prolog\_term) V)} \index{{\tt c2p\_float}}
    {\tt c2p\_float} binds the prolog\_term V (which must be a variable)
    to the (double) float value F.

\ouritem{bool c2p\_string((char *) S, (prolog\_term) V)} \index{{\tt c2p\_string}}
    {\tt c2p\_string} binds the prolog\_term V (which must be a
    variable) to the atom whose name is the value of S, which must be
    of type char *.
\end{description}
%%

The following functions create Prolog data structures within a C
program. This is usually done in order to pass these structures back to
the Prolog side.
%%
\begin{description}
\ouritem{bool c2p\_functor((char *) S, (int) N, (prolog\_term) V)} \index{{\tt c2p\_functor}}
    {\tt c2p\_functor} binds the prolog\_term V (which must be a
    variable) to an open term whose main functor symbol is given by S
    (of type char *) and whose arity is N.  An open term is one with
    all arguments as new distinct variables.

\ouritem{bool c2p\_list((prolog\_term) V)} \index{{\tt c2p\_list}}
    {\tt c2p\_list} binds the prolog\_term V (which must be a variable)
    to an open list term, i.e., a list term with both car and cdr as
    new distinct variables. Note: to create an empty list use the function
    {\tt c2p\_nil} described below.

\ouritem{bool c2p\_nil((prolog\_term) V)} \index{{\tt c2p\_nil}}
    {\tt c2p\_nil} binds the prolog\_term V (which must be a
    variable) to the atom \verb|[]| (nil).
\ouritem{prolog\_term p2p\_new()} \index{{\tt p2p\_new}}
    Create a new Prolog variable. This is sometimes needed when you want to
    create a Prolog term on the C side and pass it to the Prolog side.
\end{description}
%%



To use the above functions, one must be able to get access to the
components of the structured Prolog terms.
This is done with the help of the following functions:

\begin{description}
\ouritem{prolog\_term p2p\_arg((prolog\_term) T, (int) A)} \index{{\tt p2p\_arg}}
    Argument T must be a prolog\_term that is a structured term (but
    not a list).  A is a positive integer (no larger than the arity of
    the term) that specifies an argument position of the term {\tt T}.  {\tt
    p2p\_arg} returns the A$^{th}$ subfield of the term {\tt T}.

\ouritem{prolog\_term p2p\_car((prolog\_term) T)} \index{{\tt p2p\_car}}
    Argument T must be a prolog\_term that is a list (not nil).  {\tt
    p2p\_car} returns the car (i.e., head of the list) of the
    term T.

\ouritem{prolog\_term p2p\_cdr((prolog\_term) T)} \index{{\tt p2p\_cdr}}
    Argument T must be a prolog\_term that is a list (not nil).  {\tt
    p2p\_car} returns the cdr (i.e., tail of the list) of the
    term T.
\end{description}
%%

It is very important to realize that these functions return the actual
Prolog term that is, say, the head of a list or the actual argument of a
structured term. Thus, assigning a value to such a prolog term also
modifies the head of the corresponding list or the relevant argument of the
structured term. It is precisely this feature that allows passing
structured terms and lists from the C side to the Prolog side.
For instance, 
%%
\begin{verbatim}
   prolog_term plist,        /* a Prolog list           */
               structure;    /* something like f(a,b,c) */
   prolog_term tail, arg;
   ..........
   tail = p2p_cdr(plist);         /* get the list tail  */
   arg  = p2p_arg(structure, 2);  /* get the second arg */

   /* Assume that the list tail was supposed to be a prolog variable */
   if (is_var(tail))
      c2p_nil(tail);  /* terminate the list */
   else {
      fprintf(stderr, "Something wrong with the list tail!");
      exit(1);
   }
   /* Assume that the argument was supposed to be a prolog variable */
   c2p_string("abcdef", arg);
\end{verbatim}
%%

In the above program fragment, we assume that both the tail of the list and
the second argument of the term were supposed to be bound to Prolog variables.
In case of the tail, we check if this is, indeed, the case. In case of the
argument, no checks are done; XSB will issue an error (which might be hard
to track down) if the second argument is not currently bound to a variable.

The last batch of functions is useful for passing data in and out of the
Prolog side of XSB. The first function is the only way to get a
{\tt prolog\_term} out of the Prolog side; the second function is
sometimes needed in order to pass complex structures from C into Prolog.
%%
\begin{description}
  \ouritem{prolog\_term reg\_term((int) R)} \index{{\tt reg\_term}}
    Argument R is an argument number of the Prolog predicate implemented by
    this C function (range 1 to 255). The function {\tt
    reg\_term} returns the prolog\_term in that predicate argument.
\ouritem{bool p2p\_unify(prolog\_term T1, prolog\_term T2)}
    \index{{\tt p2p\_unify}}
    Unify the two Prolog terms. This is useful when an argument of the
    Prolog predicate (implemented in C) is a structured term or a list,
    which acts both as input and output parameter.
\end{description}
%%

For instance, consider the Prolog call {\tt test(X, f(Z))},
which is implemented by a C function with the following fragment:
%%
\begin{verbatim}
    prolog_term newterm, newvar, z_var, arg2;
    .....
    /* process argument 1 */
    c2p_functor("func",1,reg_term(1));
    c2p_string("str",p2p_arg(reg_term(1),1));
    /* process argument 2 */
    arg2 = reg_term(2);
    z_var = p2p_arg(arg2, 1);  /* get the var Z */
    /* bind newterm to abc(V), where V is a new var */
    c2p_functor("abc", 1, newterm);
    newvar = p2p_arg(newterm, 1);
    newvar = p2p_new();
    ....
    /* return TRUE (success), if unify; FALSE (failure) otherwise */
    return p2p_unify(z_var, newterm);
\end{verbatim}
%%
On exit, the variable $X$ will be bound to the term {\tt func(str)}.
Processing argument 2 is more interesting. Here, argument 2 is used both
for input and output. If {\tt test} is called as above, then on exit $Z$
will be bound to {\tt abc(\_h123)}, where {\tt \_h123} is some new Prolog
variable. But if the call is {\tt test(X,f(1))} or {\tt test(X,f(Z,V))}
then this call will \emph{fail} (fail as in Prolog, {\it i.e.}, it is not
an error), because the term passed back, {\tt abc(\_h123)}, does not unify
with {\tt f(1)} or {\tt f(Z,V)}. This effect is achieved by the use of
{\tt p2p\_unify} above.

We conclude with two real examples of functions that pass complex data in
and out of the Prolog side of XSB. These functions are part of the Posix
regular expression matching package of XSB. The first function uses
argument 2 to accept a
list of complex prolog terms from the Prolog side and does the processing
on the C side. The second function does the opposite: it constructs
a list of complex Prolog terms on the C side and passes it over to the
Prolog side in argument 5.

%%
\begin{verbatim}
/* XSB string substitution entry point: replace substrings specified in Arg2
   with strings in Arg3.
   In: 
       Arg1: string
       Arg2: substring specification, a list [s(B1,E1),s(B2,E2),...]
       Arg3: list of replacement string
   Out:
       Arg4: new (output) string
   Always succeeds, unless error.
*/
bool do_regsubstitute__(void)
{
  /* Prolog args are first assigned to these, so we could examine the types
     of these objects to determine if we got strings or atoms. */
  prolog_term input_term, output_term;
  prolog_term subst_reg_term, subst_spec_list_term, subst_spec_list_term1;
  prolog_term subst_str_term=(prolog_term)0,
    subst_str_list_term, subst_str_list_term1;
  char *input_string=NULL;    /* string where matches are to be found */
  char *subst_string=NULL;
  prolog_term beg_term, end_term;
  int beg_offset=0, end_offset=0, input_len;
  int last_pos = 0; /* last scanned pos in input string */
  /* the output buffer is made large enough to include the input string and the
     substitution string. */
  char subst_buf[MAXBUFSIZE];
  char *output_ptr;
  int conversion_required=FALSE; /* from C string to Prolog char list */

  input_term = reg_term(1);  /* Arg1: string to find matches in */
  if (is_string(input_term)) /* check it */
    input_string = string_val(input_term);
  else if (is_list(input_term)) {
    input_string =
      p_charlist_to_c_string(input_term, input_buffer, sizeof(input_buffer),
                             "RE_SUBSTITUTE", "input string");
    conversion_required = TRUE;
  } else
    xsb_abort("RE_SUBSTITUTE: Arg 1 (the input string) must be an atom or a character list");

  input_len = strlen(input_string);

  /* arg 2: substring specification */
  subst_spec_list_term = reg_term(2);
  if (!is_list(subst_spec_list_term) && !is_nil(subst_spec_list_term))
    xsb_abort("RE_SUBSTITUTE: Arg 2 must be a list [s(B1,E1),s(B2,E2),...]");

  /* handle substitution string */
  subst_str_list_term = reg_term(3);
  if (! is_list(subst_str_list_term))
    xsb_abort("RE_SUBSTITUTE: Arg 3 must be a list of strings");

  output_term = reg_term(4);
  if (! is_var(output_term))
    xsb_abort("RE_SUBSTITUTE: Arg 4 (the output) must be an unbound variable");

  subst_spec_list_term1 = subst_spec_list_term;
  subst_str_list_term1 = subst_str_list_term;

  if (is_nil(subst_spec_list_term1)) {
    strncpy(output_buffer, input_string, sizeof(output_buffer));
    goto EXIT;
  }
  if (is_nil(subst_str_list_term1))
    xsb_abort("RE_SUBSTITUTE: Arg 3 must not be an empty list");

  /* initialize output buf */
  output_ptr = output_buffer;

  do {
    subst_reg_term = p2p_car(subst_spec_list_term1);
    subst_spec_list_term1 = p2p_cdr(subst_spec_list_term1);

    if (!is_nil(subst_str_list_term1)) {
      subst_str_term = p2p_car(subst_str_list_term1);
      subst_str_list_term1 = p2p_cdr(subst_str_list_term1);

      if (is_string(subst_str_term)) {
        subst_string = string_val(subst_str_term);
      } else if (is_list(subst_str_term)) {
        subst_string =
          p_charlist_to_c_string(subst_str_term, subst_buf, sizeof(subst_buf),
                                 "RE_SUBSTITUTE", "substitution string");
      } else 
        xsb_abort("RE_SUBSTITUTE: Arg 3 must be a list of strings");
    }

    beg_term = p2p_arg(subst_reg_term,1);
    end_term = p2p_arg(subst_reg_term,2);

    if (!is_int(beg_term) || !is_int(end_term))
      xsb_abort("RE_SUBSTITUTE: Non-integer in Arg 2");
    else{
      beg_offset = int_val(beg_term);
      end_offset = int_val(end_term);
    }
    /* -1 means end of string */
    if (end_offset < 0)
      end_offset = input_len;
    if ((end_offset < beg_offset) || (beg_offset < last_pos))
      xsb_abort("RE_SUBSTITUTE: Substitution regions in Arg 2 not sorted");

    /* do the actual replacement */
    strncpy(output_ptr, input_string + last_pos, beg_offset - last_pos);
    output_ptr = output_ptr + beg_offset - last_pos;
    if (sizeof(output_buffer)
        > (output_ptr - output_buffer + strlen(subst_string)))
      strcpy(output_ptr, subst_string);
    else
      xsb_abort("RE_SUBSTITUTE: Substitution result size %d > maximum %d",
                beg_offset + strlen(subst_string),
                sizeof(output_buffer));
    
    last_pos = end_offset;
    output_ptr = output_ptr + strlen(subst_string);

  } while (!is_nil(subst_spec_list_term1));

  if (sizeof(output_buffer) > (output_ptr-output_buffer+input_len-end_offset))
    strcat(output_ptr, input_string+end_offset);

 EXIT:
  /* get result out */
  if (conversion_required)
    c_string_to_p_charlist(output_buffer,output_term,"RE_SUBSTITUTE","Arg 4");
  else
    /* DO NOT intern. When atom table garbage collection is in place, then
       replace the instruction with this:
                  c2p_string(output_buffer, output_term);
       The reason for not interning is that in Web page
       manipulation it is often necessary to process the same string many
       times. This can cause atom table overflow. Not interning allws us to
       circumvent the problem.  */
    ctop_string(4, output_buffer);
  
  return(TRUE);
}


/* XSB regular expression matcher entry point
   In:
       Arg1: regexp
       Arg2: string
       Arg3: offset
       Arg4: ignorecase
   Out:
       Arg5: list of the form [match(bo0,eo0), match(bo1,eo1),...]
             where bo*,eo* specify the beginning and ending offsets of the
             matched substrings.
             All matched substrings are returned. Parenthesized expressions are
             ignored.
*/
bool do_bulkmatch__(void)
{
  prolog_term listHead, listTail;
  /* Prolog args are first assigned to these, so we could examine the types
     of these objects to determine if we got strings or atoms. */
  prolog_term regexp_term, input_term, offset_term;
  prolog_term output_term = p2p_new();
  char *regexp_ptr=NULL;      /* regular expression ptr               */
  char *input_string=NULL;    /* string where matches are to be found */
  int ignorecase=FALSE;
  int return_code, paren_number, offset;
  regmatch_t *match_array;
  int last_pos=0, input_len;
  char regexp_buffer[MAXBUFSIZE];

  if (first_call)
    initialize_regexp_tbl();

  regexp_term = reg_term(1);  /* Arg1: regexp */
  if (is_string(regexp_term)) /* check it */
    regexp_ptr = string_val(regexp_term);
  else if (is_list(regexp_term))
    regexp_ptr =
      p_charlist_to_c_string(regexp_term, regexp_buffer, sizeof(regexp_buffer),
                             "RE_MATCH", "regular expression");
  else
    xsb_abort("RE_MATCH: Arg 1 (the regular expression) must be an atom or a character list");

  input_term = reg_term(2);  /* Arg2: string to find matches in */
  if (is_string(input_term)) /* check it */
    input_string = string_val(input_term);
  else if (is_list(input_term)) {
    input_string =
      p_charlist_to_c_string(input_term, input_buffer, sizeof(input_buffer),
                             "RE_MATCH", "input string");
  } else
    xsb_abort("RE_MATCH: Arg 2 (the input string) must be an atom or a character list");

  input_len = strlen(input_string);
  
  offset_term = reg_term(3); /* arg3: offset within the string */
  if (! is_int(offset_term))
    xsb_abort("RE_MATCH: Arg 3 (the offset) must be an integer");
  offset = int_val(offset_term);
  if (offset < 0 || offset > input_len)
    xsb_abort("RE_MATCH: Arg 3 (=%d) must be between 0 and %d", input_len);

  /* If arg 4 is bound to anything, then consider this as ignore case flag */
  if (! is_var(reg_term(4)))
    ignorecase = TRUE;

  last_pos = offset;
  /* returned result */
  listTail = output_term;
  while (last_pos < input_len) {
    c2p_list(listTail); /* make it into a list */
    listHead = p2p_car(listTail); /* get head of the list */

    return_code = xsb_re_match(regexp_ptr, input_string+last_pos, ignorecase,
                               &match_array, &paren_number);
    /* exit on no match */
    if (! return_code) break;

    /* bind i-th match to listHead as match(beg,end) */
    c2p_functor("match", 2, listHead);
    c2p_int(match_array[0].rm_so+last_pos, p2p_arg(listHead,1));
    c2p_int(match_array[0].rm_eo+last_pos, p2p_arg(listHead,2));

    listTail = p2p_cdr(listTail);
    last_pos = match_array[0].rm_eo+last_pos;
  }
  c2p_nil(listTail); /* bind tail to nil */
  return p2p_unify(output_term, reg_term(5));
}
\end{verbatim}
%%


%%% Local Variables: 
%%% mode: latex
%%% TeX-master: "manual2"
%%% End: 
