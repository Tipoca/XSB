\chapter{System Description} \label{chap:system}
\label{system}

% TLS: changing object file extensions.
% TLS: discuss relative paths.

Throughout this chapter, we use \verb'$XSB_DIR' to refer to the
directory in which XSB was installed.

\section{Entering and Exiting XSB from the Command Line}
%=================================

After the system has been installed, the emulator's executable code appears 
in the file:
\begin{verbatim}
                     $XSB_DIR/bin/xsb
\end{verbatim}
If, after being built, XSB is later installed  at a central location,
\verb'$SHARED_XSB', the emulators executable code appears in
\begin{verbatim}
                     $SHARED_XSB/bin/xsb
\end{verbatim}
Either of these commands invokes XSB's top-level interpreter, which is
the most common way of using XSB.

XSB can also directly execute object code files from the command line
interface.  Suppose you have a top-level routine {\tt go} in a file
{\tt foo.P} that you would like to run from the UNIX or Windows
command line.  As long as {\tt foo.P} contains a directive, e.g. {\tt
  :- go.}, and {\tt foo.P} has been compiled to an object file ({\tt
  foo.xwam}), then
\begin{verbatim}
                     $XSB_DIR/bin/xsb -B foo.xwam
\end{verbatim}
will execute {\tt go} (and any other directives), loading the
appropriate files as needed~\footnote{In XSB, all extensions except
  '.pl' --- (default '.P', '.H', '.xwam', '.D' (output by mode
  inferencing), and '.A' (assembly dump) --- are defined in C and
  Prolog code using macros in {\tt \$XSB\_DIR/emu/extensions\_xsb.h}
  and can be changed by a user if desired.  Of course, such a step
  should not be taken lightly, as it can cause severe compatibility
  problems.}.
%
In fact the command
\verb'$XSB_DIR/bin/xsb' is equivalent to the command:
\begin{verbatim}
             $XSB_DIR/bin/xsb -B $XSB_DIR/syslib/loader.xwam
\end{verbatim}
%%$
There is one other way to execute XSB from a command line.  Using the
{\tt -e} command-line option any goal can be can be executed, up to
1024 characters.  For instance 
\begin{verbatim}
             $XSB_DIR/bin/xsb -e "writeln('hello world'),halt."
\end{verbatim}
%%$
writes ``hello world'' and exits XSB.  Within the 1024 character
limit, any query or command can be executed, including consulting
files, so this method is actually quite general~footnote{Various
  options can suppress XSB's startup and end messages, as discussed
  below.}.

There are several ways to exit XSB.  A user may issue the
command \verb'halt.' or \verb'end_of_file.', or simply type
\verb'CTRL-d' at the XSB prompt.  To interrupt XSB
while it is executing a query, strike \verb'CTRL-c'.

\section{The System and its Directories}
%=======================================
When installed, the XSB system resides in a single directory that
contains several subdirectories.  For completeness, we review the
information in all subdirectories.  Normally, only the documentation
and files in the Prolog subdirectories, particularly {\tt examples},
{\tt lib}, and {\tt packages} will be of interest to users.
\begin{enumerate}
\item {\tt bin} contains scripts that call XSB executables
for various configurations.
%
\item {\tt build} contains XSB configuration scripts.  You may
already be familiar with the {\tt build} directory, which is used to
build XSB.
%
\item {\tt config} contains executables and other files specific to
particular configurations.
%
\item {\tt docs} contains the user manuals and other documentation,
including the technical documentation manual for developers.  
%
\item {\tt emu} contains the C source code for the XSB emulator, for
I/O and for various interfaces.
%
\item {\tt etc} contains miscellaneous files used by XSB.
%
\item {\tt examples} contains some examples for Prolog, tabling,
HiLog and various interfaces.
%
\item {\tt cmplib} contains Prolog source and object code for the
compiler. 
%
\item {\tt gpp} contains a copy of the Gnu pre-processor used to
preprocess Prolog files.
%
\item {\tt lib} contains Prolog source and object code for extended
libraries. 
%
\item {\tt packages} The directory {\tt packages} contains the various
applications, such as FLORA, the XMC model checker and many others.
These applications are written in XSB and can be quite useful, but are
not part of the XSB system per se.
%
\item {\tt Prolog\_includes} contains include files for the Prolog
libraries, which are preprocessed using GPP.
%
\item {\tt syslib} contains Prolog source and object code for core XSB
libraries. 
\end{enumerate}

\noindent
All Prolog source programs are written in XSB, and all object (byte
code) files contain SLG-WAM instructions that can be executed by the
emulator.  These byte-coded instructions are machine-independent, so
usually no installation procedure is needed for the byte code files.

If you are distributing an application based on XSB and need to cut
down space, the {\tt packages}, {\tt examples} and {\tt docs}
directories are not usually needed (unless of course you are using one
of the packages in your application).  {\tt lib} may not be needed,
(most core system files are in syslib) nor are Prolog source files
necessary.  Unless your application needs to rebuild XSB, the {\tt
  emu} and {\tt build} directories do not need to be distributed.

\section{File Names and XSB}  \label{sec:filenames}
%
\index{base file name}
\index{file designator}
Three files are associated with Prolog source code in
XSB~\footnote{Other types of files may be associated with foreign code
--- see Volume 2.}.
\begin{itemize}
\item A single {\it source} file, whose name is the {\em base file
  name} plus the an extension suffix {\tt .P} or if a {\tt .P} file
  cannot be found , {\tt .pl}.
\item An {\it object (byte-code)} file, whose name consists of the
  base file name plus the suffix {\tt .xwam}.
\item An optional {\it header} file, whose name is the base file name
  plus the suffix ``{\tt .H}''.  When used, the header file normally
  contains file-level declarations and directives while the source
  file usually contains the actual definitions of the predicates
  defined in that module.  However, such information can be
  equivalently put into the {\tt .P} (or {\tt .pl} file.
\end{itemize}
%
Most of the XSB system predicates, such as {\tt consult/[1,2]}, {\tt
compile/[1,2]} and others are somewhat flexible in how they allow
file designation.  A file may be designated by a {\em file designator}
which can be a base file name, a source file name; or the relative or
absolute paths to a base or source file name.  Unfortunately, the
exact semantics of a file designator differs among system predicates
in \version, as well as among platforms.  

\section{The Module System of XSB} \label{Modules}
%========================================================
XSB has been designed as a module-oriented Prolog system.  Modules
provide a small step towards {\em logic programming ``in the large''}
that facilitates the construction of large programs or projects from
components that are developed, compiled and tested separately.  Also
module systems support the {\em principle of information hiding} and
can provide a basis for {\em data abstraction}.  The module system of
XSB is {\em file based} -- one module per file -- and {\em flat} --
modules cannot be nested.  In addition, XSB's module system is to some
extent {\em atom-based}, where any symbol in a module can be imported,
exported or be a local symbol, as opposed to the predicate-based ones
where this can be done only for predicate symbols~\footnote{Operator
  symbols can be exported as any other symbols, but their precedence
  must be redeclared in the importing module.}.  As we will discuss,
this leads to certain differences of XSB's module system from those of
some other Prologs.

\paragraph*{Module Syntax}
\index{modules!name} 
%
By default, files are not treated as modules.  In order for a file to
be treated as a module, it must contain one or more {\tt export/1} or
{\tt module/1} declarations, which specify that a set of symbols
appearing in that module is visible and therefore can be used by any
other module.  In XSB, the {\em module name} is equal to the base file
name in which the module is defined.  Any file (either module or not)
may also contain {\tt import/1} or {\tt use\_module/2} declarations,
which allow symbols defined in and exported by other modules to be
used in the current module.  In addition, a module can also contain
{\it local declarations}, which specify that a set of symbols is {\it
  visible by this module only}, and therefore cannot be accessed by
any other module.  Module declarations can appear anywhere in the
source or header files and have the following forms:

\index{modules!XSB syntax} \index{\texttt{export/1}}
\index{\texttt{import/2}} \index{\texttt{local/1}} 
\index{modules!compatability syntax}
\index{\texttt{module/2}}  \index{\texttt{use\_module/2}}    

\demo{:- export $sym_1$, \ldots, $sym_l$. }

\demo{:- import $sym_1$, \ldots, $sym_n$ from $module$. }

\demo{:- local $sym_1$, \ldots, $sym_m$. }

\noindent
where $sym_i$ has the form $functor/arity$, and $module$ is a module
name.

In XSB, the declaration

\demo{:- module($filename$,[$sym_1$, \ldots, $sym_l$.]).}

\noindent
can be seen as syntactic sugar for 

\demo{:- export $sym_1$, \ldots, $sym_l$. }

\noindent
as long as the $filename$ is the same as the name of the file in which
it was contained.  Similarly,

\demo{:- use\_module($module$,[$sym_1$, \ldots, $sym_l$.]).}

\noindent
is treated as semantically equivalent to 

\demo{:- import $sym_1$, \ldots, $sym_n$ from $module$. }

\noindent
Accordingly, {\tt use\_module/2} and {\tt module/1} can be used
interchangibly with {\tt import/2} and {\tt export/1}.  However the
declaration

\demo{:- use\_module($module$).}

\noindent
which is often used in other Prolog systems, is {\em not} equivalent
to an XSB import statement, as each XSB import statement must
explicitly declare a list of predicates that are used from each
module.  Such a declaration will raise a compilation error.

For modules, the base file name is stored in its byte code file, so
that renaming a byte-code file for a mule may cause problems, as the
renaming will not affect the information within the byte-code file.
However, byte code files generated for non-modules can be safely
renamed.

\paragraph*{Module Semantics} 
%
In XSB's atom-based module system, the name of each predicate and
function symbol {\tt p/n} is identified as if it were prefixed with
its module name (i.e. base file name).  Hence the occurrence of {\tt
  p/n} in two different modules, {\tt m1} and {\tt m2} are distinct
symbols that can be denoted as {\tt m1:p/n} and {\tt m2:p/n}.  

Normally, only exported symbols can be imported; if a non-exported
symbol {\tt p/2} is imported from a module {\tt m1} by module {\tt m2}
an environment conflict warning will be issued as soon as {\tt m1} and
{\tt m2} are loaded in the same session -- i.e. the conflict is
detected at run-time.  When a non-module file is loaded, its
predicates and symbols are loaded into the module {\tt usermod}, which
is the working module of the XSB command-line interpreter and
C-calling XSB interface.  Dynamically asserted code is also loaded
into {\tt usermod} by default.
%
Currently the following set of rules is used to determine the module
prefix of a symbol:
\begin{itemize}
%
\item A predicate symbol {\tt p/n} is {\em defined} in a module {\tt
  m} if {\tt m} contains a clause with head {\tt p/n} or a dynamic
  declaration for {\tt p/n}.  Any predicate symbol {\tt p/n} defined
  in a module {\tt m}, whether exported or not, can be called by
  prepending the module prefix using the {\tt :/2} functor, e.g. {\tt
    m:p(A,...)}.  For brevity, we call this an {\em explicit module
    call} to {\tt p/n}.  The following example illustrates these
  principles.
%
\begin{center}
\begin{tabular}{cc}\hline \hline
\multicolumn{2}{c}{\rule{0ex}{2.5ex}\textbf{Exported and Non-Exported Predicates}} \\ \hline \hline
\rule[-2ex]{0ex}{5ex} \texttt{m1} &  \\
\begin{minipage}{18.5em}
\begin{verbatim}
:- export p/2.

exported(a,b).

local(c,d).

\end{verbatim}
\end{minipage}
&
\begin{minipage}{14.5em}
\begin{verbatim}
| ?- exported(X,Y).

X = a
Y = b

yes
| ?- local(X,Y).
/* Existence Error */
| ?- m1:local(X,Y).

X = c
Y = d

yes

\end{verbatim}
\end{minipage}
%\vspace*{-2ex} 
\\ \hline \hline
\end{tabular}
\end{center}
%
\item Every predicate symbol defined in a module is assumed by default
  to be {\em local} to a module unless it is declared otherwise by an
  export or import declaration.  Symbols that are local to a given
  module are not visible to other modules except through explicit
  module calls.  The following example shows how different
  declarations for dynamic predicates within a module may be global
  ({\tt usermod}) or local.  Calls to statically defined predicates
  behave similarly.
%
\begin{center}
\begin{tabular}{cc}\hline \hline
\multicolumn{2}{c}{\rule{0ex}{2.5ex}\textbf{Visibility of Dynamic Predicates}} \\ \hline \hline
\rule[-2ex]{0ex}{5ex} \texttt{m1} & \texttt{m2} \\
\begin{minipage}{20.5em}
\begin{verbatim}
:- export p1/2, p2/2, p3/2, p4/2.
:- dynamic d1/2.
:- import d2/2 from usermod. 

p10:- a1.

p1(X,Y):- assert(d1(X,Y)).
p2(X,Y):- d1(X,Y).

p2(X,Y):- assert(d2(X,Y)).
p3(X,Y):- d2(X,Y).
\end{verbatim}
\end{minipage}
&
\begin{minipage}{14.5em}
\begin{verbatim}
| ?- [m1].
[m1 loaded]

yes
| ?- p1(a,b).

yes
| ?- p2(X,Y).

X = a
Y = b

yes
| ?- d1(X,Y).
/* Existence Error */
| ?- m1:d1(X,Y).

X = a
Y = b

yes
| ?- p3(1,2).

yes
| ?- p4(X,Y).

X = 1
Y = 2

yes
| ?- d2(X,Y).

X = 1
Y = 2

yes
\end{verbatim}
\end{minipage}
%\vspace*{-2ex} 
\\ \hline \hline
\end{tabular}
\end{center}
%
\item Functors that occur as literals in the bodies of clauses, are
  treated as predicate symbols.  
\bi
\item {\em Standard} predicates are taken to be a part of {\tt
  usermod}, and are implicitly imported into user-defined modules.
  Standard predicates include ISO predicates along with many other XSB
  predicates for tabling, indexing and other functions.  The current
  listing of standard predicates can be found in the index of this
  manual under {\em Standard predicates}.
%
\item Other predicates are taken to be local to the module in which
  they occur.
\ei
\item Functors that do not occur as literals in the body of clauses in
  a module are taken to be structure symbols.  These symbols are
  assumed to be global and do not require an explicit module call to
  be used, unless declared otherwise through a {\tt local/1}
  declaration.  In addition, terms that are dynamically created by
  standard predicates such as {\tt read/1, functor/3, '=..'/2}, etc)
  are taken to be structure symbols and are contained in {\tt
    usermod}.  
%
%\item If a symbol is imported from another module (via an explicit
%  import declaration), the module prefix of the symbol is the module
%  it is imported from; any other symbol takes the module where the
%  symbol occurs as its module prefix.
%
\item All atoms are assumed to be global and do not require an
  explicit module call to be used.  This can occasionally lead to
  unexpected results if a token is used both as an atom and a 0-ary
  function symbol.  In the following table, the query {\tt ?- p10}
  will call {\tt a1/0}, while {\tt ?- p11.} will throw an existence
  error.

\begin{center}
\begin{tabular}{cc}\hline \hline
\multicolumn{2}{c}{\rule{0ex}{2.5ex}\textbf{Atoms and 0-ary Predicates}} \\ \hline \hline
\rule[-2ex]{0ex}{5ex} \texttt{m1} & \texttt{m2} \\
\begin{minipage}{18.5em}
\begin{verbatim}
:- export p10/0, p11/0.
:- import a1/0 from m2.

p10:- a1.

p11:- atom_chars(A1,[a,'1']),
      call(A1).

\end{verbatim}
\end{minipage}
&
\begin{minipage}{14.5em}
\begin{verbatim}
:- export a1/0.




a1:- writeln(found_a1).

\end{verbatim}
\end{minipage}
%\vspace*{-2ex} 
\\ \hline \hline
\end{tabular}
\end{center}
%
\end{itemize}
%
For clarity, we state a few consequences of these rules.
\begin{itemize}
%\item If users want to use a symbol (e.g. {\tt p/2}) from another
%  module, {\tt mod}, they must explicitly import it otherwise the two
%  symbols are different even if they are of the same $functor/arity$
%  form.  Alternately, a call can be prepended with a module name, as
%  {\tt mod:p(A,B)}.
%
\item In \version{}, a module {\em cannot} export predicate symbols
  that are imported from other modules.  This happens because an {\tt
    import} declaration is considered a request for permission to use
  a symbol from a module where its definition and an {\tt export}
  declaration appear. 
%
\item The implicit module for a particular symbol appearing in a
  module must be uniquely determined.  As a consequence, a symbol of a
  specific $functor/arity$ {\em cannot} be declared as both exported
  and local, or (as just discussed) both exported and imported from
  another module, or declared to be imported from more than one
  module, etc.  These types of environment conflicts are detected at
  compile-time and abort the compilation.
%
% Perhaps discuss symbols, if that isn't too confusing.
\item If a module {\tt m1} imports a predicate {\tt p/n} from a module
  {\tt m2}, but {\tt m2} does not export {\tt p/n}, nothing is
  detected at the time of compilation.  As discussed above, if {\tt
    p/n} is defined in {\tt m2} a runtime warning about an environment
  conflict will be issued.  However, if {\tt p/n} is not defined in
  {\tt m2}, a runtime existence error will be thrown~\footnote{This
    behavior can be altered through the Prolog flag {\tt unknown}.}.
%
\item Only one definition of a symbol {\tt p/n} can appear in a
  module, without being explicitly associated with a module using the
  {\tt :/2} functor.  Accordingly only one default definition of {\tt
    p/n} can be loaded into the interpreter's module ({\tt usermod}).
  An attempt to load a module that redefines {\tt p/n} results in a
  warning to the user and the newly loaded symbol {\em redefines} the
  definition of the previously loaded one.
\end{itemize}
% 
\paragraph*{Usage inference and the module system}
The import and export statements of a module $M$ are used by the
compiler for inferring usage of predicates.  At compilation time, if a
predicate $P/N$ occurs as callable in the body of a clause defined in
$M$, but $P$ is neither defined in $M$ nor imported into $M$ from some
other module, a warning is issued that $P/N$ is undefined.  Here
``occurs as callable'' means that $P/N$ is found as a literal in the
body of a clause, or within a system meta-predicate, such as {\tt
assert/1}, {\tt findall/3}, etc.  Currently, occurrences of a term
inside user-defined meta-predicates are not considered as callable by
XSB's usage inference algorithm.  Alternatively, if $P/N$ is defined in
$M$, it is {\em used} if $P/N$ is exported by $M$, or if $P/N$ occurs
as callable in a clause for a predicate that is used in $M$.  The
compiler issues warnings about all unused predicates in a module.  On
the other hand, since all modules are compiled separately, the usage
inference algorithm has no way of checking whether a predicate
imported from a given module is actually exported by that module.

\index{xsbdoc}
\index{\texttt{document\_export/1}}
\index{\texttt{document\_import/1}}
Usage inference can be highly useful during code development for
ensuring that all predicates are defined within a set of files, for
eliminating dead code, etc.  In addition, import and export
declarations are used by the {\tt xsbdoc} documentation system to
generate manuals for code~\footnote{Further information on {\tt
xsbdoc} can be found in {\tt \$XSB\_DIR/packages/xsbdoc}.}.  For these
reasons, it is sometimes the case that usage inference is desired even
in situations where a given file is not ready to be made into a
module, or it is not appropriate for the file to be a module for some
other reason.  In such a case the directives {\tt document\_export/1}
and {\tt document\_import/1} can be used, and have the same syntax as 
{\tt export/1} and {\tt import/1}, respectively.  These directives
affect only usage inference and {\tt xsbdoc}.  A file is treated as a
module if and only if it includes an {\tt export/1} statement, and
only {\tt import/1} statements affect dynamic loading and name
resolution for predicates.

\section{Standard Predicates in XSB} \label{sec:standard}
%========================================================

Whenever XSB is invoked, a large set of {\em standard} predicates are
defined and can be called from the interpreter or other
interface~\footnote{Such predicates are sometimes called ``built-ins''
  in other Prologs.}.  These predicates include the various ISO
predicates~\cite{ISO-Prolog}, along with predicates for tabling, I/O,
for interaction with the operating system, for HiLog, and for much
other functionality.  Standard predicates are listed in this manual
under the index heading {\em Standard predicates} and at an
implementation level are declared in the file {\tt
  \$XSB\_DIR/syslib/std\_xsb.P}.  If a user wishes to redefine a
standard predicate, she has several choices.  First, the appropriate
fact in {\tt \$XSB\_DIR/syslib/std\_xsb.P} should be commented out.
Once this is done, a user may define the predicate as any other user
predicate.  Alternately, the compiler option {\tt allow\_redefinition}
can be used to allow the compiler to redefine a standard predicate
(Section~\ref{sec:CompilerOptions}).  If a user wants to make a new
definition or new predicate standard, the safest course is to put the
predicate into a module in the {\tt lib} directory, and add or modify
an associated fact in {\tt \$XSB\_DIR/syslib/std\_xsb.P}.

%========================================================
\section{The Dynamic Loader and its Search Path} \label{LibPath}
\index{load search path}

XSB differs from some other Prolog system in its ability to {\tt
dynamically} load modules.  In XSB, the loading of user modules Prolog
libraries (such as the XSB compiler) is delayed until predicates in
them are actually needed, saving program space for large Prolog
applications.  Dynamic loading is done by default, unlike other
systems where it is not the default for non-system libraries.

When a predicate imported from another module (see
Section~\ref{Modules}) is called during execution, the dynamic loader
is invoked automatically if the module is not yet loaded into the
system, The default action of the dynamic loader is to search for the
byte code file of the module first in the system library directories
(in the order {\tt lib, syslib}, and then {\tt cmplib}), and finally
in the current working directory.  If the module is found in one of
these directories, then it will be loaded ({\em on a first-found
basis}). Otherwise, an error message will be displayed on the current
error stream reporting that the module was not found.  Because system
modules are dynamically loaded, the time it takes to compile a file is
slightly longer the first time the compiler is invoked in a session
than for subsequent compilations.


\subsection{Changing the Default Search Path and the Packaging System}
%=========================================================

The default search path of the dynamic loader can easily be changed by
having a file named {\verb|.xsb/xsbrc.P|} in the user's home
directory.  User-supplied library directories are searched by the
dynamic loader {\em before} searching the default library directories.
\index{\texttt{library\_directory/1}}  The {\verb|.xsb/xsbrc.P|} file,
which is automatically consulted by the XSB interpreter, might look
like the following:
\begin{verbatim}
             :- assert(library_directory('./')).
             :- assert(library_directory('~/')).
             :- assert(library_directory('/usr/lib/sbprolog')).
\end{verbatim}
After loading the module of the above example, the current working
directory is searched first, then the user's home directory,, then
{\tt "/usr/lib/sbprolog/"}, and finally XSB's system library
directories ({\tt lib, syslib}, and {\tt cmplib}).  XSB also uses {\tt
library\_directory/1} for internal purposes.  For instance, before the
user's {\verb|.xsb/xsbrc.P|} is consulted, XSB puts the {\tt packages}
directory and the directory \verb|.xsb/config/$CONFIGURATION| on the
library search path.  The directory
\verb'.xsb/config/$CONFIGURATION' is used to store user libraries that
are machine or OS dependent. (\verb'$CONFIGURATION' for a machine is
something that looks like {\tt sparc-sun-solaris2.6} or {\tt
pc-linux-gnu}, and is selected by XSB automatically at run time).  

Note that the file {\verb|.xsb/xsbrc.P|} is not limited to setting the
library search path.  In fact, arbitrary Prolog code can go there.

We emphasize that in the presence of a {\verb|.xsb/xsbrc.P|} file {\em
it is the user's responsibility to avoid module name clashes with
modules in XSB's system library directories}.  Such name clashes can
cause unexpected behavior as system code may try to load a user's
predicates.  The list of module names in XSB's system library
directories can be found by looking through the directories {\tt
\$XSB\_DIR/\{syslib,cmplib,lib\}}.

\index{packages}
Apart from the user libraries, XSB now has a simple packaging system.
A {\em package\/} is an application consisting of one or more files that
are organized in a subdirectory of one of the XSB system or user libraries.
The system directory \verb|$XSB_DIR/packages| has several examples
%%$
of such packages, many of which are documented in Volume 2 of this
manual, or contain their own manuals.  Packages are convenient
as a means of organizing large XSB applications, and for simplifying
user interaction with such applications.  User-level packaging is
implemented through the predicate
%%
\begin{verbatim}
     bootstrap_userpackage(+LibraryDir, +PackageDir, +PackageName).
\end{verbatim}
%%
\index{\texttt{bootstrap\_userpackage/3}}
which must be imported from the {\tt packaging} module. 

To illustrate, suppose you wanted to create a package, {\tt foobar}, inside
your own library, {\tt my\_lib}. Here is a sequence of steps you can
follow:  
%%
\begin{enumerate}
\item Make sure that {\tt my\_lib}\ is on the library search path by putting
  an appropriate assert statement in your {\tt xsbrc.P}.
\item Make a subdirectory \verb|~/my_lib/foobar| and organize all the
  package files there. Designate one file, say, {\tt foo.P}, as the
  entry point, {\it i.e.}, the application file that must be loaded first.
\item Create the interface program \verb|~/my_lib/foobar.P| with the
  following content:
    %%
    \begin{verbatim}
           :- bootstrap_userpackage('~/my_lib', 'foobar', foobar), [foo].
    \end{verbatim}
    %%
  The interface program and the package directory do not need to have the
  same name, but it is convenient to follow the above naming schema.
\item Now, if you need to invoke the {\tt foobar} application, you can
  simply type \verb|[foobar].| at the XSB prompt. This is because both and
  \verb|~/my_lib/foobar| have already been automatically added to the
  library search path.
\item If your application files export many predicates, you can simplify
  the use of your package by having \verb|~/my_lib/foobar.P| import all
  these predicates, renaming them, and then exporting them. This provides a
  uniform interface to the {\tt foobar} module, since all the package
  predicates are can now be imported from just one module, {\tt foobar}.
\end{enumerate}
%%
In addition to adding the appropriate directory to the library search
path, the \verb|bootstrap_userpackage/3| predicate also adds
information to the predicate \verb|package_configuration/3|, so that
other applications could query the information about loaded packages.

Packages can also be unloaded using the predicate
\verb|unload_package/1|. For instance, 

%%
\begin{verbatim}
       :- unload_package(foobar).  
\end{verbatim}
%%
removes the directory \verb|~/my_lib/foobar| from the library search path
and deletes the associated information from \verb|package_configuration/3|.
\index{\texttt{unload\_package/1}}
\index{\texttt{package\_configuration/2}}

Finally, if you have developed and tested a package that you think is
generally useful and you would like to distribute it with XSB, please
contact {\tt xsb-development@sourceforge.net}.


\subsection{Dynamically loading predicates in the interpreter}
%=============================================================
Modules are usually loaded into an environment when they are consulted
(see Section~\ref{Consulting}).  Specific predicates from a module can
also be imported into the run-time environment through the standard
predicate {\tt import PredList from Module}\index{\texttt{import/1}}.
Here, {\tt PredList} can either be a Prolog list or a comma list.
(The {\tt import/1} can also be used as a directive in a source module
(see Section~\ref{Modules}). \index{standard predicates}

We provide a sample session for compiling, dynamically loading, and
querying a user-defined module named {\tt quick\_sort}.  For this
example we assume that {\tt quick\_sort.P} is a file in the current
working directory, and contains the definitions of the predicates {\tt
concat/3} and {\tt qsort/2}, both of which are exported.

{\footnotesize
\begin{verbatim}
             | ?- compile(quick_sort).
             [Compiling ./quick_sort]
             [quick_sort compiled, cpu time used: 1.439 seconds]

             yes
             | ?- import concat/3, qsort/2 from quick_sort. 

             yes
             | ?- concat([1,3], [2], L), qsort(L, S).

             L = [1,3,2]
             S = [1,2,3]

             yes.
\end{verbatim}
}

The standard predicate {\tt import/1} does not load the module 
containing the imported predicates, but simply informs the system 
where it can find the definition of the predicate when (and if) the
predicate is called.


\section{Command Line Arguments} \label{sec:EmuOptions}
%========================================================
\index{emulator!command line options}
\index{options!command line arguments}
\index{stacks!default sizes}
\index{stacks!expanding}
%========================================================

There are several command line options for the emulator. The general 
synopsis obtained by the command {\tt \$XSB\_DIR/bin/xsb --help} is: 
{\small 
\begin{verbatim}
xsb [flags] [-l] [-i]
xsb [flags] -n
xsb [flags] module
xsb [flags] -B boot_module [-D cmd_loop_driver] [-t] [-e goal]
xsb [flags] -B module_to_disassemble -d
xsb -[h | v]
xsb --help | --version | --nobanner | --quietload | --noprompt

memory management flags:
    -c tcpsize [unit] | -m glsize [unit] | -o complsize [unit] 
         | -u pdlsize [unit] | -r | -g gc_type
      unit: k/K memory in kilobytes; m/M in megabytes; g/G in gigabytes
    -c tcpsize | -m glsize | -o complsize | -u pdlsize | -r | -g gc_type
miscellaneous flags:
    -s | -S | -T

module:
    Module to execute after XSB starts up.
    Module should have no suffixes, no directory part, and
    the file module.xwam must be on the library search path.
boot_module:
    This is a developer's option.
    The -B flags tells XSB which bootstrapping module to use instead
    of the standard loader.  The loader must be specified using its
    full pathname, and boot_module.xwam must exist.
module_to_disassemble:
    This is a developer's option.
    The -d flag tells XSB to act as a disassembler.
    The -B flag specifies the module to disassemble.
cmd_loop_driver:
    The top-level command loop driver to be used instead of the
    standard one.  Usually needed when XSB is run as a server.
 
                 -i : bring up the XSB interpreter
            -e goal : evaluate goal when XSB starts up
                 -l : the interpreter prints unbound variables using letters
                 -n : used when calling XSB from C
                 -B : specify the boot module to use in lieu of the standard loader
                 -D : Sets top-level command loop driver to replace the default
                 -t : trace execution at the SLG-WAM instruction level
                       (for this to work, build XSB with the --debug option)
                 -d : disassemble the loader and exit
         -c N [unit]: allocate N units (def. KB) for the trail/choice-point stack
         -m N [unit]: allocate N units (def. KB) for the local/global stack
         -o N [unit]: allocate N units (def. KB) for the SLG completion stack
         -u N [unit]: allocate N units (def. KB) for the SLG unification stack
                 -r : turn off automatic stack expansion
         -g gc_type : choose heap garbage collection ("none", "sliding", 
                                                      "indirection", or "copying") 
                 -p : enable Prolog profiling thru use of profile_call/1
                 -s : maintain detailed statistical information
                 -S : set default tabling method to subsumption-based
                 -T : print a trace of each called predicate
    --max_threads N : maintain information for up to N threads
      -v, --version : print the version and configuration information about XSB
         -h, --help : print this help message
         --nobanner : don't show the XSB banner on startup
        --quietload : don't show the `module loaded' messages
         --noprompt : don't show prompt (for non-interactive use)
--shared_predicates : make predicates thread-shared by default

\end{verbatim}
}
The order in which these options appear makes no difference.

\begin{description}
\item[{\tt -i}] Brings up the XSB interpreter.  This is the normal
  use, and the option is kept only for backward compatability.
\item[{\tt -e goal}] Pass {\tt goal}  to XSB at startup. This goal is evaluated
    right before the first prompt is issued. For instance, 
    \verb'xsb -e "write('Hello!'), nl."'
    will print a heart-warming message when XSB starts up.
\item[{\tt -l}] Forces the interpreter to print unbound variables as
	letters, as opposed to the default setting which prints
	variables as memory locations prefixed with an underscore.
	For example, starting XSB's interpreter with this option will
	print the following:
        \begin{verbatim}
                  | ?- Y = X, Z = 3, W = foo(X,Z).

                  Y = A
                  X = A
                  Z = 3
                  W = foo(A,3)
	\end{verbatim}
	as opposed to something like the following:
	\begin{verbatim}
                  | ?- Y = X, Z = 3, W = foo(X,Z).

                  Y = _10073976
                  X = _10073976
                  Z = 3
                  W = foo(_10073976,3);
	\end{verbatim}
\item[{\tt -n}] used in conjunction with the {\tt -i} option, to
    indicate that the usual read-eval-print top-loop is not to be
    entered, but instead will interface to a calling C program.  See
    the chapter {\it Calling XSB from C} in Volume 2  for details.
\item[{\tt -d}] Produces a disassembled dump of {\tt byte\_code\_file} to 
    {\tt stdout} and exits.
\item[{\tt -c} {\em size [units]}] Allocates {\em initial size\/} units of
  space to the trail/choice-point stack area.  The trail stack grows
  upward from the bottom of the region, and the choice point stack
  grows downward from the top of the region.  If units is not provided
  or is {\tt k} or {\tt K}, the size is allocated in kilobytes; if
  {\tt m} or {\tt M} in megabytes; and if {\tt g} or {\tt G} in
  gigabytes.  Because this region is expanded automatically from
  Version 1.6.0 onward, this option is rarely needed.  If this option
  is not speficied the default initial size is 768 KBytes.
%
\item[{\tt -m} {\em size [units]}] Allocates {\em initial size\/}
  units of space to the local/global stack area.  The global stack
  grows upward from the bottom of the region, and the local stack
  grows downward from the top of the region.  If units is not provided
  or is {\tt k} or {\tt K}, the size is allocated in kilobytes; if
  {\tt m} or {\tt M} in megabytes; and if {\tt g} or {\tt G} in
  gigabytes.  Because this region is expanded automatically from
  Version 1.6.0 onward, this option is rarely needed.  If this option
  is not specified the default initial size is 768 KBytes.
\item[{\tt -o} {\em size [units]}] Allocates {\em initial size\/} units of
  space to the completion stack area.  If units is not provided or is
  {\tt k} or {\tt K}, the size is allocated in kilobytes; if {\tt m}
  or {\tt M} in megabytes; and if {\tt g} or {\tt G} in gigabytes.
  Because this region is expanded automatically from Version 1.6.0
  onward, this option is rarely needed. If this option is not
  specified the default initial size is 768 KBytes.
\item[{\tt -u} {\em size [units]}] Allocates {\em initial size} KBytes of
  space to the unification (and table copy) stack.  If units is not
  provided or is {\tt k} or {\tt K}, the size is allocated in
  kilobytes; if {\tt m} or {\tt M} in megabytes; and if {\tt g} or
  {\tt G} in gigabytes.  Default 64 KBytes.  (This option is rarely
  needed.)
\item[{\tt -D}] Tells XSB to use a top-level command loop driver specified
  here instead of the standard XSB interpreter. This is most useful when
  XSB is used as a server.
\item[{\tt -t}] Traces through code at SLG-WAM instruction level.
  This option is intended for developers and is not fully supported.
  It is also not available when the system is being used at the
  non-debug mode (see Section~\ref{debugging}).
\item[{\tt -r}] Turns off automatic stack expansion.
%
\index{garbage collection}
\item[{\tt -g gc\_type}] Chooses the heap garbage collection strategy
  that is employed; choice of the strategy is between {\tt "none"}
  (meaning perform no garbage collection), or garbage collection based
  on {\tt sliding} on {\tt copying}, or on {\tt indirection} (See
  \cite{CATmem@ISMM-98} for descriptions of the first two garbage
  collectors, and \cite{CaSC01} for a description of the third).
%
\item[{\tt -p}] Enables the engine to collect information that can be
  used for profiling.  See Volume 2 of this manual for details.
%
\item[{\tt -s}] Maintains information on the size of program stacks
  for the predicate {\tt statistics/0}.  This option may be expected
  to slow execution by around 10\%.  Default: off.  Note: this option
  is not available with the multi-threaded engine.
\item[{\tt -S}] Indicates that tabled predicates are to be evaluated
    using subsumption-based tabling as a default for tabled predicates
    whose tabling method is not specified by using {\tt
    use\_variant\_tabling/1} or {\tt use\_subsumptive\_tabling/1} (see
    Section \ref{sec:TablePred:Decl&Mod}).  If this option is not
    specified, variant-based tabling will be used as the default tabling
    method by XSB\@.
\item[{\tt -T}] Generates a trace at entry to each called predicate
    (both system and user-defined).  This option is available mainly
    for people who want to modify and/or extend XSB, and it is
    {\em not\/} the normal way to trace XSB programs.  For the
    latter, the standard predicates {\tt trace/0} or {\tt debug/0}
    should be used (see Chapter~\ref{debugging}).

    Note: This option is not available when the system is being used
    at the non-tracing mode (see Section~\ref{debugging}).
  \item[{\tt --maxthreads N}] Allows XSB to maintain information for
    up to {\tt N} threads.  This means that XSB can currently run {\tt
      N} threads that are active, or that are inactive, non-detached,
    and not yet joined.
  \item[{\tt --nobanner}] Start XSB without showing the startup banner.
    Useful in batch scripts and for interprocess communication (when XSB is
    launched as a subprocess).
  \item[{\tt --quietload}] Do not tell when a new module gets loaded. Again, is
    useful in non-interactive activities and for interprocess communication.
  \item[{\tt --noprompt}] Do not show the XSB prompt. This is useful only in batch
    mode and in interprocess communication when you do not want the prompt
    to clutter the picture.
%
\index{shared\_predicates}
  \item[{\tt --shared\_predicates}] In the multi-threaded engine, makes
    all predicates thread-shared by default; has no effect in the
    single-threaded engine.
\end{description}

As an example, if you would like to use XSB as part of a script, where
XSB loads up a file {\tt foo.P}, and execute {\tt goal} simply
use the command: 
\begin{verbatim}
xsb --nobanner --quietload --noprompt -e "[foo], goal, halt."
\end{verbatim}


\section{Memory Management}\label{memory_management}
\index{memory management} \index{garbage collection}
%===================================================
All execution stacks are automatically expanded in \version{},
including the local stack/heap region, the trail/choice point region,
and the completion stack region.  For the main thread, each of these
regions begin with an initial value set by the user at the
command-line or with a default value -- see
Section~\ref{sec:EmuOptions}).  Execution stacks increase their size
until it is not possible to do so with available system memory.  At
that point XSB tries to find the maximal amount of space that will
still fit in system memory.  When a thread is created within an XSB
process, the size of the thread's execution stacks may be set by {\tt
  thread\_create/3}, otherwise the default values indicated in
Section~\ref{sec:EmuOptions} are used.  Once XSB is running, these
default values may be modified using the appropriate XSB flags (see
Section~\ref{State}).  In addition, whenever a thread exits, memory
specific to that thread is reclaimed.

Heap garbage collection is automatically included in XSB
\cite{CaSC01,CATmem@ISMM-98}.  (To change the algorithm used for heap
garbage collection or to turn it off altogether, see the predicate
{\tt garbage\_collection/1} or Section~\ref{sec:EmuOptions} for
command-line options).  In \version{} the default behavior is indirect
garbage collection.  Starting with \version{}, heap garbage collection
may automatically invokes garbage collection of XSB's atom table,
which stores Prolog's atomic constants.  Expansion and garbage
collection of execution stacks can occur when multiple threads are
active; however atom garbage collection will not be invoked if there
is more than one active XSB thread.

The program area (the area into which XSB byte-code is loaded) is also
dynamically expanded as needed.  For dynamic code (created using {\tt
  assert/1}, or the standard predicates {\tt load\_dyn/1} and {\tt
  load\_dync/1}) index size is also automatically reconfigured.  Space
reclaimed for dynamic code depends on several factors.  If there is
only one active thread, space is reclaimed for retracted clauses and
abolished predicates as long as (1) there are no choice points that
may backtrack into the retracted or abolished code, and (2) if the
dynamic predicate is tabled, all of its tables are completed.
Otherwise, the code is marked for later garbage collection.  If more
than one thread is active, private predicates behave as just
described, however space reclamation for shared predicates will be
delayed until there is a single active thread.  See
Section~\ref{sec:assert} for details.

Space for tables is dynamically allocated as needed and reclaimed
through use of the predicates {\tt abolish\_all\_tables/0}, {\tt
  abolish\_table\_pred/1}, {\tt abolish\_table\_call/1} and other
predicates.  As with dynamic code, space for tables may be reclaimed
immediately or marked for later garbage collection depending on
whether choice points may backtrack into the abolished tables, on the
number of active threads, etc.  See
Section~\ref{sec:TablePred:Deleting} for details.

\section{Compiling, Consulting, and Loading} \label{Consulting}
%====================================================
Like other Prologs, XSB provides for both statically compiled code and
dynamically asserted code.  Unlike some Prologs, however, there is no
difference between compiled and consulted code.  Abstractly,
``Compiling'' in XSB means creation of a file containing SLG-WAM
byte-code; ``Consulting'' means loading such a byte-code file, (and
perhaps creating the file itself, via compilation).  Compiled code may
be more optimized than asserted code, but certain types of indexing,
such as trie and star indexing are (currently) available only for
dynamically asserted predicates (see {\tt index/2}).

\index{\texttt{consult/[1,2]}}
The standard predicate {\tt consult/[1,2]} is the most convenient
method for entering static source code rules into XSB's
database~\footnote{In XSB, {\tt reconsult/[1,2]} is defined to have
the same actions as {\tt consult/[1,2]}.}.  {\tt consult/1} has the
form: 
%
\begin{center}
{\tt consult(+Files)}
\end{center}
%
in which {\tt Files} is either a file designator (see
Section~\ref{sec:filenames}) or a list of file designators.  {\tt
consult/1} compiles any files specified in {\tt Files} that have been
touched later than their byte-code files, and loads them into XSB.  In
addition the form
%
\begin{center}{\tt
	    [Files]. 
}\end{center}
%
Is an alternative syntax for {\tt consult(Files)}.

{\tt consult/2} has the form.
\begin{center}{\tt
	consult(+Files, +CompilerOptionList) }
\end{center} 
{\tt Files} is as in {\tt consult/1} and {\tt CompilerOptionList}
is a list of options that are to be passed to the compiler if it is
invoked (alternatively, these compiler options can be declared in the
files themselves, as in Section~\ref{the_compiler}.  {\tt consult/2}
is more general than {\tt consult/1}, and {\tt consult(Files)} is
defined as {\tt 	consult(FileName, [])}.

Consulting a file (module) conceptually consists of the following five
steps which are described in detail in the following paragraphs.
\begin{description}
\item[Name Resolution:] determine the file to be consulted, including
	directory and drive location and extension.
\item[Compilation:]  if the source file has changed later than the
	object file, compile the module using predicate {\tt
            compile/2} with the options specified.
\item[Loading:] load the object code of the file into memory. 
\item[Importing:] if the file is a module, import any the exported
	predicates of that module to {\tt usermod}.
\item[Query Execution:] execute any queries that the file may contain.
\end{description}

There are two steps to name resolution: determining the proper
directory prefix and determining the proper file extension.  When {\tt
FileName} is absolute (i.e. it contains a path from the file to the
root of the file system) determining the proper directory prefix is
straightforward.  If {\tt FileName} is relative, i.e. it contains a
{\tt '/'} in Unix or {\tt '/'} in Windows, {\tt FileName} is expanded
to a directory prefix in an OS-dependent way, resolving symbols like
{\tt '.'}, {\tt '..'} and {\tt '\~{}'} when applicable.  However, the
user may also enter a name without any directory prefix. In this case,
the directory prefix is a directory in the dynamic loader path (see
Section~\ref{LibPath}) where the source file exists.  Once the
directory prefix is determined, the file name is checked for an
extension.  If there is no extension the loader first checks for a
file in the directory with the {\tt .P} extension, (or {\tt .c} for
foreign modules) before searching for a file without the extension,
and finally for a file with a {\tt .pl} extension.  Note that since
directories in the dynamic loader path are searched in a predetermined
order (see Section~\ref{LibPath}), if the same file name appears in
more than one of these directories, the compiler will consult the
first one it encounters.

Compilation is performed if the update date of a source file ({\tt
*.P/.pl} or {\tt *.H}) is later than that of the the object file ({\tt
*.xwam}).
%%
%----------------------------------------------------------------------
\comment{
{\em and} if the source file is not larger than the default compile
size.  This default compile is set to be 5000 words (via a declaration
in {\tt cmplib/config.P}), but can be reset by the user.  If the
source file is larger than the default compile size, the file will be
loaded using {\tt load\_dyn/1}, and otherwise it will be compiled
({\tt load\_dyn/1} can also be called separately, see the Section {\it
Asserting Dynamic Code} for details).  While {\tt load\_dyn} gives
reasonably good execution times, compilation can always be done by
using {\tt compile/[1,2]} explicitly.  }
%----------------------------------------------------------------------
%%
Currently (\version), a foreign language module is compiled when at
least one of files {\tt *.c} or {\tt *.H} has been changed from the
time the corresponding object files have been created.

Once the file is compiled into byte-code, the byte-code for the file
is loaded into XSB's database.  After loading the file all exported
predicates of that module are imported into the current environment
(the current working module {\tt usermod}) if the file is a module.
For non-modules, all predicates are imported into the current working
module.  As the last part of loading, any queries --- that is, any
terms with principal functor {\tt '?-'/1}, or with the principal
functor {\tt ':-'/1} and that are not directives like the ones
described in Section~\ref{the_compiler} are executed in the order that
they are encountered.

Error conditions for {\tt consult(+File)} are as follows: 
\bi
\item 	{\tt File} is not instantiated
\bi
\item 	{\tt instantiation\_error}
\ei
%
\item 	{\tt File} is not an atom
\bi
\item 	{\tt type\_error(atom,File)}
\ei
\item 	{\tt File} does not exist in the current set of library directories
\bi
\item 	{\tt existence\_error(file,File)}
\ei
%
\item 	{\tt File} has an object code extension (e.g. {\tt .xwam})
\bi
\item 	{\tt permission\_error(compile,file,File)}
\ei
%
\item {\tt File} has been loaded previously in the session {\em and}
  there is more than one active thread.  \bi
\item 	{\tt misc\_error}
\ei
\ei

\paragraph*{The multifile directive}
\index{\texttt{multifile/2}}
The default action upon loading a file or module is to delete all
previous byte-code for predicates defined in the file.  If this is not
the desired behavior, the user may add to the file a declaration
\begin{center}
{\tt :- multifile Predicate\_List .} \\
\end{center}
where {\tt Predicate\_List} is a list of predicates in {\em
functor/arity\/} form.  The effect of this declaration is to delete
{\em only\/} those clauses of {\tt predicate/arity} that were defined
in the file itself.  Issues to remember when using {\tt multifile} in
XSB \version{} are:
%
\begin{itemize}
\item If a predicate {\tt p/n} is to be defined in more than one file,
{\tt p/n} must be declared as multifile in each file in which it is
defined. 
%
\item The {\tt multifile} declaration should be trusted only for
static code.  Multifile declarations are not supported for code that
is dynamically loaded (via {\tt load\_dyn/[1,2]} or {\tt
load\_dync/[1,2]}).
%
\end{itemize}

Other predicates for loading files are {\tt ensure\_loaded/[1,2]}.  

\begin{description}
\standarditem{ensure\_loaded(+FileName)}{ensure\_loaded/1}
%
This predicate checks to see whether the object file for {\tt
FileName} is newer than the source code and header files for {\tt
FileName}, and compiles {\tt FileName} if not.  If {\tt FileName} is
loaded into memory, {\tt ensure\_loaded/1} does not reload it, unlike
{\tt consult/1} which will always reload
\end{description}
% 
{\tt ensure\_loaded/2} is documented in Section~\ref{sec:LoadDyn}.

\section{The Compiler} \label{the_compiler} \index{Compiler}
%========================================================

The XSB compiler translates XSB source files into
byte-code object files.  It is written entirely in Prolog.
Both the sources and the byte code\index{byte code!files!compiler}
for the compiler can be found in the XSB system directory
{\tt cmplib}\index{Compiler!\texttt{cmplib}}.
%
\index{GPP}
Prior to compiling, XSB filters the programs through \emph{GPP}, a 
preprocessor written by Denis Auroux (auroux@math.polytechnique.fr).
This preprocessor maintains high degree of compatibility with the C
preprocessor, but is more suitable for processing Prolog programs.
The preprocessor is invoked with the compiler option \verb|xpp_on|
as described below. The various features of GPP are described in
Appendix~\ref{gpp-man}.

XSB also allows the programmer to use preprocessors other than GPP.
However, the modules that come with XSB distribution require GPP.
This is explained below (see \verb|xpp_on| compiler option).

The following sections describe the various aspects of the compiler 
in more detail.


\subsection{Invoking the Compiler} \label{compiler_invoking}
\index{invoking the Compiler}\index{Compiler!invoking}
%=====================================================

The compiler is invoked directly at the interpreter level (or in a 
program) through the Prolog predicates {\tt compile/[1,2]}.  

\index{\texttt{compile/[1,2]}}
 {\tt compile/2} has the form: 
\begin{center}{\tt	
	compile(+Files, +OptionList)
}
\end{center}
in which {\tt Files} is either an absolute or relative filename or a
ground list of absolute or relative file names; and {\tt OptionList}
is a ground list of compiler options.  {\tt compile/2} compiles all
files specified, using the compiler options specified in {\tt
OptionList} (see Section~\ref{sec:CompilerOptions} below for the
precise details.)

\demo{$|$ ?- compile(Files).} 

\noindent
is just a notational shorthand for the query:

\demo{$|$ ?- compile(Files, []).}

The standard predicates {\tt consult/[1,2]} call {\tt compile/1} (if
necessary).  Error conditions for {\tt compile/[1,2]} are similar to
those for {\tt consult/[1,2]}.

The list of compiler options {\tt OptionList}, if specified, 
should be a proper Prolog list, i.e.\ a term of the form:
\begin{center}
	{\tt [ $option_1$, $option_2$, $\ldots$, $option_n$ ].}
\end{center}
where $option_i$ is one of the options described in
Section~\ref{sec:CompilerOptions}.

The source file name corresponding to a given module is obtained by 
concatenating a directory prefix and the extension {\tt .P} (or other
filenames, see Section~\ref{sec:filenames}) 
to the module name.  The directory prefix must be in the
dynamic loader path (see Section~\ref{LibPath}).
Note that these directories are searched in a predetermined
order (see Section~\ref{LibPath}), so if a module with the same name
appears in more than one of the directories searched, the compiler 
will compile the first one it encounters.  In such a case, the user can 
override the search order by providing an absolute path name.

If {\tt File} contains no extension, an attempt is made to compile the
file {\tt File.P} (or other extensions) before trying compiling the
file with name {\tt File}.

We recommend use of the extension {\tt .P} for Prolog source file to
avoid ambiguity.  Optionally, users can also provide a header file for
a module (denoted by the module name suffixed by {\tt .H}).  In such a
case, the XSB compiler will first read the header file (if it
exists), and then the source file.  Currently the compiler makes no
special treatment of header files.  They are simply included in the
beginning of the corresponding source files, and code can, in
principle, be placed in either.  
\comment{
In future versions of XSB the
header files may be used to check interfaces across modules, hence it
is a good programming practice to restrict header files to
declarations alone.
}
 
The result of the compilation (an SLG-WAM object code file) is stored
in a ($\langle$filename$\rangle$.xwam), but {\tt compile/[1,2]} does {\em
not\/} load the object file it creates.  (The standard predicates {\tt
consult/[1,2]} and {\tt reconsult/[1,2]} both recompile the source
file, if needed, and load the object file into the system.)  The
object file created is always written into the directory where the
source file resides (the user must therefore have write permission
in that directory).
 
If desired, when compiling a module (file), clauses and directives can be
transformed as they are read.  This is indeed the case for definite clause
grammar rules (see Chapter~\ref{DCGs}), but it can also be done for clauses
of any form by providing a definition for predicate {\tt term\_expansion/2}
(see Section~\ref{DCG_builtins}).

Predicates {\tt compile/[1,2]} can also be used to compile foreign
language modules.  In this case, the names of the source files should
have the extension {\tt .c} and a {\tt .P} file must {\em not\/}
exist.  A header file (with extension {\tt .H}) {\em must} be present
for a foreign language module (see the chapter {\it Foreign Language
Interface} in Volume 2).


\subsection{Compiler Options}\label{sec:CompilerOptions}
\index{Compiler!options}\index{options!Compiler}
%=================================================

Compiler options can be set in three ways: from a global list of
options ({\tt set\_global\_compiler\_options/1}), from the
compilation command ({\tt compile/2} and {\tt consult/2}), and
from a directive in the file to be compiled (see compiler directive
{\tt compiler\_options/1}).

\begin{description}
\standarditem{set\_global\_compiler\_options(+OptionsList)}{set\_global\_compiler\_options/1}
    {\tt OptionsList} is a list of compiler options (described below).
    Each can optionally be prefixed by \verb|+| or \verb|-|,
    indicating that the option is to be turned on, or off,
    respectively.  (No prefix turns the option on.)  This evaluable
    predicate sets the global compiler options in the way indicated.
    These options will be used in any subsequent compilation, unless
    they are 
    reset by another call to this predicate, overridden by options
    provided in the compile invocation, or overridden by options in
    the file to be compiled.
\end{description}

The following options are currently recognized by the compiler:
\begin{description}
\item[{\tt optimize}]\index{\texttt{optimize}}
	When specified, the compiler tries to optimize the object code.
	In \version, this option optimizes predicate calls, among other
	features, so execution may be considerably faster for recursive
	loops.  However,
	due to the nature of the optimizations, the user may not be able to
	trace all calls to predicates in the program.  
%%
	\comment{TLS: can no longer assert to static code
	Also the Prolog code
	should be {\em static}.  In other words, the user is {\em not} allowed
	to alter the entry point of these compiled predicates by asserting new
	clauses.  }
%%
	As expected, the compilation phase will also be slightly
	longer.  For these reasons, the use of the {\tt optimize} option may
	not be suitable for the development phase, but is
	recommended once the code has been debugged.


\item[{\tt xpp\_on}]\index{\texttt{xpp\_on}}\index{GPP}
  Filter the program through a preprocessor  before sending it to the XSB 
  compiler. By default (and for the XSB code itself), XSB uses GPP, a
  preprocessor developed by Denis Auroux (auroux@math.polytechnique.fr)
  that has high degree of compatibility with the C preprocessor, but is
  more suitable for Prolog syntax. In this case, the source code can
  include the usual C 
  preprocessor directives, such as \verb|#define|, \verb|#ifdef|, and
  \verb|#include|. This option can be specified both as a parameter to
  compile/2 and as part of the {\tt compiler\_options/1} directive inside
  the source file. See Appendix~\ref{gpp-man} for more details on GPP.

  When an \verb|#include "file"| statement is encountered,
  XSB directs GPP preprocessor to search for the files to include in the
  directories \verb|$XSB_DIR/emu| 
  %%$
  and \verb|$XSB_DIR/prolog_includes|.
  %%$
  \index{\texttt{xpp\_include\_dir}}
  %%
  However, additional directories can be added to this search path by
  asserting into the predicate \verb|xpp_include_dir/1|, {\bf which must
    be imported from module} {\tt parse}.
  
  Note that when compiling XSB programs, GPP searches the current directory
  and the directory of the parent file that contains the include-directive
  \emph{last}. If you want additional directories to be searched, then the
  following statements must be executed:
%%
\begin{verbatim}
    :- import xpp_include_dir/1 from parse.
    :- assert(xpp_include_dir('some-other-dir')).
\end{verbatim}
%%
  If you want Gpp to search directories in a different order,
  {\tt xpp\_options/1} can be used (see below).

   Note: if you assert something into this predicate then you must also {\tt
    retractall(xpp\_include\_dir(\_))} after that or else subsequent Prolog
  compilations might not work correctly.

  XSB predefines the constant {\tt XSB\_PROLOG}, which can be used for
  conditional compilation. For instance, you can write portable program
  to run under XSB and and other prologs that support C-style
  preprocessing and use conditional compilation to account for the
  differences: 
  %%
  \begin{samepage}
  \begin{verbatim}
#ifdef XSB_PROLOG
    XSB-specific stuff
#else
    other Prolog's stuff
#endif
    common stuff
  \end{verbatim}
  \end{samepage}
  %%

  \index{xpp\_program}
  %%
  However, as mentioned earlier, XSB lets the user filter programs (except
  the programs that belong to XSB distribution) through any preprocessor
  the user wants. To this end, one only needs to assert the appropriate
  command into the predicate \verb|xpp_program|, which should be imported
  from module {\tt parse}. The command should not include the file
  name---XSB appends the name of the file to be compiled to the command
  supplied by the user. For instance, executing
  %%
  \begin{verbatim}
   :- assert(xpp_program('/usr/bin/m4 -E -G')).
  \end{verbatim}
  %%
  before calling the compiler will have the effect that the next XSB
  program passed to the compiler will be first preprocessed by the M4 macro
  package. Note that the XSB compiler automatically clears out the
  {\tt xpp\_program} predicate, so there is no need to tidy up each time.
  But this also means that if you need to compile several programs with a
  non-standard preprocessor then you must specify that non-standard
  preprocessor each time the program is compiled.
  
\item[{\tt xpp\_options}] \index{\texttt{xpp\_options}} This dynamic predicate
  must be imported from module {\tt parse}.  If some atom is asserted into
  {\tt xpp\_options} then this atom is assumed to be the list of command
  line options to be used by the preprocessor (only the first asserted atom
  is ever considered). If this predicate is empty, then the default list of
  options is used (which is {\tt '-P -m -nostdinc -nocurinc'}, meaning: use
  Prolog mode and do not search the standard C directories and the
  directory of the parent file that contains the include-instruction).
  
  As mentioned earlier, when XSB invokes Gpp, it uses the option {\tt
    -nocurinc} so that Gpp will not search the directory of the parent file. 
  If a particular application requires that the parent file directory
  must be searched, then this can be accomplished by executing 
  {\tt assert(xpp\_options('-P -m -nostdinc'))}.
  
  Note: if you assert something into this predicate then you must also {\tt
    retractall(xpp\_options(\_))} after that or else subsequent Prolog
  compilations might not work correctly.
  
\item[{\tt xpp\_dump}] \index{\texttt{xpp\_dump}}
  %%
  This causes XSB to dump the output from the GPP preprocessor into a file.
  If the file being compiled is named {\tt file.P} then the dump file is
  named {\tt file.P\_gpp}. This option can be included in the list of
  options in the {\tt compiler\_options/1} directive, but usually it is
  used for debugging, as part of the {\tt compile/2} predicate. If {\tt
    xpp\_dump} is specified directly in the file using {\tt
    compiler\_options/1} directive, then it should \emph{not} follow the
  {\tt gpp\_on} option in the list (or else it will be ignored).

\item[{\tt quit\_on\_error}] \index{\texttt{quit\_on\_error}}
  This causes XSB to exit if compilation of a program end with an error.
  This option is useful when running XSB from a makefile, when it is
  necessary to stop the build process after an error has been detected. For
  instance, XSB uses this option during its own build process.


%-------------------------------
\index{tabling!compiler options}
\index{tabling!automatic}
%-------------------------------
\item[{\tt auto\_table}]\index{\texttt{auto\_table}} When specified as a
  compiler option, the effect is as described in
  Section~\ref{tabling_directives}.  Briefly, a static analysis is made to
  determine which predicates may loop under Prolog's SLD evaluation.  These
  predicates are compiled as tabled predicates, and SLG evaluation is used
  instead.
\index{tabling!supplemental}
\item[{\tt suppl\_table}]\index{\texttt{suppl\_table}} The intention of this
  option is to direct the system to table for efficiency rather than
  termination.  When specified, the compiler uses tabling to ensure that no
  predicate will depend on more than three tables or EDB facts (as
  specified by the declaration {\tt edb} of
  Section~\ref{tabling_directives}).  The action of {\tt suppl\_table} is
  independent of that of {\tt auto\_table}, in that a predicate tabled by
  one will not necessarily be tabled by the other.  During compilation,
  {\tt suppl\_table} occurs after {\tt auto\_table}, and uses table
  declarations generated by it, if any.
%--------------------------------------
\index{specialization!compiler options}
%--------------------------------------
\item[{\tt spec\_repr}]\index{\texttt{spec\_repr}}
	When specified, the compiler performs specialization of partially
	instantiated calls by replacing their selected clauses with the
	representative of these clauses, i.e. it performs {\em folding\/}
	whenever possible.  We note in general, the code replacement
	operation is not always sound; i.e. there
	are cases when the original and the residual program are not
	computationally equivalent.  The compiler checks for sufficient (but
	not necessary) conditions that guarantee computational equivalence.
	If these conditions are not met, specialization is not performed
	for the violating calls.
\item[{\tt spec\_off}]\index{\texttt{spec\_off}}
	When specified, the compiler does not perform specialization of
	partially instantiated calls.
\item[{\tt unfold\_off}]\index{\texttt{unfold\_off}}
	When specified, singleton sets optimizations are not performed
	during specialization.  This option is necessary in \version\
	for the specialization of {\tt table} declarations that select
	only a single chain rule of the predicate.
%qtls ??
\item[{\tt spec\_dump}]\index{\texttt{spec\_dump}}
	Generates a {\tt module.spec} file, containing the result of
	specializing partially instantiated calls to predicates defined
	in the {\tt module} under compilation.  The result is in Prolog
	source code form.

%---------------------------------------------
\index{unification factoring!compiler options}
%---------------------------------------------
\item[{\tt ti\_dump}]\index{\texttt{ti\_dump}}
	Generates a {\tt module.ti} file containing the result of applying
	unification factoring to predicates defined in the {\tt module}
	under compilation.  The result is in Prolog source code form.
	See page~\pageref{transformational_indexing} for more information
	on unification factoring.
\item[{\tt ti\_long\_names}]\index{\texttt{ti\_long\_names}}
	Used in conjunction with {\tt ti\_dump}, generates names for
	predicates created by unification factoring that reflect the
	clause head factoring done by the transformation.
\comment{
\item[{\tt init\_var\_off}]\index{\texttt{init\_var\_off}}
	When specified, the compiler will give a warning (instead of an
	error) upon finding that a potentially uninitialized variable is
	being used.  {\em Potentially uninitialized variables\/} are
	variables that appear in only one branch of an {\sf or} or an
	{\sf if-then-else} goal in the body, and, furthermore, are used
	after that goal.
	In certain clauses, the variable may always be initialized after
	the {\sf or} or the {\sf if-then-else} goal, because the execution 
	cannot continue through the path of the branch that does not initialize
	the variable.  In these cases, the {\tt init\_var\_off} option can be
	useful, though the user is cautioned against careless use of this
	option.

	{\sc Warning:} The object-file generated by the compiler when this
		option is used may not execute correctly (or even cause
		XSB to core dump!) if the variable is indeed
		uninitialized when used.
}
%---------------------------------------------
\index{mode analysis!compiler options}
%---------------------------------------------
\item[{\tt modeinfer}]\index{\texttt{modeinfer}}
	This option is used to trigger mode analysis. For each module
	compiled,  the mode analyzer creates a  {\tt {\em module}.D} file
	that contains the mode information.

	{\sc Warning:}
	Occasionally, the analysis itself may take a long time. 
	As far as we have seen,
	the analysis times are longer than the rest of the compilation time
	only when the module contains recursive predicates of arity $\geq 10$.
	If the analysis takes an unusually long time
	(say, more than 4 times as long as the rest of the compilation)
	you may want to abort and restart compilation without {\tt modeinfer}.
	
\item[{\tt mi\_warn}]\index{\texttt{mi\_warn}}
	During mode analysis, the {\tt .D} files corresponding to the
	imported modules are read in. The option {\tt mi\_warn} is used
	to generate warning messages if these {\tt .D} files are 
	outdated --- {\em i.e.}, older
	than the last modification time of the source files.

\item[{\tt mi\_foreign}] This option is used {\em only\/} when mode analysis
	is performed on XSB system modules. This option is
	needed when analyzing {\tt standard} and {\tt machine} in
	{\tt syslib}.


\item[{\tt sysmod}] \index{standard predicates}
	Mainly used by developers when compiling system modules. If
	specified, standard predicates (see {\tt
	/\$XSB\_DIR/syslib/std\_xsb.P}) are automatically available
	for use only if they are primitive predicates (see the file
	{\tt \$XSB\_DIR/syslib/machine.P} for a current listing of
	primitive predicates.  When compiling in this mode,
	non-primitive standard predicates must be explicitly imported
	from the appropriate system module.  Also standard predicates 
	are permitted to be defined.
\item[{\tt allow\_redefinition}] \index{standard predicates}
	By default the compiler refuses to compile a file that
	contains clauses that would redefine a standard predicate
	(unless the {\tt sysmod} option is in effect.)  By specifying
	this option, the user can direct the compiler to quietly allow
	redefinition of standard predicates.

\item[{\tt verbo}] Compiles the files (modules) specified in ``verbose'' mode, 
	printing out information about the progress of the compilation of each 
	predicate.
\item[{\tt profile}] This option is usually used when modifying the
	XSB compiler.  When specified, the compiler prints out
	information about the time spent in each phase of the
	compilation process.

\item[{\tt asm\_dump, compile\_off}] Generates a textual representation of 
	the SLG-WAM assembly code and writes it into the file {\tt module.A}
	where {\tt module} is the name of the module (file) being compiled.  
	
	{\sc Warning:} This option was created for compiler debugging and is
		not intended for general use.  There might be cases where
		compiling a module with these options may cause generation
		of an incorrect {\tt .A} and {\tt .xwam} file.  In such cases,
		the user can see the SLG-WAM instructions that are
		generated for a module by compiling the module as usual and
		then using the {\tt -d module.xwam} command-line
		option of the 
		XSB emulator (see Section~\ref{sec:EmuOptions}).
\item[{\tt singleton\_warnings\_off}] Does not print out any warnings
	for singleton variables during compilation.  This option can
	be useful for compiling generated programs.
\item[{\tt index\_off}] When specified, the compiler does not generate indices
	for the predicates compiled.  
\end{description}


\subsection{Specialization}\label{specialization}
\index{Compiler!specialization}\index{specialization!Compiler}
%=============================================================

From Version 1.4.0 on, the XSB compiler automatically performs
specialization of partially instantiated calls.  Specialization can be
thought as a source-level program transformation of a program to a
residual program in which partially instantiated calls to predicates
in the original program are replaced with calls to specialized versions
of these predicates.  The expectation from this process is that the
calls in the residual program can be executed more efficiently that
their non-specialized counterparts.  This expectation is justified
mainly because of the following two basic properties of the
specialization algorithm:
\begin{description}
\item[Compile-time Clause Selection] The specialized calls of the
	residual program  directly select (at compile time) a subset
	containing only the clauses that the corresponding calls of the
	original program would otherwise have to examine during their
	execution (at run time).  By doing so, laying down unnecessary
	choice points is at least partly avoided, and so is the need to
	select clauses through some sort of indexing.
\item[Factoring of Common Subterms] Non-variable subterms of partially
	instantiated calls that are common with subterms in the heads
	of the selected clauses are factored out from these terms
	during the specialization process.  As a result, some head
	unification ({\tt get\_*} or {\tt unify\_*}) and some argument
	register ({\tt put\_*}) WAM instructions of the original
	program become unnecessary.  These instructions are eliminated
	from both the specialized calls as well as from the specialized
	versions of the predicates.
\end{description}
Though these properties are sufficient to get the idea behind
specialization, the actual specialization performed by the XSB
compiler can be better understood by the following example.  The
example shows the specialization of a predicate that checks if a list
of HiLog terms is ordered:
\begin{center}
\tt
\begin{tabular}{ccc}
\begin{tabular}{l}
ordered([]). \\
ordered([X]). \\
ordered([X,Y|Z]) :- \\
\ \ \ \ X @=< Y, ordered([Y|Z]). 
\end{tabular}
& $\longrightarrow$ &
\begin{tabular}{l}
ordered([]). \\
ordered([X]). \\
ordered([X,Y|Z]) :- \\
\ \ \ \ X @=< Y, \_\$ordered(Y, Z). \\
\\
:- index \_\$ordered/2-2. \\
\_\$ordered(X, []). \\
\_\$ordered(X, [Y|Z]) :- \\
\ \ \ \ X @=< Y, \_\$ordered(Y, Z).
\end{tabular}
\end{tabular}
\end{center}
The transformation (driven by the partially instantiated call
{\tt ordered([Y|Z])}) effectively allows predicate {\tt ordered/2}
to be completely deterministic (when used with a proper list as its
argument), and to not use any unnecessary heap-space for its
execution.  We note that appropriate {\tt :- index} directives are
automatically generated by the XSB compiler for all specialized
versions of predicates.

The default specialization of partially instantiated calls is without
any folding of the clauses that the calls select.  Using the {\tt
spec\_repr} compiler option (see Section~\ref{sec:CompilerOptions})
specialization with replacement of the selected clauses with the
representative of these clauses is performed.  Using this compiler
option, predicate {\tt ordered/2} above would be specialized as follows:
%%
\begin{center}
\begin{minipage}{4.1in}
\begin{verbatim}
ordered([]).
ordered([X|Y]) :- _$ordered(X, Y).

:- index _$ordered/2-2.
_$ordered(X, []).
_$ordered(X, [Y|Z]) :- X @=< Y, _$ordered(Y, Z).
\end{verbatim}
\end{minipage}
\end{center}
%%$
%%
We note that in the presence of cuts or side-effects, the code
replacement operation is not always sound, i.e.  there are cases when
the original and the residual program are not computationally equivalent
(with respect to the answer substitution semantics).  The compiler
checks for sufficient (but not necessary) conditions that guarantee
computational equivalence, and if these conditions are not met,
specialization is not performed for the violating calls.

The XSB compiler prints out messages whenever it specialises
calls to some predicate.  For example, while compiling a file
containing predicate {\tt ordered/1} above, the compiler would print
out the following message:
\begin{center}
{\tt	\% Specialising partially instantiated calls to ordered/1}
\end{center}
The user may examine the result of the specialization transformation
by using the {\tt spec\_dump} compiler option
(see Section~\ref{sec:CompilerOptions}).

Finally, we have to mention that for technical reasons beyond the scope of
this document, specialization cannot be transparent to the user; predicates
created by the transformation do appear during tracing.


\subsection{Compiler Directives}\label{compiler_directives}
\index{Compiler!directives}\index{directives!Compiler}
%=====================================================

Consider a directive
\begin{verbatim}
:- foo(a).
\end{verbatim}
That occurs in a file that is to be compiled.  There are two logical
interpretations of such a directive.
\begin{enumerate}
\item {\tt foo(a)} is to be executed upon loading the file; or
%
\item{\tt foo(a)} provides information used by the compiler in
compiling the file. 
\end{enumerate}

By default, the interpretation of a directive is as in case (1) {\em
except} in the case of the compiler directives listed in this section,
which as their name implies, are taken to provide information to the
compiler.  Some of the directives, such as the {\tt mode/1} directive,
have no meaning as an executable directive, while others, such as {\tt
import/2} do.  In fact as an executable directive {\tt import/2}
imports predicates into {\tt usermod}.  For such a directive, a 
statement beginning with {\tt ?-}, such as 
\begin{verbatim}
?- import foo/1 from myfile.
\end{verbatim}
indicates that the directive should be executed upon loading the file,
and should have no meaning to the compiler.  On the other hand, the
statement
\begin{verbatim}
:- import foo/1 from myfile.
\end{verbatim}
Indicates that {\tt foo/1} terms in the file to be compiled are to be
understood as {\tt myfile:foo/1}.  In other words, the statement is
used by the compiler and will not be executed upon loading.  For
non-compiler directives the use of {\tt ?-} and {\tt :-} has no effect
--- in both cases the directive is executed upon loading the file.

The following compiler directives are recognized in \version\ of XSB
\footnote{Any parallelisation directives ({\tt parallel}) are simply
ignored by the compiler, but do not result in syntax errors to enhance
compatibility with various other earlier versions of PSB-Prolog.}.

\subsubsection{Mode Declarations}\label{mode_declarations}
\index{modes!directives}\index{directives!modes}
%-----------------------------------------------------

The XSB compiler accepts {\tt mode} declarations of the form:

\demo{:- mode $ModeAnnot_1, \ldots, ModeAnnot_n$.}

\noindent
where each $ModeAnnot$ is a {\em mode annotation\/} (a {\em term
indicator\/} whose arguments are elements of the set {\tt
$\{$+,-,\#,?$\}$}).  From Version 1.4.1 on, {\tt mode} directives are
used by the compiler for tabling directives, a use which differs from
the standard use of modes in Prolog systems\footnote{The most common
uses of {\tt mode} declarations in Prolog systems are to reduce the
size of compiled code, or to speed up a predicate's execution.}.  See
Section~\ref{tabling_directives} for detailed examples.

Mode annotations have the following meaning:
\begin{description}
\item[{\tt +}]
	This argument is an input to the predicate.  In every invocation
	of the predicate, the argument position must contain a non-variable
	term.  This term may not necessarily be ground, but the 
	predicate is guaranteed not to alter this argument).

	\demo{:- mode see(+), assert(+).}
\item[{\tt -}]
	This argument is an output of the predicate.  In every
	invocation of the predicate the argument
	position {\em will always be a variable\/} (as opposed to 
	the {\tt \#} annotation below).
	This variable is unified with the value returned by the predicate.
	We note that Prolog does not enforce the requirement that output
	arguments should be variables; however, output unification is not
	very common in practice.

	\demo{:- mode cputime(-).}
\item[{\tt \#}]
	This argument is either:
	\begin{itemize}
	\item	An output argument of the predicate for which a non-variable
		value may be supplied for this argument position.  If such a
		value is supplied, the result in this position is unified with
		the supplied supplied value.  The predicate fails if this
		unification fails.  If a variable term is supplied, the
		predicate succeeds, and the output variable is unified with
		the return value.

		\demo{:- mode '='(\#,\#).}
	\item	An input/output argument position of a predicate that has
		only side-effects (usually by further instantiating that
		argument).  The {\tt \#} symbol is used to denote the $\pm$
		symbol that cannot be entered from the keyboard.
	\end{itemize}
\item[{\tt ?}]
	This argument does not fall into any of the above categories. 
        Typical cases would be the following:
	\begin{itemize}
	\item	An argument that can be used both as input and as output
		(but usually not with both uses at the same time).

		\demo{:- mode functor(?,?,?).}
	\item	An input argument where the term supplied can be a variable
		(so that the argument cannot be annotated as {\tt +}), or is
		instantiated to a term which itself contains uninstantiated
		variables, but the predicate is guaranteed {\em not\/} to
		bind any of these variables.

		\demo{:- mode var(?), write(?).}
	\end{itemize}
\end{description}
We try to follow these mode annotation conventions throughout this manual.

Finally, we warn the user that {\tt mode} declarations can be error-prone,
and since errors in mode declarations do not show up while running the
predicates interactively, unexpected behavior may be witnessed in compiled
code, optimized to take modes into account (currently not performed by
XSB)\@.  However, despite this danger, {\tt mode} annotations can be
a good source of documentation, since they express the programmer's
intention of data flow in the program.


\subsubsection{Tabling Directives}\label{tabling_directives}
\index{tabling!directives}\index{directives!tabling}
%-----------------------------------------------------
\index{\texttt{auto\_table}}
Memoization is often necessary to ensure that programs terminate, and
can be useful as an optimization strategy as well.  The underlying
engine of XSB is based on SLG, a memoization strategy, which,
in our version, maintains a table of calls and their answers for each
predicate declared as {\em tabled}.  Predicates that are not declared
as tabled execute as in Prolog, eliminating the expense of tabling
when it is unnecessary.

The simplest way to use tabling is to include the directive

\demo{:- auto\_table.}

\noindent
anywhere in the source file.  {\tt auto\_table} declares predicates
tabled so that the program will terminate.

To understand precisely how {\tt auto\_table} does this, it is
necessary to mention a few properties of SLG.  For programs which have
no function symbols, or where function symbols always have a limited
depth, SLG resolution ensures that any query will terminate after it
has found all correct answers.  In the rest of this section, we
restrict consideration to such programs.

Obviously, not all predicates will need to be tabled for a program to
terminate.  The {\tt auto\_table} compiler directive tables only those
predicates of a module which appear to static analysis to contain an
infinite loop, or which are called directly through {\tt tnot/1}.  It
is perhaps more illuminating to demonstrate these conditions through
an example rather than explaining them.  For instance, in the program.

%tls commented out minipage because latex was formatting badly,
\begin{center}
%\begin{minipage}{3in}
\begin{verbatim}
:- auto_table. 

p(a) :- s(f(a)). 

s(X) :- p(f(a)).

r(X) :- q(X,W),r(Y).

m(X) :- tnot(f(X)).

:- mode ap1(-,-,+).
ap1([H|T],L,[H|L1]) :- ap1(T,L,L1).

:- mode ap(+,+,-).
ap([],F,F).
ap([H|T],L,[H|L1]) :- ap(T,L,L1).

mem(H,[H|T]).
mem(H,[_|T]) :- mem(H,T).
\end{verbatim}
%\end{minipage}
\end{center}

\noindent
The compiler prints out the messages
\begin{verbatim}
% Compiling predicate s/1 as a tabled predicate
% Compiling predicate r/1 as a tabled predicate
% Compiling predicate m/1 as a tabled predicate
% Compiling predicate mem/2 as a tabled predicate
\end{verbatim}

Terminating conditions were detected for {\tt ap1/3} and {\tt ap/3}, but
not for any of the other predicates.

{\tt auto\_table} gives an approximation of tabled programs which we
hope will be useful for most programs.  The minimal set of tabled
predicates needed to ensure termination for a given program is
undecidable.  
\comment{
Practically, refining the set of tabled predicates
deduced by {\tt auto\_table} is still an open research problem.
}
It should be noted that the presence of meta-predicates such as {\tt
call/1} makes any static analysis useless, so that the {\tt
auto\_table} directive should not be used in such cases.

Predicates can be explicitly declared as tabled as well, through the
{\tt table/1}.  When {\tt table/1} is used, the directive takes the
form

\demo{:- table(F/A).}

\noindent
where {\tt F} is the functor of the predicate to be tabled, and {\tt A} its
arity.  

\index{\texttt{suppl\_table}}
Another use of tabling is to filter out redundant solutions for
efficiency rather than termination.  In this case, suppose that the
directive {\tt edb/1} were used to indicate that certain predicates were
likely to have a large number of clauses.  Then the action of the declaration
{\tt :- suppl\_table} in the program:
\begin{verbatim}
:- edb(r1/2).
:- edb(r2/2).
:- edb(r3/2).

:- suppl_table.

join(X,Z):- r1(X,X1),r2(X1,X2),r3(X2,Z).
\end{verbatim}
would be to table {\tt join/2}.  The {\tt suppl\_table} directive is
the XSB analogue to the deductive database optimization, {\em
supplementary magic templates} \cite{BeRa91}.  {\tt suppl\_table/0} is
shorthand for {\tt suppl\_table(2)} which tables all predicates
containing clauses with two or more {\tt edb} facts or tabled
predicates.  By specifying {\tt suppl\_table(3)} for instance, only
predicates containing clauses with three or more {\tt edb} facts or
tabled predicates would be tabled.  This flexibility can prove useful
for certain data-intensive applications.


\subsubsection{Indexing Directives}\label{indexing_directives}
\index{indexing!directives}\index{directives!indexing}
%-------------------------------------------------------------

The XSB compiler by default generates an index on the principal 
functor of the first argument of a predicate.  Indexing on the appropriate 
argument of a predicate may significantly speed up its execution time.  
In many cases the first argument of a predicate may not be the most
appropriate argument for indexing and changing the order of arguments
may seem unnatural.  In these cases, the user may generate an index
on any other argument by means of an indexing directive.  This is a
directive of the form:

\demo{:- index Functor/Arity-IndexArg.}

\noindent
indicating that an index should be created for predicate 
{\tt Functor}/{\tt Arity} on its ${\tt IndexArg}^{\rm th}$ argument.
One may also use the form:

\demo{:- index(Functor/Arity, IndexArg, HashTableSize).}

\index{\texttt{index/2}}
\noindent
which allows further specification of the size of the hash table to use for
indexing this predicate if it is a {\em dynamic} (i.e., asserted) predicate.
For predicates that are dynamically loaded, this directive can be used to
specify indexing on more than one argument, or indexing on a combination
of arguments (see its description on page~\pageref{index_dynamic}).
For a compiled predicate the size of the hash table is computed automatically,
so {\tt HashTableSize} is ignored.

All of the values {\tt Functor}, {\tt Arity}, {\tt IndexArg} (and possibly
{\tt HashTableSize}) should be ground in the directive.  More specifically,
{\tt Functor} should be an atom, {\tt Arity} an integer in the range 0..255,
and {\tt IndexArg} an integer between 0 and {\tt Arity}.  If {\tt IndexArg}
is equal to~0, then no index is created for that predicate. An {\tt index}
directive may be placed anywhere in the file containing the predicate it
refers to.

As an example, if we wished to create an index on the third argument 
of predicate {\tt foo/5}, the compiler directive would be:

\demo{:- index foo/5-3.}


\subsubsection{Unification Factoring}\label{transformational_indexing}
\index{indexing!transformational}
When the clause heads of a predicate have portions of arguments common
to several clauses, indexing on the principal functor of one argument
may not be sufficient.  Indexing may be improved in such cases by the
use of unification factoring.  Unification Factoring is a program
transformation that ``factors out'' common parts of clause heads,
allowing differing parts to be used for indexing, as illustrated by
the following example:
\begin{center}
\tt
\begin{tabular}{ccc}
\begin{tabular}{l}
p(f(a),X) :- q(X). \\
p(f(b),X) :- r(X).
\end{tabular}
& $\longrightarrow$ &
\begin{tabular}{l}
p(f(X),Y) :- \_\$p(X,Y). \\
\_\$p(a,X) :- q(X). \\
\_\$p(b,X) :- r(X).
\end{tabular}
\end{tabular}
\end{center}
The transformation thus effectively allows $p/2$ to be indexed
on atoms $a/0$ and $b/0$.  Unification Factoring is transparent
to the user; predicates created by the transformation are internal
to the system and do not appear during tracing.

The following compiler directives control the use of unification
factoring \footnote{Unification factoring was once called
transformational indexing, hence the abbreviation {\tt ti} in the
compiler directives}:
\begin{description}
\item[{\tt :- ti(F/A).}] Specifies that predicate $F/A$ should be
	compiled with unification factoring enabled.
\item[{\tt :- ti\_off(F/A).}] Specifies that predicate $F/A$ should be
	compiled with unification factoring disabled.
\item[{\tt :- ti\_all.}] Specifies that all predicates defined in the
	file should be compiled with unification factoring enabled.
\item[{\tt :- ti\_off\_all.}] Specifies that all predicates defined in
	the file should be compiled with unification factoring disabled.
\end{description}
By default, higher-order predicates (more precisely, predicates named
{\it apply\/} with arity greater than 1) are compiled with unification
factoring enabled.  It can be disabled using the {\tt ti\_off}
directive.  For all other predicates, unification factoring must be
enabled explicitly via the {\tt ti} or {\tt ti\_all} directive.  If
both {\tt :- ti(F/A).} ({\tt :- ti\_all.}) and {\tt :- ti\_off(F/A).}
({\tt :- ti\_off\_all.}) are specified, {\tt :- ti\_off(F/A).} ({\tt
:- ti\_off\_all.}) takes precedence.  Note that unification factoring
may have no effect when a predicate is well indexed to begin
with.  For example, unification factoring has no effect on the
following program:
\begin{center}
\tt
\begin{tabular}{l}
p(a,c,X) :- q(X). \\
p(b,c,X) :- r(X).
\end{tabular}
\end{center}
even though the two clauses have $c/0$ in common.  The user may
examine the results of the transformation by using the {\tt ti\_dump}
compiler option (see Section~\ref{sec:CompilerOptions}).

\subsubsection{Other Directives} \label{other-directives}
%==============================================

XSB has other directives not found in other Prolog systems.

\begin{description}
\desc{:- hilog $atom_1, \ldots, atom_n$.}
	Declares symbols $atom_1$ through $atom_n$ as HiLog symbols.
	The {\tt hilog} declaration should appear {\em before} any use of
	the symbols.  See Chapter~\ref{Syntax} for a purpose of this
 	declaration.
\desc{:- ldoption($Options$).}
        This directive is only recognized in the header file ({\tt .H} file) 
	of a foreign module. See the chapter {\it Foreign Language
Interface} in Volume 2 for its explanation.
\desc{:- compiler\_options($OptionsList$).}
	Indicates that the compiler options in the list $OptionsList$
	should be used to compile this file.  This must appear at the
	beginning of the file.  These options will override any others,
	including those given in the compilation command.  The options
	may be optionally prefixed with \verb|+| or \verb|-| to
	indicate that they should be set on or off.  (No prefix
	indicates the option should be set on.)

\end{description}

\subsection{Inline Predicates}\label{inline_predicates}
\index{Compiler!inlines}\index{inlines!Compiler}
%======================================================

{\em Inline predicates} represent ``primitive'' operations in the
(extended) WAM.  Calls to inline predicates are compiled into a
sequence of WAM instructions in-line, i.e. without actually making a
call to the predicate.  Thus, for example, relational predicates (like
{\tt >/2}, {\tt >=/2}, etc.) compile to, essentially, a subtraction
followed by a conditional branch.  As a result, calls to inline
predicates will not be trapped by the debugger, and their evaluation
will not be visible during a trace of program execution.  Inline
predicates are expanded specially by the compiler and thus {\em cannot
  be redefined by the user without changing the compiler}.  The user
does not need to import these predicates from anywhere.  There are
available no matter what options are specified during compiling.

Table~\ref{inlinepredicatetable} lists the inline predicates of
XSB \version.  Those predicates that start with \verb|_$|
are internal predicates that are also expanded in-line during
compilation.

\begin{table}[htbp]\centering{\tt
\begin{tabular}{lllll}
\verb|'='/2|    &\verb|'<'/2|	&\verb|'=<'/2|  &\verb|'>='/2| &\verb|'>'/2| \\
\verb|'=:='/2|  &\verb|'=\='/2|	&is/2           &\verb|'@<'/2| &\verb|'@=<'/2|\\
\verb|'@>'/2|	&\verb|'@>='/2|	&\verb|'=='/2|	&\verb|'\=='/2|&fail/0  \\
true/0		&var/1		&nonvar/1	&halt/0	       &'!'/0 \\
min/2           &max/2          &\verb|'><'|/2  &\verb|**|/2   &sign/1 \\
'\_\$cutto'/1	&'\_\$savecp'/1	&'\_\$builtin'/1
\end{tabular}}
\caption{The Inline Predicates of XSB}\label{inlinepredicatetable}
\end{table}

We warn the user to be cautious when defining predicates whose functor
starts with {\tt \_\$} since the names of these predicates may
interfere with some of XSB's internal predicates.  The situation may
be particularly severe for predicates like {\tt '\_\$builtin'/1} that
are treated specially by the XSB compiler.

\section{A Note on ISO Compatibility} \label{sec:iso}
\index{ISO Compatability}

In \version, an effort has been made to ensure compatibility with the
core Prolog ISO standard~\cite{ISO-Prolog}.  In this section, we
summarize the differences with the ISO standard.  XSB implements
almost all ISO built-ins and evaluable functions, although there are
certain semantic differences between XSB's implementation and that of
the ISO standard in certain cases.

The main difference of XSB with the ISO standard is in terms of
parsing.  \version{} of XSB does not support full ISO
syntax~\footnote{XSB also does not support multiple character sets or
  Unicode, which is perhaps a bigger limitation than the lack of full
  ISO syntax.}.  In addition, XSB supports only the ASCII character
set for atoms, predicates and functions, so that ISO predicates
relating to different character sets, such as {\tt
  char\_conversion/2}, {\tt current\_char\_conversion/2} and others
are not supported.

Another difference is that XSB does not support the logical update
semantics for assert and retract, but instead supports an immediate
semantics.  Despite the patholical examples that can be devised using
the immediate semantics, the logical semantics for assert is not often
critical for single-threaded applications.  It is however, critical
for multi-threaded applications, and XSB will support this in the
future.

A somewhat more minor difference involves XSB's implementation of ISO
streams.  XSB can create streams from several Firstzd class objects,
including pipes, atoms, and consoles in addition to files.  However by
default, XSB opens streams in binary mode, rather than text mode in
opposition to the ISO standard, which opens streams in text mode.
This makes no difference in UNIX or LINUX, for which text and binary
streams are identical, but does make a difference in Windows, where
text files are processed more than binary files.

% TLS: how about floats?
% public/private

Most other differences with the core standard are mentioned under
portability notes for the various predicates.  However, we do not
always note when there are deviations in the types of errors reported
by predicates.  While XSB throws errors whenever the standard throws
errors, and vice-versa sometimes XSB may throw a different error than
specified by the standard.  This is often due to an effort in XSB to
minimize the amount of error checking performed, and should rarely if
ever cause a problem for applciations.

XSB supports most new features mentioned in the revisions to the core
standard~\cite{ISO-Revision}, including {\tt call\_cleanup/2} and
various library predicates such as {\tt subsumes/2}, {\tt
  numbervars/3} and so on.  XSB also has strong support for the
working multi-threading Prolog standard~\cite{Prolog-MT-ISO}, and XSB
has been one of the first Prologs to support this standard.  However,
because XSB has an atom-based module system it does {\em not} support
the ISO standard for Prolog modules.

%%% Local Variables: 
%%% mode: latex
%%% TeX-master: "manual1"
%%% End: 
