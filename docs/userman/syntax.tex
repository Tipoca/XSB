\chapter{Syntax} \label{Syntax}
%==============================

The syntax of XSB is taken from C-Prolog with extensions to support
HiLog~\cite{ChKW93}~\footnote{Sporadic attempts are made to make XSB
ISO-compliant, contact us if you have a problem with syntax.}, which
adds certain features of second-order syntax to Prolog.

\section{Terms} \label{TermSyntax}
%=================================
The data objects of the HiLog language are called {\em terms}.
A {\em HiLog term} can be constructed from any logical symbol or a term
followed by any finite number of arguments.  In any case, a {\em term}
is either a {\em constant}, a {\em variable}, or a {\em compound term}.

A {\em constant} is either a {\em number} (integer or floating-point) or an
{\em atom}.  Constants are definite elementary objects, and correspond to
proper nouns in natural language.


\subsection{Integers}
The printed form of an integer in HiLog consists of a sequence of digits
optionally preceded by a minus sign ({\tt '-'}).  These are normally 
interpreted as base $10$ integers.  It is also possible to enter integers
in other bases ($2$ through $36$); this can be done by preceding the digit
string by the base (in decimal) followed by an apostrophe ({\tt '}).  If a
base greater than $10$ is used, the characters {\tt A-Z} or {\tt a-z} are
used to stand for digits greater than $9$.

Using these rules, examples of valid integer representations in XSB are:
\begin{verbatim}
           1    -3456    95359    9'888    16'1FA4    -12'A0    20'
\end{verbatim}
representing respectively the following integers in decimal base:
\begin{verbatim}
           1    -3456    95359     728       8100      -120      0
\end{verbatim}

Note that the following:
\begin{verbatim}
                   +525     12'2CF4     37'12     20'-23
\end{verbatim}
are not valid integers of XSB.

A base of $0$ (zero) will return the ASCII code of the (single) character
after the apostrophe; for example,
\begin{verbatim}
                                 0'A = 65
\end{verbatim}


\subsection{Floating-point Numbers}
A HiLog floating-point number consists of a sequence of digits with an
embedded decimal point, optionally preceded by a minus sign ({\tt '-'}), and
optionally followed by an exponent consisting of uppercase or lowercase
{\tt 'E'} and a signed base $10$ integer.

Using these rules, examples of HiLog floating point numbers are:
\begin{verbatim}
              1.0    -34.56    817.3E12    -0.0314e26    2.0E-1
\end{verbatim}
Note that in any case there must be at least one digit before, and one digit
after, the decimal point.


\subsection{Atoms}
A HiLog atom is identified by its name, which is a sequence of up to 1000
characters (other than the null character).  Just like a Prolog atom, a HiLog
atom can be written in any of the following forms:
\begin{itemize}
\item Any sequence of alphanumeric characters (including {\tt '\_'}), starting
      with a lowercase letter.

\item Any sequence from the following set of characters (except of the
      sequence {\tt '/*'}, which begins a comment):
      \begin{verbatim}
                     + - * / \ ^ < > = ` ~ : . ? @ # &
      \end{verbatim}

\item Any sequence of characters delimited by single quotes, such as:
      \begin{verbatim}
                         'sofaki'    '%'    '_$op'
      \end{verbatim}
      If the single quote character is to be included in the sequence it must
      be written twice. For example:
      \begin{verbatim}
                             'don''t'      ''''
      \end{verbatim}

\item Any of the following:
      \begin{verbatim}
                                 !  ;  []  {}
      \end{verbatim}
      Note that the bracket pairs are special. While {\tt '[]'} and
      {\tt '$\{\}$'} are atoms, {\tt '['}, {\tt ']'}, {\tt '$\{$'},
      and  {\tt '$\}$'} are not.
      Like Prolog, the form {\tt [X]} is a special notation for lists
      (see Section~\ref{Lists}), while the form {\tt $\{$X$\}$} is
      just ``syntactic sugar'' for the term {\tt '$\{\}$'(X)}.
\end{itemize}

Examples of HiLog atoms are:
\begin{verbatim}
               h   foo   ^=..   ::=   'I am also a HiLog atom'   []
\end{verbatim}


\subsection{Variables}
Variables may be written as any sequence of alphanumeric characters
(including {\tt '\_'}) beginning with either a capital letter or {\tt '\_'}.
For example:
\begin{verbatim}
                      X   HiLog   Var1   _3   _List
\end{verbatim}

If a variable is referred to only once in a clause, it does not need to be
named and may be written as an {\em anonymous variable}, represented by a
single underscore character {\tt '\_'}.  Any number of anonymous variables
may appear in a HiLog clause; all of these variables are read as distinct
variables.  Anonymous variables are not special at runtime.


\subsection{Compound Terms}
Like in Prolog, the structured data objects of HiLog are {\em compound terms}
(or {\em structures}).  The external representation of a HiLog compound term
comprises a {\em functor} (called the {\em principal functor} or the
{\em name} of the compound term) and a sequence of one or more terms called
{\em arguments}.  Unlike Prolog where the functor of a term must be an atom,
in HiLog the functor of a compound term {\em can be any valid HiLog term}.
This includes numbers, atoms, variables or even compound terms.  Thus, since
in HiLog a compound term is just a term followed by any finite number of
arguments, all the following are valid external representations of HiLog
compound terms: 
\label{some_compound_terms}
\begin{verbatim}
          foo(bar)             prolog(a, X)              hilog(X)       
       123(john, 500)        X(kostis, sofia)         X(Y, Z, Y(W))   
      f(a, (b(c))(d))       map(double)([], [])     h(map(P)(A, B))(C)
\end{verbatim}

Like a functor in Prolog, a functor in HiLog can be characterized by
its {\em name} and its {\em arity} which is the number of arguments this
functor is applied to.  For example, the compound term whose principal functor
is {\tt 'map(P)'} of arity 2, and which has arguments {\tt L1}, and {\tt L2},
is written as:
\begin{verbatim}
                            map(P)(L1, L2)
\end{verbatim}

As in Prolog, when we need to refer explicitly to a functor we will normally
denote it by the form $Name/Arity$.  Thus, in the previous example, the functor
{\tt 'map(P)'} of arity 2 is denoted by:
\begin{verbatim}
                                 map(P)/2
\end{verbatim}
Note that a functor of arity 0 is represented as an atom.

In Prolog, a compound term of the form $p(t_1, t_2, \ldots, t_k)$ is usually
pictured as a tree in which every node contains the name $p$ of the functor
of the term and has exactly $k$ children each one of which is the root of the
tree of terms $t_1, t_2, \ldots, t_k$.

For example, the compound term
\begin{verbatim}
                 s(np(kostis), vp(v(loves), np(sofia)))
\end{verbatim}
would be pictured as the following tree:

\begin{minipage}{4.0in}
\begin{verbatim}
                                  s
                                /   \
                             np       vp
                             |       /  \
                             |      v     np
                             |      |     |
                          kostis  loves  sofia
\end{verbatim}
\end{minipage}

\noindent
The principal functor of this term is {\tt s/2}.  Its two arguments are also
compound terms.  In illustration, the principal functor of the second
argument is {\tt vp/2}.

Likewise, any external representation of a HiLog compound term
$t(t_1, t_2, \ldots, t_k)$ can be pictured as a tree in which every node
contains the tree representation of the name $t$ of the functor of the term
and has exactly $k$ children each one of which is the root of the tree of
terms $t_1, t_2, \ldots, t_k$.

Sometimes it is convenient to write certain functors as {\em operators}.
{\em Binary functors} (that is, functors that are applied to two arguments)
may be declared as {\em infix operators}, and {\em unary functors} (that is,
functors that are applied to one argument) may be declared as either 
{\em prefix or postfix operators}.  
Thus, it is possible to write the following:
\begin{verbatim}
                    X+Y     (P;Q)     X<Y      +X     P;
\end{verbatim}
More about operators in HiLog can be found in section~\ref{Operators}.


\subsection{Lists}\label{Lists}
As in Prolog, lists form an important class of data structures in HiLog.
They are essentially the same as the lists of Lisp: a list is either the atom
{\tt '[]'}, representing the empty list, or else a compound term with functor
{\tt '.'}  and two arguments which are the head and tail of the list
respectively, where the tail of a list is also a list.
Thus a list of the first three natural numbers is the structure:
\begin{verbatim}
                                  .
                                 / \
                                1    .
                                    / \
                                   2    .
                                       / \
                                      3   []
\end{verbatim}
which could be written using the standard syntax, as:
\begin{verbatim}
                             .(1,.(2,.(3,[])))
\end{verbatim}
but which is normally written in a special list notation, as:
\begin{verbatim}
                                  [1,2,3]
\end{verbatim}
Two examples of this list notation, as used when the tail of a list is a
variable, are:
\begin{verbatim}
                       [Head|Tail]      [foo,bar|Tail]
\end{verbatim}
which represent the structures:
\begin{verbatim}
                            .                .
                           / \              / \
                       Head   Tail        foo   .
                                               / \
                                             bar  Tail
\end{verbatim}
respectively.

Note that the usual list notation {\tt [H|T]} does not add any new power
to the language; it is simply a notational convenience and improves
readability. The above examples could have been written equally well as:
\begin{verbatim}
                      .(Head,Tail)      .(foo,.(bar,Tail))
\end{verbatim}

For convenience, a further notational variant is allowed for lists of
integers that correspond to ASCII character codes.  Lists written in this
notation are called {\em strings}.  For example,
\begin{verbatim}
                            "I am a HiLog string"
\end{verbatim}
represents exactly the same list as:
\begin{verbatim}
    [73,32,97,109,32,97,32,72,105,76,111,103,32,115,116,114,105,110,103]
\end{verbatim}


\section{From HiLog to Prolog} \label{HiLog2Prolog}
%==================================================
From the discussion about the syntax of HiLog terms, it is clear that the
HiLog syntax allows the incorporation of some higher-order constructs in
a declarative way within logic programs.  As we will show in this section,
HiLog does so  while retaining a clean first-order declarative semantics.
The semantics of HiLog is first-order, because every HiLog term (and formula)
is automatically {\em encoded (converted)} in predicate calculus in the way
explained below.

Before we briefly explain the encoding of HiLog terms, let us note that the 
HiLog syntax is a simple (but notationally very convenient) encoding for Prolog
terms, of some special form.  In the same way that in Prolog:
\begin{center}
{\tt	1 + 2}
\end{center}
is just an (external) shorthand for the term:
\begin{center}
{\tt  +(1, 2)} 
\end{center}
in the presence of an infix operator declaration for {\tt +} 
(see section~\ref{Operators}), so:
\begin{center}
{\tt  X(a, b)}
\end{center}
is just an (external) shorthand for the Prolog compound term:
\begin{center}
{\tt    apply(X, a, b)}
\end{center}
Also, in the presence of a {\tt hilog} declaration (see
section~\ref{other-directives}) for {\tt h}, the HiLog term whose external
representation is:
\begin{center}
{\tt  h(a, h, b)} 
\end{center}
is a notational shorthand for the term:
\begin{center}
{\tt apply(h, a, h, b)}
\end{center}
Notice that even though the two occurrences of {\tt h} refer to the same 
symbol, only the one where {\tt h} appears in a functor position is encoded
with the special functor {\tt apply/}$n, n \geq 1$.

The encoding of HiLog terms is performed based upon the existing declarations
of {\em hilog symbols}.  These declarations (see section~\ref{other-directives}),
determine whether an atom that appears in a functor position of an external 
representation of a HiLog term, denotes a functor or the first argument of a 
set of special functors {\tt apply}.  The actual encoding is as follows:
\begin{itemize}
\item	The encoding of any variable or parameter symbol (atom or number) that
	does not appear in a functor position is the variable or the symbol
	itself.
\item	The encoding of any compound term {\tt t} where the functor {\em f}
	is an atom that is not one of the hilog symbols (as a result of a
	previous {\tt hilog} declaration), is the compound term that has
	{\em f} as functor and has as arguments the encoding of the arguments
	of term {\em t}.  Note that the arity of the compound term that results
	from the encoding of {\em t} is the same as that of {\em t}.
\item	The encoding of any compound term {\tt t} where the functor {\em f}
	is either not an atom, or is an atom that is a hilog symbol, is a 
	compound term that has {\tt apply} as functor, has first argument
	the encoding of {\em f} and the rest of its arguments are obtained
	by encoding of the arguments of term{\em t}.  Note that in this case
	the arity of the compound term that results from the encoding of
	{\em t} is one more than the arity of {\em t}.
\end{itemize}

Note that the encoding of HiLog terms described above, implies that even
though the HiLog terms:
\begin{center}
\begin{minipage}{1.0in}
\begin{verbatim}
	p(a, b)
	h(a, b)
\end{verbatim}
\end{minipage}
\end{center}
externally appear to have the same form, in the presence of a {\tt hilog}
declaration for {\tt h} but not for {\tt p}, they are completely different.
This is because these terms are shorthands for the terms whose internal 
representation is: 
\begin{center}
\begin{minipage}{1.2in}
\begin{verbatim}
	    p(a, b)
	apply(h, a, b)
\end{verbatim}
\end{minipage}
\end{center}
respectively.  Furthermore, only {\tt h(a,b)} is unifiable with the HiLog term
whose external representation is {\tt X(a, b)}.

We end this short discussion on the encoding of HiLog terms with a small
example that illustrates the way the encoding described above is being done.
Assuming that the following declarations of parameter symbols have taken place,
\begin{center}
\begin{minipage}{1.6in}
\begin{verbatim}
 :- hilog h.
 :- hilog (hilog).
\end{verbatim}
\end{minipage}
\end{center}
before the compound terms of page~\pageref{some_compound_terms} were
read by XSB, the encoding of these terms in predicate calculus using
the described transformation is as follows:
\begin{center}
\begin{minipage}{4.5in}
\begin{verbatim}
      foo(bar)                    prolog(a,X) 
   apply(hilog,X)             apply(123,john,500)
apply(X,kostis,sofia)       apply(X,Y,Z,apply(Y,W))
  f(a,apply(b(c),d))       apply(map(double),[],[])  
        apply(apply(h,apply(map(P),A,B)),C)
\end{verbatim}
\end{minipage}
\end{center}


\section{Operators} \label{Operators}
%====================================
From a theoretical point of view, operators in Prolog are simply a notational
convenience and add absolutely nothing to the power of the language.
For example, in most Prologs {\tt '+'} is an infix operator, so
\begin{verbatim}
                                 2 + 1
\end{verbatim}
is an alternative way of writing the term {\tt +(2, 1)}.  That is, {\tt 2 + 1}
represents the data structure:
\begin{verbatim}
                                   +
                                  / \
                                 2   1
\end{verbatim}
and not the number 3.  (The addition would only be performed if the structure
were passed as an argument to an appropriate procedure, such as {\tt is/2}).
% tls we havent done this yet. described in section~\ref{Arithmetic}).

However, from a practical or a programmer's point of view, the existence of
operators is highly desirable, and clearly handy.

Prolog syntax allows operators of three kinds: {\em infix}, {\em prefix}, and
{\em postfix}.  An {\em infix} operator appears between its two arguments,
while a {\em prefix} operator precedes its single argument and a {\em postfix}
operator follows its single argument.

Each operator has a precedence, which is an integer from 1 to 1200.  The
precedence is used to disambiguate expressions in which the structure of the
term denoted is not made explicit through the use of parentheses.  The
general rule is that the operator with the highest precedence is the
principal functor.  Thus if {\tt '+'} has a higher precedence than {\tt '/'},
then the following
\begin{verbatim}
                           a+b/c     a+(b/c)
\end{verbatim}
are equivalent, and both denote the same term {\tt +(a,/(b,c))}. Note that
in this case, the infix form of the term {\tt /(+(a,b),c)} must be written
with explicit use of parentheses, as in:
\begin{verbatim}
                               (a+b)/c
\end{verbatim}

If there are two operators in the expression having the same highest
precedence, the ambiguity must be resolved from the {\em types} (and 
the implied {\em associativity}) of the operators.  The possible types
for an infix operator are
\begin{verbatim}
                          yfx     xfx     xfy
\end{verbatim}
Operators of type {\tt 'xfx'} are not associative.  Thus, it is required that
both of the arguments of the operator must be subexpressions of lower
precedence than the operator itself; that is, the principal functor of each
subexpression must be of lower precedence, unless the subexpression is written
in parentheses (which automatically gives it zero precedence).

Operators of type {\tt 'xfy'} are {\em right-associative}:  only the
first (left-hand) subexpression must be of lower precedence; the right-hand
subexpression can be of the same precedence as the main operator.
{\em Left-associative} operators (type {\tt 'yfx'}) are the other way around.

An atom named {\tt Name} can be declared as an operator of type {\tt
Type} and precedence {\tt Precedence} by the command;

\begin{description}
\isoitem{op(+Precedence,+Type,+Name)}{op/3}
\end{description}
%
The same command can be used to redefine one of the predefined XSB
operators (obtainable via {\tt current\_op/3}).  However, it is not
allowed to alter the definition of the comma ({\tt ','}) operator.  An
operator declaration can be cancelled by redeclaring the {\tt Name}
with the same {\tt Type}, but {\tt Precedence} 0.

As a notational convenience, the argument {\tt Name} can also be a list of
names of operators of the same type and precedence.

It is possible to have more than one operator of the same name, so
long as they are of different kinds: infix, prefix, or postfix.  An
operator of any kind may be redefined by a new declaration of the same
kind.  For example, the built-in operators {\tt '+'} and {\tt '-'} are
as if they had been declared by the command:
\begin{verbatim}
                       :- op(500, yfx, [+,-]).
\end{verbatim}
so that:
\begin{verbatim}
                                  1-2+3
\end{verbatim}
is valid syntax, and denotes the compound term:
\begin{verbatim}
                                 (1-2)+3
\end{verbatim}
or pictorially:
\begin{verbatim}
                                    +
                                   / \
                                  -   3
                                 / \
                                1   2
\end{verbatim}

In XSB, the list functor {\tt '.'/2} is one of the standard operators,
that can be thought as declared by the command:
\begin{verbatim}
                          :- op(661, xfy, .).
\end{verbatim}
So, in XSB,
\begin{verbatim}
                                  1.2.[]
\end{verbatim}
represents the structure
\begin{verbatim}
                                    .
                                   / \
                                  1   .
                                     / \
                                    2   []
\end{verbatim}
Contrasting this picture with the picture above for {\tt 1-2+3} shows
the difference between {\tt 'yfx'} operators where the tree grows to
the left, and {\tt 'xfy'} operators where it grows to the right.  The
tree cannot grow at all for {\tt 'xfx'} type operators.  It is simply
illegal to combine {\tt 'xfx'} operators having equal precedences in
this way.

If these precedence and associativity rules seem rather complex, remember
that you can always use parentheses when in any doubt.

In \version{} of XSB the possible types for prefix operators are:
\begin{verbatim}
                      fx       fy       hx       hy
\end{verbatim}
and the possible types for postfix operators are:
\begin{verbatim}
                               xf       yf
\end{verbatim}

We end our discussion about operators by mentioning that prefix
operators of type {\tt hx} and {\tt hy} are {\em proper HiLog
  operators}.  The discussion of proper HiLog operators and their
properties is deferred for the manual of a future version.

