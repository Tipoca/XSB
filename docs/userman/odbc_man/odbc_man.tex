\chapter{XSB-ODBC Interface} \label{odbc_interface}
%====================================================

\begin{center}
{\Large {\bf By Baoqiu Cui and Lily Dong }}
\end{center}

\section{Introduction}
%=====================

The XSB-ODBC interface is the PC platform counterpart to XSB-Oracle
interface on UNIX systems. It allows XSB users to access data in any
ODBC compliant database management system (DBMS). Using this uniform
interface, information in different DBMS's can be accessed as though it
existed as Prolog facts. Similar to its counterpart on UNIX platforms,
XSB-ODBC interface provides users with two levels of interaction: a {\it
relation level} and a {\it view level}.  The former offers a
tuple-at-a-time retrieval of information from ODBC data sources while
the latter can translate an entire Prolog clause into a single SQL
query, including joins and aggregate operations, which gives XSB users
all the features of Prolog as a query language such as intentional
database specification, recursion etc.  A listing of the features that
XSB-ODBC interface provides is as follows:
\begin{itemize} 
\item Concurrent access from multiple XSB processes to a single DBMS
\item Full data access and cursor transparency including support for
        \begin{itemize}
        \item Full data recursion through XSB's tabling mechanism
        \item Runtime type checking
        \item Automatic handling of NULL values for insertion, 
                deletion and querying
        \item Partial recovery for cursor losses due to cuts
        \end{itemize}
\item Full access to data source including
        \begin{itemize}
        \item Transaction support
        \item Cursor reuse for cached SQL statements 
                with bind variables (by avoiding re-parsing and re-declaring).
        \item Caching compiler generated SQL statements with bind variables 
                and efficient cursor management for cached statements
        \end{itemize}
\item A powerful Prolog / SQL compiler based on \cite{Drax92}.
\item Full source code availability
\item Independence from database schema by employing a {\it relation level}
\item Performance as SQL by employing a {\it view level} 
\item No mode specification is required for optimized view compilation
\end{itemize}

We use the {\tt Hospital} database as our example to illustrate 
the usage of XSB-ODBC interface in this manual. We assume the basic 
knowledge of Microsoft ODBC interface and its ODBC administrator 
throughout the text.  Please refer to ``Inside Windows$^{TM}$ 95''
 for information on this topic.

\section{Using the Interface}
%========================================

The XSB-ODBC module has to be loaded before the interface can be used.  
To load it, type in ``{\tt [odbc\_call].}'' at the XSB prompt.

\subsection{Connecting to and Disconnecting from Data Sources}
%=================================================

Assuming that the data source to be connected to is available, i.e. it has an 
entry in {\tt ODBC.INI} file which can be checked by running Microsoft 
ODBC Administrator, it can be connected to in the following way: 

\begin{center}
{\tt  | ?- odbc\_open(data\_source\_name, username, passwd).}
\end{center}

If the connection is successful, the system will give a positive response of 
{\tt yes}.  This step is necessary before anything can be done with the data 
sources since it gives XSB the opportunity to initialize system resources for 
the session.

To close the current session use:
\begin{center}
{\tt  | ?- odbc\_close.}
\end{center}

and XSB will give all the resources  it allocated for this session back to the
system.

\subsection{Accessing Tables in Data Sources}
%=======================================

There are several ways that can be used to extract information from or modify 
a table in a data source.  Users can access a table using the relation level interface 
or view level interface which XSB provides or they can have their SQL statements 
executed directly without having XSB process it.  In general,  uses are required 
to firstly use {\tt odbc\_import/2} to give XSB the information about columns 
in the table of interest, expect for the cases such as direct execution of SQL 
statements and data dictionary operations, etc. and XSB will automatically assert 
data dictionary information$($some other ODBC interface predicates can 
cause XSB to do this too$)$. The Prolog predicate {\tt listing/2} can be used
$($if it's available$)$ to see the asserted data dictionary information at any time.  

The syntax of  {\tt odbc\_import/2} is as follows:

\begin{center}
{\tt | ?- odbc\_import('TableName'('FIELD1', 'FIELD2', ..., 'FIELDn'), 'PredicateName').}
\end{center}

where {\tt 'TableName'} is the name of the table that is desired for accessing
and {\tt 'PredicateName'} is the name of the predicate for future table operations 
from XSB. {\tt 'FIELD1'}, {\tt 'FIELD2'}, ... , {\tt 'FIELDn'} are
the exact attribute names$($case sensitive$)$ as defined in the table schema.  The
chosen columns define the view and the order of arguments for the
database predicate {\tt 'PredicateName'}.  
 
For example, to create a link to the {\tt  Test} table through the {\tt 'test'} predicate:
\begin{verbatim}
| ?- odbc_import('Test'('TId','TName','Length','Price'),test).

yes
\end{verbatim}

\subsection{Using the Relation Level Interface}
%===========================================

Once the links between tables and predicates have been successfully established, 
information can then be extracted from these tables using the corresponding 
predicates.   Continuing from the above example, now rows from the table 
{\tt Test} can be obtained:
 \begin{verbatim}

| ?- test(TId, TName, L, P).

TId = t001
TName = X-Ray
L = 5
P = 100 
\end{verbatim}

Backtracking can then be used to retrieve the next row of the table {\tt Test}.

Records with particular field values may be selected in the same way
as in Prolog; no mode specification for database predicates is
required. For example:
\begin{center}

{\tt | ?- test(TId, 'X-Ray', L, P).}
\end{center}
generates the query:
\begin{verbatim}

SELECT rel1.TId, rel1.TName, rel1.Length, rel1.Price
FROM Test rel1
WHERE rel1.TName = ?
;
\end{verbatim}
and 
\begin{center}

{\tt | ?- test('NULL'(\_), 'X-Ray',  L, P).}
\end{center}

generates: (See Section \ref{NULL-values})

\begin{verbatim}
SELECT NULL , rel1.TName, rel1.Length, rel1.Price
FROM Test rel1
WHERE rel1.TId IS NULL AND rel1.TName = ?
;
\end{verbatim}

During the execution of this query the bind variable {\tt ?} will be bound
to {\tt 'X-Ray'}.\newline

Note that if a field includes a quote $(')$ then this should be represented by
using two quotes.

Also as a courtesy to Quintus Prolog users we have provided
compatibility support for some PRODBI predicates which access tables
at a relational level.

\begin{verbatim}
i) | ?- odbc_attach(PredicateName, table(TableName)).
\end{verbatim}

eg. invoke 
\begin{center}
{\tt | ?- odbc\_attach(test2, table('Test')).} 
\end{center}
and then execute 
\begin{center}
{\tt | ?- test2(TId, TName, L, P).}
\end{center}
to retrieve the rows.
\begin{verbatim}

ii) | ?- odbc_record('Test', R).

    R = [t001, X-Ray, 5, 100];

    R = ...

\end{verbatim}
    You can use {\tt odbc\_record/2} to treat the whole database row as a 
single list structure.

\subsection{The View Level Interface}
%===========================================

The view level interface can be used for the definition of rules
whose bodies includes only imported database predicates (by using the
relation level interface) described above and aggregate predicates
(defined below).  When they are invoked, rules are translated into complex
database queries, which are then executed taking advantage of the query
processing ability of the DBMS's.

One can use the view level interface through the predicate {\tt odbc\_query/2}:  
\begin{center}
{\tt | ?- odbc\_query('RuleName'(ARG1, ..., ARGn), DatabaseGoal).}
\end{center}
All arguments are standard Prolog terms.  {\tt ARG1}, {\tt ARG2}, ...,
{\tt ARGn} defines the attributes to be retrieved from the database,
while {\tt DatabaseGoal} defines the selection restrictions and join
conditions.

The compiler is a simple extension of \cite{Drax92} which generates SQL
queries with bind variables and handles NULL values as described in
Section \ref{NULL-values}.  It allows negation, the expression
of arithmetic functions, and higher-order constructs such as grouping,
sorting, and aggregate functions.

Database goals are translated according to the following rules
from \cite{Drax92}:
\begin{itemize}
\item Disjunctive goals translate to distinct SQL queries
        connected through the UNION operator.
\item Goal conjunctions translate to joins.
\item Negated goals translate to negated EXISTS subqueries.
\item Variables with single occurrences in the body are not
          translated.
\item Free variables translate to grouping attributes.
\item Shared variables in goals translate to equi-join conditions.
\item Constants translate to equality comparisons of an attribute and
          the constant value.
\item Nulls are translated to {\tt IS NULL} conditions.
\end{itemize}
For more examples and implementation details see \cite{Drax92}.
 
In the following, we show the definition of a simple join view between the 
two database predicates {\it Room} and {\it Floor}.

Assuming the declarations:
\begin{verbatim}

| ?- odbc_import('Room'('RoomNo','CostPerDay','Capacity','FId'),room).

| ?- odbc_import('Floor'('FId','','FName'),floor).
\end{verbatim}

use

\begin{verbatim}

| ?- odbc_query(rule1(RoomNo,FName),
                  (room(RoomNo,_,_,FId),floor(FId,_,FName))).
yes

| ?- rule1(RoomNo,FloorName).
\end{verbatim}

Prolog/SQL compiler generates the SQL statement:
\begin{verbatim}

SELECT rel1.RoomNo , rel2.FName FROM Room rel1 , Floor rel2 
WHERE rel2.FId = rel1.FId;

RoomNo = 101
FloorName = First Floor
\end{verbatim}

Backtracking can then be used to retrieve the next row of the view.
\begin{center}

{\tt | ?- rule1('101','NULL'(\_)).}
\end{center}

generates the SQL statement:
\begin{verbatim}

SELECT rel1.RoomNo, NULL
FROM Room rel1 , Floor rel2
WHERE rel1.RoomId = ? AND rel2.FId = rel1.FId AND rel2.FName IS NULL;
\end{verbatim}

The view interface also supports aggregate functions predicates such as sum, avg,
count, min and max.  For example
\begin{verbatim}

| ?- odbc_import('Doctor'('DId', 'FId', 'DName','PhoneNo','ChargePerMin'),doctor).

yes
| ?- odbc_query(avgchargepermin(X),
                (X is avg(ChargePerMin, A1 ^ A2 ^ A3 ^ A4 ^ 
                          doctor(A1,A2, A3,A4,ChargePerMin)))).

yes
| ?- avgchargepermin(X).

SELECT AVG(rel1.ChargePerMin)
FROM doctor rel1;

X = 1.64

yes
\end{verbatim}


A more complicated example:
\begin{verbatim}

| ?- odbc_query(nonsense(A,B,C,D,E),
                (doctor(A, B, C, D, E), 
                 not floor('First Floor', B), 
                 not (A = 'd001'), 
                 E > avg(ChargePerMin, A1 ^ A2 ^ A3 ^ A4 ^
                         (doctor(A1, A2, A3, A4, ChargePerMin))))).

| ?- nonsense(A,'4',C,D,E).

SELECT rel1.DId , rel1.FId , rel1.DName , rel1.PhoneNo , rel1.ChargePerMin
FROM doctor rel1
WHERE rel1.FId = ? AND NOT EXISTS
(SELECT *
FROM Floor rel2
WHERE rel2.FName = 'First Floor' and rel2.FId = rel1.FId
) AND rel1.Did <> 'd001' AND rel1.ChargePerMin >
(SELECT AVG(rel3.ChargePerMin)
FROM Doctor rel3
);


A = d004
C = Tom Wilson
D = 516-252-100
E = 2.5
\end{verbatim}

All database rules defined by odbc\_query can be queried with any mode.

Note that at each call to a database relation or rule, the
communication takes place through bind variables.  The corresponding
restrictive SQL query is generated, and if this is the first call with
that adornment, it is cached.  A second call with same adornment would
try to use the same database cursor if still available, without
parsing the respective SQL statement.  Otherwise, it would find an
unused cursor and retrieve the results.  In this way efficient access
methods for relations and database rules can be maintained throughout
the session.

Also the relation level interface can be used to define and
access simple project views of single tables.  For example:
\begin{center}

{\tt | ?- odbc\_import('Room'('RoomNo','Capacity'),roomview).}
\end{center}
defines {\tt roomview/2}.

\subsection{Insertions and Deletions of Rows}
%=========================================

Insertion and deletion operations can also be performed on an imported
table.  The two predicates to accomplish these operations are 
{\tt odbc\_insert/2} and {\tt odbc\_delete/2}.  The syntax of 
{\tt odbc\_insert/2} is as follows: the first argument is the declared 
database predicate for insertions and the second argument
is some imported data source relation.  The second argument can be 
declared with some of its arguments bound to constants.  For example 
after {\tt Room} is imported through {\tt odbc\_import}:
\begin{verbatim}

|?- odbc_import('Room'('RoomNo','CostPerDay','Capacity','FId'), room).
yes 
\end{verbatim}

Now we can do

\begin{verbatim}
| ?- odbc_insert(room_ins(A1,A2,A3),(room(A1,A2,A3,'3'))).

yes
| ?- room_ins('306','NULL'(_),2).

yes
\end{verbatim}
This will insert the row: ('306',NULL, 2,'3') into the table {\tt Room}. Note that 
any call to {\tt room\_ins/7} should have all its arguments bound.

See Section \ref{NULL-values}) for information about NULL value
handling.

The first argument of {\tt odbc\_delete/2} predicate is the declared delete
predicate and the second argument is the imported data source relation
with the condition for requested deletes, if any.  The condition is
limited to simple comparisons.  For example assuming
{\tt Room/3} has been imported as above:
\begin{verbatim}

| ?- odbc_delete(room_del(A), (room('306',A,B,C), A > 2)). 

yes
\end{verbatim}

After this declaration you can use:
\begin{verbatim}

| ?- room_del(3).
\end{verbatim}

to generate the SQL statement:
\begin{verbatim}

DELETE From Room rel1 
WHERE rel1.RoomNo = '306' AND rel1.CostPerDay = ? AND ? > 2
;
\end{verbatim}

Note that you have to commit your inserts or deletes to tables to make
them permanent.  (See section \ref{TransactionManagement}).

\subsection{ Direct Execution of SQL statements}
%==========================================
It is also possible to execute SQL statements directly. 
{\tt odbc\_sql\_select/2} and {\tt odbc\_sql/1} predicates provide this 
feature.  The former takes an SQL query string as its input and returns 
a list of field values.  For example:
\begin{verbatim}

| ?- odbc_sql_select('SELECT * FROM Test', R).

R = [t001, X-Ray, 5, 100];

...

\end{verbatim}
The latter, {\tt odbc\_sql/1}, can be used for any other non-query SQL statement request:
\begin{verbatim}

| ?- odbc_sql('Create Table MyTable( Column1 DataType1, Column2 DataType2)').

yes
\end{verbatim}

\subsection{Access to Data Dictionaries}
%======================================

The following utility predicates provide users the tools to access data 
dictionaries.  Users of Quintus Prolog may note that these predicates are all 
PRODBI compatible.  A brief description of these predicates is as follows:
\begin{description}

\item[odbc\_show\_schema(accessible)]
         Shows all accessible table names for the user.  This list can be long!

\item[odbc\_show\_schema(user)]
        Shows just those tables that belongs to user.

\item[odbc\_show\_schema(tuples('Table'))]
        Shows the contents of the base table named {\tt 'Table'}.

\item[odbc\_show\_schema(arity('Table'))]
        The number of fields in the table {\tt 'Table'}.

\item[odbc\_show\_schema(columns('Table'))]
        The field names of a table.
\end{description}

For retrieving above information use:
\begin{itemize}

\item odbc\_get\_schema(accessible,List)
\item odbc\_get\_schema(user,List)
\item odbc\_get\_schema(arity('Table'),List)
\item odbc\_get\_schema(columns('Table'),List)
\end{itemize}

The results of above are returned in List as a list.

\subsection{Other Database Operations}

\begin{description}

\item[odbc\_create\_table('TableName','FIELDs')]
        {\tt FIELDS} is the field specification as in SQL.
\begin{verbatim}

eg. odbc_create_table('MyTable', 'Col1 NUMBER,
                             Col2 TEXT(50),
                             Col3 TEXT(13)').
\end{verbatim}


\item[odbc\_create\_index('TableName','IndexName', index(\_,Fields))]
        {\tt Fields} is the list of columns for which an index
        is requested.  For example:
\begin{center}

{\tt odbc\_create\_index('Doctor', 'DocKey', index(\_,'DId')).}
\end{center}

\item[odbc\_delete\_table('TableName')] To delete a table named {\tt 'TableName'}

\item[odbc\_delete\_view('ViewName')] To delete a view named {\tt 'ViewName'}

\item[odbc\_delete\_index('IndexName')] To delete an index named {\tt 'IndexName'}
\end{description}

These following predicates are the supported PRODBI syntax for deleting and 
inserting rows:
\begin{description}

\item[\mbox{odbc\_add\_record('Floor',['Seventh Floor','7'])}]
         arguments are a list composed of field values and the table name to
         insert the row.

\item[\mbox{odbc\_delete\_record('Floor', ['Seventh Floor',\_])}]
         to delete rows from {\tt 'Floor'} matching the list of values mentioned in
         second argument. 
\end{description}

For other SQL statements use {\tt odbc\_sql/1} with the SQL statement as the 
     first argument.  For example:
\begin{center}
{\tt odbc\_sql('grant connect to fred identified by bloggs')).}
\end{center}

\subsection{Transaction Management}\label{TransactionManagement}
%===================================================
Depending on how the transaction options are set in ODBC.INI for data sources, 
any changes to the data source tables may not be committed$($changes become 
permanent$)$  until the user expicitly issues a commit statement. 
The predicate {\tt odbc\_transaction/1} is provided in this sense. 
\begin{description}
\item[odbc\_transaction(commit)]
        Commits all transactions up to this point.
\item[odbc\_transaction(rollback)]
        Rolls back all transactions$($discard the changes made$)$  since 
the last commit.
\end{description}

\subsection{Handling NULL Values}
%=======================================
Null value is handled in the same way as that of XSB Oracle interface.
Please refer to Section \ref{NULL-values}) for details.

\subsection{Interface Flags}
%====================================

Users are given the option to monitor the SQL queries generated 
by the interface and their execution status by using the 
predicate {\tt db\_flag/3}.  The first parameter indicates the function  to be
changed.  The second argument is the old value, and the third argument specifies
the new value.  For example:
\begin{verbatim}

| ?- odbc_flag(show_query, Old, on).

     Old = off
\end{verbatim}

SQL statements will now be displayed for all SQL queries $($the default$)$.
To turn it off use {\tt odbc\_flag(show\_query,on, off)}.  The default value 
of {\tt show\_query} is on.

To control the error behavior of either the interface or
data sources use {\tt odbc\_flag/3} with fail\_on\_error as first argument.
For example:

\begin{description}

\item[\tt | ?- odbc\_flag(fail\_on\_error, on, off)] Gives all the error control 
to users, hence all requests to data sources return true.  It's users' 
responsibility to check each of their actions and do error handling.

\item[\tt | ?- odbc\_flag(fail\_on\_error, off, on)] Interface fails whenever 
error occurs.
\end{description}

The default value of fail\_on\_error is on.


\subsection{Datalog}
%===================
Users can write recursive Datalog queries with exactly the same
semantics as in XSB using imported database predicates or database
rules.  For example assuming {\tt odbc\_parent/2} is an imported database
predicate, the following recursive query computes its transitive closure.

\begin{verbatim}

:- table(ancestor/2).
ancestor(X,Y) :- odbc_parent(X,Y).
ancestor(X,Z) :- ancestor(X,Y), odbc_parent(Y,Z).
\end{verbatim}


\section{Limitation and Guidelines for Application Developers} \label{Guide}
%=======================================================
Since XSB-ODBC interface is a simulation of XSB-ORACLE interface on UNIX 
platform, it inherits all limitations of the XSB-ORACLE interface, i.e. limited 
number of usable cursors, cursor leaking when using cuts and etc.  Hence 
the guidelines for XSB-ORACLE interface application 
developers are also for XSB-ODBC interface application developers.  
Please refer to Sections \ref{oracle:use} and \ref{oracle:limitations}
for datails. 

\section{Error messages}
%============================
\begin{description}
\item[ERR - DB: Connection failed] For some reason the attempt to connect
        to data source failed.
        \begin{itemize}
        \item   Diagnosis: Try to see if the data source has been registered 
                                  with Microsoft ODBC Administrator, the username and 
                                  password are correct and MAXCURSORNUM is not set 
                                  to a very large number.
        \end{itemize}

\item[ERR - DB: Parse error] The SQL statement generated by the
                          Interface or the first argument to {\tt odbc\_sql/1} or 
                       {\tt odbc\_sql\_select/2} can not be parsed by the data 
                              source driver.
        \begin{itemize}
        \item   Diagnosis: Check the SQL statement.  If our interface
                generated the erroneous statement please contact us at
                {\tt xsb-contact@cs.sunysb.edu}.
        \end{itemize}

\item[ERR - DB: No more cursors left] Interface run out of non-active
cursors either because of a leak (See Section \ref{oracle:use}) or no
more free cursors left. 
\begin{itemize}

\item Diagnosis: System fails always with this error.  odbc\_transaction(rollback) or
           odbc\_transaction(commit) should resolve this by freeing all cursors.
\end{itemize}

\item[ERR - DB: FETCH failed] Normally this error should not occur if the 
interface running properly.
\begin{itemize}

\item Diagnosis: Please contact us at xsb-contact@cs.sunysb.edu
\end{itemize}

\end{description}

